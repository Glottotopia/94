\documentclass[output=paper]{LSP/langsci}
\author{Jimm Goodtracks, Bryan James Gordon, and Saul Schwartz}
\title{Perspectives on Chiwere revitalization}
\abstract{This chapter examines the Ioway, Otoe-Missouria Language Project's Chiwere revitalization efforts from three perspectives. Jimm Goodtracks provides an account of how he came to be involved in Chiwere language preservation in the 1960's and of the shifting strategies for documentation and revitalization that he has employed in subsequent decades. Bryan James Gordon presents a list of phrases in Chiwere that he prepared while living in Jimm's household language nest in order to communicate with Jimm's grandson. Saul Schwartz contextualizes the Project's activities by describing their role within the Ioway and Otoe-Missouria communities and comparing them to strategies for revitalizing other Siouan and Native American languages. What emerges is a sense of language revitalization as a social process that involves personal dedication, support from family and friends, collaboration between linguists and community members, and a sense of responsibility to previous and future generations. KEYWORDS: [language documentation, language revitalization, language revival, language nest, language classes, immersion, dictionary, translation, pedagogical materials, memoir]}

\maketitle

\begin{document}

This chapter describes the Ioway, Otoe-Missouria Language Project's recent Chiwere language revitalization activities. Chiwere is a Siouan language with three historically attested dialects --- Ioway (\emph{Báxoje}), Otoe (\emph{Jiwére}), and Missouria (\emph{\~Nút\^{ }achi})\footnote{\emph{Ioway}, as a term for an American Indian people and their associated language and culture, is also known as \emph{Iowa}, as in the names of the two federally recognized Iowa Tribes. Similarly, \emph{Otoe} is sometimes spelled <Oto>, and \emph{Missouria} can be called \emph{Missouri}. Except where otherwise indicated by angle brackets (<>), all Chiwere words in this chapter are written with the orthography used by the Ioway, Otoe-Missouria Language Project, which is described on the Project's website \citep{GoodtracksND}.} --- and is a heritage language for three federally recognized tribes: the Iowa Tribe of Kansas and Nebraska, the Iowa Tribe of Oklahoma, and the Otoe-Missouria Tribe of Indians, also in Oklahoma. Most authoritative scholarly sources place the last fluent Chiwere speakers in the 1990's \citep{LewisSimonsFennig2013, ParksRankin2001}. Though no fluent speakers have been identified since then, a handful of semispeakers remain, and some Ioways and Otoe-Missourias continue to use Chiwere in certain contexts. Nevertheless, most tribal members have few opportunities to hear Chiwere in their daily lives, and written Chiwere constitutes their primary form of access to the language.

The Ioway, Otoe-Missouria Language Project is a community-based Chiwere documentation and revitalization effort headed by Jimm Goodtracks. Though the Project is not formally affiliated with academic or tribal institutions, it has collaborated with both over the years. The Project's documentary work has been funded by the National Science Foundation's Documenting Endangered Languages program since 2007, first through a grant to the Iowa Tribe of Kansas and Nebraska to prepare a dictionary and then through a grant to the Project itself to support ongoing work on an annotated corpus.\footnote{Work on the dictionary was funded by NSF award number BCS-0553585 \emph{Ioway Otoe-Missouria Dictionary Project}. Work on the corpus is funded by NSF award number BCS-1160665 \emph{Chiwere (ISO 639-3: iow) Audio Archive Project (CAAP)}.}  The Project makes its dictionary, pedagogical materials, and other resources available online through its website \citep{GoodtracksND}.

In addition to these public activities, the Project also supports a more private revitalization effort: Jimm's house is a \textsc{language nest}, in which he raised his grandson, \iai{Sage}, to speak Chiwere as his first language. During the course of their graduate studies in anthropology and linguistics, Bryan James Gordon and Saul Schwartz had the privilege of living with Jimm and his grandson. They participated in the Project's documentary and revitalization activities as well as the household language nest.

Our chapter begins with Jimm's account of how he came to be involved in Chiwere language preservation in the 1960's and of the shifting strategies for documentation and revitalization that he has pursued in subsequent decades. Bryan's section presents a list of phrases in Chiwere that he prepared while living in the language nest in order to facilitate communication with Jimm's grandson. Saul's section seeks to contextualize the Project's activities by describing their role within the Ioway and Otoe-Missouria communities and comparing them to strategies for revitalizing other Siouan and Native American languages. Our hope is that by combining our perspectives, a richer and more complex picture of Chiwere revitalization will emerge. As can be seen from what we have written, we are each indebted in our own ways to \iai{Bob Rankin}, to whose memory this volume is dedicated, for his contributions to our work.

\section{The Ioway, Otoe-Missouria Language Project (Jimm)}
I left Pawnee, Oklahoma, after my graduation from Oklahoma State University in May 1965 to begin my first genuine employment in social welfare and to live as an independent adult. My work in southern Colorado would initiate my course into linguistics and the committed study of our Elders' Báxoje, Jiwére-\~Nút\^{ }achi (`Ioway, Otoe-Mis\-sou\-ria') Language.

I cannot really say when it was that I realized that in southern Colorado I had arrived in an encapsulated leftover corner of early Spanish colonization. The semi-arid desert was irrigated by a system of antiquated hand dug waterways that nourished small family rancheros that surrounded many aldeas (`villages') off the highways and produced lush green fields for small herds of cattle and sheep and individual gardens for the valley residents. The people lived in flat roofed homes made of adobe, each having its own dug well, and some homes had archaic hornos (clay ovens still in use by the Pueblo Tribes in New Mexico). Only in the highway towns, like Alamosa, Antonito, La Jara, and Manasa, could one enjoy a city water system. The area residents referred to themselves in English as \emph{Spanish}, but when speaking their antiquated Spanish, they referred to themselves as \emph{Chicanos} or \emph{la raza}. They were said to be descendants of los conquistadores of some 400 or so years earlier, who in turn married into the Native Peoples (Utes, Apache, and Pueblos) and settled in the mountain valleys, making them their home, while the indigenous people were driven further and deeper into the mountains that are characteristic of the State of Colorado. The area was rich in radio programs in Spanish, Navaho, Taos Pueblo, etc. I was surrounded by non-English speakers everywhere.

During these same years in Oklahoma, the late 1960's, I recall being around many tribal languages usually spoken by the Elders: Pawnee, Otoe, Ioway, O\-sage, Ponca, Sauk, Kiowa, Creek, Cherokee, and Seminole were just a few of the languages heard frequently. There were a few monolinguals who required the assistance of a translator for their Native Language. It was common to hear Ponca, Ioway-Otoe, and Osage spoken and sung in prayer services, ceremonies, and conversation at the various community dances, handgames, and prayer meetings -- both in traditional Native American Church services held in tipis and in the Indian Methodist and Indian Baptist churches. The languages were always present in ceremonials, as in the Iroshka Society, the AsaKipiriru (`Young Dog') Dances, and Pipe Blessings. It was only natural that any of us young people would have learned and used some of the Native Language of the Elders, and yet it was a knowledge all too often taken for granted. It was overspun by the lure of ``fun'' and partying, driving youthful passions, and the need to ``fit in'' for that special interaction that can only be achieved from one's peer group.

Now, as a novice social caseworker, I had arrived back in time in the immense Valle de Sangre de Cristo. The area stirred my interest in the historical, earlier times of our Oklahoma Grandfathers. I became enthralled with the uncommon culture and language of the Chicanos. A younger brother noted how similar these people were to Native People in Oklahoma. Well, not quite, but close enough.

I quickly learned that young Chicanos of my age that lived in the outlying communities could not be relied upon to speak English or know sufficient English to carry on business. Most of their Elders had never traveled beyond the mountain ranges that surrounded the Valley. I learned to speak Spanish and thought of how I came from a small family of relatives who spoke indigenous language(s) of the Land, just as these People. And then I thought to myself: Why did I not pursue developing my limited knowledge from and with the Elders there?

Thus, I was inspired to write down every word and phrase that I knew and remembered of our Grandmothers' Báxoje, Jiwére-\~Nút\^{ }achi Language. I arranged during my travels home to Oklahoma to spend an increasing amount of time with our old Uncles and Aunties. I asked for the words and terms for plants, trees, and articles of dance clothing. I became more interested in the personal histories of the old People, of their early days in Oklahoma, the changes they have been obliged to conform to, and their Spirituality, which saw them through their transformation from resident possessors and spiritual keepers of the land to a contemporary disparaged minority which seemingly endlessly needed to bend to the will of the dominant controlling Society. On one occasion, at the Iroshka Society ceremonial dances of the O\-sage in Gray Horse, Oklahoma, Mama introduced me to an Osage woman, saying to her: ``My son is an old man made over.'' She was referring to my passion to know language, old traditions, teachings, and ways, which contrasted with most adolescents, whose interest was minimal at best.

During the years of 1965--1987, I worked to preserve and research the Ioway, Otoe-Missouria Language, oral tradition, history, and customs. I researched all manner of documentation on the Ioway, Otoe-Missouria culture, oral literature, language, lifeways, and ceremonials made by early explorers, fur traders, and missionaries. At the same time, I was mentored by several Elders in the Ioway, Otoe-Missouria communities and one other up in the Northern Plains.

In 1970--1972, I received a National Mental Health Scholarship to attend graduate school at the University of Kansas School of Social Welfare in Lawrence. It was there that I acquainted myself with the KU Department of Linguistics and met \iai{Robert Rankin} and \iai{Ken Miner}. I attended the Siouan and Caddoan Languages Conference and met \iai{John Koontz} from Colorado. I learned that these three men had done extensive research and fieldwork in the Kansa, Quapaw, Omaha, and Winnebago [Ho-Chunk] communities. Through their personal and professional association, they provided me with much insight into the study of academic linguistics and how it could be useful in community or applied linguistics. They provided assistance for my personal work with the Ioway-Otoe Language and Elders of both communities that I had begun about 1965.  

Bob and John were particularly helpful in showing how linguistic analysis of words provides insight into the grammar of the language and the construction and root meanings of words. The utility and exactness of linguistic orthography was pointed out to me. When I began serious work on a corpus dictionary for the language, I would share files with them and seek their advice and recommendations for more complete entries. After I worked out an acceptable standard orthography with my Elder informants and advisors, I intended to include a word sounding out such as was the common practice among community members. One of the community Elders wrote a brief dictionary in the 1970's of words and phrases in this folk manner \citep{Murray1977}. For example, he wrote as in the first column of \tabref{Murray}:

\begin{table}
\begin{tabular}[t]{ lll }\lsptoprule
Murray orthography & English translation & Goodtracks orthography \\ \midrule
knee & `water' & ñí \\
the-gray & `footprint' & thígre \\
wah-yeeng-eh-the & `yellow bird' & wayí\textipa{N}e dhí \\
khoo-me-khan-jay & `big stink' & xúmi xánje \\ \lspbottomrule
\end{tabular}
\caption{Murray's \citeyearpar{Murray1977} orthography compared with Ioway, Otoe-Missouria Language Project orthography.}
\label{Murray}
\end{table}

Since the communities had become accustomed to view their language written in such a format, I began to include duplicate dictionary word entries in a similar manner. However, \iai{Bob Rankin} advised that this double format was inappropriate and a disservice to the people and language per se. He remarked how many of the younger community people had already undertaken the study of other languages, such as Spanish, French, etc., and at the outset, they were obliged to learn the particular alphabet and phonics of the particular language. Thus it followed that language students need to learn the standard language orthography based on a linguistic model. He spoke to the imperfect, changing nature of using English phonics as a model for unwritten indigenous languages, which results in incorrect pronunciations and enunciations of words and spoken conversation. Also, he pointed out that to write indigenous languages in such a ``Morse code'' manner was to demean the languages as being something less than the highly developed and complex languages that they are, the equal of any of the contemporary dominant languages that presently exist. Observations among the people in the communities gave truth to his admonition, as there were heard and still persist mispronounced words and simply incorrect phonetics. 

In the summer of 1971, while still in graduate school at KU, I engaged the services of a professional linguist at Kansas State University, \iai{Lila Wistrand-Robinson}, who secured a grant for a study of Ioway, Otoe-Missouria Language with the Elders. She produced a lexicon and a set of basic and intermediate level language study books, which were published in the 1970's and combined my personal research with her own \citep{OtoeIowaWistrandRobinson1977, OtoeIowaWistrandRobinson1978}. The materials were distributed gratis to the students and families of the three Ioway, Otoe-Missouria communities. Meanwhile, I continued to work on my own dictionary files.  

During July 1985, the Otoe-Missouria Johnson O'Malley Program, after being referred to me by several local Elders, employed me to instruct the children in the summer program on the Báxoje-Jiwére Language. A number of the children had been raised in the presence of their grandparents and were familiar with words and phrases from the language. I utilized word games based on such popular games as Bingo and developed songs designed to teach numbers. I used the tune from the old racist nursery song ``One little, two little, three little Indians,'' and replaced the words with \emph{Iyá\textipa{N}ki, núwe, dáñi nampóiñe} `One, two, three little fingers'. Then we practiced basic oral conversations.  

From that time, various individuals have engaged me in several capacities to assist them to learn the songs, language, and the traditional teachings and lifeways of the Elders (traditional stories, sweat lodge, Native American Church, etc.), in which I was mentored over a period of thirty years. On several occasions, the Otoe-Missouria Tribe asked me to return and provide language classes for the community.  

Meanwhile, through the years, I compiled a collection of wékaⁿ (traditional stories) and developed comprehensive dictionary files, which combined my own knowledge of the language with that of tribal Elders and manuscript sources going back to the 1830's \citep{Goodtracks1992}. In 2002, I was asked to provide an introduction to the Báxoje Language in White Cloud, Kansas, in conjunction with the annual Baxoje Fall Encampment held every September by the Iowa Tribe of Kansas and Nebraska, and I was encouraged afterwards to continue the sessions at several locations, including in Lawrence. 

In 2003, \iai{Bob Rankin} and \iai{John Koontz} encouraged me to write a grant proposal to develop my dictionary files into an online encyclopedic dictionary that would serve as a resource for the Native communities and the work of professional linguists. Bob and John both wrote letters in support of my proposal. The proposal was awarded a grant, and I commenced to edit my files into full entries for both sections of the bilingual dictionary. And during 2004--2005, I partnered with \iai{Marge Schweitzer}, a retired Oklahoma State University anthropologist, to transfer the aging, deteriorating cassette recordings of the Elders to a digital format, which provided an opportunity to edit and correct various flaws and errors in the original books for basic and intermediate Ioway-Otoe language study \citep{OtoeIowaWistrandRobinson1977, OtoeIowaWistrandRobinson1978}. The cassettes for the first book followed the original material closely but not so the remaining cassettes, which contained previously undocumented materials. Thus all additional words, phrases, and the same contained on these recordings were added to newly written bilingual booklets \citep{Goodtracks2004a, Goodtracks2004b}.

By this time, I had retired from my social welfare career, and in 2005, I unexpectedly became the legal guardian and adoptive parent of my grandson. I resolved to have a language nest for him as, to date, after all the community classes, programs, and highly vocalized encouragements to ``Talk Your Native Language,'' there were no new speakers to be found anywhere (\emph{new speakers} being defined as individuals who daily communicated in the community heritage language).  

So I spoke nothing but Báxoje to my grandson, reading to him and telling stories to him in Báxoje, and when he began to speak, he spoke nothing but Báxoje. It is his first language. My family encouraged and supported my efforts, and they would speak to him as well as assist in his care. In the summer of 2008, Bryan Gordon, a graduate student, volunteered to assist with the work on the dictionary edits. In the course of staying with us that summer, he too learned to speak in Báxoje with my grandson and wrote an interesting and comical list of ``Phrases in Báxoje Ich\^{ }é Indispensible to Living with a Three-Year-Old'' (see Figures \ref{phraselist1}--\ref{phraselist4} below).

In 2009, the Siouan and Caddoan Languages Conference was held at the University of Nebraska -- Lincoln campus. One of the Ho-Chunk community participants from Wisconsin desired to compare the relative closeness between Báxoje and Ho-Chunk languages. Thus, he proceeded to say Ho-Chunk words to my grandson. My grandson did not comprehend his exercise with the word comparisons. So when the individual said, ``súúch'' (`red'), Hiⁿtágwa (`my grandson') replied: ``Hiñego (`no'), šúje,'' correcting his pronunciation to the Báxoje word. Again, the person said ``woonáⁿ\v{z}iⁿ'' (`shirt'), and my grandson responded, ``Hiñego, wónayiⁿ,'' he corrected him. ``Warúch'' (`eat something'), the man said. ``Hiñego, warúje,'' my grandson continued to correct him and so on. The Ho-Chunk delegate thought it amusing, as did I.

Following the conference, another graduate student, Saul Schwartz from Ohio, who attended graduate school at Princeton University in New Jersey, volunteered to stay the summer. He assisted me with the dictionary edits and also with the home schooling of my grandson. He had prepared himself before arriving by learning basic conversation in Báxoje, and he continued to hone his capacity to speak and understand the language during that summer. For home-schooling classes, he continued with several materials I had developed and composed, then he added other purchased learning aides, such as lettered blocks, dice, picture books, and innovative books and drawings. He became a ``big brother'' of sorts, and a constant companion for Grandson, taking him on walks, bike rides, and swimming pool visits. He preferred that I continue to sing the teaching songs, such as ``Iyá\textipa{N}ki, núwe, dáñi nampóiñe'' (`One, two, three little fingers') and the ``ABC Uyáⁿwe'' (`ABC Song'). Saul returned the following summer to continue his assistance and take on a new mission, namely field research on Native language study and regeneration.  

When Grandson was about four years of age, he had noticed that he was the only child speaking Báxoje, and adults outside the immediate family who he was aware could speak some Báxoje-Jiwére would invariably talk English to him rather than speak whatever they knew in the language. Thus, he acquired basic English, and he began to speak it more often than he would his first language. Nevertheless, for some years afterwards, he would often approach me with new English words he heard spoken and ask: ``What do they mean when they say `diarrhea'? Say it in Báxoje!''

So I would respond: ``Tahéda rayéthri ra\^{ }úⁿna.''
	
``Ohh, ok, pí ke,'' he would respond upon comprehending the new word. Now, it is not so much that he asks, but still every now and then, he will ask. Further, every day I continue to speak in part in Báxoje, his first language. And we continue to use it as a part of family gatherings and prayer ceremonials, such as in the YúgweChí (`Purification Lodge') and Wanáxi Kigóñe (`Spirit Feasts'). The language does not have the prominent place in the home as it did in Grandson's preschool years, but it remains a permanent presence in the home. He is now ten years old.

Initially, I had pursued this language interest to satisfy my own knowledge. Later, I desired to have knowledge of the Ioway-Otoe Elders for the benefit of my family and my children. And now I desire to record and share the knowledge I keep as a resource in print for the younger generations that may desire to know about their heritage and language. The Elders are all gone on, and my own generation grows older and fewer. I am now seventy-two. So, it will be good when the taped voices of those traditional Elders have been saved in order to instruct the contemporary and unborn generations. And that will be my final project and contribution to the late Elders, the communities, and those individuals who are inspired to speak the language in the manner of their great-grandparents.

I, Jimm G. Goodtracks, on October 29, 2014.

\section{Phrases in \emph{Báxoje Ich\^{ }é} Indispensible to Living with a Three-Year Old (Bryan)}
	In the summer of 2008 Hiⁿtáro (`my friend')\footnote{See \sectref{saul} for further explanation and interpretation of the Báxoje kinship terms employed.} Jimm Goodtracks hired me to assist his work on his Báxoje, Jiwére-\~Nút\^{ }achi, Ma\^{ }ú\textipa{N}ke (`Ioway, Otoe-Missouria, English') dictionary. He was raising Itágwa (`his grandson') \iai{Sage Goodtracks}, then three years old, in Báxoje Ich\^{ }é. Out of respect for the immersion environment, and out of my love for languages stretching back to when I was Hiⁿtóšge's (`my nephew's') age, I committed to doing my part. I created this list as a practical guide and a learning tool for myself, checking a lot of it with Hiⁿtáro along the way, but not all of it. It needs many corrections yet. Hiⁿtáro and Saul requested the list for inclusion in this chapter. I had originally posted it on a social-media site popular at the time and it was lost, so it was a relief that it could be turned up again. 

What you see in the following lists of phrases
% in Figures \ref{phraselist1}--\ref{phraselist4} 
at first glance reads more authoritarian than a person ought to be with a child. This is because I usually added to this list during the workday, during which Hiⁿtóšge (`my nephew') provided me with constant company and friendship, plenty of play breaks, walks, shared meals, and many teaching and learning opportunities -- but I also wished to stay productive and offer alternative behaviors often. I used the praising phrases more often than the critical ones. I feel bashful at putting phrases invented by a non-fluent learner out in publication, and disclaim that I learned more about Báxoje Ich\^{ }e from Hiⁿtóšge's and Itúgaⁿ's (`his grandfather's') corrections than from my faulty practice. 

To my recollection, the section breaks here reflect those in the original, but they are not consistently thematic, so I have not titled them. I was probably trying to maintain a sort of thematic structure while adding new phrases on the fly. Instead of reorganizing, I've kept the section breaks where they were.

% \subsection[``Hiñégo, míne wá\^{ }uⁿnachi mínahiⁿ bé re hó.'']{``Hiñégo, míne wá\^{ }uⁿnachi mínahiⁿ bé re hó.''\newline (`No, please leave me sitting alone because I am working.')\newline Phrases in Báxoje Ich\^{ }é Indispensible to Living with a Three-Year-Old}
\subsection*{``Hiñégo, míne wá\^{ }uⁿnachi mínahiⁿ bé re hó.''\newline
(`No, please leave me sitting alone because I am working.')\newline
Phrases in Báxoje Ich\^{ }é Indispensible to Living with a Three-Year-Old}
\begin{list}{}{} \itemsep1pt \parskip0pt \parsep0pt
\item{Wabúhge hiⁿñí\textipa{N}e ke. `I'm out of bread.'}
\item{Ritúgaⁿ igwáⁿxe re. `Ask your grandfather.'}
\item{Inúhaⁿ Ritúgaⁿ ahósege škúñi ne. `Don't talk to your grandfather that way again.' (lit. `Second time your grandfather don't talk saucily to him.')}
\item{Wapópoge škúñi ne. `Don't throw stuff around.'}
\item{Wójiⁿ škúñi ne. `Don't hit.'}
\item{\^{ }\'Uⁿ škúñi ne. `Don't do that.'}
\item{}
\item{Ra\^{ }úⁿ ramáñišge rakích\^{ }e hñašgu. `If you keep doing that you might get hurt.'}
\item{(Chúⁿhaⁿwe) Táⁿgrigi gasúⁿ wanáⁿxi ñí\textipa{N}eñe ke. Gasúⁿ yáⁿ híwe re. `Now there are no ghosts outside (the window), so go to sleep now.'}
\item{Arábechi maⁿgrída mína hñe ke. `Since you threw it up there, it's going to stay up there.'}
\item{Gasúⁿ srudhéda asríⁿ jí re. `Ok, now go and bring that back here.'}
\item{}
\item{Wá\^{ }uⁿ hamína ke.	`I'm working.'}
\item{Dagú ra\^{ }úⁿ je? `What are you doing?'}
\item{Dagúre raína (je)? `What are you eating?'}
\item{Dagúrehsji ragúⁿsda (je)? `What exactly do you want?'}
\item{Dagúre uráje je? `What are you looking for?'}
\item{Wagísdóxi škúñišge srúdhe škúñi hñe ke. `If you don't ask for it you won't get it.'}
\item{Wayére (je)? `Who is that?'}
\item{Dagúra (je)? `What is it?'}
\item{Wé? `What!?'}
\item{\underline{\hspace{1em}} taⁿdáre iwéhiⁿnegi  je? `Where did you put my [horizontal] \underline{\hspace{1em}}?'}
\item{\underline{\hspace{1em}} taⁿdáre iwéjehiⁿnegi je? `Where did you put my [vertical] \underline{\hspace{1em}}?'}
\item{\underline{\hspace{1em}} taⁿdáre iwénahiⁿnegi je? `Where did you put my [round] \underline{\hspace{1em}}?'}
\item{\'Uⁿnek\^{ }uⁿ hagúⁿta ke. `I want you to give it back.'}
\end{list} 
% \caption{}
% \label{phraselist1}
% \end{figure}
% \FloatBarrier

\subsection*{``Hiñégo, míne wá\^{ }uⁿnachi mínahiⁿ bé re hó.'' (cont'd)\newline
(`No, please leave me sitting alone because I am working.')\newline
Phrases in Báxoje Ich\^{ }é Indispensible to Living with a Three-Year-Old}
% \begin{figure}[p]
\begin{list}{}{} \itemsep1pt \parskip0pt \parsep0pt

\item{Rixráñi je? `Are you hungry?'}
\item{Ridáxraⁿ je? `Is that too hot for you?'}
\item{\~Ní srátaⁿ ragúⁿsda je? `Do you want to drink water?'}
\item{Pagráⁿda agúje wórataⁿ hñe ke. `First you have to put on your shoes.'}
\item{Dókirašdaⁿ.	`Just a little.'}
\item{}
\item{Iyáⁿkišdaⁿ. `Just one.'}
\item{Hókithre{s}daⁿ. `Just half a piece.'}
\item{(Wabúhgehgu / Warók\^{ }iⁿ / núxechebáñi / chebáñiwébri) ragúⁿsdasge Ritúgaⁿ iwáⁿxe re. `If you want (cookies / pie / ice cream / cheese) go ask your grandfather.'}
\item{Yáⁿ híwe ragúⁿsda je? `You want to go to bed?'}
\item{Išdáⁿ rich\^{ }éšge ke. `You must be tired.'}
\item{}
\item{Urídus\^{ }adaⁿna ke. `I'm getting fed up with you.'}
\item{Wayíⁿ uríxwáñi ke. `You're crazy.' (lit. `Your mind fell down.')}
\item{Xáp\^{ }a re. `Quiet down.'}
\item{Amína ne. `Sit down.'	}
\item{Uyéchi wará je. `Go to the bathroom.'}
\item{}
\item{Dá.	`I don't know.'}
\item{Gasúⁿda áñiⁿ ke.	`I already have it.'}
\item{Dagúre šé asríⁿ je?	 `What do you have there?'}
\item{Wáji hadúšdaⁿ ke. `I already ate.'}
\item{Waráji iríbraⁿ ke. `You've already eaten enough.'}
\item{}
\item{\'Ata ihádušdaⁿ ke.	`I already saw that.'}
\item{Ra\^{ }úⁿna aríta ke.	`I already saw you do it.'}
\item{Gasúⁿda ihápahu\textipa{N}e ke.	 `I already know that.'}
\item{}
\end{list} 
% \caption{}
% \label{phraselist2}
% \end{figure}

% \begin{figure}[p]
\subsection*{``Hiñégo, míne wá\^{ }uⁿnachi mínahiⁿ bé re hó.'' (cont'd)\newline (`No, please leave me sitting alone because I am working.')\newline Phrases in Báxoje Ich\^{ }é Indispensible to Living with a Three-Year-Old}
\begin{list}{}{} \itemsep1pt \parskip0pt \parsep0pt
\item{Pi ra\^{ }úⁿ ke! `You did good / well!'}
\item{\'E\^{ }o! `Watch out!'}
\item{Námañí huhe ke. `There's a car coming.'}
\item{Nabráhge: ``\v{S}uⁿkéñi chéxi ke,'' ériwana ke. `That sign says there's a dangerous dog.'}
\item{Ná\^{ }uⁿ asrúchena náwe úⁿk\^{ }uⁿ ne. `Give me your hand when you cross the street.'}
\item{Xámi amáñi ne. `Walk on the grass.'}
\item{Danáñida chínada hiⁿwámañi ke; héroda eswéna hiⁿné ho. `We walked into town yesterday. Let's go maybe tomorrow.'}
\item{Háⁿwe Waxóñidaⁿ naⁿkérida Jíwere \~Nút\^{ }achi Wóyaⁿwe hiⁿnáwi hiⁿnúšdaⁿwi ke. `We already went to the Otoe-Missouria Encampment a week ago.' (lit. `We already went to where the Otoe and Missouria sing a Sunday into the past.')}
\item{Háⁿwegi míšdaⁿ hamáñi hajé.  `Today I'm going for a walk by myself.'}
\item{Rí\^{ }e jegí ramína hñe ke. `You stay here.'}
\item{Aréhga je? `Are you telling the truth?' / `Are you serious?' / `Is that so?'}
\item{Hiⁿwáha re. `Show me.'}
\item{Náⁿje giwáha re. `Go show your uncle.' (lit. `father / father's brother')}
\item{Chúgwa jí re.	`Come back inside the house.'}
\item{}
\item{Axéwe šgáje re. `Go play outside.'}
\item{Núwerút\^{ }ana amína re.	`Go play on your bicycle.'}
\item{Grúya re.	`Clean up.'}
\item{Rixúmi ke aréchi kipídha re. `You smell bad, so go take a bath.'}
\item{\v{S}é miⁿtáwe ke;  ritáwe škúñi. `That's mine, not yours.'}
\item{Wagúñetáⁿiⁿ wanáⁿp\^{ }i hiⁿnágirusdajena rusdáⁿ ne. `Stop pulling on my rainbow necklace.'}
\end{list} 
% \caption{}
% \label{phraselist3}
% \end{figure}
% \FloatBarrier

\subsection*{``Hiñégo, míne wá\^{ }uⁿnachi mínahiⁿ bé re hó.'' (cont'd)\newline (`No, please leave me sitting alone because I am working.')\newline Phrases in Báxoje Ich\^{ }é Indispensible to Living with a Three-Year-Old}
% \begin{figure}[p]
\begin{list}{}{} \itemsep1pt \parskip0pt \parsep0pt
\item{Ga\^{ }ída. `Over there.'}
\item{Akína ne.	`Wait.' / `Watch out.'}
\item{Gasúⁿhsji ke. `Right now!'  }
\item{Tóriguⁿ.	`[See you / Not until] Later.'}
\item{Uxré. `Soon.'}
\item{Náhehiⁿna gigré re. `Leave me alone.'}
\item{Ruhdá skúñi re. `Don't touch that.'}
\item{Ritúgaⁿ nigírixogešge ke. `Your grandfather might scold you. '}
\item{Ritúgaⁿ uhágidagešge ke. `I'll tell your grandfather.'}
\item{}
\item{Húⁿche.	`Yes.'}
\item{Hiñégo. `No.'}
\item{Ahó.	`Hi there.' / `Thanks.' / `Good.' / `Ok.' / `I acknowledge that.'}
\item{Ahó warigróxi ke. `Thank you very much.'}
\item{Pi ke! `It's good.' / `Wow, cool!' / `Ok.'}
\item{Pi je? `Is that good?'}
\item{Ripí je? `Are you okay?'}
\item{Daríhga? `How are you?'}
\item{Pí škúñi ke. `That's bad.'}
\item{Wayíⁿthwe skúñi ke. `You're misbehaving.'}
\item{Uwáreri škúñi ke. `I didn't understand you.'}
\item{Gasúⁿ ke. `Enough!'}
\item{Rúšdaⁿ ne. `Stop it.'}
\item{Dákaⁿhina uránadhena rúšdaⁿ ne. `Stop turning the light off and on.'}
\item{\'Aⁿwithruje škúñi ne. `Don't squirt me.'}
\item{}
\item{Jehéhgana šéhe ke. `I told you so.'}
\item{Gasúⁿ srúšdaⁿ ke. `Ok, you're done.'}
\end{list} 
% \caption{}
% \label{phraselist4}
% \end{figure}
% \FloatBarrier

\section{Strategies and Challenges in Chiwere Revitalization (Saul)}\label{saul}
In this section, I would like to provide a broader context for understanding the Project's activities by describing their role within the Ioway and Otoe-Missouria communities and comparing them to strategies for revitalizing other Siouan and Native American languages. Following a long process of domain contraction, Chiwere was used primarily in religious contexts by 1950 \citep{Davidson1997, FurbeeStanley1996, FurbeeStanley2002}. This resonates with my own observations since I began working with Jimm in 2009. Besides interactions explicitly framed as language learning (e.g. tribal language classes), I have seen and heard Chiwere used for endonyms, salutations, valedictions, alimentation (especially water and common or traditional foods), elimination, kinship terms, personal names, and tribal programs. Some tribal members incorporate Chiwere into their personal or professional activities, especially if they are involved in art, music, and/or activism.\footnote{Examples of tribal programs with Chiwere titles include the Otoe-Missouria tribal newsletter, \emph{Wórage: Stories of the People}, and the Iowa Tribe of Oklahoma's eagle sanctuary, Bah Kho-je Xla Chi (\emph{Báxoje Xrá Chí}). For examples of Chiwere use by individual tribal members, see art, essays, and websites by Lance \citet{Foster1989, Foster1996, Foster1999, Foster2009, FosterNDa, FosterNDb, FosterNDc}, an album by artist and musician Reuben \citet{Kent2004}, \citet{Jones2004}, and Brett Ramey's \citeyearpar{RameyND} website.} 

Nevertheless, as in many Native American communities \citep[see e.g.][]{Kroskrity1998}, ritual contexts are the most prestigious domains for Chiwere and other indigenous language use for Ioways and Otoe-Missourias, including in Native American Church meetings, ceremonial dances (e.g. Iroška), sweat lodges, and personal prayers. Since access to these activities can be limited or restricted, not all tribal members are exposed to Chiwere in these settings; furthermore, participants themselves do not always know the meaning of the Chiwere words they use or else they memorize their meanings \citep[see e.g.][520--521]{Davidson1997}.

The association between Chiwere and religious settings has fostered a sense that the language itself is sacred and that its circulation should be restricted like the distribution of other kinds of ceremonial knowledge \citep{Davidson1997}. As a result, tribal members have few or no opportunities to hear or use Chiwere in their daily lives unless they have family members who make a special effort to use the language on a regular basis or participate in language classes. Currently, two of the three tribes for whom Chiwere is a heritage language have active language programs that offer classes and other educational resources: the Iowa Tribe of Kansas and Nebraska, whose language program is directed by the Tribal Historic Preservation Officer, \iai{Lance Foster}, and the Otoe-Missouria Tribe of Indians, where \iai{Sky Campbell} is the tribal language coordinator.

One strategy for reversing language shift in such contexts involves expanding the domains available for heritage language use. Jon Reyhner's \citeyearpar[vii]{Reyhner1999} adaptation of Fishman's \citeyearpar{Fishman1991} Graded Intergenerational Disruption Scale, for example, measures vitality in large part by whether a language is used in communal public spaces like educational and governmental institutions, businesses, and mass media. Jimm and his counterparts in tribal language programs try to raise awareness about language preservation by making Chiwere more visible in public places and encouraging people to use it more often. The last time I visited the Iowa Tribe of Kansas and Nebraska's offices, for example, Lance had put up signs in Chiwere identifying different offices and the restrooms. Sky has made similar signs for the Otoe-Missouria offices, and stop signs in the tribal complex parking lot are also in Chiwere.\footnote{I am told that the Iowa Tribe of Oklahoma also has Chiwere signage in their offices.}

The Ioway, Otoe-Missouria Language Project has pursued a similar strategy of increasing Chiwere's public presence. For example, the Project designed and printed a tee shirt that includes a beadwork-inspired floral design, the Chiwere endonyms for the Iowa, Otoe-Missouria, and closely related Ho-Chunk peoples; an image of an elder and a child wearing ceremonial dance clothes' and a sentence in Chiwere that translates, `The language honors our elders and teaches our children.' The Project also designed mugs that include the Chiwere phrase for `I love my coffee' with the image of an Oneota-style ceramic vessel superimposed over a medicine wheel.\footnote{\emph{Oneota} refers to an archaeological culture ancestral to a number of historical groups, including the Iowa and Otoe-Missouria. Some tribal members claim elements of Oneota culture as part of their cultural heritage, e.g., in Oneota-style ceramics by Ioway artist Reuben Kent (\citeyear{KentND}; \citealt{RundleRundle2007}). Members of related Siouan groups, such as the Omaha, also see themselves as descendants of Oneota communities \citep{Buffalohead2004}.}

In both cases, these objects were designed to set up educational interactions through question and answer routines. Since both objects include only Chiwere and no English text, those who do not know Chiwere have to ask someone who does if they want to know what the tee shirt or mug ``says.'' When I have modeled the tee shirt at powwows and been approached by curious tribal members, Jimm has used the opportunity to explain the shared histories of the Iowa, Otoe, Missouria, and Ho-Chunk peoples and to discuss the aesthetic principles and symbolism of floral designs. Similarly, when I have witnessed interactions surrounding the coffee mug, Jimm has explained that the Chiwere word for `coffee,' \emph{mákaⁿthewe}, literally means `black medicine,' just like the word for `tea,' \emph{xámi mákan}, literally means `herb medicine.' This often leads into a discussion of traditional notions of medicine, healing, substance abuse, etc.

To take a more humorous example, Jimm's license plate reads \emph{DAGWISA}, the Chiwere word for `What did you say?' When people approach Jimm to ask about his license plate, a typical dialogue goes something like this:

	``What does your license plate mean?''
	
	``What did you say?''
	
	``I said, what does your license plate mean?''
	
	``What did you say?'' etc.
	
While in this example the Chiwere language token does not appear in a culturally significant environment, it is still designed to promote an interactional context involving Jimm, who uses the opportunity to impart grammatical and cultural knowledge.

Bringing Chiwere out of religious settings and into the public sphere also carries risks. One of the primary dangers is that the language may lose its indexical associations with the traditional cultural values that motivate its revitalization in the first place. This is a phenomenon that other scholars have described for Alaskan Native languages \citep{DauenhauerDauenhauer1998}, Apache \citep{Nevins2013, Samuels2007}, California Indian languages \citep{Ahlers2006}, Kaska \citep{Meek2010}, and Maliseet \citep{Perley2011}. Some communities, particularly in the Southwest, restrict the circulation of language materials in order to prevent the decontextualization of heritage languages from what are considered to be proper settings and forms of language use \citep{Debenport2015, Whiteley2003}. Jimm's language revitalization strategies represent another approach to the problem of decontextualization: rather than limiting access to language materials, Jimm promotes the circulation of materials that associate Chiwere with other symbols of traditional Ioway and Otoe-Missouria culture (e.g. the tee shirt and mug described above) and resists the circulation of materials that associate Chiwere with what he believes are dominant society practices and values.

Translation requests are one domain in which Jimm exercises discretion in order to control the cultural associations of Chiwere language tokens. Jimm often receives requests to calque English idioms -- for example, `Go green!' (for a tribal environmental awareness program); `I $\heartsuit$ boobies!' (for tribal breast-cancer awareness bracelets); `Bigg Rigg' (for fans of Otoe-Missouria mixed-martial-arts fighter \iai{Johny ``Bigg Rigg'' Hendricks}); and `They are in a Warthog' (a phrase used in playing the video game \emph{Halo}\footnote{\emph{Warthog} is a type of vehicle in the game.}). These requests are often met with ambivalence since they are seen as having no connection with traditional culture, and Jimm often declines to provide translation services for efforts that would increase Chiwere language use if he believes that such use would undermine traditional values.

Once, for example, a tribal member sent Jimm a list of English terms that she wanted translated into Chiwere. The list focused on terms for genitalia and bodily functions that are considered ``bad words'' in English. Jimm declined to provide the requested translations since he felt that the corresponding Chiwere terms lacked the negative associations of their English counterparts and worried that English speakers would project those associations onto the Chiwere terms. Many tribal members also believe that there are no ``bad words'' in Chiwere. Thus, swearing is one domain of language use where there is considerable resistance to using Chiwere. (For a description of an inverse situation, where an indigenous language is used almost exclusively to swear, see \citealt{Muehlmann2008}.) In a similar vein, Jimm's response to the request to translate `I $\heartsuit$ boobies!' for a tribal breast cancer awareness program included a long description of traditional attitudes toward sexuality, which he felt ran counter to the slogan's sexual innuendo. Jimm was also concerned by the request to translate phrases used in playing the video game \emph{Halo}, which was intended to enable tribal members to communicate in Chiwere while playing the game. When I explained to Jimm what \emph{Halo} is (a first-person shooter, i.e., a rather violent video game), he expressed reservations that Chiwere be associated with it and replied to the request by saying: ``If I cannot contribute to peace and harmony, what the old people called \emph{wapána}, then I cannot contribute at all.''

Jimm's responses to these translation requests reflect the attitudes of some tribal members who have strong views about appropriate and inappropriate contexts for Chiwere language use. Occasionally, mere proximity between Chiwere and objectionable content can trigger concerns. Once, Jimm and I were approached by an elder who objected to Chiwere language lessons being posted on YouTube by a tribal member whose account also linked to music videos that contained suggestive and/or violent imagery. The elder was also concerned that the language lessons, which featured the voice of her deceased relative, were now publically available to unrestricted audiences. Of course, younger generations, particularly those who live far from tribal reservations, may appreciate the increased access to their heritage language that online platforms like YouTube provide, and they may have no qualms about viewing a Chiwere language lesson followed by a popular music video.

The challenge of giving an endangered heritage language a wider public presence while maintaining its traditional cultural associations is one faced by many working to revitalize Siouan languages. For example, a request on the Siouan List to translate a line from \emph{Alice in Wonderland} (``curiouser and curiouser, cried Alice'') for a polyglot compilation produced multiple responses. While some found the intellectual challenge of translating a Victorian neologism into Siouan languages intriguing, others were less receptive to the request because of its perceived triviality and irrelevance to Native American communities. Bryan wrote: ``It's a more distinguished request than pet names\footnote{Requests, many from non-Native people, to translate names for pets and children or stock English phrases into Siouan languages are so common that John \citet{Koontz2003a} posted his general responses to such questions on the FAQ section of his website. Once, he was even asked (presumably as a joke) for a Native American name for an RV; he responded in kind with \emph{Hotanke}, an Anglicized spelling of the Dakotan word for the Winnebago [Ho-Chunk] people, from one of whose English names the Winnebago brand of RV's took its name \citep{Koontz2003b}.} and such, but it's not the kind of translation work I would prefer to spend my time on. Why don't people ask us to translate Microsoft Word or a K-12 curriculum or something important?'' \citep{Gordon2014}. Jimm concurred: ``I have other priorities and am unclear on the need for [a translation of] the particular quote from a story which has nothing in common with Native American Culture.... To spend time on the translation of materials that have no immediate application to the language communities is nonsensical and, for my part, a waste of time'' \citep{Goodtracks2014}. Willem de Reuse shared his general guidelines for responding to such requests: ``One has to pick and choose. If it is short and culturally appropriate, I generally agree to it.... Then other requests have to be nixed, like the set of `Spring Break' phrases I once was asked to translate, things like `I am so drunk,' and `Where is the bathroom?'\thinspace'' \citep{deReuse2014a}. As \citet{deReuse2014b} explained, part of the reason he objected to translating spring-break phrases is because the translations could be circulated in a way that would trivialize indigenous languages. In short, most linguists and activists working on Native American language revitalization face the question of what exactly is a ``culturally appropriate'' application of a heritage language -- and how to prevent indigenous languages from appearing in culturally inappropriate contexts.

One solution to this problem is to embed heritage languages in contexts rich with other traditional cultural symbols. The tee shirt and coffee mug described above are two examples.\footnote{Many other materials produced by the Project also seek to embed language in culturally rich environments. In the past, for example, the Project has published calendars with historical photographs and the names of the months in Chiwere \citep{Goodtracks1985}. The Otoe-Missouria Tribe recently published a similar calendar \citep{OtoeMissouriaLD2014}.}  Another example is a board game produced by the Kaw Nation language department, \emph{Wajíphaⁿyiⁿ}, which Jimm and I adapted for Chiwere and played with his grandson and other relatives. The game encourages players to imagine themselves as camp criers, who move among the traditional moieties of the tribe answering vocabulary questions to accumulate clan counting sticks \citep{KanzaLP2004}. Many pedagogical materials produced for Siouan languages also qualify as culturally rich to the extent that their content addresses traditional themes (\citealt[e.g.][]{HartmannMarschke2010, KanzaLP2010}). Chiwere pedagogical materials, for example, feature late nineteenth- and early twentieth-century photographs of Ioways and Otoe-Missourias in traditional dress and emphasize speech genres like prayer and oral histories of elders \citep{OtoeIowaWistrandRobinson1977, OtoeIowaWistrandRobinson1978}. \Citet{Debenport2015} notes that a ``nostalgic'' mood also permeates many Pueblo pedagogical materials.

Jimm's dictionary also emphasizes a connection between Chiwere and traditional cultural practices and values by including elaborate encyclopedia-style entries for terms with particular cultural significance. The entry for \emph{mihxóge}, for example, begins by giving a range of more or less word-for-word English translations for the term, including ``blessed person; a spiritual person or intermediary; gay, lesbian, homosexual, bisexual; two spirits person; transvestite; transsexual; berdache'' \citep[6]{Goodtracks2008}. Following the standard definition field is a long note that begins with a more literal definition of the word as ``an individual who has some natural female-like aspect of their character, personality or nature, which is of a mysterious divine origin'' based on a morphological analysis of \emph{mihxóge} as derived from \emph{mi\textipa{N}e}, indicating a feminine quality, \emph{xóñitaⁿ}, which refers to something sacred, blessed, or mysterious, and -\emph{ge}, a suffix that indicates an innate or natural gift, ability, or state \citep[6]{Goodtracks2008}.

The note then frames this analysis as a reflection of a traditional understanding of homosexuality associated with the elders, who respected \emph{mihxóge} as spiritual leaders. The note includes quotations from elders (``They are \emph{waxóbriⁿ} [`holy'], and they kinda know that and use it,'' and, ``They're not crazy. They just got that born in them. Born in their nature,'' etc.), which are interpreted for readers in relation to current social conditions \citep[7]{Goodtracks2008}. The note contrasts the elders' traditional views with Judeo-Christian attitudes, in which homosexuality may be seen as a choice or sin. The note also suggests that returning to traditional views would enable \emph{mihxóge} to once again cultivate ``their dormant `medicine powers' and abilities'' for the benefit of all \citep[7]{Goodtracks2008}. The note concludes with a final quotation from an elder who gives instructions on how to behave toward \emph{mihxóge}: ``Talk to them, be good to them and that's all. But don't hurt them --- it'll come back on you. They got medicine.'' \citep[7]{Goodtracks2008}

Thus, through a series of metapragmatic framings, the dictionary identifies the semantics of \emph{mihxóge} (as reflected in its morphological composition) with its ancestral and potential future pragmatics. In other words, in defining the word \emph{mihxóge}, the dictionary also teaches readers a set of traditional attitudes and behaviors associated with the elders. Jimm's desire to include the kind of information that readers could use to live the language in this way, which reflects a commitment to meet the needs of community audiences as well as linguists, explains some of the dictionary's unconventional formatting.

Jimm's language nest (discussed in more detail below) provided an opportunity for me and other participants to live the language. Within the language nest, for example, we never addressed or referred to each other as ``Jimm,'' ``\iai{Sage},'' or ``Saul.'' Instead, we always put a kinship term before the proper name. This is evident in Bryan's preface to his list of Chiwere phrases, where he refers to Jimm as \emph{Hiⁿtáro} `my friend' and \emph{Itúgaⁿ} `his [Sage's] grandfather' and to Sage as \emph{Itágwa} `his [Jimm's] grandson.' When I first came to live with Jimm and his grandson, Jimm explained to me that it is impolite to address someone without a kinship term that expresses a consanguinal, affinal, or ``fictive'' relation because relatedness is the basis of Ioway and Otoe-Missouria society. Historically, families in Ioway, Otoe-Missouria, and other Native American communities often made strangers, including anthropologists, into relatives through adoption \citep[see e.g.][]{Kan2001}. Jimm locates the origin of such practices in clan origin myths, which describe how clans met each other and formed larger societies by establishing relationships in which they refer to each other as \emph{hiⁿtáro} `my friend\footnote{In the clan origin myths, this relationship is established through formal gift exchange and pipe ceremonies. \emph{Itáro} also refers to what are known as \emph{Indian friends} in certain varieties of American Indian English, in which two people are bound to each other by mutual ritual obligations. Between Jimm, his grandson, and me, the semantic sense of the term is closer to the English word \emph{friend} since we are not bound to each other by ritual obligations, but our pragmatic use of the term to address or refer to each other differs from the conventional English usage of \emph{friend} in a way that reflects how Ioways and Otoe-Missourias use kinship terms to express the social and cultural value of relatedness.}.' Like Bryan, I refer to Jimm and his grandson, and they to me, as \emph{hiⁿtáro}. Sometimes I refer to Jimm's grandson as \emph{hiⁿthúñe} `my little brother', since after all these years he has become like a little brother to me. By using these kinship terms with each other, we foster and express relationships that have become part of our lives as lived. The examples of \emph{mihxóge} and \emph{hiⁿtáro} reflect how the Project and language nest link Chiwere language with cultural values and social action.

Of course, just as restricting Chiwere to religious settings limits opportunities for language use, associating the language exclusively with a nostalgic conception of ``traditional culture'' risks alienating those who struggle to see how a heritage language could be relevant to their modern lives. Linguists have noted on the Siouan List that some community members reject games as productive learning activities in language classes because the language is ``sacred'' or oppose colorful pictures and contemporary vocabulary in pedagogical materials because they are not ``traditional'' \citep{deReuse2014b, Ullrich2014a, Ullrich2014b}. Similar attitudes have been described for other Native American languages: \Citet[463--464]{Moore1988} mentions an example of how everyday talk in Wasco has become ``mythologized'' and subject to restrictions once applied only to a specific set of myths. Clearly, an exclusive association between indigenous languages and ``sacred'' or ``traditional'' domains may present an obstacle to revitalizing languages as means of everyday communication.

In short, while language revitalization seeks to create new opportunities for heritage language use by adapting the language to current conditions, this quest for relevance is tempered with the recognition that codes can become disassociated from the traditional values that motivate their revitalization in the first place. If what we care about is not only preserving linguistic diversity (in the sense of grammatical structures) but also preserving distinctive cultural worldviews and lifeways, we will have won the battle while losing the war if people are learning and using heritage languages primarily to participate in activities associated with the dominant society. As an Apache bilingual teacher wonders, if children are only learning how to use Apache to order a cheeseburger, what's the point? \citep[551]{Samuels2006} Revitalizing Chiwere, Apache, and other endangered Native American languages seems to require a balance between enabling language learners to order cheeseburgers or play video games and encouraging them to pursue a deeper engagement with ancestral cultural traditions that they will simultaneously transmit and reinvent for future generations.

A second danger that accompanies attempts to make heritage languages more visible in public spaces is that the languages tend to be used as emblems of indigenous identities without actually producing speakers (\citealt{Ahlers2006}; \citealt[98]{DauenhauerDauenhauer1998}; \citealt[715]{Whiteley2003}). This reflects the fact that as much as some language-revitalization efforts are motivated by a desire to reconnect community members with aspects of traditional culture, they are also informed by a thoroughly modern, nationalist notion of the role of language in political and social life \citep{KroskrityField2009, Nevins2013}. In other words, people are not necessarily interested in learning the language; rather, they may want to display the code in order to accomplish nonlinguistic symbolic goals. It is often disappointing for linguists to discover, for example, that people may want a dictionary of their heritage language to have, but rarely if ever to take off the shelf to read. Community language classes also have a tendency to serve symbolic and social needs rather than being effective in producing speakers. 

A sense of disillusionment with community classes and other forms of language use that affirm identities but rarely produce speakers is common among linguists who work on Siouan languages. In an interview, for example, \iai{Bob Rankin} told me that accurate documentation is more important than pedagogical materials for producing new speakers:

\begin{quote} So, if we get it right, those of us who are lucky enough to have been able to work with this last group of speakers, then the materials will be available to future scholars, or I always hope that ---. You know, every 10,000 kids there's some language genius who's born, there's some little kid who can just pick up languages, and I was one of those, so I know they exist. And there'll be some little Kaw kid or some little Omaha kid who'll pick this up and just attack it hammer and tongs and actually learn it someday. I'm not one of those people who optimistically believes that language-retention programs or language-teaching programs are going to resuscitate these languages. People are too busy to learn languages. Language learning is not easy for most people. It's hard work. Everybody wants to know a foreign language, but nobody wants to study a foreign language, and I completely understand that from all those Romance irregular verbs that I had to memorize, but the materials will be there. If the tribe needs them, if scholars need them, they'll at least be there.\end{quote}

Bob presents himself here as something of a messianic realist: he distances himself from ``optimists'' who believe that classes will produce adults who learn their heritage language as a second language and instead emphasizes the importance of documentation, which will allow those with a gift for picking up languages to teach themselves. 

Similarly, as Jimm points out above, the inspiration for his language nest came in part from his disillusionment with community language classes and similar programs that produced a substantial discourse about the importance of language revitalization but failed to produce new speakers who used the language on a daily basis. A disconnect between discourse supporting language preservation and a lack of effective action to reverse language shift is unfortunately a common phenomenon \citep[see e.g.][]{DauenhauerDauenhauer1998}. Like Bob, Jimm has focused on documentation in recent decades. ``Even if no one is interested now,'' he told me, ``it will all be there in the dictionary for anyone who comes along and wants to learn.''

In addition to putting more energy into documentation, Jimm also established his home as a language nest, in which he raised his grandson to speak Chiwere as his first language. The term language nest originates from and is a translation of a M\=aori language revitalization program called Te K\=ohanga Reo, which focuses on early-childhood language immersion \citep{King2008}. Te K\=ohanga Reo inspired similar programs in Hawai`i \citep{Warner2008, WilsonKamana2008} and other indigenous communities in North America and beyond. Whereas Te K\=ohanga Reo and similar programs are usually communal or corporate in nature, Jimm's language nest is domestic and includes only Jimm, his grandson, and graduate students who have lived with them, such as Bryan and myself.

Jimm's household is the only one I know of where Chiwere is used as a primary language of communication and is one of the few places where Chiwere is spoken on a regular basis at all. \iai{White Cloud} (or Chína \iai{Maxúthga}, as we called the small town where we lived) is near the reservation of the Iowa Tribe of Kansas and Nebraska. Chiwere had not been a native language there for many years. One of the last fluent speakers, \iai{Arthur Lightfoot}, was born there in 1902, but he moved to Oklahoma in 1935. When he died in 1996, he was one of two or three fluent speakers left and the only one of his dialect. The town itself, named after an Ioway chief, runs from the banks of the Missouri River up into the neighboring bluffs. It was a regional center of commerce and culture during the steamboat era, and the downtown district is listed on the National Register of Historic Places for its ``sense of historic time and place as a nineteenth century river town'' \citep[1]{Wolfenbarger1996}. For that reason, a few scenes from the film \emph{Paper Moon} \citep{Bogdanovich1973}, set in Great-Depression-era Kansas and Missouri, were filmed there. Today, the area feels like a ghost town. From a population of 1,000 or so in 1868, the 2010 census reported 176 residents, almost 20\% Native Americans. Abandoned buildings and houses outnumber those permanently or seasonally occupied. When cars come through, they are usually on their way to the nearby reservation bingo hall and casino. The town is only busy twice a year for semiannual flea markets, but at the last one I went to, in 2012, I heard people complain that numbers were down because vendors and shoppers prefer to go to markets in towns closer to where they lived.

For many years, Jimm only spoke to his grandson in Chiwere, and Jimm told me that Bryan had done the same when he had lived with them for a summer. Since everyone else spoke to his grandson in English, however, his grandson had become bilingual and would use English unless addressed in Chiwere, which did not happen outside the home. Before I went to live with them, I prepared as well as I could to be a productive participant in their language nest. I made and memorized a few hundred flashcards, studied pedagogical materials, memorized Bryan's list of ``Phrases in Báxoje Ich\^{ }é Indispensible to Living with a Three-Year-Old,'' and walked around campus listening to Chiwere recordings on my iPod.

The first summer, I spent a lot of time with Jimm's grandson. We were friends since there were few other children around and none that knew or seemed interested in learning Chiwere despite our proximity to the reservation. He was also my primary language teacher. I knew more grammar, but he knew more words, so I would constantly ask him for the names of things: ``Dagúra? Jé\^{ }e ráye dagúra? Sé\^{ }e dagwígana je?'' (`What is it? What's the name for this? What do they call that?'). During a walk through the woods, I asked about the Chiwere word for `leaf', which got us confused as we tried to sort out \emph{náwo} `path', \emph{náawe} `leaf', \emph{náwe} `hand', and \emph{núwe} `two'. Since we only communicated in Chiwere, we often played in relative silence. When we went to the playground, we would chase each other and jab each other with our fingers. We did not jab each other hard, but the person who was jabbed was supposed to yell, ``Ow, pahíⁿ, gích\^{ }e ke!'' (`Ow, that's sharp, it hurts!'). Or, we would climb to the top of the slide and ask each other what we saw:

	``Díno xáñe arásda je?' (`Do you see a big dinosaur?'), he would say.
	
	``Húⁿje, áta ke,'' I would say. (`Yeah, I see it.')

Our Chiwere repertoires consisted primarily of play routines and household phrases. `Good morning.' `Pass the salt.' `Have you seen my keys?' `Where's the dog?' `Good night.' In novel situations, we communicated with varying degrees of success through a combination of Chiwere, English, and body language. My own oral Chiwere skills peaked quickly that summer because Jimm and I usually spoke to each other in English. The topics we discussed were often technical (related, for example, to the formatting of dictionary entries, computer issues, and the like), and neither of us felt comfortable addressing them in Chiwere. From time to time we would talk about talking more in Chiwere with each other, but we never kept it up for long.

After the first summer, I returned and lived with Jimm and his grandson on and off for two years during my fieldwork. I volunteered to help home school his grandson in Chiwere. Chiwere is the first language he learned to read. When we began matching letters to sounds, his favorite activity was going through the consonant and glottal stop combinations: \emph{k\^{ }a, k\^{ }e, k\^{ }i, k\^{ }o, k\^{ }u}. One problem was that there were only two books in Chiwere for us to read. One, called \emph{Hó Gíthige}, was a story about an uncle and nephew going fishing that Jimm translated from Lakota. The other was \emph{Mischíñe na Náthaje}, a wékaⁿ (`myth') Jimm had hand-illustrated. His grandson was soon bored of reading those two books over and over. On the math side, I was able to teach addition and subtraction in Chiwere but had a hard time trying to explain more advanced topics like multiplication or division.

Pressures to improvise gave our Chiwere novel features, including codeswitching and lexical and grammatical innovation. For example, Jimm's grandson would say ``Wanna sgáje?'' for `Do you wanna play?' We were also forced to come up with new words for a number of household objects, some of which found their way into Jimm's dictionary. When Jimm's grandson wanted something to drink, he would say ``Dagúra sráhdaⁿ?'' which means `What do you want to drink?' He associated this phrase with receiving something to drink because this is what Jimm would ask him before giving him something to drink. Jimm's grandson was unaware that \emph{rahdaⁿ} and other \emph{ra}- initial verbs follow an irregular conjugation pattern for first- and second-person forms. Sometimes, I responded by saying what I wanted to drink rather than giving him something to drink to try to help him understand that \emph{sráhdaⁿ} is a second-person form -- that is, I would respond to the semantic rather than pragmatic meaning of his utterance. He would say, ``Hiñégo, mí\^{ }e hasráhdaⁿ!'' (`No, me, I-you-drink!'), \emph{hasráhdaⁿ} being an ungrammatical form that includes both the regular first-person prefix \emph{ha}- and the irregular second-person prefix \emph{s-}. Jimm's grandson seems not to have recognized the irregular pronoun prefixes at all and thus treated the \emph{s-} as part of the word for `drink'. \Citet{Moore1988} describes a similar tendency to lexicalize already inflected forms as stems available for further inflection among younger speakers and semispeakers of Wasco. 

One challenge that faced the language nest was a lack of reinforcement beyond the household. All of Jimm's grandson's favorite television shows, movies, and books were in English, and he could rarely if ever use Chiwere to communicate with anyone besides his grandfather, Bryan, and me. Once when we visited another reservation to help Head Start teachers incorporate Chiwere into their classrooms, the tribal language coordinator asked Jimm's grandson to say something in Chiwere. ``Won't you say a little something?'' he said. ``Even a word or two?'' Jimm's grandson just stared at him with a shy smile and shook his head. ``He doesn't do performances,'' Jimm said, ``he uses the language to communicate.'' Unfortunately, Jimm's grandson had few opportunities to communicate in Chiwere outside the household, and supporters in the community treated him as something of a spectacle. Others expressed concern that his language acquisition would be delayed or that he would never learn to speak proper English. Over time, Jimm's grandson began speaking to me in English even when I would address him in Chiwere, and I incorporated more English into the home school. We were just as likely to read \emph{Harry Potter} as \emph{Hó Gíthige} and \emph{Mischíñe na Náthaje}, and all our math was in English, which suited me well because my Chiwere abilities found their limit in trying to explain multiplication and division. Chiwere went from being the medium of instruction to a special subject. As Jimm puts it, Chiwere may not have the prominent place that it once did, but it is still a permanent presence.

Reflecting on my experiences, Jimm's language nest has much in common with other home-based attempts to revive languages that are no longer spoken. One of the most famous examples of language revival, often presented as an inspirational model to indigenous communities whose heritage languages are ``sleeping'' \citep[see][]{Hinton2008}, is Hebrew. According to the popular narrative of Hebrew revival, \iai{Eliezer Ben-Yehuda}, a late nineteenth-century Zionist, raised his children to be the first native speakers of modern Hebrew. For the previous 2,000 years, Jews had studied Hebrew for religious purposes but used vernacular languages for everyday communication. In the beginning, Ben-Yehuda's own Hebrew skills were less than fluent, and his lexicon lacked terms for many household objects and activities. Thus, when he wanted his wife to prepare a cup of coffee, ``he was at a loss to communicate words such as `cup,' `saucer,' `pour,' `spoon,' and so on, and would say to his wife, in effect: `Take such and such, and do like so, and bring me this and this, and I will drink.'\thinspace'' \citep[37--38]{Fellman1973} Over time, however, Ben-Yehuda came up with words to fill these gaps. Furthermore, family friends urged Ben-Yehuda and his wife to speak other languages to their children because they feared that they would grow up unable to speak at all \citep[50--53]{BenAvi1984}. Like Ben-Yehuda's household, Jimm's language nest also involved dealing with gaps between communicative goals and abilities, innovating lexical and grammatical forms, and resisting external pressure to speak dominant languages.\footnote{While Ben-Yehuda plays a central role in popular accounts of Hebrew revival, the emergence of modern Hebrew was a complex process that went far beyond his efforts (Harshav 1999). Nevertheless, his household still provides an interesting comparative context for examining home-based strategies for language revival.}

There are also parallels between Jimm's household and a more recent case: the revival of Miami, an Algonquian language, which was labeled ``extinct'' when the last fluent speaker died in the 1960's. In the 1990's, however, \iai{Daryl Baldwin}, a tribal member, learned Miami from historical documentation and then taught his wife and four children. Like Jimm and his grandson, the Baldwins also lived in a rural area and home-schooled their children in order to create some separation from outside influences. Nevertheless, all of the Baldwins are bilingual and bicultural. The family's language is characterized by a focus on domestic topics, relatively simple grammar, and changes from classical Miami reflecting some influence from English as well as lexical innovation. The Baldwins agreed to use Miami with each other whenever possible, but in practice around 30\% of their conversation time was in Miami, though there was considerable variation depending on the topic \citep[14]{Leonard2007}.

Leonard argues that the Miami case shows the reclamation of sleeping languages as languages of daily communication is possible, but he is careful to note that the measure of success in the Baldwin household is not Miami language fluency. Instead, the goal is to develop language proficiency as a means of enhancing the family's connection to traditional worldviews and their modern Miami identities. \iai{Daryl Baldwin} refers to the goal as ``cultural fluency'' (\citealt[36--37]{Leonard2007}; \citeyear[139--140]{Leonard2011}).\footnote{Another example of a similar language nest can be found in the documentary film \emph{We Still Live Here} \citep{Makepeace2011}, which addresses efforts by \iai{Jessie Little Doe} to teach her daughter Wampanoag. Accounts of household language nests by Baldwin, Little Doe, and others are included in the recent edited volume \emph{Bringing Our Languages Home} \citep{Hinton2013}.} 

Similarly, Jimm's goal was never for his grandson to speak only Chiwere forever. Jimm expected his grandson to switch to English as his primary language of communication but hopes that his life will always be enriched by Chiwere language and associated cultural traditions. Like Leonard, my experience living in a household working to revive an indigenous heritage language has led me to believe that such efforts are entirely possible, especially when the heritage language has a strong connection to distinctive cultural practices and values, when there is a collective commitment within the household to use the language for communication as often as possible, and when there is an openness to the new words and grammatical structures that arise when speaking a language that has not been spoken for many years. A supportive broader community or state apparatus may be necessary for language revival on a large scale, but much can be accomplished by an individual ``language genius'' or family if they dedicate themselves to the task.

\section{Conclusion}
In this chapter, we have tried to present a multifaceted account of the Ioway, Otoe-Missouria Language Project's Chiwere revitalization activities, including personal accounts of our motivations and histories, a textual artifact from Jimm Goodtracks' Chiwere language nest, and a broader context for understanding how the Project's approach compares to other approaches to language revitalization in Siouan communities and beyond.

There is no way to tell what the future for Chiwere and other sleeping Siouan languages will hold. As our accounts demonstrate, successful language revitalization often depends on a fortuitous and potentially fragile combination of individual dedication, support from family and friends, collaboration with linguists and communities, and a sense of responsibility to previous and future generations. In the words of the Ioway, Otoe-Missouria Language Project's mission statement, the goal of the Project is ``to share in part the Elders' desire to continue their language and traditional culture and knowledge,'' but ``ultimately, it is the role of each descendant to continue it further among their own family and relations.''

\emph{Hó, Náwo Pí ramáñišge tahó!}  May you walk a good road!

\section*{References}


\begin{reflist}

Ahlers, Jocelyn C. 2006 Framing Discourse: Creating Community through Native Language Use. Journal of Linguistic Anthropology 16(1):58--75.

Ben-Avi, Ittamar 1984 Ittamar Ben Avi, the ``Guinea Pig Infant'' of Modern Spoken Hebrew, Tells How He Uttered His First Word in \citet{Hebrew1885}. In Momentous Century: 	Personal and Eyewitness Accounts of the Rise of the Jewish Homeland and State, 1875-	1978. Levi Soshuk and Azriel Eisenberg, eds. New York: Herzl Press.

Bogdanovich, Peter 1973 Paper Moon. Hollywood, CA: Paramount Pictures.

Buffalohead, Eric 2004 Dhegihan History: A Personal Journey. Plains Anthropologist 49(192):327--343.

Dauenhauer, Nora Marks, and Richard Dauenhauer 1998 Technical, Emotional, and Ideological Issues in Reversing Language Shift: Examples from Southeast Alaska. In Endangered Languages: Current Issues and Future Prospects. Lenore A. Grenoble and Lindsay J. Whaley, eds. Pp. 57--98. Cambridge: Cambridge University Press.

Davidson, Jill D. 1997 Prayer-Songs to Our Elder Brother: Native American Church Songs of the Otoe-Missouria and Ioway. Ph.D. dissertation, Department of Anthropology, University of Missouri-Columbia.

Debenport, Erin 2015 Fixing the Books: Secrecy, Literacy, and Perfectibility in Indigenous New Mexico. Santa Fe: School for Advanced Research Press.

Fellman, Jack 1973 The Revival of a Classical Tongue: Eliezer Ben Yehuda and the Modern Hebrew Language. The Hague: Mouton.

Fishman, Joshua A. 1991 Reversing Language Shift: Theoretical and Empirical Foundations of Assistance to Threatened Languages. Clevedon: Multilingual Matters.

Foster, Lance M. 1989 Mankánye Washi. http://ioway.nativeweb.org/iowaylibrary/mankanye\underline{ }washi\underline{ }1.htm, accessed June 28, 2015.

Foster, Lance M. 1996 The Ioway and the Landscape of Southeast Iowa. Journal of the Iowa Archeological Society 43: 1--5.

Foster, Lance M. 1999 Tanji na Che: Recovering the Landscape of the Ioway. In Recovering the Prairie. Robert F. Sayre, ed. Pp. 178--190. Madison: University of Wisconsin Press.

Foster, Lance M. 2009 The Indians of Iowa. Iowa City: University of Iowa Press.

Foster, Lance M. N.d.a Ioway Cultural Institute. http://ioway.nativeweb.org/, accessed June 27, 2015.

Foster, Lance M. N.d.b Nemaha. http://nemaha.blogspot.com/, accessed June 27, 2015.

Foster, Lance M. N.d.c The Sleeping Giant. http://hengruh.livejournal.com/, accessed June 27, 2015.

Furbee, N. Louanna, and Lori A. Stanley 1996 Language Attrition and Language Planning in Accommodation Perspective. Southwest Journal of Linguistics 15: 45--62.

Furbee, N. Louanna, and Lori A. Stanley 2002 A Collaborative Model for Creating Heritage Language Curators. International Journal of the Sociology of Language 154: 113--128.

Goodtracks, Jimm G. 1985 Banyi Baxoje-Jiwere Nut?achi: An Indian New Year Calendar Featuring Iowa-Otoe Missouria People, May 1985-April 1986. Perkins, OK: Jimm G. Goodtracks.

Goodtracks, Jimm G. 1992 Baxoje-Jiwere-Nyut'aji--Ma'unke Iowa-Otoe-Missouria Language to English. Boulder: Program for the Study of Indigenous Languages, Department of Linguistics, University of Colorado.

Goodtracks, Jimm G. 2004a Text for Baxoje \textasciitilde Jiwere \textasciitilde Ich\^{ }e Ioway \textasciitilde Otoe Language CD I. Lawrence, KS: Baxoje \textasciitilde Jiwere Ich\^{ }e Ioway \textasciitilde Otoe Language Project.

Goodtracks, Jimm G. 2004b Text for Baxoje \textasciitilde Jiwere \textasciitilde Ich\^{ }e Ioway \textasciitilde Otoe Language CD II. Lawrence, KS: Baxoje \textasciitilde Jiwere Ich\^{ }e Ioway \textasciitilde Otoe Language Project.

Goodtracks, Jimm G. 2008 Mihxóge. In BJM Ich\^{ }e Wawagaxe: IOM Dictionary. Lawrence, KS: Ioway, Otoe-Missouria Language Project. http://iowayotoelang.nativeweb.org/pdf/m\underline{ }btoe\underline{ }feb2008.pdf, accessed June 28, 2015.

Goodtracks, Jimm G. 2014 Here Is a Little Curious Inquiry for All of Us to Ponder. Post to the Siouan List, April 11. http://listserv.linguistlist.org/pipermail/siouan/

Goodtracks, Jimm G. 2014-April/009473.html, accessed June 18, 2015.

Goodtracks, Jimm G. N.d. Welcome to the Ioway, Otoe-Missouria Language Website. Ioway, Otoe-Missouria Language Project. http://iowayotoelang.nativeweb.org/goals.htm, accessed June 27, 2015.

Gordon, Bryan James 2014 Here Is a Little Curious Inquiry for All of Us to Ponder. Post to the Siouan List, April 11. http://listserv.linguistlist.org/pipermail/siouan/2014-April/009468.html, accessed June 28, 2015.

Harshav, Benjamin 1999 Language in Time of Revolution. Stanford: Stanford University Press.

Hartmann, Iren, and Christian Marschke, eds. 2010 Hoc\k{a}k Teaching Materials, Volume 2: Texts with Analysis and Translation and an Audio-CD of Original Hoc\k{a}k Texts. Albany: State University of New York Press.

Hinton, Leanne 2008 Sleeping Languages: Can They Be Awakened? In The Green Book of Language Revitalization in Practice. 3rd edition. Leanne Hinton and Ken Hale, eds. Pp. 413--418. Bingley: Emerald.

Hinton, Leanne  ed. 2013 Bringing Our Languages Home: Language Revitalization for Families. Berkeley: Heyday.

Jones, Matthew L. 2004 Wahtohtana héda \~Nyút\^{ }achi Mahín Xánje Akípa: The Year the Otoe and Missouria Meet the Americans. Wicazo Sa Review 19(1):35--46.

Kan, Sergei, ed. 2001 Strangers to Relatives: The Adoption and Naming of Anthropologists in Native North America. Lincoln: University of Nebraska Press.

Kanza Language Project 2004 Wajíphaⁿyiⁿ: A Kanza Language Game. Kaw City, OK: Kaw Nation.

Kanza Language Project 2010 Kaáⁿze Wéyaje - Kanza Reader: Learning Literacy through Historical Texts. Kaw City, OK: Kaw Nation.

Kent, Reuben 2004 Bidanwe. Compact disc.

Kent, Reuben N.d. Turtle Island Storyteller Reubin Kent: These Native Ways. Wisdom of the Elders. http://wisdomoftheelders.org/turtle-island-storyteller-reubin-kent/, accessed June 4, 2015.

King, Jeanette 2008 Te K\=ohanga Reo: M\=aori Language Revitalization. In The Green Book of Language Revitalization in Practice. 3rd edition. Leanne Hinton and Ken Hale, eds. Pp. 119--128. Bingley: Emerald.

Koontz, John 2003a FAQ. http://spot.colorado.edu/\textasciitilde koontz/faq.htm, accessed June 28, 2015.

Koontz, John 2003b Names. http://spot.colorado.edu/\textasciitilde koontz/faq/names.htm, accessed June 28, 2015.

Kroskrity, Paul V. 1998 Arizona Tewa Kiva Speech as a Manifestation of a Dominant Language Ideology. In Language Ideologies: Practice and Theory. Bambi B. Schieffelin, Kathryn A. Woolard, and Paul V. Kroskrity, eds. Pp. 103--122. New York: Oxford University Press.

Kroskrity, Paul V. and Margaret C. Field, eds. 2009 Native American Language Ideologies: Beliefs, Practices, and Struggles in Indian Country. Tucson: University of Arizona Press.

Leonard, Wesley Y. 2007 Miami Language Reclamation in the Home: A Case Study. Ph.D. dissertation, Department of Linguistics, University of California, Berkeley.

Leonard, Wesley Y. 2011 Challenging ``Extinction'' through Modern Miami Language Practices. American Indian Culture and Research Journal 35(2):135--160.

Lewis, M. Paul, Gary F. Simons, and Charles D. Fennig. eds. 2013 ``Iowa-Oto.'' In Ethnologue: Languages of the World. 17th edition. Dallas: SIL International. http://www.ethnologue.com/language/iow, accessed June 28, 2015.

Makepeace, Anne 2011 We Still Live Here. Lakeville, CT: Makepeace Productions.

Meek, Barbra A. 2010 We Are Our Language: An Ethnography of Language Revitalization in a Northern Athabaskan Community. Tucson: University of Arizona Press.

Moore, Robert E. 1988 Lexicalization Versus Lexical Loss in Wasco-Wishram Language Obsolescence. International Journal of American Linguistics 54(4):453--468.

Muehlmann, Shaylih 2008 ``Spread Your Ass Cheeks'': And Other Things That Should Not Be Said in Indigenous Languages. American Ethnologist 35(1):34--48.

Murray, Franklin 1977 Iowa Tribe of Oklahoma Language. Perkins: Iowa Tribe of Oklahoma. http://bahkhoje.com/wp-content/uploads/2014/11/Franklin-M.-Murray-An-Original-Ioway-Speaker-1896-1983-First-Ioway-Dictionary-created-by-him.pdf, accessed April 22, 2015.

Nevins, M. Eleanor 2013 Lessons from Fort Apache: Beyond Language Endangerment and Maintenance. Chichester, West Sussex: Wiley-Blackwell.

Otoe and Iowa Language Speakers with Lila Wistrand-Robinson 1977 Jiwele-Baxoje Wan'shige Ukenye Ich'e Otoe-Iowa Indian Language Book I: Alphabet, Conversational Phrases, and Drills. Park Hill: Christian Children's Fund American Indian Project.

Otoe and Iowa Language Speakers with Lila Wistrand-Robinson 1978 Jiwele-Baxoje Wan'shik'okenye Ich'e Otoe-Iowa Indian Language Book II: Simple, Compound and Complex Sentences; Songs and Stories for Practice. Park Hill: Christian Children's Fund American Indian Project.

Otoe-Missouria Language Department 2014 Jiwere-Nut'achi Banyi Koge Nuwe Agr\k{i} Grebr\k{a} Agr\k{i} That\k{a}: 2015 Otoe-Missouria Calendar. Red Rock, OK: Otoe-Missouria Language Department.

Parks, Douglas R., and Robert L. Rankin 2001 Siouan Languages. In Handbook of North American Indians, vol. 13: Plains. Raymond J. DeMallie, ed. Pp. 94--114. Washington, DC: Smithsonian Institution.

Perley, Bernard C. 2011 Defying Maliseet Language Death: Emergent Vitalities of Language, Culture, and Identity in Eastern Canada. Lincoln: University of Nebraska Press.

Ramey, Brett N.d. Hóragewi. http://www.horagewi.com/, accessed June 27, 2015.

de Reuse, Willem 2014a Alice Translation. Post to the Siouan List, April 11. http://listserv.linguistlist.org/pipermail/siouan/2014-April/009475.html, accessed June 28, 2015.

de Reuse, Willem 2014b Saul and Jan's Discussion. Post to the Siouan List, April 25. http://listserv.linguistlist.org/pipermail/siouan/2014-April/009557.html, accessed June 28, 2015.

Reyhner, Jon 1999 Introduction: Some Basics of Language Revitalization. In Revitalizing Indigenous Languages. Jon Reyhner, Gina Cantoni, Robert N. St. Clair, and Evangeline Parsons Yazzie, eds. Pp. v-xx. Flagstaff: Northern Arizona University.

Rundle, Kelly, and Tammy Rundle 2007 Lost Nation: The Ioway 1. Moline, IL: Fourth Wall Films.

Samuels, David W. 2006 Bible Translation and Medicine Man Talk: Missionaries, Indexicality, and the ``Language Expert'' on the San Carlos Apache Reservation. Language in Society 35(4):529--557.

Ullrich, Jan 2014a Citation/Quotation Conventions for List? Post to Siouan List, April 25. http://listserv.linguistlist.org/pipermail/siouan/

Ullrich, Jan 2014-April/009556.html, accessed June 28, 2015.

Ullrich, Jan 2014b Saul and Jan's Discussion. Post to Siouan List, April 25. http://listserv.linguistlist.org/pipermail/siouan/2014-April/009559.html, accessed June 28, 2015.

Warner, Sam L. No`eau 2008 The Movement to Revitalize Hawaiian Language and Culture. In The Green Book of Language Revitalization in Practice. 3rd edition. Leanne Hinton and Ken Hale, eds. Pp. 133--144. Bingley: Emerald.

Whiteley, Peter 2003 Do ``Language Rights'' Serve Indigenous Interests? Some Hopi and Other Queries. American Anthropologist 105(4):712--722.

Wilson, William H., and Kauanoe Kaman\=a 2008 ``Mai Loko Mai O Ka `I`ini: Proceeding from a Dream?: The `Aha P\=unana Leo Connection in Hawaiian Language Revitalization. In The Green Book of Language Revitalization in Practice. 3rd edition. Leanne Hinton and Ken Hale, eds. Pp. 147--176. Bingley: Emerald.

Wolfenbarger, Deon 1996 White Cloud Historic District. National Register of Historic Places Registration Form. National Park Service, United States Department of the Interior. http://pdfhost.focus.nps.gov/docs/NRHP/Text/96000701.pdf, accessed June 28, 2015.

\end{reflist}


% \printbibliography[heading=subbibliography,notkeyword=this]

\end{document}