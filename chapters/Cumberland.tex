\documentclass[output=paper]{LSP/langsci}
\author{Linda Cumberland}
\title{In his own words: Robert Rankin recalls his work with the Kaw people and their language}

\abstract{In this edited transcript of a 2011 interview, Robert Rankin discusses his early train- ing in linguistics, his first contacts with the Kaw people and language, and his subsequent lifelong involvement with the Kaw Language Project. KEYWORDS: [Kaw, Kaw Language Project, Robert Rankin, fieldwork]}

\maketitle

\begin{document}


Robert Rankin was fundamental to the development of the Kaw Language Project, an office maintained by the Kaw Nation at its headquarters in Kaw City, Oklahoma. Under Bob's supervision and with his tireless assistance, the KLP has produced an array of language materials, including teaching materials, a volume of Kaw language texts (\textit{Kaa\textsuperscript{n}ze Weyaje}---\textit{Kanza Reader}; McBride \& Cumberland 2010) and a dictionary (\textit{Kaa\textsuperscript{n}ze Ie Wayaje: An Annotated Dictionary of Kaw (Kanza)}; Cumberland \& Rankin 2012). In the early 2000s, he and then Language Director Justin McBride worked together to assemble a comprehensive collection of all Kaw language data known to exist and archived it at tribal headquarters. His work with Kaw people and their native language extended over four decades and produced the only available sound recordings of the language, collected in the 1970s, from the last native speakers of the language. In December 2011, I, as Language Director at that time, sat down with Bob to ask him to recall those early days of his work and the speakers he worked with. This is an edited version of that conversation.

\textbf{\underline{Linda}}: It's December first, 2011. I'm in the Kaw Nation Language Office with Dr. Robert Rankin to talk about his field experience in the 1970s, recording the last first-language native speakers of the Kaw language. So, Dr. Rankin ---Bob---, why don't I just ask you to tell in your own words how you started doing fieldwork down here and about your fieldwork.

\textbf{\underline{Bob}}:  Well, it's hard to know exactly where to start. I was trained in European languages in my college and university work, and I stayed with European languages right on down through my doctoral dissertation at the University of Chicago. I did two years of work recording Romanian language dialects in Romania---which at that time was a Communist country---under a Fulbright Student Grant in the mid-1960s and had gotten the job at the University of Kansas on the basis of my linguistic training and my knowledge of those European languages. But I had a colleague at the University of Kansas, Dale Nicklas, who had worked most of his life with the Choctaw people in southeastern Oklahoma, and he kept reminding me that it was the duty of every American linguist to try to document at least one Native American language. 

I'd always kind of ignored him, but he finally got to me, so I'd begun work in northeastern Oklahoma trying to find people who could still speak Quapaw in the summer of 1972. I was really unable to find more than one or two people who remembered much Quapaw. I familiarized myself with the literature on the Siouan language family, the family of languages related to Dakota Sioux, of which Kaw was one, and I had determined that after Quapaw, the Kaw language was the least well documented in the archival materials available in the Smithsonian and other archives. 

And so, about mid-1973, I thought that I would try to find speakers of Kaw, or as I called it at that time, Kansa.\footnote{There are no hard and fast rules for the use of the terms \textit{Kansa} (or \textit{Kanza}) and \textit{Kaw}. Both are used interchangeably in reference to the language, while the people generally refer to themselves as \textit{Kaw}. -LC} Of course, one of the things you have to realize is that it was hard to find people who could talk about a project like this. I had no idea who would know the Kaw language, how many people spoke it, where they were, what their names were, where they were located. This was all sort of a mystery. I remember asking the woman behind the desk at the Osage Museum over in Pawhuska if she knew anybody who could still speak the Kansa language and she gave me the names of three or four families. And so I came over here to see what I could find out. 

The gentleman I located lived in the village of Washunga, which at that time was not yet flooded by the Army Corps of Engineers [to create Kaw Dam and Kaw Lake]. I think it was Clyde Monroe. I went to his home. He was bedridden at that time. His family permitted me to talk with him for a few minutes, and I told him the kind of project I was interested in. He said I should talk with Walter Kekahbah or Maude Rowe, Ralph Pepper or Tom Conn. He may have named one or two other individuals, maybe one or more of the Mehojahs. So I did that. He told me that Walter Kekahbah lived in the nursing home up in Newkirk [Oklahoma]. I went up there and I was able to interview him. He claimed that he had forgotten most of his Kaw but, really, he hadn't. He made some recordings for me, some brief recordings, and they turned out to match very well with the Kaw that was documented in the late 19\textsuperscript{th} century. So, he remembered a lot more Kaw than he thought he did. But he told me that I really should talk with Maude Rowe, that she had spoken [Kaw] more recently than he had. 

I took his advice and met with Maude Rowe's son, Elmer Clark. Elmer invited me over that evening for some watermelon. He said his mom was going to be there, so I got together with her. She agreed to make recordings for me and so I set up a schedule to visit her weekday afternoons at 2:00 where she was living in Pawhuska. So we did that for part of, uh, it would have been the summer of '73, the latter part of the summer of '73. 

Well, I don't usually talk about this sort of thing with people, but I was really kind of scared of this whole project. I guess I'm a sort of timid sort of guy and I{\ldots} it{\ldots} I don't know. I felt like I was intruding or, uh---I don't know. But anyway, I got over my fear, my reticence, and started making recordings with Maude Rowe. She moved up to Shidler, Oklahoma shortly thereafter. I think she moved the next year. And so I would meet her there. 

We would meet for about two hours, from about 2:00 p.m. to 4:00 p.m. every weekday while I was down here. I would come down for about two weeks at a time and then I would go back up to Lawrence [Kansas] and do all the other things that had piled up, waiting for me to attend to them, and then I'd make another trip and we would go over more material. At that time, I acquired a lot of copies of archival materials stored in the Smithsonian Institution, by 19\textsuperscript{th} century amateur linguist James Owen Dorsey. He had worked recording stories and vocabulary in the Kaw language in the late 1880s, early 1890s. So I acquired copies of the various Kaw traditional stories that he had recorded back then and I went over each one of these, word by word, with Mrs. Rowe. She would conjugate the verbs for me and explain portions of the stories that I couldn't understand. So she was a big help all along, and she provided more vocabulary. So we did that on and off in summer vacations. Sometimes I'd come down in the Christmas holidays, the Easter holidays, for a couple days and we'd do more recording.\footnote{All of Rankin's recordings with Mrs. Rowe during these sessions were sound recordings, nearly all of which he later transcribed in his field notebooks. Copies of these sounds recordings and notebooks are archived at the Kaw Nation. James Owen Dorsey's recordings, of course, are written only, although Rankin did make sound recordings of Mrs. Rowe repeating or elaborating on some sections of Dorsey's work.}

She suggested that I talk to Ralph Pepper, in Tulsa, so I got in touch with one of Ralph Pepper's daughters, Hattie Lou Pepper, and she introduced me to her dad. Her dad was very hard of hearing when I met him, and to record with Mr. Pepper, I had to basically write out my questions ahead of time so he could read them because he had trouble hearing my voice. He could always understand his daughter very easily but when I would try to talk to him, he had problems hearing, so he would respond to written questions and translate material. Occasionally he would include a little prayer. This would have been already in the mid-1970s, I think. So we continued. I did some recording with Mr. Pepper, although his hearing loss made it difficult. And I did quite a lot of recording with Maude Rowe. Oh, Levi Choteau --- he was the other speaker that she mentioned. I had talked to him. I had gone down to Reno, Oklahoma and talked briefly with him, but he was bedridden at the time and wasn't able to do much recording for me.

So, that basically went on. I would come down during vacations, whenever I had some spare time, and do recordings [with Maude Rowe]. This lasted until I started having back problems, sort of in the late 70s. I had to have some surgery then and so that sort of broke things off. Mrs. Rowe passed away in 1978 and Ralph Pepper, in the early 1980s. So at that point, I think, that was pretty much the last of the individuals who could speak fluent Kaw. That's about the story [of work with Kaw speakers].

\textbf{\underline{Linda}}:  What were your impressions of Mrs. Rowe, her personality? What was a typical session like? Just reminisce about Mrs. Rowe a bit.

\textbf{\underline{Bob}}:  Well, that's always a pleasure. She was a delightful woman. Of course at that time I was this 30-year-old and she was a woman in her 70s, I guess, so again I guess I felt a little reticent, a little timid, you know, but she was always a pleasure to work with, to record with. She had a wonderful sense of humor. I would generally spend the mornings at my motel room, going over the material that I was going to ask her about that day. I'd prepare in the mornings, and then I'd go get lunch and to Shidler and record with her a couple of hours, going over the materials I had prepared. 

One morning I had gotten lazy and--- I don't know what I did, overslept or ate a long breakfast, or something. Anyway, I was not prepared. So I went up with my tape recorder and I opened my notebook of James Owen Dorsey traditional tales that he had tried to write down in the 1800s. So, what I would do is read what he had written down out loud and then I'd get her to tell me what it meant to her, and verify with her that Dorsey had gotten the meaning right and had gotten the pronunciation right. I'd get her to pronounce the sentences, too. And so I was reading along --- it was the story of the two raccoons --- and I got to one point that I hadn't noticed because I hadn't prepared that day, where underneath the sentences written in the Kaw language, where normally Dorsey would write the English translation, in this particular instance he had written Latin, not English. So he'd written these Latin words there, and I hadn't bothered to check. So I got to this sentence and read it out loud. And Mrs. Rowe just sat there for a minute. And then she kind of made this funny noise. She kind of went, ``Pff, pff, pff,'' you know, through her lips, and then she just burst out laughing, laughing as hard as she could. I couldn't figure out what the deal was, and then I looked to see the translation and I saw the Latin---and I can read Latin---and it was, uh, shall we say, very questionable language, really ripe. Raccoon had done something really naughty and then had done it again. And I had read this right out loud to Mrs. Rowe! And if she hadn't had a terrific sense of humor, that probably would have been the end of my fieldwork, right that day. I don't think either one of us would have gotten over it. I was so embarrassed! I had used language with her in her native language that I would hesitate to use to my own mom or grandma---maybe even my own wife. It was pretty raw language, but she, heh, just laughed like crazy. We got together with Elmer Clark that evening and we were talking about the day's recording and she said, ``Oh, we had a humdinger! It was a humdinger!'' [Bob laughs.] And it was.

\textbf{\underline{Linda}}: Wasn't it [the use of Latin] sort of a product of the times when Dorsey did that? It was my understanding that there were censorship laws that kept them from writing certain things in English, so they wrote it in Latin to avoid putting stuff in print that would have violated [American] censorship laws.

\textbf{\underline{Bob}}: That's entirely possible. I am not aware of particular laws, but it wouldn't surprise me at all because those were the days when Hollywood kisses couldn't last longer than three seconds and if you were sitting on a bed you had to have at least one foot on the floor. Heh! And that was just movies. So, yeah, I suppose if those stories were going to be published, probably they wouldn't have wanted to write those words in English.

\textbf{\underline{Linda}}: That was the era of ``banned in Boston.''

\textbf{\underline{Bob}}: Yes, absolutely. But you're right. I suspect Dorsey was under some constraints to write the translations of blue vocabulary and used Latin. \footnote{A law and subsequent judgment that had been put in place just as Dorsey was doing his fieldwork on Dhegiha languages would have led Dorsey and other ethnographers to use Latin in place of English when translating anything that might have been construed as obscene.  First, the federal Comstock Act of 1873, ``which grants the post office the power to censor mail containing material that is `obscene, lewd, and/or lascivious' {\ldots} The law and its state-level counterparts are initially used primarily to target information on contraception rather than pornography, but soon become the primary vehicles for obscenity-related prosecutions. The Comstock Act allowed warrantless searches of the mail for `obscene' materials.'' [{http://civilliberty.about.com/od/freespeech/tp/History-of-Censorship.htm}] The judgment came in 1897. ``In \textit{Dunlop v. United States}, the Supreme Court rules that print literature---including but not limited to risqu\'e fiction and information on birth control---may be considered obscene under the Comstock law. The \textit{Dunlop} standard is still current law, based on the 2005 prosecution of several Internet sites specializing in written online erotica graphically depicting illegal acts. As Chief Justice Warren Burger wrote in \textit{Kaplan v. California} (1975), ``[o]bscene material in book form is not entitled to any First Amendment protection merely because it has no pictorial content.''  [ibid.] }

\textbf{\underline{Linda}}: What was Ralph Pepper like?

\textbf{\underline{Bob}}: Like I said, he was very hard of hearing and at the time that I was able to speak with him, he had also been in an accident. He had been hit by a truck, so he was bedridden. But he was eager to speak the Kaw language. He was happy to record for me. It was just a matter of interacting with him, not being able to speak to him directly and have him understand most of what I said. I really had to write everything out, so it was kind of laborious for him, I think. He would sit there while I scribbled things out and showed him and he would say something. By that time I had done quite a lot of recording with Mrs. Rowe, so I could understand what he said. So that worked out really well. It was very important to get the pronunciation and the sentence grammar that male speakers used because the Siouan languages, unlike the English language, have one grammar for men and a different grammar for women. They actually conjugate verbs a little differently. So it's really important to get both male and female speakers of Kaw, or Quapaw, or Omaha, or Osage any of those languages.

\textbf{\underline{Linda}}: Was the work with Walter Kekahbah similarly laborious?

\textbf{\underline{Bob}}: It was difficult for me. I only met with him once or twice. I was just beginning then, so I didn't have much of any notion what the Kaw language was like. He was the first Kaw speaker I met. At the time, when he said that he had forgotten all his Kaw, I took him at his word. But after I had gotten to know something of the language after working with Mrs. Rowe, I went back and listened to the recordings of Walter Kekahbah. They were quite fluent and had a lot of the male speech features that I was looking for in Ralph Pepper's speech, also.

\textbf{\underline{Linda}}: What was the attitude of these Kaw speakers when they knew what the project was: that you wanted to make recordings and that you were going to document the language? Were they actively supportive or were they reluctant? Sometimes speakers in that generation were reluctant to share their language.

\textbf{\underline{Bob}}: I didn't find a reluctance to share. Everybody that I approached was happy to do what they could. They were interested in seeing the language documented, for posterity, you know, for their grandkids and great grandkids. I think in the beginning people didn't know what to expect from me, and for good reason. I didn't know what to expect from \textit{myself} at that time, much less what to expect from them. This was just something that sort of grew, as we worked together. With Mrs. Rowe, I recorded enough with her over the period of several years that she had gotten interested in it, herself, by the time the project began to wind down and she seemed always very eager to help out.

\textbf{\underline{Linda}}: ``\ldots preserving this for kids and grandkids\ldots'' How many members of these people's families did you meet?

\textbf{\underline{Bob}}:  I met a lot of Maude Rowe's relatives. I met a couple of Walter Kekahbah's relatives---I think his nephew and his wife. I'd met Ralph Pepper's daughter. I'd met some of the other family, some of his nephews or great nephews and so forth. But Mrs. Rowe's family, goodness! I've met her son, his daughter, \textit{her} son and daughters, and one of \textit{her} daughters has kids now, so I guess I've known four generations or so of that family. Ah---Stormy and Dewey [Storm Brave and Dewey Donelson].

\textbf{\underline{Linda}}: They keep winning prizes at [the annual Native American Youth] Language Fair. That's a really strong line of language persistence through that family.

\textbf{\underline{Bob}}: Mm-hm, mm-hm.

\textbf{\underline{Linda}}: So, once the project wound down with the data collection and you were back at the [KU] campus, what did you do with the data? Obviously, you've stayed engaged with [the Kaw Nation] for decades, but what did you do with it first?

\textbf{\underline{Bob}}: Well, the first thing you need to do, of course, is transcribe all those tapes. Most of my recordings were on old tape recorders, reel-to-reel tape machines, so that had to be written down, and I got through most of it. I may still have a little bit that still needs to be written down, even after all these years because, you know, you get busy doing other things. I had a teaching position nine months out of the year and it's almost impossible to do a lot of tape transcription during that period. Things just get in the way of writing it down and analyzing the data, you know, and papers for various linguistic conferences{\ldots} . Then I set about compiling a dictionary. I bought a computer---bought my first computer in 1984---and began typing all the vocabulary into a dictionary, which is now pretty much complete. And I went through the various stories, trying to figure out different points of grammar, how the grammar fit together in sentences, and all that sort of thing. It's very time consuming. It's what I enjoy doing, so I can't complain. I do a lot of comparative research, too, comparing material from the Kaw language to material from related, sister languages. So that's another thing that occupies quite a lot of my time.

\textbf{\underline{Linda}}: I'm curious to know how you linked up again with the Kaw tribe directly. Was that because Justin McBride got in touch with you? How did your very productive collaboration with Justin come about?

\textbf{\underline{Bob}}: Actually, it began somewhat before that. The Kaw Nation decided at some point to have a language program. I can't remember precisely what year this was. It would have been the 80s or early 90s? I'm not quite sure. Anyway, I got a phone call from Lonnie Burnett. Lonnie was, I think, the first Language Coordinator. He came up to Lawrence and brought a video-cam with him. He wanted me to record the alphabet and some sample vocabulary, some sample words containing the various letters of the Kaw alphabet. We did that, and I think the tape still exists somewhere.

\textbf{\underline{Linda}}: [pointing] Right in that drawer, right there [in the Language Director's office].

\textbf{\underline{Bob}}: Well, it was kind of an amateur production on my part, I'm afraid, but it was ok. He did that and then at some point, let's see, Kelly Test was involved---Test or Estes, I don't remember which name she was using. She was quite young at the time. She was involved with the program and then at some point Justin McBride took over the program. 

But actually, before Lonnie Burnett started up the language program here, one morning, I was sitting in my office at KU, there suddenly appears in my door this group of people: Little Carol [Clark] and Jim Pepper, Henry and Curtis Kekahbah. Oh, and Johnny McCauley. They just appeared in my office door. I recognized Curtis, and I recognized Little Carol, so, heck, I invited them in --- Old Home Week! --- and they said, ``We know you have these recordings of the Kaw language, and we wonder if you'd like to share them with the Tribe---if you'd make copies of them.'' That sounded like a fine idea to me and so I said, ``Sure.'' They arranged to have the CDs made from my copies of my tape recordings at a television studio, a professional sound studio in Wichita. So I took all my tapes, all fifty of 'em, or however many there were---'cause I don't remember how many of them there are---at least fifty \textit{hours}, anyway---and I drove 'em down to Wichita and we met down there, and they copied the tapes and then mastered several copies---one for me, one for the tribe, one for Elmer Clark's family, maybe one or two others for archival purposes. 

Basically, that provided the Tribe with copies of everything that I had done, and that served as the database for Justin's work with the language. He came to the tribe with a BA in Linguistics from OU [University of Oklahoma], and a dedication to seeing the language preserved and documented, and so he went right to work here, and the rest is history.

\textbf{\underline{Linda}}: Yes. He invented the board game\footnote{The board game, \textit{Waj\'ipha\textsuperscript{n}yi\textsuperscript{n}}, Camp Crier, is a Kaw language learning game that also introduces players to the clan structure of the Kaw tribe. The box version might still be available through the Kaw Nation museum store; an electronic version is included in the Kaw language CD set, available through the Kaw language web site, WebKanza, {http://www.kawnation.com/langhome.html}.} and certainly mapped out the structure for the language CD, which I then completed when I joined the language department. I came in January of 2006. Justin had already been there three years, possibly four.

\textbf{\underline{Bob}}: I would think so. I would think at least four. I remember he used to be in the tribal headquarters across the street, with the scorpions---heh! I think he and Kelly had been there at least three years. And then they came over here [to the Maude McCauley Clark Rowe Social Services Building].

\textbf{\underline{Linda}}: Justin hosted the annual Siouan Language Conference here in 2005. 

\textbf{\underline{Bob}}: Ok, well, you can count backwards from there. I think my recordings for Lonnie Burnett had been a year or two before Justin came, so you can count backwards from there and figure out when I had done my initial recordings for him. The delegation to Lawrence that appeared in my office would have been a few months to a year before that, I think. We were just getting the CDs available when he began his work, I think.

\textbf{\underline{Linda}}: Now, Justin left just a little over a year ago. In fact, he earned a Master's while he was here, while he was working full time, but then he left to begin work on a PhD. I think he had been here 9 years when he left. So he's left quite a legacy.

\textbf{\underline{Bob}}: Yes.

\textbf{\underline{Linda}}: And he overlapped with me---I'd been here five years when he left.

\textbf{\underline{Bob}}: So we probably did the conversion of the tapes to CDs in the mid-90s. I think that's about right. It's amazing now, that the entire stack of tape recordings that I made in the 70s, at slooow tape speed to preserve tape, all fits on one CD now.

\textbf{\underline{Linda}}: You know, it's worth noting that if you hadn't made those recordings in the 70s there would be no recordings of the Kaw language, as far as we know. There might have been some informal recordings, but yours are the only known record of how the language actually sounded. It's extremely important for the history of the Kaw language and the documentation, linguistically, just knowing what the language sounded like from fluent, first-language speakers of it.

\textbf{\underline{Bob}}: Let me add to that, since you brought it up. Tape recorders became pretty popular in the United States already in the mid-50s and so tons of people had tape recorders of various kinds in the 60s and 70s, certainly, and it may well be that in people's closets or attics or old dusty bookshelves, maybe they made some casual recordings of their grandparents---Grandma, Granddad, uncles---speaking the Kaw language. People, families, may not even remember they have those, but if they do, those recordings, if anyone has any casual recordings, no matter how short, of their elders speaking in the Kaw language, those would be extremely valuable to the language program. If they would be willing to make those available for copies, I would be happy to go through them and make sure that they get translated so that people can follow them.\footnote{Sadly, of course, such recordings would not have the benefit of Bob's knowledge today, but it would still be of great value for such recordings to be added to the archive.}

\textbf{\underline{Linda}}: Yes. So, the interesting thing, when I hear this whole history, from the moment when you just asked the question about speakers who still spoke the language to the present day, many people have come and gone, but \textit{you} are the constant. You started it; you've been with it every step of the way. You've trained any number of people in this office to work with it; you've seen it through the three essential elements of language documentation: a dictionary, a grammar (your grammatical sketch [Rankin 1989]), and texts. You are the constant throughout all work on Kaw language [in recent times]. So I know that the Tribe is extraordinarily grateful to you and you certainly deserve this sort of aura of being able to walk on water. [Both laugh.]

\textbf{\underline{Bob}}: I just consider myself lucky to have been able to get paid for doing what I really enjoy doing in life. It's a big thing in life, you know, to have a job that you really enjoy, and this is what I like.

\textbf{\underline{Linda}}: So now that you're retired, what kind of plans do you have---partly for Kaw language, but what else are you working on?

\textbf{\underline{Bob}}: Well, I'm still doing a lot of work---comparative vocabulary, that sort of thing. But I think that, especially for the Kaw language, we have a dictionary that's nearing completion, we have the Reader, the set of Kaw texts, that represent most of the traditional stories in connected texts that we've inherited from our predecessors. The main thing that's left now is a written grammar, so that's something that I think we all hope we'll be able to work on.

\textbf{\underline{Linda}}: Did you continue with your work on European languages at all?

\textbf{\underline{Bob}}: Not really. Actually, when I got into the Kaw language and really enjoying my work with the Kaw people, I never looked back. I left my European studies back in about 1975 and just pretty much came over to what I do now.

\textbf{\underline{Linda}}: Now, you've had several students who have picked up on certain aspects of Siouan research. Who has done that? Have any of them worked on Kaw?

\textbf{\underline{Bob}}: Mm, no, not really. There was one young lady whose help I enlisted at KU who analyzed some of the recordings to check the status of long and short vowels, that kind of thing. She did a brief project on that, but she didn't follow up on it. She wound up doing something else, instead. One of my students, Sara Trechter, has taken up the study of Mandan, which is up in North Dakota, and she's done a lot of work with the last living speaker of Mandan. Another of my students, Giulia Oliverio, got interested in a Siouan language that was originally spoken in Virginia in the 1700s---the 18th and 19th centuries, and has been extinct now for quite a while. She collected all the archival material that had ever been done on the Tutelo language and she drew from it as a doctoral dissertation---a lexicon and a dictionary. I've had a couple of students who have followed up. And then Dave Kaufman has been working on collecting the Biloxi material and the Biloxi dictionary.\footnote{David Kaufman served briefly as Kaw Language Director from August 2013 to December 2014. At this writing (early 2015), Justin McBride ---now Dr. Justin McBride--- has stepped in to re-assemble the Kaw language archives following a physical move to a new location.}

\textbf{\underline{Linda}}: So, all these are Siouan languages.

\textbf{\underline{Bob}}: These are all Siouan languages. Tutelo was originally spoken in Virginia---Virginia and West Virginia. Biloxi was spoken in Louisiana, and Mandan is up in North Dakota, so the speakers of the original Siouan languages got around quite a bit. They migrated over much of North America.

\textbf{\underline{Linda}}: Just to close out, for the record: Siouan languages---just informally, what makes them so closely related?

\textbf{\underline{Bob}}: The original, single Siouan language was probably spoken back in Kentucky---or back East, and the speakers broke off in groups and moved westward. And, well, one of the groups became the Mandan, one group became the Crows, one of the groups became the Dakota Sioux; a different group became the Ioway, Otoe, and Winnebago [Ho-Chunk] tribes, and then there was an additional group that split up into five sister tribes: the Kaw, the Osage, the Quapaw, the Omahas, and the Poncas, languages that are still pretty similar to each other after several hundred or a thousand years. And to a certain extent, they can still understand each other somewhat, whereas if they had to meet up with someone who spoke Dakota Sioux, Mandan, Biloxi, or one of those other languages, they wouldn't be able to understand each other.

The Kaws, the people who could speak fluent Kaw, could understand what Osage people said in their own language. Mrs. Rowe used to get her beadwork supplies from Mrs. Robinson, over in Pawhuska, who sold the beads for beadwork. Now, Mrs. Robinson was an Osage speaker and Maude Rowe could go in and speak to her in Kaw and she would speak back in Osage and they would understand each other. So those two languages are quite close. And the speakers of those languages must have diverged from what must have been a single tribe relatively recently. Earlier divisions would have been the Omahas, the Poncas, and the Quapaws. So those five tribes are closely related; their languages are very similar; they have similar vocabularies, whereas the speakers of the other language --- the Sioux, the Crows, the Mandans and so forth --- they must have split off at a much, much, much earlier date---several centuries---and their languages are very different now.

\textbf{\underline{Linda}}: A couple of the sources we're working with, George Morehouse and Addison Stubbs, were involved with the [Kaw] tribe pretty closely and did documentation, each in his own amateur way, and each said that Osage and Kaw are the ``same language.'' Maybe you could just put that notion to rest, officially.

\textbf{\underline{Bob}}: Well, they're similar enough that they could understand what one another was saying. The pronunciation is certainly different. I've heard spoken Osage, I've worked with a group of Osages back around 1980 devising an alphabet at that time for teaching the language (they've developed their own alphabet since then), so I know something about Osage --- it's different [from Kaw].

====

\noindent This is where the existing record of the interview ends. It is unclear how the original recording was lost, but in all of the existing copies of the recording, the final minute is missing. Fortunately, as I recall, it consisted only of our concluding expressions of thanks to each other for the interview. At the conclusion of the interview, we went to lunch and then went back to work on the project at hand, whatever it was that day. Bob was unfailingly generous with his time and knowledge, and refused to let his health issues deter him from continuing his work on the language. When he could no longer make the trip to Kaw City, I made a series of week-long trips to Lawrence over the following year and a half, where he could more easily work around his regular medical appointments, putting in a full eight hours each day. The pleasure he took in his work on Kaw and comparative Siouan was evident and inspiring to all of us who were privileged to work with him. 

\section*{References}
\printbibliography 

\begin{reflist}

Cumberland, Linda A. and Robert L. Rankin. 2012. Kaa\textsuperscript{n}ze Ie wayaje: An annotated dictionary of Kaw (Kanza)). Kaw City, Ok: Kaw Nation.

McBride, Justin T., and Linda A. \citet{Cumberland2010}. Ka\'a\textsuperscript{n}ze wey\'aje --- Kanza reader: Learning literacy through historical texts. Kaw City, OK: Kaw Nation.

Rankin, Robert L. (1989). Kansa (A. Yamamoto, Trans.). In T. Kamei, R. Kono, and E. Chino (Eds.), The Sanseido encyclopaedia of linguistics (Vol 2): Languages of the world, pt. 2 (pp. 301-307). Tokyo: Sanseido Press.
\end{reflist}

\end{document}