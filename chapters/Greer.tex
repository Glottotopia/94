\documentclass[output=paper]{LSP/langsci}
\author{Jill D. Greer}
\title{Baxoje-Jiwere grammar sketch }

\abstract{This synchronic grammar follows a descriptive approach to Jiwere-Baxoje of the Mississippi Valley branch of the Siouan language family.  It expands upon prior published and unpublished documentation, based upon fieldwork conducted from 1987-96. Jiwere-Baxoje-Nyut'achi is a ``sleeping language,'' with no fully fluent speakers at present, but with revitalization efforts underway in each of the three native communities of Red Rock and Perkins, Oklahoma, and White Cloud, Kansas. The sketch begins with phonology, morphology, then syntax, with special attention to the complex system of verbal affixes; the interesting phenomenon of noun-incorporation within certain verbs; verb classes (regular stems, irregular stems in \textit{r/l, w}, and \textit{d}, and the causative construction); positional verbs, which may serve as auxiliary verbs; and SOV word order, with clause-final and utterance final enclitics marking relation to the following clause, source of evidence, sentence type, and gender of speaker. The topic of language variation concludes the sketch, with gender differences documented for greetings and interjections; brief tables illustrate phonological and lexical distinctions associated with both tribal dialects. KEYWORDS: [Jiwere-Baxoje, Ioway, Otoe-Missouria, Chiwere, descriptive grammar]}

\maketitle

\begin{document}

\section{Introduction}

Baxoje-Jiwere belongs to the Mississippi Valley branch of the Siouan language family, and is the native language of the Plains/Prairie tribes known today as the Otoe-Missouria and Iowa (\citealt[3,8]{Goddard1996}).  While their original homelands were in northern Missouri, southeastern Nebraska, and the state of Iowa, during the late 19th century, the two tribes relocated to a north-central portion of Indian Territory in an attempt to avoid Euro-Americans' increasing encroachment on their reservations and the assimilation policies of the BIA.  One segment of the Iowa chose to stay on a portion of their original reservation near the Missouri River in northeastern Kansas (\citealt{Wedel2001}; \citealt{Schweitzer2001}).  

The following sketch is based upon fieldwork in central Oklahoma which I conducted mainly between 1987 and 1996 while a graduate student and research assistant within a larger team, led by Dr. Louanna Furbee, and including Lori A. Stanley.\footnote{Stanley's 1990 Ph.D. dissertation includes a life history of Dr. Truman W. Dailey, one of the primary speakers and contacts within the Otoe-Missouria Tribe, available at the University of Missouri-Columbia Library.} The research was conducted with the permission of the 1987 Otoe-Missouria Tribal Council, among members of both the Otoe-Missouria and the Ioway Tribes living in an approximately 100 mile radius of Red Rock, OK.  It was funded initially by a University of Missouri Faculty Development Grant, then generously supported by the National Science Foundation Documenting Endangered Languages Program and the American Philosophical Society's Phillips Fund.

Báxoje is the Ioway tribe's name for their people and language.  J\'iwere is the native Otoe term for themselves (and the language), while Nyút'achi refers to the Missouria people/language.\footnote{Because the Missouria language was not recorded, I omit the name Nyút'achi  when referencing the language in general, although the Missouria people and history are remembered in Otoe tribal heritage in the conjoined name today.} The native language spoken by these two tribes has frequently been called \textit{Chiwere} in the existing literature (\citealt{Whitman1947}; Marsh n.d.; \citealt{Wedel2001}; \citealt{Schweitzer2001}).  However, because this spelling makes it more likely for English speakers to mispronounce the first sound of the Otoes' self-name, I prefer to use <J> instead, because the voiced allophone is far less likely to be aspirated by language learners with English as their first language.  GoodTracks also follows this orthographic shift.  

In addition to the two contemporary communities centered in Red Rock and Perkins, Oklahoma, respectively, there is also a Northern Ioway Nation located on their original Reserve in White Cloud, Kansas.  Sadly, there are no L1 speakers of Baxoje-Jiwere, but a few individuals may be semi-speakers.  Language renewal efforts are underway in each of the small communities, so there is hope that while yet sleeping, the ancestral tongue may still be awakened.    

Many factors led to this particular effort to document Jiwere-Baxoje, but the original impetus was the collegial friendship between two University of Chicago linguists (both students of Eric Hamp), the late Robert L. Rankin and N. Louanna Furbee. These two scholars both landed jobs in the Midwest, the former at KU in Lawrence, Kansas, and the latter just a few hours away at Mizzou.  They remained in touch throughout the 1970s and 80s.  As Bob adopted Siouan languages as his primary research focus, he saw the urgent need for more linguists. He would tease Louanna that since she was employed by the primary research university in the state that was named for one dialect of this highly endangered Siouan language, it was her duty to start doing research on it.  

His good-natured urging came to fruition in 1987, when a critical mass of graduate students interested in language study surrounded Louanna and she offered a special seminar on Siouan languages.  About nine eager students enrolled in the course, myself included.  Bob came to Mizzou to give a beginning workshop to Louanna's class, with stacks of handouts full of concrete suggestions such as questions to ask and topics to cover in the field.  

His help did not end there, but continued throughout the years, giving feedback on papers, guiding our elicitation of forms for the Comparative Siouan Dictionary, reading much earlier versions of this sketch, and countless other generous acts on his part. Thus, without the initial friendship between Robert L. Rankin and N. Louanna Furbee,\footnote{The essential role of Louanna Furbee as major professor, grant writer, P.I., fieldworker, editor, friend, and all around pillar of strength cannot be overemphasized. The MCLP (Missouri Jiwere Language Project) original materials are archived at Luther College, Decorah, Iowa.} there would have been no Missouri Jiwere Language Project grammar.  This work is dedicated to them both.  All errors are of course my own.  

\section {Sound system (phonology)}	
		
\subsection{Consonants} 		     	                     	
\subsubsection{Stops} There are three sets of stops distinguished by these features: 	     
\begin{description}
\item[a.] Aspiration  /p\textsuperscript{h}, t\textsuperscript{h} , k\textsuperscript{h}/
\item[b.] Glottalization /p', t', k'/
\item[c.] Plain (neither aspirated nor glottalized) /b, d, g/	
\end{description}   		         		
The ``plain" sounds can be either voiced or voiceless, but the two allophones would have been heard by native speakers as the ``same.''  Different scholars of Baxoje-Jiwere have used either or both [p/b, k/g, t/d] for the plain (lenis) series. Variation may have existed between closely related forms within the three historic speech communities, within some families, or even with particular speakers.  Notes by earlier researchers suggest that individuals' speech did display such tendencies, but the data are too limited to address such topics at present (\citealt{Whitman1947}). In addition, the glottal stop /\textipa{P}/ does appear in word-initial, medial, and word-final positions, but in the first two instances, it serves primarily to prevent amalgamation and preserve semantic content before certain vowel-initial morphemes such as verb stems.  In those settings, its function is morphological, rather than phonemic per se. Likewise, it tends to appear in word-final position only for a limited set of morphemes, namely interjections and sentence final particles/enclitics.  In those instances, its phonetic abruptness carries an iconic meaning of emphasis, doubt, or even impatience (c.f. Tables \ref{sentencefinalparticles} and \ref{interjections}). 

\subsubsection{Affricates}  As with the stop series, there are three contrasts:  plain affricates /\v{c}/-/\v{j}/, aspirated /\v{c}\textsuperscript{h}/,  and glottalized /\v{c}'/. 

\subsubsection{Fricatives}  The plain series has a larger set of sounds than the glottalized versions.  				
\begin{description}
\item[a.] Plain: 	  /\textipa{T} \textipa{D} s  \v{s}  x   h/   								        	
\item[b.] Glottalized:  /\textipa{T}'     s'    x'/	
\end{description}
 						    
\subsubsection{Nasals}  The four nasal consonants are /m n \~n \textipa{N}/.  The latter two phonemes /\~n/ and /\textipa{N}/ were significant as indices of tribal identity.  Baxoje speakers favored \textit{\~n} in words where Jiwere speakers typically said \textit{\textipa{N}}, such as `horse': \textit{\v{s}u\textbf{\~n}e} in Ioway vs. \textit{su\textbf{\textipa{N}}e} in Jiwere.\footnote{That example also illustrates another common pronunciation difference between the distinct versions of this language, namely the plain /s/ at the beginning of words for Otoe, where Ioway produces /\v{s}/ instead.}   However, there are clear cases of  /\~n/ in both dialects, such as the shared indefinite plural \textit{-\~ne}.  Word-initial /n/ often palatalized to [\~n] before front high vowels /i, \k{i}/.  

The /\textipa{N}/ cannot occur word initially, and probably is historically derived from phonological environments where a velar stop followed a nasal vowel.  Note that there is a very strong tendency to pronounce an  epenthetic homorganic nasal consonant when nasal vowels precede stops, probably for economy of effort, or making the word ``smoother,'' as some elders liked to put it, as in the /m/ in \textit{n\k{a}\textbf{m}p\textsuperscript{h}o} `finger'.\footnote{Amelia \citegen{Susman1943} work on Hooc\k{a}k (Winnebago) mentioned the same tendency in that very closely related Siouan language.}

\subsubsection{Liquids}  There has been some difficulty defining and representing the liquid sound in Baxoje-Jiwere.  Phonetically, it has been described as resembling an unreleased ``flap'' [d] like the medial sound in \textit{latter}, the plain [r] found in Spanish, and a variation upon the [l] sound \citep[235]{Whitman1947};  Rankin also included [\textipa{D}, n, y] as possible phonetic reflexes (\citealt[346]{Wedel2001}; \citealt[447]{Schweitzer2001}). For orthographic consistency, the symbol /r/ will be used. 

\subsubsection{Glides} Glides include /w/ and /y/.  

\subsection{Vowels} 										      	         	          
\subsubsection{Oral vs. Nasal.}  There are both oral and nasal vowels in Baxoje-Jiwere.   They include /a i o u e/ and /\k{a} \k{i} \k{u}/. Frequently /\k{a}/ would be realized as a nasalized schwa. 

\subsubsection{Vowel allophones as gender indexicals.}  Phonetic vowel quality sometimes differs significantly in particular words used by female speakers especially; in those contexts, there is also an [\textipa{E}] and sometimes an [\ae].  These variations are limited to a particular small domain of the overall vocabulary of the language, and serve a social-indexical function. (Cf. section 5.2. on sentence final particles and interjections.)   

\subsubsection{Vowel length.}  Robert Rankin transcribed long vowels from a recording of a key word list by a Jiwere speaker, but I have been unable to perceive length on the same recording. No minimal pairs clearly establish phonemic significance of vowel length between etymologically unrelated words.\footnote{John Boyle's student presented a brief paper on this topic based on spectrographic analysis of MCLP recordings, but that paper has not been published.}  Thus, at present there is scant evidence to support the idea of \textit{phonemic} vowel length, although the revised Plains volume of the Handbook of North American Indians presents a list of long and short vowels, based on Rankin's analysis (\citealt[432]{Wedel1996}; \citealt[447]{Schweitzer1996}).       	         

However, there are very prolonged vowels that occur when morphological boundaries have been ``blurred'' during amalgamation.  The greatly extended length preserves the mora from the contracted morpheme, and sometimes affects the stress pattern as well.  It seems to be primarily a morphological rather than phonological process.  

\subsection{Stress/accent} 
Stress is both volume and pitch-based, with phonemic value in Baxoje-Jiwere, as in \textit{ráwe} `beaver' and \textit{raw\'e} `to count' \citep{GoodtracksND}, or \textit{g\'isa} `to laugh at another (v.)' vs. \textit{gisá}  `a knot (n.)'  \citep{DorseyND}   When a root word with two syllables has additional affixes attached to it, the basic stress (and pitch) pattern can change, typically with primary stress shifting to the left in the case of prefixation, and addition of a secondary stress in the case of infixes or suffixes.  An adequate prediction of stress patterns is beyond the scope of this grammar.\footnote{Cf. discussions of Dorsey's Law in \citet{Miner1979} and \citet{HaleWhiteEagle1980}.} 

\subsection{Syllable structure} There is a strong tendency to end all syllables with a vowel,\footnote{The few exceptions to the preference for vowel-final syllables would be represented as a CVC structure. However, such instances only appear in informal speech and seem to be elision. During quick speech, the final unstressed vowel disappears, yet speakers give the full ``precise'' pronunciation with final vowel if asked to repeat or clarify what they said. This seems to have been a major aspect of the historical sound changes separating Hooc\k{a}k from Jiwere.} thus (V) and (CV) are very frequent syllable shapes. Initial consonant clusters are allowed (CCV), but examples of CCCV have not been discovered, nor have (VCC).  The consonant clusters shown in \tabref{clusters} may begin a syllable. 

\begin{table}
\caption{Syllable-initial consonant clusters} \label{clusters}
\begin{tabular}[t]{ l l l }
\lsptoprule
a.  stop + liquid: & br- & \textit{bra}-`separated, spread in layers, sliced, flat' \\
& gr- & \textit{gr\k{u}} `to curse' \\
b.  stop + glide: & p\textsuperscript{h}y- & \textit{p\textsuperscript{h}yúbr\k{a}} `mint, medicine tea, Indian perfume  \\
& & herb' \\
& gw- & \textit{gwák'\k{u}}   `to wipe off, scrape off, dry one's self \\
& &  (body)' \\
c.  fricative + stop: & sd- & \textit{sd\k{a}} `to stop, cease, leave off' \\
& sg- & \textit{sga} `to be white, shiny' \\
& \v{s}g- & \textit{\v{s}g\k{ú}\~n\k{i}}  `no; not; (does) not' \\
& \textipa{T}g-	& \textit{\textipa{T}ga} `to be white' (old form) \\
& hg- & \textit{hga} `to be white'  (Ioway) \\
d.  fricative + liquid: & sr- & \textit{sroge} `to remove object from inside hole' \\
& \textipa{T}r- & \textit{\textipa{T}r\'ije}  `easily, softly, slowly' \\
& xr- & \textit{xra}  `eagle' \\
e.  fricative + glide: &  sw- & \textit{sw\k{a}hi} `to soften' (flesh, leather, stale bread) \\
& \v{s}w- & \textit{\v{s}w\k{a}ra} `soft (buckskin, flesh, cloth)' \\
f.  fricative + nasal: & sn- & \textit{sni} `cold' \citep{RobinsonND}   \\
& \textipa{T}n-	& \textit{\textipa{T}ni} `cold' (possibly archaic; Dorsey in \citep{GoodtracksND}  \\
\lspbottomrule
\end{tabular}
\end{table}

\subsection{Longer sound  patterns/prosody}  
For length constraint, phrase level prosody is included under the later section labeled Syntax.

\subsection{Phonological processes} 
\subsubsection{Elision.}  One of the most common changes, elision is characteristic of rapid speech, such as the final vowels mentioned in Footnote 8 which frequently are deleted.  			

\subsubsection{Vowel harmony and nasal spread}  The nasal quality of a nasal vowel may ``spread'' regressively (from right to left) to nearby vowels.  (Hooc\k{a}k scholars have documented such nasality spread not just to directly adjacent vowels, but also across the consonants /h/ and /w/ to the closest non-adjacent vowel  (\citealt[7--8]{HelmbrechtLehmann2010}).  

\subsubsection{Vowel ablaut.}  This well-known phenomenon within Siouan languages involves /a/ and /e/ which may alternate in a variety of settings, especially before particular verbs or certain suffixes, suggesting it is morphologically conditioned.  Motion verbs are one set of verbs that trigger ablaut.  Some verbs ending in \textit{-e} such as \textit{ugwe} `to enter' and \textit{re} `to go' will also ablaut to final \textit{-a}  before \textit{-wi} `definite plural' as does the indefinite plural \textit{\~ne} > \textit{na} before the definite \textit{-wi} also.  Conversely, verbal prefixes with final /a/ will ablaut to /e/ before the possessive \textit{gra-} and the verb \textit{udwá\~n\k{i}} `to fail to reach, fail to come up to' (\citealt[239-40]{Whitman1947}), as well as \textit{doye} `to break'.  The instrumental prefix \textit{gi-} `by hitting' (with an ax, hammer, or other object in the hand) will trigger ablaut from /a/ to /e/ in the pronominal prefixes which attach directly to it.  The dative \textit{gi-}, however, will not trigger the same vowel change, despite the identical phonetic shape, supporting the idea that it is not a purely phonological process.  

Examples:

\begin{exe}
\ex \gll \v{C}\textsuperscript{h}úgw\textbf{á}-wi    re. \\
house.enter-\textsc{def.pl}  \textsc{imp}(male speaker) \\

[from \v{c}\textsuperscript{h}i `house'+ ugw\underline{e} `to enter' + -\underline{wi} `\textsc{def.pl}']
\trans `Come in the house, you-all.'  [Marsh `Giants' Bk2 LN49]
\ex \gll Iwál\textbf{\`a}-wi          ho. \\
yonder.go-\textsc{def.pl} \textsc{hort}(male speaker)\\ 

[from i-`there'+ wa-`directional' + r\underline{e} `to go'+ -\underline{wi} `\textsc{def.pl}'] 		
\trans `Let's go over there!' 	[Marsh The Twins LN65]
\ex \gll  H\textbf{e}-grahi k\textsuperscript{h}i.  \\
1\textsc{p.agt}-love   \textsc{decl}(female speaker) \\

[from h\underline{a}-1\textsc{p.agt} + \underline{gra}hi `love'] 
\trans `I love him.'  
\end{exe} 

An alternative analysis accounts for the vowel change before \textit{-gra} `\textsc{poss}' and \textit{-gi} `\textsc{dat/ben}' as two vowels coalescing.  Since the key morphemes in question are consonant initial, there would have to be an underlying vowel, either /e/ or /i/.   The /e/ matches the target vowel, and parallels the \textsc{3pl} form found in the independent pronoun \textit{e\textipa{P}e}, and the possessive \textit{etháwe} `his/hers/its' and \textit{eth\'ewi} `theirs'.  But there is precedent within Baxoje-Jiwere for /a/ + /i/ to become /e/, which Whitman (1946:239) called `amalgamation.'  The volume reviewer likewise suggested that possibility, \textit{igra-}.  That shape/meaning resembles the \textsc{3pl} inalienable prefix on kin terms (\tabref{inalienable}), which parallels cognate Lakota forms and matches the reconstructed Proto-Siouan *i- Possessive (on non-verbs) (\citealt{Rood1979}).  This analysis also reserves the term `ablaut' to stem-final vowels, as has been the norm within Siouan scholarship. \footnote{Unfortunately, \textit{udwá\~n\k{i}} `to fail to reach, come up to' and other verbs with separable prefixes preceding \textsc{pron} prefixes also show the shift from final /a/ to /e/, [\textit{uh\'edwa\~n\k{i}} `I fail to reach'] and these cases do not fit neatly into the proposed explanation \citep[240]{Whitman1946}.}
           
\section{Words/morphology}
\subsection{Nouns}   Many nouns can function fully as verbs, complete with the extensive system of prefixes and suffixes described later in the verbal template. Siouan languages are classified as strongly verb oriented, with very few prefixes or suffixes limited only to nouns.\footnote{\citet{Helmbrecht2002} gives an extended discussion of ways to distinguish between nouns and verbs in Hooc\k{a}k (Winnebago).} Certain verbal prefixes transform that state/action into something more noun-like, as in the following example, wherein the verb `to eat' becomes `something to eat upon':  \textit{wá:ru\v{j}e} `table' < \textit{wa-} `indefinite object' + \textit{a-} `upon' + \textit{ru\v{j}e} `eat'. Without the locative \textit{a-} `upon', the first vowel is not lengthened, and the stress remains on the second syllable: \textit{warú\v{j}e} `something to eat, food.'  Because there is a $\varnothing$ third person pronominal prefix, `food' sounds identical to the third person singular sentence `He ate (something).'

\subsubsection{Possessing: inalienable vs. alienable}  Native American languages often distinguish people and things extremely close to a person's identity and self (\textsc{inalienable}) versus other entities that separate more easily (\textsc{alienable}). The former category includes kinship terms and in Baxoje-Jiwere, the formal social ties of friendship and parenthood.\footnote{While body parts may be inalienably possessed in other languages, it is not the case in Jiwere-Baxoje. Frozen remnants of such a system are evidenced if one interprets the initial \textit{i}- in the following body parts as representing the third person \textit{i}- inalienable prefix found in kin terms and other life-long social relationships like formal friendship and parenthood (`(his/her) child' \textit{i\v{c}i\v{c}\k{i}\textipa{N}e} (Otoe), \textit{i\v{c}i\v{c}\k{i}\~ne} (Ioway)): \textit{ihd\'oge} `elbow', \textit{ir\'eje} `shoulder', \textit{isd\k{a}} `eye'.} The prefixes meaning inalienable possession are bound morphemes similar in shape to first and second singular person patient  pronouns, but they differ in having an expressed third person form (which is sometimes dropped in fast speech), as shown in \tabref{inalienable}. (See \tabref{regularverb} for personal pronominal prefixes.)

\begin{table}
\footnotesize
\begin{tabular}[h!]{ l l l l }
\lsptoprule
Kin term\footnotemark  &	 \multicolumn{3}{c}{Inalienable possessive prefix}  \\
\midrule
& 1st person sg. & 2nd person sg. & 3rd person sg. \\
\midrule
Father & h\k{\'i}-ka 	& & \\		
&`my father/FaBr'  & & \\								            
& h\k{i}-daje	 &  naje <ri-\k{a}je & aje < i-aje  \\
&(old) `my father' & `your father' & 'his/her father' \\
Mother	 & h\k{\'i}-na 	 & & \\
& `my mother/MoZ' & & \\
& h\k{\'i}-h\k{u} & \textcrd i-h\k{u} /ri-h\k{u} &  i-h\k{u}  \\   
& archaic `my mother' &`your mother/MoZ'  & `his/her mother/MoZ' \\
Man's elder brother & h\k{i}-y\k{\'i}na  & ri-y\k{\'i}na   & i-y\k{\'i}na   \\
& `my elder brother' & `your elder brother' & `his elder brother' \\
Woman's brother & h\k{i}-\v{c}\k{i}do  &  ri-\v{c}\k{i}do   & i-\v{c}\k{i}do 	\\
& `my(\textsc{fem}) brother' & 'your(\textsc{fem}) brother' & 'her brother' \\
Grandfather & h\k{i}-t\textsuperscript{h}úga  & ri-t\textsuperscript{h}úga   & i-t\textsuperscript{h}úga   \\
&  'my grandfather' & 'your grandfather' & 'his/her grandfather'\\
\lspbottomrule
\end{tabular}
\caption{Inalienable possession} \label{inalienable}
\end{table}
\footnotetext{See \citet{Goodtracks1992} dictionary for complete inventory.}

\subsubsection{Address form -o `speaking to this one'} 

While \textit{h\k{i}t\textsuperscript{h}ára} `my friend' is the unmarked referential form, a person would switch to \textit{h\k{i}t\textsuperscript{h}áro} `my friend (address form)' while speaking directly to the special friend (formally established as cultural role).\footnote{The friendship would have been initiated by parents of two children of the same sex, formalized with a ceremonial feast, and thereafter a lifelong bond of reciprocity and obligation existed between the two, to be recognized by this word \textit{-t\textsuperscript{h}ara} `friend'. The ultimate duty came at the death of one friend, when the other would sit with the deceased's body for the duration of the wake, traditionally 4 days before burial would take place (\citealt{Whitman1936}, \citealt{Davidson1997}).}  Kin terms also take the same address morpheme when speaking directly \textbf{to} that person. The identical substitution of /o/ for  final vowel affects line-final words in songs as well (\citealt{Davidson1997}).  There is no vowel variation by gender for this morpheme.         
  
\subsubsection{Names}  A proper name uniquely identifies someone, for both address and referential purposes. It also may encode key identity features (gender, clan membership, personal attributes/characteristics, or significant events relating to that person).\footnote{The Reverend James Owen Dorsey collected names, their meanings, and clan identification during his brief fieldwork in the late 19th century. The Smithsonian Institution has his field notes, truly a rich resource for individuals interested in discovering more about names, now available in their digital archive.}  Both dogs and horses were named also (Cf. \citealt{Whitman1936} for traditional Otoe-Missouria dog names).  

\textbf{Gender} Some names were identical for both genders within the same clan, but often a woman's form differed by the addition of \textit{-m\k{i}} `feminine' suffix.  A nickname could be coined to tease someone, as when one elder told another they should call me \textit{Tosk\'e-m\k{i}} `Quick/Speedy-Woman,' because I had done something so quickly that it surprised them. While names for men were not specially marked, there was a masculine morpheme \textit{-do} that occurs in words for male noun referents such as `boy,' `buck,' and `bull'; see \tabref{genderaffixes}.  

\begin{table}
\caption{Gender affixes: \textit{-do} `\textsc{masc}';  \textit{-m\k{i}} `\textsc{fem}' }	 \label{genderaffixes}				
\begin{tabular}[h!]{ l l }	 
\lsptoprule                                 
a)  \textit{ i\v{c}\textsuperscript{h}i\textbf{d\'o}\k{i}\~ne} `boy-child' & < i- `at/around'+ \v{c}\textsuperscript{h}i `house'+ -do `\textsc{masc}'+   \\
& \hspace{2em} -\k{i}\~ne `small/\textsc{dim}' [Ioway] \\
b)  \textit{i\v{c}\textsuperscript{h}i\textbf{m\k{i}}\textipa{N}(e)} `girl-child'	 & < i- `at/around'+ \v{c}\textsuperscript{h}i `house' + -m\k{i} \textsc{fem} + \\
& \hspace{2em} -\k{i}\textipa{N}e `small/\textsc{dim}' [Otoe] \\
c) \textit{ t\textsuperscript{h}a\textbf{do}} `buck, male deer'	& < t\textsuperscript{h}a `deer'\footnote{With white-tailed deer, a buck is clearly the ``marked form'' if the visible feature of antlers was the primary basis for assigning group membership.}  + -do `\textsc{masc}' \\		                      
d) \textit{\v{c}\textsuperscript{h}\'e\textbf{do}} `bull buffalo' & < \v{c}\textsuperscript{h}\'e `buffalo, bison' + -do `\textsc{masc}' \\
\lspbottomrule
\end{tabular}
\end{table}

\textbf{Diminutive suffix}  \textit{-\k{i}\textipa{N}e, -\v{s}\k{\'i}\textipa{N}e} [O-M];  \textit{\k{i}\~ne,  \v{s}\k{\'i}\~ne} `small /\textsc{dim} [Ioway]'.  There are also cases in Ioway tales where the protagonist's name is created from a verb + diminutive suffix: [V + \textsc{dim} > Name].
 
\begin{exe}
\ex
\begin{xlist}
\ex \gll B\'e-\~ne-\k{\'i}\textipa{N}e \\
 throw.out-\textsc{indef.pl}-\textsc{dim} \\
\trans `The Outcast' < `Little One(They)Threw Away' [Marsh `The Outcast' Ln. 141]

\ex \gll H\k{i}n\k{ú}-\v{s}\k{\'i}\textipa{N}e  \v{c}\'ila \\						     	 	
my.first.son-\textsc{dim}   dear \\				 	       	
\trans `My dear Little-Son'   [Marsh `The Wanderer' Ln. 200]
\end{xlist}
\end{exe}

\subsubsection{Number}  Nouns do not inflect for plural or case; numerals may follow the noun to give an exact number, or verbal suffixes reveal plural information instead.  Numbers may act as stative verbs, with  patient inflection, as also happens in other Siouan languages such as Quapaw \citep[481]{Rankin2008} and Lakota \citep[708]{Ullrich2008}.  

\vspace{1em}
\textbf{Numerals}  One through ten are the basics from which other numbers are expressed. Eleven through nineteen are formed using the formula `X over ten' {lit. `ten-over-one'}: \textit{grebr\k{a} agri (i)y\k{a}k\textsuperscript{h}i}, `ten over two', etc. Multiples of ten become `two tens' (lit. `ten (be) two') \textit{grebr\k{a} núwe}, `two tens over one',  up to ninety-nine.  An interesting example of word coinage is the large quantity `one thousand'; it is expressed by the word \textit{k\'oge} `box or trunk', because shipments of money (presumably annuity payments from Washington, D.C.) arrived in packing boxes, each of which held one thousand dollars.
											
\vspace{1em}
\textbf{Ordinal numbers}  Baxoje-Jiwere may use either a prefix \textit{i-} or a suffix \textit{-y\k{a}}.

\begin{exe}
\ex  \textit{i-} `ordinal marker':    [Marsh n.d. `The Giant' Book 2]
\begin{xlist}	
\ex 
\gll  walúxawe i-\textipa{T}át\textsuperscript{h}\k{a}  dahá\textipa{P}-e   \\
bundle      \textsc{ord}-five     it.is.standing-that.one \\
\trans `that fifth upright bundle' (LN 25) 
			
\ex 
\gll  walúxawe i-\v{s}ágwe dahá\textipa{P}-e  \\
bundle      \textsc{ord}-six     it.is.standing-that.one \\
\trans `that sixth upright bundle' (LN 30)

\ex 
\gll walúxawe  i-\v{s}áhm\k{a}  dahá\textipa{P}-e \\  
bundle     \textsc{ord}-seven    it.is.standing-that.one \\
\trans {}`that seventh upright bundle' (LN 34)
\end{xlist}
\ex (\textit{i-} `ordinal marker') + -\textit{y\k{a}} `indefinite article' [Marsh `The Wanderer']
\begin{xlist}
\ex \gll Dá\~n-\k{\'i}=y\k{a}     (ut\textsuperscript{h}\k{a}\textipa{P}\k{i}wagi  a\v{s}k\k{u}).   \\
	three-\textsc{ord}=\textsc{indef} {} {} \\
\trans `A third time (he makes them appear to him, it seems).'	 (LN 34)  
		 
\ex 
\glll Hetále    id\'oy\k{a} dahági   s\'ige   al\'e   g\k{ú}\textipa{P}wa\v{s}k\k{u}. \\
hetále i-dowe=y\k{a}  dahági s\'ige   al\'e   g\k{ú}\textipa{P}wa\v{s}k\k{u} \\		   
then.it.is \textsc{ord}-four=\textsc{indef} time.it.is  again it.is.this he.do.it.it.seems.' \\ 
\trans `And then, he does it again for the 4th time, it seems' (LN 35)
\end{xlist}
\end{exe}
\subsubsection{Compound nouns} 
Jiwere-Baxoje compound nouns,\footnote{In other Siouan languages, e.g. Lakota and Crow, there can be a greater degree of noun incorporation. See Ullrich (2008:738); \citet{DeReuse1994}; \citet{Gracyzk1991}.} shown in \tabref{compound} often have the modifying word precede the base noun, while other times the modifier(s) follow it. These words can also include names, i.e. \textit{m\k{a}k\textsuperscript{h}á  ru\v{j}\`e} `medicine eaters' denoting those who participate in the religious traditions surrounding the sacred sacrament peyote.  

\begin{table}
\caption{Compound nouns} \label{compound}
\begin{tabular}[t]{ l l }
\lsptoprule
a.  \textit{\v{c}\textsuperscript{h}\'ina} & `village'     < \v{c}\textsuperscript{h}i `house' + -na `horizontal?' \\
b. \textit{\v{c}\textsuperscript{h}\'ina wan\`axi} & `cemetery' < \v{c}\textsuperscript{h}ina `village' + wanaxi `spirit,  \\		  
& \hspace{2em}ghost'  \\         
c.  \textit{walú\v{s}ge  \v{c}\textsuperscript{h}\`ina} & `giant(s) village'  [Marsh n.d. `The Wanderer' 	\\        
& \hspace{2em}Ln 100] \\  
d.  \textit{h\k{\'i}d\k{ú}\textipa{N}e-n\k{\`a}w\k{u}} & `mouse + paths'   [Marsh n.d. `The  \\                    
& \hspace{2em}Wanderer' Ln 67] \\
e.  \textit{wanáxi wax\`o\~nit\textsuperscript{h}\k{a}} & `spirit/ghost + be holy/sacred'  [\citealt{Davidson1997}] \\
f.  \textit{m\k{á}y\k{a} uh\k{\`a}we} & `heaven'   < `land + full.of.light' [\citealt{Davidson1997};   \\
&  \hspace{2em} Good Tracks n.d.] \\
g. \textit{m\k{á}y\k{a} w\`atahe} & `Wanderer' < m\k{á}y\k{a} `land' +wa- `directional' +   \\
& \hspace{2em} dahe `be standing' \\
h. \textit{w\k{á}\textipa{N}eg\'ihi} & `Chief/Headman' <w\k{a}\textipa{N}e `man' + gi-`\textsc{ben/dat}' +  \\          
& \hspace{2em} -hi `\textsc{caus}' \\ 
i. \textit{w\k{a}\textipa{P}kwás'ose}  & `warrior/veteran/soldier'< w\k{a}\textipa{N}e `man' + was'ose \footnote{\citet{Whitman1947} noted glottal stop marking morpheme juncture. It seems especially prevalent when the deleted sounds/syllable involves /\textipa{N}/.}  \\
& \hspace{2em} `brave' \\
j.  \textit{w\k{á}\textipa{P}\v{s}ige}   &  `person' < w\k{a}\textipa{N}e `man' +  \v{s}ige `again'+/or  -ge  `\textsc{nom}'  \\                                                                              
k. \textit{w\k{a}\textipa{P}\v{s}\'i k'u\v{c}'e} & `man-hunter' <  w\k{a}\textipa{P}\v{s}ige  `person' + k'u\v{c}'e  `to kill'   \\                                           
l.  \textit{t\textsuperscript{h}\`a wa\textipa{T}l\k{u}} & `roasted deer'  < deer + to roast      [Marsh n.d. `The \\         
& \hspace{2em}Wanderer' Ln. 175]	\\
m. \textit{ist\k{a} \v{c}\textsuperscript{h}i}  & `(menstrual) period'  [literally `be alone-house'] \\
\lspbottomrule \end{tabular}
\end{table}


\subsubsection{Culture contact and word coinage}
					   		     	     
There was strong resistance to borrowing from European languages throughout Plains tribes in general,\footnote{Cf. \citet{Brown1999}, also \citet{LarsonND}} so it is not surprising that Jiwere-Baxoje speakers also chose to coin new words, or extend the meaning of existing words. For instance, the Ioways chose the part of a bird that powers its motion to name that revolutionary object, the wheel: \textit{ahu} `wing' > wheel (wagon/car).\footnote{Keith Basso described the Western Apache (Athabaskan) words for automobiles in similar ways, but in that case it was a hand/arm = front wheel and foot = rear wheel set extension (\citeyear[17]{Basso1990}).}   

\begin{exe}
\ex
\begin{xlist}
\ex wagon = \textit{nám\k{a}\~n\k{i}} < na `wood' + m\k{á}\~n\k{i} `moving/walking' 			       	         	
\ex train = \textit{nám\k{a}\~n\k{i} d\`ak'o}  <  nám\k{a}\~n\k{i} `wagon' + dák'o `thunder/fire'		       	          	
\ex photographs/pictures = \textit{\k{i}\v{j}e wagaxe}  < \k{i}\v{j}e `face' + wagáxe `writing'	        	         	
\ex Saturday = \textit{h\k{á}we uk\textsuperscript{h}\`i\textipa{T}re} `day-half' < h\k{á}we `day' + uk\textsuperscript{h}\'i\textipa{T}re `half, be split into two' 	[because the Tribal Agency was open from morning to noon on Saturdays]        	    	
\ex piano = \textit{nay\k{a}we}  `wood sings' < na `wood' + y\k{á}we  `to sing'
\end{xlist}
\end{exe}

The existing word for `metal' \textit{m\k{a}\textipa{D}e} originally referred to copper, available from the Great Lakes region in particular, and found throughout late Woodland through Mississippian periods in the Mississippi River valley and tributaries.  European silver and gold coins were called `white/light' or `shiny' metal, \textit{m\k{a}\textipa{D}\'e \textipa{T}ka}. The different types of coins led to this unique descriptor for `penny' < `coin (white/shiny-metal)+ red' \textit{m\k{a}\textipa{D}\'e \textipa{T}ka \v{s}\`u\v{j}e}.  This new unit formed a single compound noun, as shown by the phrase \textit{m\k{a}\textipa{D}\'e \textipa{T}ka \v{s}\`u\v{j}e iy\k{a}} `a penny, one penny.'

\subsubsection{Degrees of  noun incorporation}
\tabref{conjugating} demonstrates various ways that the words now functioning as compound verbs are conjugated.  The left-most column represents the least degree of noun incorporating into the verb, because the personal pronominal prefixes still attach directly to the verb: [Noun [Pronominal prefix + Verb]]. Or a speaker might prefer to add an auxiliary verb to carry person/number inflections, rather than inflect the main verb; see center column. Finally, a fully fused/incorporated noun-verb lexeme accepts the pronominal prefixes attaching directly to the left-most edge of the word, as represented in the far right-hand column [Pronoun + [Noun + Verb]].  The table shows some variation, and speaker preference seems to have been involved. Forms with \textit{ho} `voice' (11-13) appear to be more fully fused than other nouns were.    

\begin{table}
\begin{footnotesize}
\begin{tabular} { l l l l }
\lsptoprule
Jiwere gloss & [N+[\textsc{pron}-V]]  & [N+V] \textsc{pron}-\textsc{aux}	& \textsc{pron}-[N+V] \\
\midrule
1)  h\'o\textipa{T}ige `to fish' & ho-\textbf{he}-\textipa{T}ige & & \\
{[`fish + split']}  & `\textbf{I} am fishing'	& - & 	- \\
     
2) n\k{a}s\v{j}e p\textsuperscript{h}isk\k{u}\~n\k{i} & n\k{a}s\v{j}e-\textbf{h\k{i}}-p\textsuperscript{h}isk\k{u}\~n\k{i} & - & - \\
`to be unkind' & `\textbf{I} am unkind'	& & \\
{[`heart be good-not']} & & & \\   
 
3) n\k{a}s\v{j}e p\textsuperscript{h}i & n\k{a}s\v{j}e \textbf{ri}-p\textsuperscript{h}i  & &\\   
`to be kind' & `\textbf{you} are kind' & - & - \\
{[`heart be-good']} & 	& & \\	
 
4) n\k{a}t'\k{ú}d\k{a}  & n\k{a}t'\k{u}-\textbf{he}-d\k{a} & & \\
`to pity' & `\textbf{I} pity him'	& -  &  - \\
 
5) irodaxra & 	iro-\textbf{h\k{i}}-daxra & irodaxra \textbf{h\k{i}}\~n\k{i}\textbf{wi} & - \\
 `to have a fever' & `\textbf{I} have a fever' & `\textbf{we} (\textsc{pl}) have a fever'  & \\
{[`body-burn/be hot']}  & iro-\textbf{ri}-daxra & [a\~n\k{i} `have'] & \\
&  `\textbf{you} have a fever'	    & & \\
 
6) iroru\textipa{T}'a  & \textbf{wawa}-roru\textipa{T}'a\textbf{wi} & roru\textipa{T}'a \textbf{h\k{i}}\~ni-\textbf{wi} & - \\                               
`to be shaken up, &  `\textbf{we}'re shook up pl.' & `\textbf{we}'re shook up' & \\
excited' & (first response) & (second response) & \\
{[`body be-pushed?']} &  roru\textipa{T}'ani & & \\
& `\textbf{I} am shook up' & & \\
 
7) iro\textipa{T}et\textsuperscript{h}\k{a} & - & - & i\textbf{ri}ro\textipa{T}et\textsuperscript{h}\k{a} \\
`to abuse' & & & `\textbf{you} were abused'  \\
{[`body + ?']} & & & (\textsc{1Psg} \& \textsc{pl} also) \\
 
8) irok\textsuperscript{h}up\textsuperscript{h}i & - & irok\textsuperscript{h}up\textsuperscript{h}i \textbf{h\k{i}}\~niwi & i-\textbf{ri}-ro\textbf{s}k\textsuperscript{h}up\textsuperscript{h}i \\
 `to be handsome' & & `\textbf{we}  look good' & `\textbf{you} are handsome' \\
 {[`body +?look+good']} & & [ <a\~ni 'to have'] & (1P\textsc{sg}  also) \\  
 
9)  rosje & & rosje-\textbf{ri}-\~ne & \textbf{wawa}-rosje\textbf{wi} \\
`to sweat' [<`body+?'] & - & `they made \textbf{you} sweat' & `\textbf{we}'re sweating' \\
& & [\textsc{caus}] 	& (1P\textsc{sg} also) \\
 
10) d\k{a}we & - & - & \textbf{ha}-d\k{a}we \\
`to awaken, open eyes'  & & &  `\textbf{I} awakened' \\
{[<isd\k{a} `eye(s) + move']} & & & \\
 
11) hohga  `to belch' & - & - & \textbf{ra}-hohga \\
{[<ho `voice' + sound} & & & `\textbf{you} belched' \\
{symbolic  hga]}	& & & (1P\textsc{sg} \& \textsc{pl}  also) \\
 
12) hoxga `to hiccup' & - & - & \textbf{ha}-hoxga m\k{a}\~ni \\
{[<ho `voice' + sound} & & & `\textbf{I} am hiccupping' \\\
{symbolic  xga]} & & & \\	 
 
13) hoxu   `to cough'\footnote{Note 12 is lexicalized, as is its Lakota cognate, relative to Biloxi, which treated `cough' still as separable, inflecting after \textit{ho} `voice' \citep[186]{RankinEtAl2003}} & - & - & \textbf{ha}-hoxu \\
{[<ho `voice' + sound} & & & `I coughed.' \\
{symbolic  xu]}  & & & \\
\lspbottomrule
\end{tabular}
\caption{Conjugating different verbs with nouns attached} \label{conjugating}
\end{footnotesize}
\end{table}

There is an intriguing case from another Marsh text in which the noun seems strongly associated with a certain verb but it was in the third person with $\varnothing$ affix, so the conjugation pattern is unknown: \textit{t\textsuperscript{h}á \v{c}'\`ehi m\k{a}\~n\`a} `he went deer-hunting' [Marsh `The Wanderer' Ln.47]  

\subsubsection{Nominalizing prefixes} 
Certain prefixes commonly attach to verb stems to form a nominal.  To illustrate, the three prefixes in \REF{wawiwo} all incorporate the basic \textit{wa-} `indefinite object' (sometimes contracted with a locative prefix also) to action word(s).

\begin{exe}
\ex \label{wawiwo}
\begin{xlist}
\ex wa-     	       											
	       	
\textit{wagáxe}       `paper'  < wa `\textsc{indef.obj}' + gaxe `to scratch, write'
		
\textit{warúwaha}  `bundle'< wa  + ruwaha `to show with hands'
            
\ex wi-															

\textit{w\'i:\k{u}}    `tool'    < wa `\textsc{indef.obj}' + i- `at, to' + \textipa{P}\k{u} `to do, make, create'
            
\textit{w\'i:ro:ha}  `kettle' < wa + i-  + r\'oh\k{a} `plenty, lots, much, many'

\textit{w\'i:k\textsuperscript{h}\k{a}h\k{i}} `bridle'  < wa + i-  + k\textsuperscript{h}\k{a}h\k{i} `blood-vessel, sinew, cord'   [Marsh `The Outsider' Ln. 65]\footnotemark

\ex wo-  															

\textit{w\'o:\v{c}\textsuperscript{h}exi}   `difficult times, trials' < wa `\textsc{indef.obj}' + u-`in' + be.cruel/stingy'
	
\textit{w\'oy\k{a}we}  `festivity' < wa + u- + y\k{a}we `sing'? \footnotemark
\end{xlist}
\end{exe}
\footnotetext {Length is presumed here from the overall language pattern.  Marsh rarely marked vowel length in the narratives, except on interjections within dialogue, when they were greatly lengthened for emphasis.}
\footnotetext{This form and derivation is from Jimm Goodtracks. Marsh `The Outcast' Ln 160 gives \textit{w\'oyawe} with non-nasal /a/, perhaps from to \textit{yawe} `stab' (which might refer to the preparation of meat for feasting or the the piercing that took place during mourning a chief).}

\subsection{The verb and its many parts}\label{sec:greer:3.2}
	
\subsubsection{The verb template}  									                      
In Siouan languages, the most complex morphology involves the verb, which may include basic verb stem, plus up to ten ``slots'' or positions for a number of possible prefixes, as well as at least four positions for potential suffixes.  Figure \ref{fig:greer:template} (at end of chapter) is the representation of all fourteen potential affix positions and which prefix/suffixes can appear in each of those places. 

Described in more detail, beginning at the front or left-most position of an inflected verb, the prefixes may occur as follows (\citealt[246]{Whitman1947}, \citealt{MarshND}, also \citealt{HopkinsFurbeeND}). Negative numbers represent positions preceding the verb root; positive numbers follow the root.

\vspace{1em}
\textbf{Position [-10] 	1st person patient pronouns:} 	

\hspace{2em} h\k{i} =singular `me' 
												
\hspace{2em} wa\textsubscript{1a} = dual  `us two'	 (first half of separable morpheme)					           

\vspace{1em}
\textbf{Position [-9] 	The second \textit{wa-}  set:} 	
										
\hspace{2em} wa\textsubscript{2a}- `them, something;' indefinitely extended object (also detransitivizes 

\hspace{3em} the verb)		
	
\hspace{2em} wa\textsubscript{2b}- `toward, directional' 	[precedes all person prefixes except \textit{h\k{i}}- 1\textsc{sg}.

\hspace{3em} patient `me'] 	 				      

\vspace{1em}
Two examples of the first meaning, \textit{wa-}\textsubscript{2a}, give an idea of its flexibility as both derivational and inflectional morpheme:    	

\begin{exe}
\ex
\begin{xlist}
\ex \textit{\textbf{wa}naxi} `spirit, ghost' 

< wa\textsubscript{2a}- `indefinitely extended object' + naxi `breath, life'.                  	
\ex \gll Hinage   \textbf{wa}-t\textsuperscript{h}a  naha   waye:re  na?  \\           			
woman   \textbf{\textsc{pl.pat}}-\textsc{1sg.}see  those.ones who-are-they Q	\\
\trans `Who are the women that I saw?'
\end{xlist}
\end{exe}

Whitman considered directional \textit{wa}- to parallel both \textit{gra-} and \textit{gi-} of template positions -3 and -4 in some functions.\footnote{Cf. \citet{Boyle2009} for a discussion of the \textit{wa-} prefixes across the Siouan languages, quoting the late Carolyn Quintero on Osage \textit{wa-}, which was especially interesting. Based on these analyses, it may be more elegant to conclude that in Baxoje-Jiwere there is only one \textit{wa-} which does a wide variety of things to the verb, including the various functions within the different glosses given above. At present, it does not seem crucial to determine whether they are best described as two distinct morphemes \textit{wa-}, or as a single \textit{wa-} quite flexible in meaning. In the future, as more work on comparative Siouan \textit{wa-} emerges, perhaps the issue can be resolved.} The next case illustrates directional \textit{wa-} frequently found in prayer songs.	         		

\begin{exe}
\ex \gll H\k{i}y\k{i}no     \textbf{wa}-h\k{i}-na-wi. \\				      	        
OurElderBrother	  \textbf{\textsc{dir}}-1\textsc{pl.agt}-go-\textsc{def.pl} \\		           	  	        
\trans `We're going toward Our Elder Brother (Jesus).' (\citealt{Davidson1997})	
\end{exe}

\textbf{Position [-8] Locatives:}  	

\hspace{2em} a- `on, upon, over', 												

\hspace{2em} u- `in, within, into', 												

\hspace{2em} i- `at, to, by' \citep[241]{Whitman1947} 			

\vspace{1em}		 	     	     
 The locatives combine with the prefix \textit{wa}\textsubscript{2a}- (indefinitely extended object) to make a ``heavy'' syllable with a longer vowel, which usually attracts stress (Cf. nominal prefixes.) 	      							

\vspace{1em}
\hspace{2em} wa:  < wa\textsubscript{2a}- + a-  `on'  	 							     	      		

\hspace{2em} wo:  < wa\textsubscript{2a}- + u-  `in'   									     		

\hspace{2em} wi:   < wa\textsubscript{2a}- + i-   `at, to, by'

\vspace{1em}
\textbf{Position [-7]  Object/patient pronouns:}

\hspace{2em} wa\textsubscript{1b}-    `us (\textsc{1pl.pat}; speaker \& another, usually listener)' 						      	   	   

\hspace{2em} ri\textsubscript{1}-       `thee (\textsc{2sg.pat})'								   	  	   

\hspace{2em} m\k{i}-      `me (\textsc{1sg.pat})'                         

\vspace{1em}
\textbf{Position [-6]  Agent pronouns (first and second person):} 	

\hspace{2em} ha-, he-  `I/\textsc{1sg.agt}'										

\hspace{2em} ra\textsubscript{1}-, re-   `thou/\textsc{2sg.agt}'				          	    			

\hspace{2em} a-, e- `\textsc{3pl.agt}'  with motion verbs only\footnote{Whitman did not list the \textit{e-/a-} prefixes within the ordering of preverbal elements, probably because they are limited to motion verbs. However, since motion verbs do occur frequently, it seems preferable to include them as possibly archaic forms. The two also occur in 3rd p. possessive pronouns \textit{et\textsuperscript{h}awe} `his (singular)', \textit{et\textsuperscript{h}ewi} `theirs (definite pl.)', and \textit{ar\'e} `it is' (independent pronoun that primarily serves as demonstrative now, loosely `that').} 	        	        

\vspace{1em}
\textbf{Position [-5]  Reflexive \textit{k\textsuperscript{h}i-}  `(to) one's self'}		
				     		     	       
This prefix relates the event/state described by the verb back to the agent, usually translated as `one's self.' If \textit{k\textsuperscript{h}i} reduplicates, giving \textit{k\textsuperscript{h}ik\textsuperscript{h}i}, it adds the sense of reciprocal action `to/with each other'.  

\vspace{1em}
\textbf{Position [-4] Possessive  \textit{gra-}  `one's own'}	
						      
The possessive prefix gives additional information about social relations between persons and things mentioned in the verb complex.				
\begin{exe}
\ex	Excerpt from the Otoe-Missouria Flag Song:  	
						           
\textit{E-gra-\~na-gri-\~ne}.

\gll e-\textbf{gra}-a\~ni+a--gri-\~ne \\		     
3\textsc{obj}[ablaut]-\textsc{poss}-have+3\textsc{pl}-come.back.(home)-\textsc{pl.indef} \\
\trans `They brought it (the flag) back home.' (\citealt{Greer2008})	 
\end{exe}

\textbf{Position [-3]  Benefactive/dative \textit{gi\textsubscript{1}-} `for, to'}    					         

\vspace{1em}
\textbf{Position [-2]  Instrumentals (describing how an action was completed):}					

\hspace{2em} ba- `by cutting' 
			
\hspace{2em} bo- `with a blow'      	

\hspace{2em} da- `by heat or cold' 	

\hspace{2em} gi\textsubscript{2}- `with object away from the body, by pushing or striking with an 

\hspace{3em} object' 			

\hspace{2em} n\k{a}- `with foot/feet'		

\hspace{2em} ra\textsubscript{2}- `by mouth, teeth'				

\hspace{2em} ri\textsubscript{2}-  `with held object, toward the body, pulling with an object/tool'    		

\hspace{2em} ru- `with hand, toward oneself, by pulling with the hand' 			

\hspace{2em} wa\textsubscript{3}- `with hand away, by pushing with the hand'
\vspace{1em}

According to Whitman (1947:246), these nine prefixes transform a passive verb into an active one, or a stative verb into a transitive one \citep[483]{Rankin2005}. They make very specific distinctions in the world of human activity. `Long horizontal object being cut in two' -\textit{gruje} is an interesting yet abstract verbal root; someone or something must do the cutting, and the various ways that action is accomplished can be encoded very precisely (and concisely) with these prefixes, as in \textit{wa}\textsubscript{3}- `with hand away (from agent's body)' \textit{-gruje > wagruje} `to saw'.  Siouan scholars have sometimes distinguished between ``inner'' and ``outer'' instrumentals, with the latter a smaller set consisting of `by extreme temperature/heat', `by cutting with a knife,' and `by shooting/blowing' (\citealt[483-485]{Rankin2005}); however, I have not found data pertaining to that distinction in Jiwere-Baxoje thus far.   		
     
\vspace{1em}
\textbf{Position [-1]  2\textsuperscript{nd} person \textit{s-} }

Archaic form that stands for `you' (second person) on a small number of specific verb stems.  Siouan scholars have found related forms in the Mississippi Valley subgroup (e.g. Quapaw allomorphs \textit{\v{s}-/\v{z}-}), even extending into Proto-Siouan, suggesting it is of ancient origin (\citealt[479-480]{Rankin2005}).  Over time, it was probably replaced in less common verbs by the regular second person forms \textit{ra-, ri-}, but remained in very frequent verbs, which are more resistant to change.        

\begin{exe}
\ex 
\glll \textit{Arastawi k\textsuperscript{h}e}.\\
 a-ra-s-da-wi  k\textsuperscript{h}e  \\
on-you.\textsc{agt}.archaic.2-see-\textsc{pl.def} \textsc{masc.decl} \\
\trans `You (all) see it.'  (Final line, Otoe-Missouria Flag Song, \citealt{Greer2008})     
\end{exe}     
					 
\textbf{Position [0]  Verb root/stem}

\vspace{1em}
\textbf{Position [+1] Post-positioned person affixes + causative suffix  \textit{-hi} `to make something happen, to cause something'}

One way to form an active verb from a stative one is by adding the causative suffix \textit{-hi}; so \textit{\v{c}'e} `to die' becomes \textit{\v{c}'ehi} `to kill' (literally `to cause to die'). Since the causative \textit{-hi} occurs after the verb stem, personal pronoun affixes also come after the verb, but immediately before the \textit{-hi}, rather than their usual pre-verbal positions.  Sometimes the \textit{-hi} itself is omitted (as in the following example), but the pronominals' marked position after the verb, plus the meaning `to cause (something)' are still present.\footnote{One possible origin of this unusual case of pronominal prefixes shifting to the end is that \textit{hi} was once truly an independent verb, and over time, the forms were re-analyzed by speakers as single unified words. Then the initial verb of the compound was no longer conjugated. In that light, it is interesting to note that there is another \textit{hi}, the motion verb meaning `arrive here' (\citealt{Taylor1976}; \citealt{Hopkins1987}). That would parallel English idioms such as `to come to pass' for `happen, take place,' or `go and X' as in `Sam went and punched the man'.}  The word \textit{nay\k{i}hi}  `to heal, cure' literally means `to cause one to stand up, to stand X up.'  The chorus of a NAC song by Edward Small (Ioway) exemplifies an instance where  \textit{-hi} does not overtly appear.  Still, the translation and the location of the PRO prefixes after the verb stem \textit{nay\k{i}} `to stand' give evidence of the causative \textit{-hi} having an underlying presence.

\begin{exe}
\ex \gll H\k{i}y\k{\'i}no | Wak\textsuperscript{h}\k{a}da- y\k{i}\textipa{N}e   | m\k{a}ya  \v{c}egi  wahire nay\k{i}-\textbf{wa-ra} na \\
Our-elder-Brother | God- son)    | land   this   sick      stand-|\textbf{\textsc{3pl.pat-2.agt}} and|\\
\trans `Elder Brother, Son of God, you heal the sick in this land.'(\citealt{Davidson1997})
 \end{exe}

Likewise, it occurs in this sentence from missionary scholars \citet[43]{HamiltonIrvin1848}, \#53):
\begin{exe}
\ex \gll \v{C}'e-wa-{\ob}$\varnothing${\cb}-hi       k\textsuperscript{h}e. \\
   	kill-3.\textsc{pl/indef-ext-obj}-3-\textsc{caus} \textsc{decl}(male.speaker)  \\
\trans `He killed them'.	
\end{exe}

\textbf{Position [+2]  Negation  \textit{-sk\k{u}\~n\k{i}}  `not' }	
\vspace{1em}
	   	      						     
\textbf{Position [+3]  General Plural suffix  \textit{-\~ne}  `they/them'} 

Usually limited to third persons, \citet{Whitman1947} called it an indefinite form; perhaps the term `general plural' is more appropriate.  

\vspace{1em}
\textbf{Position [+4] Definite Plural \textit{-wi}  `\textsc{def.pl}'}

 Usually `we' or `you-all', it may occur with any grammatical person  
 \begin{exe} 
 \ex \gll wa-wa...-wi, { }  h\k{i}-...-wi, { }     ra-...-wi, { } ri-...-wi, { } $\varnothing$...-wi).  \\
            (1\textsc{pl}-\textsc{pat}...-\textsc{pl}, { } 12\textsc{agt}...-\textsc{pl}, { }    2\textsc{agt}...-\textsc{pl}, { } 2\textsc{pat}...-\textsc{pl}, { } 3...\textsc{pl})  \\
\end{exe}            

Both suffixes can pluralize any personal pronoun, no matter if that pronoun is in the role of an actor, patient or object (direct or indirect).  They only index number, and definiteness vs. indefiniteness.  Specifically, it says there are more than one for second and third person forms, and three or more for the first person dual form. The two potential plurals above differ by whether the people or things being referenced represent given or new information.\footnote{Think of the parallel indefinite article being used in the formula which begins many English fairy tales, `Once upon \textbf{a} time, there was \textbf{a} princess...'}   

Thus, they are not interchangeable.  They reference the speaker's knowledge about the group, how specific group membership is, whether persons' identities are known, if they have already been mentioned in a story before this point or not, and so on.  It makes sense for the definite plural to appear with the first person plural for pragmatic reasons.  It is difficult to imagine a situation in which `we' might mean a group with unknown or uncertain membership.  Second person plurals also usually take the definite plural, for the same reason, although some rare exception might occur.  However, it is very possible to imagine situations involving third persons to be either definite (`the gourd dancers from Red Rock, Oklahoma') or indefinite in nature (`everyone on Earth who knew my uncle').  Just as one might expect, zero third person-inflected verbs occur with either plural suffix, depending on the meaning intended. 

\begin{exe}
\ex \glll  w\'owak'\k{u}\~nawi  \\
wa-\textbf{$\varnothing$}-u-wa-k'\k{u}-\textbf{\~na}-\textbf{wi}    \\
\textsc{12pat-3pat-loc-12pat-}gave-\textbf{\textsc{indef.pl}}(`they')-\textbf{\textsc{def.pl}}(`us')  \\     

[vowel ablaut to \textit{\~na} from \textsc{indef.pl} \textit{-\~ne} when before \textit{-wi}]
\trans `They gave it to us.'  \citep[240]{Whitman1947}
\end{exe} 

\textbf{Position [+5]} \textbf{Mood/Aspect \textit{-h\~ne, -hna} `will, shall'}

The modal suffix seems similar to a future tense, but probably is more accurately expressed as `an action that is not yet completed' according to \citet{RankinND}.\footnote{While \citet{RankinND} included auxiliary verbs, adverbial intensifiers, positionals, and more within his comparative Siouan post-verbal template, this analysis will not follow his template for those morphological elements at this time.}   The \textit{e-} itself ablauts to \textit{a-} with verbs of motion.\footnote{Comparatively speaking, there is not yet an elegant historical explanation of ablaut across the various members of the Siouan language family (\citealt[466-468]{Rankin2005}).}  

\vspace{1em}
\textbf{Position [+6]  Evidential and gender indexical particle.}

It is not clear that these enclitics are actually part of the verbal complex, rather than serving as an audible coda indexing the gender identity of the speaker of an utterance and the degree of certainty of the speaker for the information given. The enclitics are not tied absolutely to the speaker's gender, but may also reflect the gender of a character during dialogue in a narrative, or original speaker's gender in reported speech/quotatives).  They do not seem to function as truly ``free'' morphemes, as they carry only secondary stress, and there is basically no pause between the verbal complex and the sentence-final particle, which tend to form a single prosodic contour.  Because it is such a rich and complex set, with meanings that are not easy to gloss, these particles are listed in \tabref{sentencefinalparticles}, rather than being included in the verbal template per se.   

\subsection{Auxiliary verbs}
Auxiliaries may appear alone, inflected with the full variety of verbal prefixes. When they are not the main verb, they will follow it (and any verbal suffixes attached to it).  In third person and inanimate subjects, the auxiliary verbs may not be inflected, but otherwise they would still bear first and second person prefixes, which strongly tend to be animate for practical contextual reasons.  The same pattern is found in most SOV languages \citep[490]{Rankin2005}.  

\vspace{1em}
\textbf{Positionals/modals.}  After the main verb, there is often a second verb expressing the action or position of the agent, or a distinct clause describing the activity/position of the speaker.  If one witnessed an event, a proper Baxoje-Jiwere description would include whether someone was sitting, lying, standing, or moving around while it occurred.\footnote{\citealt{Davidson1997} outlined the key role these auxiliary verbs played in creating vivid images in Native American Church songs composed by Otoe-Missouria and Ioway speakers.}  Beyond clarifying the bodily orientation of the person or thing being described, there are also various aspectual meanings that may be conveyed.  One such aspect is continuative as if the action takes place over an extended time frame, rather than occurring at a single moment or limited duration.\footnote{In addition to the continuative aspect, \citet[484--485]{Rankin2005} also distinguished a habitual (`always') aspect (Quapaw \textit{n\k{a}}), an imperfective `used to X' derived from Proto-Siouan /*\textipa{P}\~o/ `do', a potential `will/would X' (Quapaw \textit{tte}). Negation, imperative, and narrative forms were grouped with the auxiliary aspects, too. More complex moods could be created with combinations of these forms, such as potential + continuative, or negative potential continuative `to not go on X-ing'. However, I have grouped the imperative and narrative particles with the general sentence-final enclitics, in \tabref{sentencefinalparticles}.} They include:

\hspace{2em} \textit{m\k{á}\~n\k{i}} `going around, moving (in the characteristic way for that creature)'
	
\hspace{2em} \textit{m\k{\'i}na}  `sitting /dwelling'

\hspace{2em} \textit{n\k{á}\textipa{N}e} `be in a sitting position'

\hspace{2em} \textit{h\k{á}\textipa{N}e}  `be in a lying or reclining position'

\hspace{2em} \textit{dáhe}  `be in a standing /upright position'

\hspace{2em} \textit{n\k{á}y\k{i}}   `to stand something/someone up'

\subsection{Pronominals}
Baxoje-Jiwere has overt prefixes for first and second persons, while third person is represented by $\varnothing$ morpheme.  There are also three numbers expressed:  one (singular), we two (dual inclusive), or more than two (distinguished by the plural suffixes discussed in \sectref{sec:greer:3.2}).  Each person's role is identified relative to the action of the verb, as agent/actor or patient/object. There is potential confusion caused by homophony between one allomorph of first person singular patient \textit{h\k{i}-} `me' and the first person dual agent \textit{h\k{i}-} `we two'.  The other allomorph for 1\textsc{pat.sg}, \textit{m\k{i}-}, mirrors the form in the independent first person pronoun, as well as the independent possessive first person  pronoun.  The first person plural can only be expressed by addition of the definite plural suffix \textit{-wi} (see (+4) above), denoting the speaker, hearer, and one or more additional people as either agents \textit{h\k{i}-...wi}, or patients \textit{wa-wa-...wi}.  

`You' is composed of second person singular agent \textit{ra\textsubscript{2}-} and patient \textit{ri\textsubscript{2}-}, and also second person plural agent and patient forms.  See \tabref{personalpronominals} for further illustration. 

\begin{table}
\begin{tabular}{ l l l l l l }
\lsptoprule
& 1\textsc{sg} &  1\textsc{dual} & 1\textsc{pl}  & 2\textsc{sg} &  2{pl}  \\
& `I/me' & `we/us two' & `we/us all' & `thou/thee' & `you' \\
\midrule
Agent & ha-  	& h\k{i}- & h\k{i}-[+ -wi] & ra-  & ra-[+ -wi] \\
& (he-) & & &(re-) & (re-) [+ -wi] \\

Patient & m\k{i}- & wa\textsubscript{1a}-	& wa\textsubscript{1a}-[+ -wi]	&  ri- & ri-[+ -wi] \\
& h\k{i}-	& wa\textsubscript{1b}- & wa\textsubscript{1b}-[+ -wi]	& & \\
\lspbottomrule \end{tabular}

\caption{Personal pronominal prefixes} \label{personalpronominals}
\end{table}

 The parenthetical forms with final /e/ show the vowel change that takes place when the prefix is followed by certain derivational morphemes such as \textit{gra-} `one's own' (possessive), represented in the verb \textit{gra-hi} `to love, have pity on someone'.  The agentive forms \textit{ha-} `I', \textit{ra-} `thou' will become \textit{he-, re-} in other complex verbs such as \textit{n\k{a}t'ud\k{a}} `to pity (someone/something)'.\footnote{\citet{Whitman1947} has the plain [u] here while I heard it as a nasal [\k{u}], perhaps just spreading from the surrounding environment (\citealt{Davidson1997}).}  A potential origin for this word is \textit{n\k{a}hje} `heart' plus \textit{u-gi-d\k{a}} `be depressed toward' \citep[243]{Whitman1946}.  If that analysis is correct, the benefactive prefix \textit{gi-} `for' would be the conditioning morpheme for that particular case.  Another example is \textit{gi-t'\k{a}} `(it) flies', despite the fact that the \textit{gi-} prefix itself only is fully apparent in the plain $\varnothing$ third person form \citep[242]{Whitman1947}.    

Third person singular is typically marked by a zero morpheme, although an \textit{e-} prefix may rarely occur, especially with the possessor prefix `one's own', and with independent possessive third person \textit{et\textsuperscript{h}awe} `his/hers (singular)' or \textit{et\textsuperscript{h}ewi} `theirs'.  The demonstrative form \textit{-\textipa{P}e} combines with many prefixes, including third person \textit{e-}, resulting in \textit{e\textipa{P}e} `it is that one.'  Motion verbs provide an exception to that rule, with an \textit{a-} prefix in plural contexts.\footnote{Marsh n.d., \citet{Taylor1976}.}   Once again, we see an /a/-/e/ alternation.  

Independent pronouns, shown in \tabref{pronouns}, appear for emphasis or clarity, but are not required grammatically to complete a sentence, provided that the verb is properly inflected.  

\begin{table}
\begin{tabular}{ l l l} 
\lsptoprule
Person  & Independent  &  Possessive    \\
\midrule
1 Singular & m\k{\'i}re & m\k{i}t\textsuperscript{h}áwe \\
1 Dual Inclusive & h\k{\'i}re & h\k{i}t\textsuperscript{h}áwe \\
1 Plural Inclusive & - & h\k{i}t\textsuperscript{h}\'ewi \\
2 Singular & r\'ire & rit\textsuperscript{h}áwe \\
2 Plural	& - &  rit\textsuperscript{h}\'ewi \\
3 Singular & \'e\textipa{P}e  & et\textsuperscript{h}áwe \\
3 Plural	& ar\'e & et\textsuperscript{h}\'ewi \\
\lspbottomrule
\end{tabular}
\caption{Personal pronouns (\citealt{HamiltonIrvin1848}, \citealt{MarshND}).} \label{pronouns}
\end{table}

\subsection{Conjugating verbs}
\subsubsection{Regular verbs}  A verb stem is considered \textsc{regular} if it follows the verbal template of prefixes in its ordering, and the stem itself does not change in form, regardless of any shift in person or number.  Verbs are grouped according to whether they are \textsc{active} or \textsc{stative}, with the agentive pronominal prefixes inflecting the active verbs, and the patient pronominal prefixes forming the subject of stative verbs, as well as the objects of transitive verbs; see \tabref{regularverb}. 

\begin{table}
\begin{tabular}{l l l l }
\lsptoprule
Person	& Active verb & Stative verb & Transitive verb \\
\midrule
\textsc{1sg} 	& ha-m\k{a}\~n\k{i} & h\k{i}-y\k{a} , m\k{i}-y\k{a} & ha-k'e \\
& `I walk/move'	& `I sleep'	 & `I dig (it)/ I dug (it)' \\

\textsc{1du.incl} & h\k{i}-m\k{a}\~n\k{i} & wawa-y\k{a} & h\k{i}-k'e \\
 & `We 2 walk'	 & `We 2 sleep' & `We 2 dig (it)' \\

\textsc{1pl.def}  & h\k{i}-m\k{a}\~n\k{i}-wi & wawa-y\k{a}-wi & h\k{i}-k'e-wi \\
 & `We-all walk (>2)  & `We-all sleep' & `We-all dig (it)' \\

\textsc{2sg}	& ra-m\k{a}\~n\k{i} &  ri-y\k{a} & ra-k'e \\
& `You(sg) walk'	 & `You(sg) sleep' & `You (sg)dig (it)' \\

\textsc{2pl.def} & ra-m\k{a}\~n\k{i}-wi & ri-y\k{a}-wi & ra-k'e-wi \\
 & `You-all walk' & `You-all sleep' & `You-all dig (it)' \\

\textsc{3sg} & $\varnothing$-m\k{a}\~n\k{i} & $\varnothing$-y\k{a} & $\varnothing$-k'e \\
& `He/she/it walks' & `He/she/it sleeps'	 & `He/she/it digs (it)' \\

\textsc{3pl.def} & $\varnothing$-m\k{a}\~n\k{i}-wi & $\varnothing$-y\k{a}-wi & $\varnothing$-k'e-wi \\
 & `They walk'  & `They sleep'  & `They dig (it)'  \\
& (known) & (known) & (known) \\

\textsc{1sg.indef} & $\varnothing$-m\k{a}\~n\k{i}-\~ne & $\varnothing$-y\k{a}-\~ne & $\varnothing$-k'e-\~ne \\
 & `They walk'	& `They sleep' &`They dig (it)'  \\
& (unknown) & (unknown) & (unknown) \\
\lspbottomrule
\end{tabular}
\caption{Regular verb paradigm} \label{regularverb}
\end{table}

\subsubsection{Irregular verbs stems in D-, R-, W-.}  All irregular verb stems begin with  \textit{d-, r-}, or \textit{w-} sounds \citep[243]{Whitman1946}.  Note that the stem-initial consonant defines the class, and determines which conjugation will be irregular; however, there may also be prefixes attached to that stem. When any of those prefixes come before the personal pronoun, they do not influence each other (no amalgamation).  These irregular verbs share another anomaly; in second person agent forms, in addition to the expected \textit{ra-}, the archaic Siouan second person \textit{s-} also appears (Slot -1 on Verbal Template). Examples of irregular compound stems include:\footnote{Twentieth century elicitations seem to exhibit a tendency toward including the regular pronominal prefixes, in addition to the verb stem changes. However, Dorsey's slip file only has one speaker who doubles the inflection on these forms; this tendency to move toward the regular pattern may reflect the decline in everyday language use, leading to a preference for the most familiar inflections to be added onto the irregular verb stem changes \citep{DorseyND}.}	

\textbf{D- stems}  The stem's initial /d/ becomes /t/ to indicate \textsc{1agt}, instead of having the regular first person agent pronominal \textit{ha-}.  The stem change does not occur in any other person; even first person patient constructions take regular \textsc{1agt} \textit{h\k{i}}-. Second person agentive form is doubly inflected;  both \textsc{2agt} \textit{ra}- and archaic \textsc{1agt} \textit{s}- attach to the stem initial consonant; see \tabref{dstem}. 				

\begin{table}
\caption{D- stem} \label{dstem}
\begin{tabular}{ l l l }
\lsptoprule
a-dá `to see' & á\underline{t}a & `I see (it/him/her)'	\\	
& ará\underline{s}da 	& `You(\textsc{sg}) see (it...)'   \\	
& \underline{h\k{á}}da  & `We two (1\textsc{sg} \& 2\textsc{sg}) see (it,...)' \\
& \underline{h\k{á}}dawi & `We (\textsc{pl}) see (it,...)'\footnote{Stress shifted left to reflect a ``heavy'' syllable resulting from two vowels 
coalesced together, \textit{h\k{i}}- 1\textsc{pat} plus \textit{a-} `on' LOC.}  \\
&  adá & `he sees her/it'\\
& a\underline{r\'i}\underline{t}a & `I see you'	\\				       
& \k{á}ra\underline{\v{s}}da & `You(\textsc{sg}) -Archaic 2P see me' \\
&  \underline{wáwa}dáwi	& `(he) sees us (\textsc{pl})' \\
\lspbottomrule \end{tabular}
\end{table}


 \textbf{R- stems}  There are two irregular verb classes beginning with /r/. In the first paradigm, shown in \tabref{rstema}, the liquid /r/ is followed by back vowels /a, u/, giving \textit{ra}- or \textit{ru}- as the stem's first syllable.  First person agent is marked by /r/ becoming /d/.  The second person form inflects twice, with regular \textsc{2agt}  /ra-/ and archaic \textsc{2agt} /s, \v{s}/.			

\begin{table}
\caption{R-stem 1} \label{rstema}
\begin{tabular}{ l l l }
\lsptoprule
rumi `to buy' & ha\underline{d}umi & `I bought (it)' \\
& ra\underline{st}umi & `You (\textsc{sg}) bought (it)' \\
& hárumi	& `We two bought (it)' \\
& rumi	 & `He/she bought (it)' \\
\lspbottomrule
\end {tabular}
\end{table}

The second subclass of irregular verb stems, shown in \tabref{rstemb}, begin with /r/ paired with front vowels /i, e/. The /ri-, re-/ verb stems demonstrate a shift from /r/ to /j/ to mark \textsc{1agt}  forms,  while the archaic \textsc{2agt} /s, \v{s}/ morpheme inflects the unchanged stem alone.   

\begin{table}
\caption{R- stem 2} \label{rstemb}
\begin{tabular}{ l l l } 
\lsptoprule
r\'e `to go'  &	ha\underline{j}\'e  	& `I go' \\
& sre    &	`You go' \\
& h\k{i}re  & `We two go' \\
& h\k{i}rewi 	& `We go (pl.)' \\
& r\'e 	&	`He/she/it goes' \\
\lspbottomrule
\end {tabular}
\end{table}


\textbf{W- stems} These verbs have an initial voiceless bilabial glide /w/ which becomes a voiceless aspirated bilabial stop /p/ in the \textsc{1agt} form. The regular \textsc{2agt} \textit{ra-} may be present with some verbs, but is absent in others, while all W-stems take the archaic \textsc{1agt} /s, \v{s}/ inflection; see \tabref{wstem}.

\begin{table}
\caption{W- stem} \label{wstem}
\begin{tabular}{ l l l }
\lsptoprule
awá\textcrd o `to point at, point to' & á\underline{p}a\textcrd o	& `I point at (it)' \\
& a\underline{\v{s}}wá\textcrd o	 & `You point at (it)' \\
& háwa\textcrd owi	& `We (\textsc{pl}) point at (it)' \\
& awá\textcrd o	& `He/she/it points at (it)' \\
\lspbottomrule
\end{tabular}
\end{table} 

Additional verbs may conjugate regularly in all other persons, but preserve the archaic \textsc{2agt} /s, \v{s}-/ inflection.  These mixed verbs include common words: \textit{e} `to say, \textit{hij\'e} `reach a standing position', \textit{á\~n\k{i}} `to have', \textit{hiw\'e} `reach  a lying position', and \textit{dah\'e} `be standing' \citep[243]{Whitman1946}.

\subsubsection{Other special conjugation patterns:  motion verbs} 	               		                   
Like all Siouan languages, the Baxoje-Jiwere system of motion verbs has a rich set of distinctions. One intriguing dimension is the vertitive, which allows a concise and powerful way of expressing the notion of leaving home or predicting a safe homecoming.\footnote{While English lacks the motion verb equivalent to the vertitive, the compound noun `homecoming' is perhaps the closest in meaning and emotional power.}  Otoe-Missouria patriotic songs often have this powerful motion verb, poetically highlighting the fear involved when soldiers leave home, and joy when they return safely to their families.\footnote{Scholars of related Siouan languages such as Assiniboine have also analyzed these verbs in terms of how they appear in traditional narratives, where the notion of `belonging'/ home location also can be used to mean the place where a person or animal was located at the beginning of the story (by the river/point A), versus where they ended up later on (inside a cave/point B) (\citealt{Cumberland2005}).} Motion verbs are also distinguished by a 3\textsuperscript{rd} person plural prefix \textit{a-} which changes to \textit{e-} in the same conditioning environments in which 1\textsuperscript{st} and 2\textsuperscript{nd} person prefix vowels also shift from /a/ to /e/, namely before the benefactive prefix \textit{gi\textsubscript{1}-} and the possessive \textit{gra-}. (See \tabref{motionverb}).

\begin{table} 
\begin{tabular}{ l l l l l }
\lsptoprule
 Destination: & \multicolumn{2}{c}{Arriving Motion} & \multicolumn{2}{c}{Motion Prior to Arrival} \\
& non-vertitive / & vertitive & non-vertitive / & vertitive \\
\midrule
here . . . 	& j\'i  & gr\'i & hú &  gú \\
there . . . 	& h\'i &   -  & rá & grá \\
\lspbottomrule
\end{tabular}
\caption{Jiwere motion verb stems \citep[293]{Taylor1976}} \label{motionverb}
\end{table}
Note the initial consonant cluster echoes the possessive prefix [\textit{gra-} `one's own']; the shared phonological shape plus semantic congruity between vertitive and possesive is surely no coincidence.  

\subsection{Adverbials}
There are basic adverbial morphemes that may combine to express a wide range of meanings, with parallels to the personal pronouns (both independent and bound) in recognizing not only distinct first and second persons (`I' vs. `you'), but also `we two (you and I),' dual inclusive.  

\subsubsection{Spatial elements}   

Baxoje-Jiwere identifies five distinct places relating to the discourse context:\footnote{My M.A. thesis details the system of deixis in Baxoje-Jiwere (\citealt{Hopkins1988}).} 1) location of speaker `my spot here' \textit{\v{j}e-},  2) location of  listener `your spot' \textit{se-},\footnote{This form \textit{se-} with initial /s/ representing second person is very likely related to the archaic \textsc{2agt} /s/ found in conjugations of some conservative (irregular) verbs also \citep[480]{Rankin2005}.}  3) shared area of persons conversing together `our here' \textit{i-} (location of both you and me),  4) `there' \textit{ga-}, beyond the immediate discourse zone, e.g. a distant but visible location, and 5) `place beyond their sight (usually far away) \textit{hari-} (similar to archaic English \textit{yonder}).  These spatial elements combine with morphemes that distinguish between a fixed spot close at hand (\textit{-gi}), a stationary spot slightly further off (\textit{-da} `at there'), and motion toward a location (\textit{-gu} `to').  The directional sense of the prefix \textit{wa\textsubscript{2b}-} `motion toward' may follow first or second person forms to complete the variety of distinctions recognized. 
	
\subsubsection{Negatives} 

 Two basic forms can negate the main verb, \textit{sk\k{u}\~n\k{i}} `not' and \textit{\~n\k{i}\textipa{N}e} `(be/have) nothing'.  Thus, while the stative verb \textit{p\textsuperscript{h}i} `be good' expresses a positive attribute, the opposite meaning results from adding \textit{sk\k{u}\~n\k{i}} `not', giving \textit{p\textsuperscript{h}i-sk\k{u}\~n\k{i}}  literally `good-not'; `no good, bad, ornery'.  At the clausal level there can be additional ways to make it clear that something is false.  (Cf. the section on syntax, especially the evidential enclitics in sentence final position.)  

\subsubsection{Time elements}  While some Baxoje-Jiwere words for space do apply metaphorically to time, there are also specific temporal adverbs.  They tend to occur at the beginning of the sentence, as in this verse from NAC prayer-song composed by the late George Washington Dailey (Otoe-Missouria):

\ea \gll Go:\v{c}\textsuperscript{h}i   H\k{i}y\k{i}no    h\k{i}-ha-wi-yiyi \\
now     Our.Elder.Brother(male.spkr) \textsc{1pl.agt}-say-\textsc{pl.def}-chant	\\
\glt `Oh, My Lord, we're calling upon Your name, now.'	(\citealt{Davidson1997})
\z

\subsection{Other morphological processes}

\subsubsection{Sound symbolism}  

In Jiwere-Baxoje, there are two characteristics of such mimetic words that attempt to recreate certain sounds or material aspects of the world:   

\begin{description}
\item[a.] Often they use fricatives,  which sometimes form sets of related words which vary only in the fricatives' place of articulation.   
\item[b.] Many also are stative verbs, especially ones related to topics of color shade, intensity of hue, or other changes in sense perception, as in volume of noise, or roughness of texture. 
\end{description}

 This phenomenon is common in most Siouan languages, and can create interesting semantic sets differing by a single consonant sound (\citealt[468-469]{Rankin2005}). The ``lighter/less intense'' word is usually associated with a front and/or upper place of articulation, while the greatest intensity of meaning is found with the ``deepest'' back sounds.  It has been documented for Hooc\k{a}k and Dakota in particular.  Baxoje-Jiwere sound symbolic vocabulary sets include those in \tabref{soundsymbolism}. 
 
\begin{table}
\begin{tabular}{ l l }
\lsptoprule
\v{s}á-kh'e & `1) swishing sound made in water 2) sound made by hitting  \\
& or dragging of a chain' \\
thá-kh'e  & (probably th = \textipa{T}) `1) rattling of a rattlesnake; 2) rattling of corn \\
& in granary or in pile outside' \\ 
 kh\'o-kh'e & `ripping of calico, roar of falling water, sawing or scraping \\
& sound of tool on wood, whizzing of a whirled stick (`a bullroarer)  \\
& \citep[3]{Dorsey1892} \\
to-t\'o-khe & `repeated sharp sounds, such as the crackling or snapping of \\
& twigs and small branches, or frequent gunshots' \\
t\'op\v{e} & `pattering sound', n\k{a}t\'otop\v{e}  no gloss given (I posit 'the sound of \\
& dancing feet') \\
 \textipa{P}\'e-ghe, &  `the sounds of filing, grating, gnawing, or scratching on metal,\\
 &  \hspace{1em} bone, hard wood, etc.' (\citealt[4-6]{Dorsey1892})\footnotemark  \\
 kh'\'e-ghe, & `crow (bird, n.)'  [initial syllable imitating crow's call] \citep[8]{Dorsey1892} \\
kh'á-ghe & \\
ká-ghe & '' \\
\lspbottomrule
\end{tabular}
\caption{Sound symbolism} \label{soundsymbolism}
\end{table}
\footnotetext{Dorsey's orthography for consonants retained here.}

Note also the terms for upper body noises with variation in the medial fricative:  \textit{hohga}  `to belch' [\textit{ho} `voice'  plus sound symbolic \textit{hga}]; \textit{hoxga} `to hiccup' [\textit{ho} `voice' plus sound symbolic \textit{xga}]. 					

Although this is not an exhaustive list, let me add my personal favorite, \textit{h\'e \textipa{P}\v{s}i} `sneeze', which beautifully imitates the sound of sneezing, and takes an active/agentive conjugation.\footnote{Dorsey gave Dhegiha \textit{h\'e-tch\k{i}} `sneeze' (Kwapa \textit{h\v{e}-sh\k{i}}), and `snore' \textit{zh\k{a}-khdhú-de} (\citeyear[8]{Dorsey1892}).}  

\subsubsection{Reduplication}		
											
\textbf{Adult/standard reduplication}	Another kind of sound symbolism is reduplication, copying part (or all) of a particular word. If a stative verb such as a color is reduplicated, it means the color is scattered here and there (as in patches, spots, stripes), rather than in a continuous or ``solid'' distribution.  For an active verb, it gives an iterative meaning, whereby \textit{gis'\'e} `drip' becomes \textit{gis'\'es'e} `drip several drops'.  For less concrete activity, the reduplication can convey that the verb's action is somehow partial or incomplete.  For example, the form \textit{up\textsuperscript{h}a'p\textsuperscript{h}arehi} `understanding only bits and pieces, imperfectly comprehending' comes from \textit{up\textsuperscript{h}arehi} `to understand, notice, investigate'.\footnote{The latter example came from the late Rev. Arthur Lightfoot and Dr. Truman W. Dailey conversing about white missionaries' partial understanding of Indian beliefs (MCLP July 1992).}  In Jiwere-Baxoje, reduplication seems to have been a very productive process.	

\vspace{1em}														
\textbf{Reduplication in baby talk} In addition to adult reduplication, there is also ``baby talk'' or caretaker speech, a simplified version of ordinary phonological forms. Based on the limited sample available, it appears to have involved producing an exact copy of a monosyllabic morpheme, such as CV-CV.  If the word is polysyllabic, then everything after the first syllable would be deleted.  Some of the morphemes have been so simplified that it is not always clear from which word the simplified ``baby'' form originated. However, the primary difference between adult reduplication and ``baby talk'' is semantic.  The latter had no notion of something being repeated or scattered. Caretaker speech must have made it easier for little ones to learn to speak. Perhaps it originated as an adult imitation of the adorable way young children pronounce things themselves. Examples are in \tabref{babytalk}. 	Other items elicited include the repeated form + the normal diminutive suffix, \textit{-i\~ne} `little one':  \textit{mamá\textbf{\k{i}\~ne}} `baby' (Ioway), \textit{hahá\textbf{\k{i}\~ne}} `baby colt, horsey' (\citealt{Davidson1998}).

\begin{table}
\caption{Baby talk reduplication} \label{babytalk}
\begin{tabular}{ l l }
\lsptoprule
\textit{dáda} & `something to eat' \\				      					
\textit{\v{j}\'i\v{j}i}  & `hot (to touch)' \\							 		
\textit{n\k{á}n\k{a}}  & `something forbidden because of potential danger or pain' \\					
\textit{bobo} & `penis' abbreviated from \textit{buje} `acorn cap, penis' \\
\lspbottomrule
\end{tabular}
\end{table}

\section{Word order/syntax}
Baxoje-Jiwere is classified as an SOV language.  However, a verb (for third person forms, a ``plain'' (uninflected) verb) may function as a grammatical sentence,\footnote{There also needs to be a final particle that tells the gender of the speaker, as well as how certain the speaker is of the information being given, and the way the listener should respond (by listening and talking, by obeying what was said, by joining in with the speaker). These S-final particles are discussed in a later part of the grammar.} since the independent pronouns are optional, and there is a $\varnothing$ third person pronominal prefix corresponding to `he, she, it.'  	
			                      
\subsection{Noun phrases} 
	
\subsubsection{Adjectival forms.}  The head noun should come first in the noun phrase, followed by everything that describes it in any way, including stative verbs showing shape, color or size (`large, round, yellow'), which may also inflect as a main verb in other contexts, demonstrating they are not true adjectives.  

\subsubsection{Determiners, demonstratives, articles and more.}  Determiners identify which person or thing is being discussed, if it is a specific individual(s) or a generic one, how many there are, and so forth.  They include quantifiers, demonstratives, and at least one definite article and an indefinite article, which all follow their ``head''.  So `a white horse' when spoken in proper Baxoje-Jiwere order would be `horse white a' \textit{\v{s}\k{u}\~ne ska iy\k{a}} Ioway / \textit{su\textipa{N}e \textipa{T}ka iy\k{a}} Otoe-Missouria.  Quantifiers would begin with specific numerals, as well as other words relating to quantity of a group for countable objects and for animate beings (`few, many, all, most, ...') or for quantities of mass nouns such as flour, soup, water and so forth (`some, much, little, ...').   

\subsubsection{Article(s)}  

Indefinite article \textit{-y\k{a}, -iy\k{a}} `a, one' is derived from the word for `one' \textit{iy\k{á}k\textsuperscript{h}i}.\footnote{Lakota also utilizes the `one' morpheme as an indefinite article (\citealt[755-756]{Ullrich2008}).} Definite article is \textit{-ge}.\footnote{Until very recently I followed Marsh's analysis of Jiwere-Baxoje, which included no definite article. I would like to thank Johannes Helmbrecht (2015 p.c.), and Iren Hartmann (2008 p.c.), whose wonderful work on Hooc\k{a}k and excellent questions about possible cognates in Jiwere have forced me to reconsider the function of \textit{-ge}. I cannot explain how it was overlooked, except that its representation in the data collected was too infrequent to attract notice. More review of the existing data is needed to confirm the current interpretation.}   \textit{Gilbert-ge dani\textipa{N}e.}  `(That) Gilbert was drunk (again)!'

While earlier researchers did not identify a definite article for Jiwere-Baxoje, it seems likely that this is an oversight, due to the relatively small amount of data collected, and its lack of frequency compared to the English definite article. There certainly needs to be further examination in this area, considering its complexity in other Siouan languages (\citealt{Rankin1977,Rankin2005}; \citealt[455]{RoodTaylor1996}).  

\subsubsection{Interrogatives}

Those words that are used to ask questions about quantity or number fall into this category.  

\hspace{2em} \textit{tah\'ena} `how many, how much?'       (\textit{tana} in \citealt{HamiltonIrvin1848})

\hspace{2em} \textit{taheda}  `how far?'  

\hspace{2em} \textit{danáha, danáhaje}  `which?'\footnote{Cf. the similarity of sound shape in the cognate set found in Lakota (\citealt[455-457]{RoodTaylor1996}).}


\ea  \gll Bi-rawe   tahena   ra-gusta         		 ja? 	\\			          		
`moon-count  how.many    2\textsc{p.agt}-want(irreg.verb.2/s-/) Q.\textsc{fem}	\\					
\glt `How many calendars do you want?'  
\z

\subsubsection{Indefinite quantifiers}  

Such words give information as to scope, for instance which members of a collective group are included (or excluded) in the utterance. For example:  

\hspace{2em} \textit{dáhi, áhi}  'each, every'

\hspace{2em} \textit{br\'oge}  'all'
       								                   	 
\tabref{demonstrative} presents the demonstrative pronouns paired with the corresponding deictic directional prefixes.  Note the latter's strong parallels to and semantic association with discourse participants/persons in the context of the speech event.

\begin{table}
\resizebox{\textwidth}{!}{
\begin{tabular}{ l l l p{5.5cm} }
\lsptoprule
\multicolumn{2}{c}{Demonstrative Pronouns} & \multicolumn{2}{c}{Deictic Directional Prefixes (\citealt{Hopkins1988})} \\
\midrule
je\textipa{P}e  & `this one'  & je-  & 1 \textsc{loc} `near me',  `this here' \\

se\textipa{P}e & `this one & se- & 2 \textsc{loc}  `near you' [also \v{s}e- Ioway] \\
& [near you]'  & & \\

& & i- & inclusive 12 \textsc{loc} `here' \\

e\textipa{P}e & `it is he/ & e- & 3 \textsc{loc} `near her/him/it' \\
& that one' & & \\

are & `it is' & a- & *unattested; possible ablaut form of /e/ with \textit{re} `to go' \\

ga\textipa{P}e	 & `that one' & ga- & `there' \\
\lspbottomrule
\end{tabular}
}
\caption{Comparison of demonstrative pronouns to deictic directional prefixes} \label{demonstrative}
\end{table}

\textit{Ar\'e} `it is' ``points'' back at something previously mentioned, and appears with great frequency in the texts collected by Gordon Marsh \citep{HopkinsFurbeeND}  It can be paired with the emphatic bound morpheme \textit{-s\k{u}} `indeed' (\textit{ar\'e\textipa{P}s\k{u}} `indeed!' (emphatic)), and even `stacked' with the first person deictic prefix \textit{je-} `this (here)' to give \textit{járe} `this one-it is', and other additional complex compounds.    

\subsection{Subordinate clauses}
Main clauses normally occur sentence-finally, while subordinators(s) transform the first clause(s) into a supporting or modifying syntactic role, signaling duration, exact sequence of events, if events were actual or potential, etc. These subordinating particles include \textit{-sge} `if', \textit{-da} `when',  \textit{-sji}  `but, although', \textit{nu\textipa{P}a} `but'.   The temporal particle fills that function as follows:

\ea \gll  H\k{i}y\k{i}no| wo-waxo\~nit\k{a}   rit\textsuperscript{h}awe  urak\textsuperscript{h}i-\~ne   \textbf{da}| wa\textipa{P}\k{u}       warup\textsuperscript{h}i  Rire  [$\varnothing$] a-\~ne  (h)na \\ 
`Our.Elder.Brother| 	ceremony-sacred  your  they.tell.about-\textsc{indef.pl}   \textbf{when}| the.work wonderful(it.does)	  you  3- say-\textsc{indef.pl}         Imperfect.' \\
\trans `Elder Brother, when they tell about Your ceremony and the wonderful work it does, they say it's You.'
\z		

This complex sentence begins with a kin tem (addressed to Jesus), a subordinate clause indicated by subordinator \textit{-da} `when', then finally the main clause (\citealt{Davidson1997} Song \#16).\footnote{Edward Small (Ioway) composed this song after being healed during a NAC worship service.}  		

\subsubsection{Relative clauses.}  
The Baxoje-Jiwere language tends to place the head noun first within the relative clause.  An optional special marker \textit{-naha}\footnote{\citet{DorseyND} gave \textit{daha} as another potential relative clause marker, in an example sentence referring to an object rolling under a tent flap that was not fastened down: \textit{t\textsuperscript{h}\k{a} gri were \textbf{daha}, ru\textipa{T}ewi re} `\textbf{That which} has gone outside, get ye' (spelling and punctuation adapted to modern conventions). Further study on Jiwere-Baxoje demonstratives' potential relationship to positional verbs in a classificatory system is very much needed (Cf. \citealt[3]{Rankin2005}).}  `the one(s) that X' immediately follows the clause it acts upon, as in \textit{hinage at\textsuperscript{h}a naha} `the woman that I saw' (lit. `woman I saw (her) that one').   

\begin{exe}
\ex
\begin{xlist}
\ex Relative clause as the object of the sentence:	
							
\gll John hinage  at\textsuperscript{h}a            naha           uk\textsuperscript{h}i\v{c}'e      k\textsuperscript{h}e.	 \\				
John woman I.saw.(her) that.one (he)spoke.with.(her) \textsc{masc.decl}	 \\
\trans `John spoke with the woman that I saw.'

\ex Relative clause as the subject of the sentence:

\gll Hinage    at\textsuperscript{h}a    naha     John          uk\textsuperscript{h}i\v{c}'e     k\textsuperscript{h}e. \\
	Woman   I.saw that.one  John (she).spoke.with.(him) \textsc{masc.decl} \\
\trans `The woman that I saw spoke with John.'

\ex Relative clause as the direct object of the verb phrase:		
				
\gll Sam wawagaxe hapagaxe    naha           araje       khe.	\\				
Sam  book         I.wrote.it   that-one   (he).read.it   \textsc{masc.decl}	\\					
\trans `Sam read the book that I wrote.'
\end{xlist}
\end{exe}

Because the relative clause marker is optional, and the 3\textsuperscript{rd} person pronoun is zero, it can be difficult to translate some sentences, even though the general meaning is clear.      	

\subsection{Conjoined clauses}	
The conjunction \textit{heda} `and' may occur at the beginning of the second sentence.  Within more rapid speech sequences, it is common to instead have the particle \textit{-na} `and' occur at the end of the first main clause, separating it from the one to come.	

\subsection{Beyond statements: Other kinds of sentences} 

\subsubsection{Directives/requests/commands.}  These ways to ``boss'' others are linguistically interesting because many languages omit both first person and verb stating `I am telling you' to do something.  Sometimes second person form is also omitted. The ``pragmatic skewing'' occurs because overt first and second person forms may be considered too direct, and thus rude (\citealt{Heath1998}). This politeness pattern holds true with Baxoje-Jiwere directives. One speaks to children in a more direct manner than adults, since few question the authority of parents/elders to tell kids what to do.  If speaking to an adult, it would be more polite to use a different form, \textit{ne/n\textipa{E}}.  However, songs demonstrate expressing a plea with the stronger command particle, \textit{re}: 	     											

\ea \gll H\k{i}y\k{i}no    		 wa-a-wa-da-wi          re 	\\                                  	 
Our.Elder.Brother 1\textsc{pat}-look.at-\textsc{def.pl}  [command (male speaker)]	 \\	   	    
\trans `Elder Brother, look at us!'  (\citealt{Davidson1997})  	
\z				             
Finally, one may make a very polite request by using the dual/first person inclusive plural form with hortative enclitic \textit{t\textsuperscript{h}o}. `Let us all call on the Creator's name', or `Let's go to the handgame!'  		

\subsubsection{Questions} There are three ways to correctly form questions:  	
	          			
\begin{description}
\item[a.] Declarative sentence + sentence-final question particle. Word order does not vary; it is an evidential ending particle that signals an answer is expected, because the speaker is asking, not telling something.  As with many of these ending particles, the exact form varies by the speaker's gender: \textit{\v{j}e} `Q (male speaker)' / \textit{\v{j}a} `Q (female speaker)'. 

\item[b.]  By using interrogatives such as \textit{way\'e:re} `who (is it)?' or \textit{dagú:re} `what (is it)?' The interrogative word receives the special question-sentence melodic contour, which includes lengthening the stressed vowel greatly and making its pitch higher, plus pronouncing the final syllable's pitch lower than usual.  	

\item[c.]  Finally, one can create a question by simply omitting all S-final particles, and using the interrogative intonation pattern. See (25b) below.  In Ioway/Otoe-Missouria speech, the question pattern is made with a much longer (and slightly higher pitched) vowel in the penultimate syllable of the sentence, and a drop to a lower pitch in the last syllable.	
\end{description}
\begin{exe}
\ex
\begin{xlist}	       		          	     
\ex \gll Wabú\textipa{T}ga ra-gústa       \v{j}\`a?  \\						 	      		
bread       2\textsc{sg}-want(it) Q.female.speaker \\					     		
\trans `Do you want any bread?' 
\ex \gll Wabú\textipa{T}ga ra-gú:sta? \\					 	 	         		
bread        2\textsc{sg}-want(it) \\			     	     		
\trans `You want some bread?'	
\ex \gll Ra-gústa    dagúre?  \\							        		
2\textsc{sg}-want(it) what(is.it) \\				                    		 	
\trans `What (do) you want?' or `You want what?'
\end{xlist}
\end{exe}

\section{Variation in speech by social group}
\subsection{Tribal identity and language use} 	The Otoe-Missouria and Ioway people spoke mutually intelligible dialects of one language.  After a devastating enemy attack in the late 18th century, most surviving Missouria fled from their village in Missouri to the Otoe village in southeast Nebraska (\citealt{Schweitzer2001}).  Geographic separation between these two tribes ended about forty years before any language records exist.  Although recognition of a leader, ``Missouri Chief,'' is documented in Indian Territory ca. 1885,\footnote{Cf. the diary of Miss Emma DeKnight, who taught at the Otoe tribal boarding school at that time (DeKnight ms., University of Oklahoma Archives, Norman, OK).}  there is no data on unique Missouria dialect features.\footnote{J. O. Dorsey identified a tiny bit of data as specifically Missouri, but it related to only a single speaker, so I prefer to avoid any discussion of the Missouri dialect at present.}  At the phonological level, general tendencies have been noted (\tabref{dialect}). It is not as simple as always substituting one sound for another, yet listeners certainly noticed the distinctions.\footnote{There has been intermarriage for a long time, so 100\% dialect consistency for a speaker would be very unlikely, regardless of tribal membership. Dialects may be a matter of tendencies, rather than always/never. Family members might use different speech within a household, such as Mr. and Mrs. Small, Ioway and Otoe respectively. The couple understood each other but didn't speak exactly the same (Marsh n.d.).}	

\begin{table}
\resizebox{\textwidth}{!}{
\begin{tabular}{ l l l  }
\lsptoprule
I.	Phonological variation & Baxoje  & Jiwere \\
\midrule
A.  Difference in fricatives:	  & &\\ 
\hspace{1em}1.  Word initially   &  [\v{s}] & [s] \\
 & \textbf{\v{s}}ú\~ne & \textbf{s}ú\textipa{N}e \\
 & `horse' & `horse' \\       
\hspace{1em}2.  In consonant clusters   & & \\  
\hspace{2em}a) before [g]  & [sg/hg] & [\textipa{T}g] \\
 & wa\textbf{h}ge & wa\textbf{\textipa{T}}ge \\
 & `dish/plate'  & `dish/plate' \\
 & \textbf{h}ga	& \textbf{\textipa{T}}ga \\                                                               
 & `(be) white'  & `(be) white'  \\
\hspace{2em}b) before [\v{j}]  & [\textipa{P}\v{j}/h\v{j}] & [s\v{j}] \\
  & n\k{a}\textbf{\textipa{P}}\v{j}e, n\k{a}\textbf{h}\v{j}e & n\k{a}\textbf{s}\v{j}e	\\
 & `heart' & `heart'   \\
B.  Difference in nasal consonants: & & \\  	
\hspace{1em}1. Medial position, esp. before final -e & [\~n]  & [\textipa{N}] \\
&  \v{c}\textsuperscript{h}id\'oi\textbf{\~n}e	& \v{c}\textsuperscript{h}id\'oi\textbf{\textipa{N}}e \\
& `little boy' & `little boy' \\    
\midrule
II.	Select lexical differences   & & \\   
\midrule                                                                         
A.	 Nouns: &  mamái\~ne	& \v{s}úwe \\
& `little baby'  & `little baby'  \\ 
B.  Interjections:  	 &  sik'	 &  d\k{a}rah \\						                                 
& `incredible!'  &  tan-rah  [Marsh] \\
&  &`incredible!'   \\
\lspbottomrule
\end{tabular}
}
\caption{Dialect differences} \label{dialect}
\end{table}

\subsection{Gender marked speech} Three distinct lexical sets signal speaker's gender.  

\textbf{1.  Kinship}  The first set is kin terms as outlined in Goodtracks' dictionary. Gender is distinguished not only of referent (mother vs. father, etc.) but certain terms vary by sex of speaker as well, especially siblings' words for each other and words for one's in-laws.  Birth order establishes seniority and thereby determines respect relationships, and is reflected in words denoting sons, daughters, and siblings, which served as familial address terms.

\textbf{2. Sentence final particles} The second set of gender-indexical terms distinguish between declarative statements, requests,\footnote{Earlier scholars have often called the ``inclusive request'' form in \tabref{sentencefinalparticles} the \textsc{hortative.} marker, related to the rather old-fashioned word to ``exhort'' someone to do something.}  commands, dubitatives, quotatives, and more.\footnote{\citet{Trechter1993} presents a thorough analysis of gender enclitics, including the circumstances where a speaker's gender was not the determining factor, for various pragmatic and contextual reasons, including quoted speech.} These important enclitics audibly punctuate the sentence, informing the listener how to interpret the speech segment.   (See \tabref{sentencefinalparticles}. The list may not be exhaustive.) These enclitics occur in combination with each other, especially when expressing emphasis: \textit{k\textsuperscript{h}e h\k{u}\textipa{P}} `Indeed!'(`This I declare! male speaker').\footnote{Dorsey's manuscripts gave the male declaratives as distinguished by tribe, with \textit{kei} as the Otoe form and \textit{ke} as Ioway, while he listed \textit{k\textsuperscript{h}i} 'Ioway female declarative', but \textit{h\k{a}} for Otoe women's equivalent.} 

Note that a narrator will use the character's gender marker during dialogue, rather than indexing his or her own identity. Based on the songs I collected, while mixed gender singing does occur (females may join in during various worship and powwow songs), it is the men who traditionally begin the songs, and texts reflect that with male forms.  

\begin{table}
% \resizebox{\textwidth}{!}{
\begin{tabularx}{\textwidth}{ X l l }
\lsptoprule
S-final particle type & Male speaker & Female speaker \\
\midrule
Declarative 1 & k\textsuperscript{h}e & k\textsuperscript{h}i \\

Declarative 2 & k'a & h\k{a} \\
Completed Action  
'not continuing into   
present' (Dorsey)	 & & \\

Inference (2nd hand source)& no & na (?) \\ 
`I think' & & \\

Command & re & r\textipa{E}, r\ae \\
Polite Command	& ne & n\textipa{E} \\
Inclusive request & t\textsuperscript{h}o , dáh\`o, hda\textipa{P}o & t\textsuperscript{h}a \\
`Let us ...' / & & \\
`Would that'  & & \\

Question marker (optional) & \v{j}e	 & \v{j}a \\

Tag question & \textipa{P}a	& k\textipa{P}a \\

Narrative marker `It seems' 	& asg\k{u}	& asg\k{u} \\ 

Quotative & \textipa{P}e	& \textipa{P}\textipa{E} \\

Emphatic & h\k{u}\textipa{P} & \ae, \textipa{P}a, \textipa{P} \\

Surprise/excitement & t'o &	t'\k{u}: \\
'Exclamation point!'(Dorsey)  	 & & \\ 
\lspbottomrule
\end{tabularx}
% }
\caption{Sentence final particles showing mood, evidentiality and gender} \label{sentencefinalparticles}
\end{table}
	
\textbf{3.  Interjections}  The final morpheme set indexes a speaker's gender, and is usually sentence initial. (See \tabref{interjections}.) It is sometimes only a subtle difference, such as a final vowel shift, while other forms show little apparent derivational relationship between the two forms at all.  

\begin{table} 
\resizebox{\textwidth}{!}{
\begin{tabular}{ l l l }
\lsptoprule
Interjection gloss & Male speaker & 	Female speaker \\
\midrule
`Oh, my!'  & h\'e:h\k{a} & in\`a:,  hina: \\
(Pity, love, sympathy,  & & \\
compassion)	& & \\

`Say! Hey!' (Change subject) & k\`ar\'o	 &  unknown \\

Joy, Happiness & \'iy\`a & \'iy\`a\footnote{Not traditionally female but some use it now.} \\
(while singing or talking)	  & & \\

Greeting/Acknowledgement, & ah\'o, h\'o	& ahá, há \\
Thank you!   & & \\
Approval/ Sanction & & \\

`Hmph! Aw, Heck!' & d\textipa{E}\textipa{P} & h\textipa{E}\textipa{P} \\
(critical/doubtful; & (Both male and female & \\
prior speaker isn't telling & forms = short vowel [\textipa{E}] & \\
it right) & & \\

`Well! (GT) Whew!' & gw\'i,  kw\'i  & h\'i \\
(Almost!; something nearly & & \\
happened, but didn't, & & \\
either good or bad)  & & \\

Interjection gloss & Male speaker & 	Female speaker \\

`Well, well [Whitman]; & h\'e:h\k{a}	& hára\textipa{P}  \\
Oh, my!' & & \\
(negative response, as in & [also glossed as & \\
niece/ nephew teasing &  doubting truth] & \\
uncle/aunt too harshly; & & \\
surprised in a bad way) & & \\
 
`My goodness! Surely not! & bá\textipa{P},  huba\textipa{P},  hú\textipa{P} húba\textipa{P} & d\'o\textipa{P}, d\'o\textipa{P}\`o \\
No way!' (Negative & (L-R in order of  & (greater emphasis) \\
response; surprise, shock) & increasing emphasis) & ga: \citep{RankinND}  \\

`Yes' (Affirmative) & hú\v{j}\'e & hú\v{j}\`{\textipa{E}} \\

`No' (Negative) & hi\~n\'ego	& hi\~n\'ega \\
\lspbottomrule
\end{tabular}
}
\caption{Interjections showing mood and gender} \label{interjections}
\end{table}

\section*{Acknowledgments}
Various sources of funding have supported this work over the years, from the initial grants which made my graduate work possible as research assistant to Dr. N. Louanna Furbee [NSF "Documenting the Chiwere Language" BNS 88-18398 and 902-1337], APS "Chiwere Oral Traditions" [Jill D. Greer, P.I.], as well as linguistic consulting work with Jimm GoodTracks, P.I. for his dictionary project [NSF "Ioway-Otoe-Missouria Dictionary Project" BNS 0553585].

\section*{Abbreviations}
1, 2, 3 = first, second, third person; 12 = first+second person (first person dual); \textsc{agt} = agent; \textsc{aux} = auxiliary; \textsc{ben} = benefactive; \textsc{caus} = causative; \textsc{dat} = dative; \textsc{decl} = declarative; \textsc{def} = definite; \textsc{dim} = diminutive; \textsc{dir} = directional; \textsc{du} = dual; \textsc{emph} = emphatic; \textsc{ext} = extended; \textsc{fem} = feminine; \textsc{hort} = hortative; \textsc{imp} = imperative; \textsc{indef} = indefinite; \textsc{incl} = inclusive; \textsc{loc} = locative; \textsc{masc} = masculine; \textsc{nom} = nominalizer; \textsc{obj} = object; \textsc{ord} = ordinal; \textsc{pat} = patient; \textsc{pl} = plural; \textsc{poss} = possessive;  \textsc{pron} = pronoun; Q = question; \textsc{sg} = singular. 

\nocite{Davidson1999, Koontz1985,Robinson1972,SchwartzND}
\todo{Works in list of references but not cited in text: Davidson 1999,Dorsey1880,Schwartz}

\todo{there are several Hopkins\&Furbee no date. Please make sure which ones are cited in the text}
\printbibliography[heading=subbibliography,notkeyword=this] 

\todo{is Trechter 1993 or 1995?}


\begin{sidewaysfigure}
\caption{Verbal template: Prefix slots in order = verb = suffix slots}
\scriptsize
% \begin{tabular}{ | c | c | c | c | c | c | c | c | c | c | c | c | c | c | c | c |}
\resizebox{\textwidth}{!}{   
\begin{tabular}{ llllllllll l lllll}
\midrule
-10 & -9 & -8 & -7 & -6 & -5 & -4 & -3 & -2 & -1 & 0 & +1 & +2 & +3 & +4 & +5 \\
\midrule
PRO & INDEF & LOC & PAT & AGT & REF & POS & DAT & INS & Arch & VERB & CAU & NEG & PL1 & PL2 & ASP \\
& & & OBJ & Actor & & & BEN & & 2Pers & STEM & & & & & MOOD \\
\midrule 
\parbox[t]{1cm}{
% \begin{tabular}[t]{p{.7cm}}
 \textit{h\k{i}-} \\
\mbox{12AGT} \\ \\
\textit{wa}\textsubscript{1a}- 1PL.PAT \\
% \end{tabular}
}
&
\parbox[t]{1cm}{
% \begin{tabular}[t]{p{1cm}}
 \textit{wa}\textsubscript{2a}-  `them,  \mbox{something};\\
 indefinitely  extended  \\
object'\\ \\
 \textit{wa}\textsubscript{2b}-  `toward', \mbox{directional} 
% \end{tabular}
}
&
\parbox[t]{1.1cm}{
% \begin{tabular}[t]{p{.4cm}}
  \textit{a-} `on' \\ \\
 \textit{i-} `at' \\ \\
 \textit{u-} `in' \\ \\
% \end{tabular}
}
&
\parbox[t]{1cm}{
% \begin{tabular}[t]{p{1cm}}
  \textit{h\k{i}-} \\ 1SG \\ \\
 \textit{wa}\textsubscript{1b}- \\ \\
 \textit{ri\textsubscript{1}}-\\  2SG  
% \end{tabular}
}
&
\parbox[t]{1cm}{
% \begin{tabular}{p{1.6cm}}
  \mbox{\textit{ha-}, \textit{he-}}\\  1SG  \\ \\
 \mbox{\textit{ra-}, \textit{re-}}\\  2SG  \\ \\
  \textit{a-} \\ 3PL  motion verbs   
% \end{tabular}
}
&
\parbox[t]{.4cm}{
%  \begin{tabular}[t]{l} 
\textit{k\textsuperscript{h}i-}  
`self'
% \end{tabular} 
}
&
\parbox[t]{1cm}{
% \begin{tabular}{p{.6cm}}
  \textit{gra-}  `one's  own'
% \end{tabular}
}
&
\parbox[t]{1cm}{
% \begin{tabular}{p{.6cm}}
  \textit{gi-} \\ `for, to' 
% \end{tabular}
}
&
\parbox[t]{1.5cm}{
\begin{tabular}[t]{p{1.3cm}}
 \textit{ba-}\\ `by cutting' \\ \\
 \textit{bo-}\\ \mbox{`with a blow'} \\ \\
 \textit{da-}\\ `by heat  or cold'  \\ \\
  \textit{gi}\textsubscript{2}-\\ `with  obj. away  from self,   pushing   with something'\\ \\
 \textit{n\k{a}}\\ `by foot' \\ \\
 \textit{ra\textsubscript{2}-}\\ \mbox{`by mouth/teeth'} \\ \\
  \textit{ri\textsubscript{2}-}\\ `with obj.  moving  toward self, \mbox{pulling with}  something' \\  \\
  \textit{ru-}\\ `with  hand, toward  self, pulling'\\ \\
  \textit{wa\textsubscript{3}-}\\ `with  hand away, by pushing  with hand'\\  \\
\end{tabular}
}
&
\parbox[t]{1cm}{
% \begin{tabular}{p{.1cm}}
  \textit{s-} 
% \end{tabular}
}
& 
 \parbox[t]{.1cm}{~}
&
\parbox[t]{1cm}{
% \begin{tabular}{p{1cm}}
 \textit{-hi} `make,  cause' [+pers. affixes] 
% \end{tabular}
}
&

\parbox[t]{.5cm}{
% \begin{tabular}{l}
  =\textit{sgú\~n\k{i}}  \\
`not'
% \end{tabular}
}
&
\parbox[t]{1cm}{
% \begin{tabular}{p{1cm}}
  =\textit{\~ne}  `gen' [Whitman's indefinite]
% \end{tabular}
}
&
\parbox[t]{1cm}{
% \begin{tabular}{p{1cm}}
  \textit{=wi}  \\
`definite' 
% \end{tabular}
}
&
\parbox[t]{1cm}{
% \begin{tabular}{p{1cm}}
  \textit{=hna} \\
`future,  incompletive'  
% \end{tabular} 
}
\\ 
\midrule
\end{tabular}
}
\end{sidewaysfigure}



\end{document}