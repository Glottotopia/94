\documentclass[output=paper]{LSP/langsci}
\author{Bryan Rosen}
\title{On the structure and constituency of Hocąk resultatives}

\abstract{Abstract: This paper explores the structure and constituency of Hocąk (Siouan) ``adjectival'' resultatives. I argue that Hocąk resultatives project a phrasal XP as the complement of the verb in a Larsonian ``VP-shell'' (Larson 1988), while the object of the resultative is in Spec,VP. First, I show that the result is an XP and is not a full clause (i.e., a CP). Second, I provide evidence that the result is in a VP-internal position. While the focus of this paper is the structure of resultatives in Hocąk, resultatives as a construction tend to highlight other important characteristics of a language's grammar. I argue that the result predicate is an AP. This puts Hocąk resultatives in line with English adjectival resultatives. The data from resultatives thus suggest that Hocąk has the lexical category adjective, contra the previous descriptions of lexical categories in Hocąk (see Lipkind 1945; Susman 1943; and Helmbrecht 2006). The goal of this paper is therefore to present new Hocąk data, provide a structural analysis of resultatives, and then explore the adjectival nature of resultative predicates in the language. KEYWORDS: [Hoocąk, resultative construction, adjective, lexical categories]}

\maketitle

\begin{document}

\section{Introduction}

This paper explores the structure and constituency of Hocąk ``adjectival'' resultatives. In Hocąk resultatives, the result predicate appears to the left of the verb, as exemplified in (1) with \textit{paras} `flat' and \textit{šuuc} `red'.\footnote{Unless noted otherwise, the data comes from  elicitation with Cecil Garvin. My methodology follows the standard techniques of translation and acceptability judgment tasks (see Matthewson 2004 for more details).}

\ea
\ea
\glll Meredithga mąąsra paras gistakšąną. \\
 Meredith-ga mąąs-ra paras {$\varnothing$}-gistak-šąną\\
Meredith-\textsc{prop} metal-\textsc{def} flat  \textsc{3s/o}-hit-\textsc{decl}\\
% \glt `Meredith hit the metal flat.'

\ex 
\glll Cecilga wažątirera šuuc hogiha. \\
Cecil-ga  wažątire-ra šuuc {$\varnothing$}-hogiha \\
Cecil-\textsc{prop} car-\textsc{def} red \textsc{3s/o}-paint\\
% \glt `Cecil painted the car red.'
\z
\z


The analysis of examples like those in (1) is as follows: I propose that Hocąk resultatives project a phrasal AP as the complement of the verb in a Larsonian ``VP shell'' (i.e., a recursive VP structure; Larson 1988). The object of the resultative is in the specifier of VP. Thus, the sentence in (1b) has the basic structure in (2).\footnote{I assume the Principles and Parameters framework (see also the Minimalist Program and X-bar theory; Chomsky 1995). A phrase in this framework consists of three basic layers. The head (X\textsuperscript{0}) specifies the syntactic type or lexical category of the phrase (e.g., V for verb, N for noun, and A for adjective). Complements are arguments (e.g., objects) of X\textsuperscript{0} and are sisters to the X head. Specifiers (Spec for short) are often reserved for subjects of the the phrase. They are sisters to X$'$.}
\begin{exe}
\ex

{\hspace{1em}}\newline
\begin{tikzpicture}
\Tree [ .vP [ .NP \edge[roof]; {Cecilga} ] [ . v$'$ [ .VP [ .NP \edge[roof]; {wažątirera\\`the car'} ] [ .V$'$ [ .AP \edge[roof]; {šuuc\\`red'} ] [ .V  hogiha\\`paint' ] ] ] [ .v ] ] ]
\end{tikzpicture}
\end{exe}

While the focus of this paper is to propose a structure of resultatives in Hocąk, resultatives as a construction tend to highlight other important characteristics of a language's grammar. Hocąk resultatives are no exception. I argue that the result predicate is an AP. This puts Hocąk resultatives in line with English adjectival resultatives. The data from resultatives
thus suggest that Hocąk has the lexical category adjective, contra the previous descriptions of lexical categories in Hocąk (see Lipkind 1945, Susman 1943, and Helmbrecht 2006). The goal of this paper is therefore to present new Hocąk data, provide a structural analysis of resultatives, and then explore the adjectival nature of resultative predicates in the language. The rest of this paper is organized as follows: \sectref{sec:rosen:2} provides background on Hocąk syntax and resultatives in Hocąk. \sectref{sec:rosen:3} examines the constituency of Hocąk resultatives. In \sectref{sec:rosen:4}, I give a syntactic representation of resultatives in Hocąk. In \sectref{sec:rosen:5}, I argue that the result predicate projects as an AP. \sectref{sec:rosen:6} concludes the paper.

\section{Overview of Hocąk Syntax}

In this section, I first present background information on word order in Hocąk, and then I discuss some preliminary characteristics of Hocąk resultatives.

\subsection{Word Order in Hocąk}

Unmarked word order in Hocąk is SOV, as in (3). Variation in word order has discourse effects: a rightward displaced noun phrase is interpreted as discourse-old in (4a), while a leftward moved noun phrase serves a different discourse function (e.g., topic or focus) in (4b). Note that the interpretation in (4b) with OSV word order is possible because there is a pause (represented by the comma) that offsets the fronted object.

\begin{exe}

\ex \glll Wijukra šųųkra haja \\
 wijuk-ra šųųk-ra {$\varnothing$}-haja\\
cat-\textsc{def} dog-\textsc{def} \textsc{3s/o}-see\\
\glt `The cat saw the dog.'

\end{exe}

\begin{exe}
\ex
\begin{xlist}

\ex \glll Wijukra  haja, šųųkra \\
 wijuk-ra {$\varnothing$}-haja  šųųk-ra\\
cat-\textsc{def}  \textsc{3s/o}-see dog-\textsc{def}\\
\glt `The cat saw something, the dog.' 

\ex \glll \v{S}ųųkra, wijukra haja  \\
 šųųk-ra wijuk-ra {$\varnothing$}-haja  \\
dog-\textsc{def} cat-\textsc{def}  \textsc{3s/o}-see \\
\glt `The dog, the cat saw (it).' 

\end{xlist}
\end{exe}

In double object constructions, the canonical word order is subject---indirect object---direct object---verb. This is shown below in (5).

\begin{exe}

\ex \glll Hinųknįkhižą hocįcįhižą wiiwagaxhižą hok'ų.\\
hinųknįk-hižą hocįcį-hižą wiiwagax-hižą {$\varnothing$}-hok'ų\\
girl-\textsc{indef} boy-\textsc{indef} pencil-\textsc{indef} \textsc{3s/o}-give\\
\glt `A girl gave a boy a pencil.'

\end{exe}

In Hocąk, word order is crucial to disambiguate the subject from the object: the first argument is interpreted as the subject. In (6), the first interpretation of the sentence (although pragmatically unlikely) is the only one with neutral intonation; however, the second interpretation is  possible if there is a pause after `car'.

\begin{exe}

\ex \glll Wažątirera hinųkra ruwį.\\
wažątire-ra hinųk-ra {$\varnothing$}-ruwį\\
car-\textsc{def} woman-\textsc{def} \textsc{3s/o}-buy\\
\glt `The car bought the lady.' \textsc{or} `The lady bought the car.'

\end{exe}

Johnson and \citet{Rosen2014} argue that Hocąk is underlying head-final, by providing evidence from quantifier scope and postverbal predicates. Thus, I represent Hocąk as head-final here.

\subsection{Resultatives in Hocąk: Some Preliminaries}

Resultatives are complex predicates that put together a means predicate (always a verb) and a result predicate, where neither is licensed by a conjunction or an adposition (Williams 2008). In (7), the result \textit{šuuc} `red' immediately precedes the means \textit{hogiha} `paint', and the direct object \textit{wažątirera} `the car' surfaces to the left of the result. Since the result is typically analyzed as the complement of the means (Li 1999, Williams 2008), the result-means order would be expected in a head-final language.

\begin{exe}

\ex \glll Cecilga wažątirera šuuc hogiha. \\
Cecil-ga  wažątire-ra šuuc {$\varnothing$}-hogiha \\
Cecil-\textsc{prop} car-\textsc{def} red \textsc{3s/o}-paint\\
\glt `Cecil painted the car red.'

\end{exe}

The word order of resultatives and sentences with object-internal attributive modifiers is similar. Compare the position of the result phrase in (7) with the position of the attributive modifier in (8).

\begin{exe}

\ex \glll Cecilga wažątire šuucra hogiha. \\
Cecil-ga  wažątire šuuc-ra  {$\varnothing$}-hogiha\\
Cecil-\textsc{prop} car red-\textsc{def}  \textsc{3s/o}-paint\\
\glt `Cecil painted the red car.'

\end{exe}

In (8), the modifier \textit{šuuc} `red' is located to the right of the noun it modifies, \textit{wažątire} `car'. This attributive modifier cannot be to the right of the definite article \textit{-ra}. This entails that \textit{šuuc} `red' in (8) in an NP-internal position. By comparison, the result in (7) (\textit{šuuc} `red') is to the right of the definite article \textit{-ra}, which indicates that the result is in an NP-external position.

Moreover, the result AP can ``scramble,'' or move leftward, to a position before the object or subject, as illustrated in (9). In contrast, attributive modifiers do not have this option, as in (10). This contrast demonstrates that resultative predicates are not treated as part of the NP-object, and provides further evidence that they are not in an NP-internal position.

\begin{exe}
\ex
\begin{xlist}

\ex \glll Cecilga \textbf{šuuc}, wažątirera woogiha. \\
 Cecil-ga šuuc  wažątire-ra wa-{$\varnothing$}-hogiha\\
 Cecil-\textsc{prop} red car-\textsc{def} \textsc{3o.pl}-\textsc{3s}-paint \\
\glt `Cecil painted the cars red.'

\ex \glll \textbf{\v{S}uuc}, Cecilga wažątirera woogiha. \\
  šuuc Cecil-ga wažątire-ra wa-{$\varnothing$}-hogiha\\
red Cecil-\textsc{prop} car-\textsc{def} \textsc{3o.pl}-\textsc{3s}-paint \\
\glt `Cecil painted the cars red.'

\end{xlist}


\ex[*] {\glll Meredithga \textbf{šuuc}, wiišgacra ruwį. \\
Meredith-ga šuuc wiišgac-ra {$\varnothing$}-ruwį\\
Meredith-\textsc{prop} red toy-\textsc{def} \textsc{3s/o}-buy\\
\glt Intended: `Meredith bought the red toy.'}


\end{exe}

It should be noted that resultative constructions have been categorized cross-linguistically based on whether the lexical semantics of the verb and the result are independent of each other. In his typology of Japanese and English resultative predicates, \citet{Washio1997} presents two types of resultatives: \textit{weak} and \textit{strong}. When the lexical semantics of the verb entails a change, it is called a weak resultative. When the verb in resultative constructions does not entail a change, Washio refers to this class as a strong resultative. In other words, the classification between weak and strong resultatives depends on whether the matrix verb denotes a result. Consider the two English examples in (11).

\begin{exe}
\ex
\begin{xlist}

\ex {Sam painted the wall red.}

\ex {Alex pounded the metal thin.}

\end{xlist}
\end{exe}

In (11a), the verb \textit{paint} entails that there is some change, since \textit{to paint} means to apply color. \textit{Paint} represents an example of a weak resultative. In (11b), however, the verb \textit{pound} does not entail that the object being pounded will become flat. That is, pounding metal could result in the metal being bumpy. Thus, there is no entailed change with \textit{pound}. The verb \textit{pound} is an example of a strong resultative. In Hocąk, resultatives are possible when the verb lexically specifies a change, as with \textit{hogiha} `paint' in (7) above and with \textit{gižap} `polish' in (12) below.

\begin{exe}

\ex \glll Meredithga mąąsra gišįnįšįnį gižapšąną.  \\
 Meredith-ga mąąs-ra gišįnįšįnį {$\varnothing$}-gižap-šąną\\
 Meredith-\textsc{prop} metal-\textsc{def} shiny \textsc{3s/o}-polish-\textsc{decl}\\
\glt `Meredith polished the metal shiny.'

\end{exe}

A verb like \textit{gižap} `polish' strongly denotes an activity whereby its object (theme) changes its state to become `shiny.' Because \textit{gižap} implies this change of state, it is considered a weak resultative. 

They can also be formed with verbs that do not specify a change, as with \textit{gistak} `hit' and \textit{rucgis} `cut' in (13).

\begin{exe}
\ex
\begin{xlist}

\ex \glll Meredithga mąąsra paras gistakšąną. \\
 Meredith-ga mąąs-ra paras {$\varnothing$}-gistak-šąną\\
Meredith-\textsc{prop} metal-\textsc{def} flat \textsc{3s/o}-hit-\textsc{decl}\\
\glt `Meredith hit the metal flat.'

\ex \glll Matejaga peešjįra žiipįk rucgisšąną.\\
Mateja-ga peešjį-ra žiipįk {$\varnothing$}-rucgis-šąną\\
Mateja-\textsc{prop} hair-\textsc{def} short \textsc{3s/o}-cut-\textsc{decl}\\
\glt `Mateja cut the hair short.'

\end{xlist}
\end{exe}

Similar to \textit{pound} in English, \textit{gistak} `hit' in Hocąk does not denote an event whereby its object results in a particular state (e.g., flat). Thus we can consider this verb a strong resultative. The verb \textit{rucgis} `cut' belongs to the class of strong resultatives for the same reasons: the event denoted by \textit{rucgis} `cut' does not contain the notion of being short. Thus, Hocąk exhibits both strong and weak resultatives. With this background in mind, I turn to the next section, where I discuss more about the constituency of Hocąk resultatives.

\section{The Constituency of Hocąk Resultatives}

This section outlines some diagnostics that support the structure presented in (2) for Hocąk resultatives. In \sectref{sec:rosen:3.1}, I provide evidence that the result is a phrase and not a clause, while in \sectref{sec:rosen:3.2} I show that the result is in a VP-internal position.

\subsection{The Result Predicate as a Phrase}

In this subsection, I show that the result is an XP and is not a full clause (i.e., a CP). First, it should be noted that the result is not a head that forms a compound with the matrix verb; that is, the verb and the result in the construction should not be considered a single lexical unit, such as V\textsuperscript{0} or A\textsuperscript{0}. The result can include adverbial modifiers, such as \textit{hikųhe} `quickly' in (14a), and the intensifier suffix \textit{-xjį} in (14b).

\begin{exe}
\ex
\begin{xlist}

\ex \glll Meredithga mąąsra paras \textbf{hikųhe} gistakšąną. \\
 Meredith-ga mąąs-ra paras hikųhe {$\varnothing$}-gistak-šąną\\
Meredith-\textsc{prop} metal-\textsc{def} flat quickly \textsc{3s/o}-hit-\textsc{decl}\\ 
\glt `Meredith hit the metal flat quickly.'


\ex \glll Meredithga mąąsra parasxjį gistakšąną.\\
 Meredith-ga mąąs-ra paras-\textbf{xjį} {$\varnothing$}-gistak-šąną\\
Meredith-\textsc{prop} metal-\textsc{def} flat-very  \textsc{3s/o}-hit-\textsc{decl}\\
\glt `Meredith hit the metal very flat.'

\end{xlist}
\end{exe}

A piece of evidence that the result predicate is not a clause comes from the fact the result phrase cannot take declarative (15a), or complementizer (15b) suffixes.

\begin{exe}
\ex
\begin{xlist}

\ex[*] {\glll Matejaga peešjįra žiipįkšąną rucgisšąną.\\
Mateja-ga peešjį-ra  žiipįk-šąną {$\varnothing$}-rucgis-šąną\\
Mateja-\textsc{prop} hair-\textsc{def}  short-\textsc{decl}  \textsc{3s/o}-cut-\textsc{decl}\\
\glt Intended: `Mateja cut the hair short.'}

\ex[*] {\glll Matejaga peešjįra  žiipįkra rucgisšąną.\\ 
Mateja-ga peešjį-ra  žiipįk-ra  {$\varnothing$}-rucgis-šąną\\
Mateja-\textsc{prop} hair-\textsc{def}  short-\textsc{comp}   \textsc{3s/o}-cut-\textsc{decl}\\
\glt Intended: `Mateja cut the hair short.'}

\end{xlist}
\end{exe}

The result also cannot take the future tense marker \textit{kjane}, as in (16), even though the hair becoming short would necessarily take place after cutting it.

\begin{exe}

\ex[*] {\glll Matejaga peešjįra  žiipįk ikjane rucgisšąną.\\
Mateja-ga peešjį-ra  žiipįk kjane {$\varnothing$}-rucgis-šąną\\
Mateja-\textsc{prop} hair-\textsc{def}  short \textsc{fut} \textsc{3s/o}-cut-\textsc{decl}\\
\glt Intended: `Mateja cut the hair short.'}

\end{exe}

In addition, the result cannot bear the negation suffix \textit{-nį}. Negation in Hocąk is bipartite: the free particle \textit{hąąke} and the suffix \textit{-nį} are both required to form the negative. The example in (17a) shows that \textit{-nį} attaches to the matrix verb, while (17b) illustrates that the result cannot appear with \textit{-nį}.

\begin{exe}
\ex
\begin{xlist}

\ex[] {\glll Meredithga hąąke mąąsra paras gistaknį.\\
Meredith-ga hąąke mąąs-ra paras {$\varnothing$}-gistak-\textbf{nį}\\
Meredith-\textsc{prop} \textsc{neg} metal-\textsc{def} flat \textsc{3s/o}-hit-\textsc{neg}\\
\glt `Meredith did not hit the metal flat.'}

\ex[*] {\glll Meredithga hąąke mąąsra parasnį gistak.\\
Meredith-ga hąąke mąąs-ra paras-\textbf{nį} {$\varnothing$}-gistak\\
Meredith-\textsc{prop} \textsc{neg} metal-\textsc{def} flat-\textsc{neg} \textsc{3s/o}-hit\\
\glt Intended: `Meredith did not hit the metal flat.' \textsc{or}\\
`Meredith hit the metal such that its surface didn't get fully flat.'}

\end{xlist}
\end{exe}

If the result could take one of these suffixes, this would mean that it would have the syntactic status of a clause. Since the examples in (15)-(17) are ungrammatical, the result must not be a clause.

Third, Hocąk resultatives respect the \textit{Direct Object Restriction} (DOR): the result predicate must be predicated on the NP in object position (Levin and Rappaport Hovav 1995). That is, the result must be predicated of a transitive object or the subject of an unaccusative, but not the subject of a transitive or an unergative verb.\footnote{Note that the DOR can also apply to so-called `fake' objects (e.g., reflexives) of unergative verbs. For example, the result phrase \textit{hoarse} can be predicated on \textit{herself} in (i). See Carrier and \citet{Randall1992}, \citet{Li1999}, and \citet{Wechsler2005} for more details on the DOR. (i) The woman sang herself hoarse.} This restriction is shown in (18) with the transitive verb \textit{gistak} `hit'.

\begin{exe}

\ex \glll Rockyga wanįra šuuc gistakšąną.\\ 
Rocky-ga wanį-ra šuuc {$\varnothing$}-gistak-šąną\\
Rocky-\textsc{prop} meat-\textsc{def} red \textsc{3s/o}-hit-\textsc{decl}\\
\glt = `Rocky hit the meat red.' \vspace{-3pt} \\ 
$\not=$ `Rocky hit the meat  and he was red as a result.'

\end{exe}

As seen in (18), since \textit{wanįra} `the meat' is in object position, it can be the subject of the result, while the subject of matrix verb \textit{Rocky} cannot. The contrast in (18) points to the fact that the result is not a clause (i.e., a CP). I follow \citet{Li1999} and assume that when the result can be linked to either the subject or the object and the result plus the means predicate is not formed in the lexicon (i.e., they do not form a compound), the resultative phrase is a clause with a \textit{pro}-controlled subject (see also Song 2005). According to \citet{Chomsky1982}, \textit{pro} is an empty category of the type [+pronominal, --anaphoric], and Binding Theory states that it cannot be bound within its governing category. Thus, \textit{pro} could be bound by either the matrix subject or object. Since the result in (18) cannot be linked to the subject, the result cannot be a clause.

Moreover, Hocąk resultatives show a contrast in availability between prototypical unaccusative and unergative verbs. \citet{Perlmutter1978} defines unaccusative verbs as ones where the single argument is an underlying object, whereas the argument of an unergative verb is an underlying subject. Typically, unaccusative verbs denote change (e.g., \textit{break, melt}) while unergative verbs indicate manner of motion (e.g., \textit{run}) or other bodily functions (e.g., \textit{cry}). In Hocąk, intransitive verbs that take stative agreement morphemes correspond to unaccusatives, and the set of intransitive verbs that bear active agreement morphemes are parallel to unergative verbs (see e.g., Williamson 1984, Woolford 2010). Prototypical unaccusatives in (19), such as \textit{ziibre} `melt' and \textit{taaxu} `burn', can serve as the matrix verb of resultatives. On the other hand, prototypical unergative verbs in (20), such as \textit{nąąwą} `sing' and \textit{nąąk} `run', cannot.

\begin{exe}
\ex
\begin{xlist}

\ex \glll Xaigirara sgaasgap {ziibre}. \\
 xaigira-ra sgaasgap {$\varnothing$}-ziibre\\
chocolate-\textsc{def} sticky \textsc{3s}-melt\\
\glt `The chocolate melted sticky.'

\ex \glll Waisgapra seep {taaxu}.\\
 waisgap-ra seep {$\varnothing$}-taaxu\\
bread-\textsc{def} black \textsc{3s}-burn\\
\glt `The bread burned black.'

\end{xlist}
\end{exe}

\begin{exe}
\ex
\begin{xlist}

\ex[*] {\glll Hinųkra nįįra teek {nąąwą}. \\
hinųk-ra nįį-ra teek {$\varnothing$}-nąąwą\\
woman-\textsc{def} throat-\textsc{def} sore 3\textsc{s}-sing\\
\glt Intended: `The woman sang her throat sore.'}

\ex[*] {\glll Henryga wagujirera paras {nąąkšąną}. \\
Henry-ga wagujire-ra paras {$\varnothing$}-nąąk-šąną\\
Henry-\textsc{prop} shoe-\textsc{def} flat \textsc{3s}-run-\textsc{decl}\\
\glt Intended: `Henry ran the shoe(s) flat.'}

\end{xlist}
\end{exe}

Note that the restriction with unergative verbs also holds when the reflexive morpheme \textit{kii-} denotes the so-called `fake' reflexive/object of the predicate; see (21).\footnote{Under Washio's (1997) typology, intransitive resultatives are a type of weak resultative. For example, resultatives with an unergative verb like `run' can form a weak resultative. Recall that Hocąk has transitive strong resultatives (see (13) above). Hocąk resultatives thus present a counterexample to Washio's typology: Hocąk has transitive strong resultatives but lacks intransitive strong resultatives. I leave further discussion of these examples with respect to Washio's typology for future work.}

\begin{exe}

\ex[*] {\glll Hunterga hoix'įk kiinąąkšąną.\\
Hunter-ga hoix'įk <\textbf{kii}>{$\varnothing$}-nąąk-šąną\\
Hunter-\textsc{prop} tired <\textsc{refl}>\textsc{3s}-run-\textsc{decl}\\
\glt Intended: `Hunter ran himself tired.'}

\end{exe}

The DOR states the result must be predicated of the object. If we assume that the subjects of the verbs in (20) are underlying objects, we can maintain the DOR. On the other hand, since unergative verbs do not have an underlying object, no resultative interpretation is possible in (20) and (21).\footnote{The DOR holds consistently in English for transitive objects. In the case of unergative verb phrases, a fake reflexive/object ensures that there is an object that the result can be linked to. (See the translations in (20) and (21)).}

\subsection{VP-Internal Status of the Result Predicate}

In this subsection, I argue that the result predicate is the complement of the verb. I first show that the result predicate must be VP-internal, and then I provide evidence that resultatives in Hockąk project as a binary structure. Levin and Rappaport \citet{Hovav1995} use VP-ellipsis in order to show that resultatives are VP-internal, and that they are part of the eventuality of the VP. Hockąk has a type of VP-ellipsis shown in (22) and (23): the light verb \textit{ųų} can replace either a minimal VP or a multi-segmental VP, resulting from adjunction to VP. Example (22) shows an example of VP-ellipsis that targets on the object and the verb, while in (23), VP-ellipsis targets a VP-level adjunct, such as \textit{xjanąre} `yesterday'.

\begin{exe}

\ex \glll Cecilga [\textsubscript{VP} wažątirehižą ruwį] kjane anąga nee šge [haųų] kjane.\\
Cecil-ga {} wažątire-hižą {$\varnothing$}-ruwį kjane anąga nee šge ha-ųų kjane.\\
Cecil-\textsc{prop} {} car-\textsc{indef} \textsc{3s/o}-buy \textsc{fut} and I also \textsc{1s}-do \textsc{fut}\\
\glt `Cecil will buy a car, and I will too.' (Johnson 2013, (5))

\ex \glll Cecilga [\textsubscript{VP} xjanąre waši] anąga Bryanga šge [ųų].\\
Cecil-ga {} xjanąre {$\varnothing$}-waši anąga Bryan-ga šge {$\varnothing$}-ųų.\\
Cecil-\textsc{prop} {} yesterday \textsc{3s}-dance and Bryan-\textsc{prop} also \textsc{3s}-do\\
\glt `Cecil danced yesterday, and Bryan did too.' (Johnson 2013, (6a))

\end{exe}

As shown in (24b), it is not possible to `strand' the result predicate \textit{šuuc} `red' under VP-ellipsis. It thus follows that the result is inside the VP, rather than adjoined to VP.

\begin{exe}
\ex
\begin{xlist}

\ex[] {\glll Hunterga [\textsubscript{VP} nąąju seep hogiha] anąga Bryanga šge [ųų].\\
Hunter-ga {} nąąju seep {$\varnothing$}-hogiha anąga Bryan-ga šge {$\varnothing$}-ųų\\
Hunter-\textsc{prop} {} hair black \textsc{3s/o}-dye and Bryan-\textsc{prop} too \textsc{3s}-do\\ 
\glt `Hunter dyed the hair black and Bryan did, too.'}

\ex[*] {\glll Hunterga nąąju seep hogiha anąga Bryanga šge \textbf{šuuc} ųų.\\
Hunter-ga nąąju seep {$\varnothing$}-hogiha anąga Bryan-ga šge šuuc {$\varnothing$}-ųų\\
Hunter-\textsc{prop} hair black \textsc{3s/o}-dye and Bryan-\textsc{prop} too red \textsc{3s}-do\\ 
\glt Intended: `Hunter dyed the hair black and Bryan did red, too.'}

\end{xlist}
\end{exe}

Example (24) contrasts with (25).  (25) contains the adverb \emph{wasisik} `energetically' as a depictive. Since depictives are typically analyzed as adjuncts that occupy a VP-external position (Levin and Rappaport Hovav 1995), they can be stranded.

\begin{exe}

\ex \glll Bryanga [\textsubscript{VP} waarucra hoix'įk waža\textsc] anąga  Meredithga \textbf{wasisik} [ųų]. \\
 Bryan-ga {} waaruc-ra hoix'įk {$\varnothing$}-waža anąga Meredith-ga  wasisik {$\varnothing$}-ųų\\
Bryan-\textsc{prop} {} table-\textsc{def} tired \textsc{3s/o}-wipe and Meredith-\textsc{prop}  energetic \textsc{3s}-do\\
\glt `Bryan wiped the table tired(ly) and Meredith did energetically.'

\end{exe}

As we saw in (22), \textit{ųų} affects the verb and its complement. Since a result predicate is not strandable with \textit{ųų}, it must be the case that the result is inside the minimal VP, and thus is part of the core eventuality of the VP. In other words, it follows that the result is inside the verb phrase.

Another option for the structure of resultatives could be that the verb, the result, and direct object are all sisters in a flat structure. Carrier \& \citet{Randall1992} provide such a ternary analysis for English resultatives. However, \citet{Bowers1997} argues that a ternary structure cannot account for structures involving Across the Board movement. This type of movement describes a situation when a syntactic element moves from multiple base positions to a single terminal position. In this conjunctive test, the object and result of both conjuncts form a single constituent (see also Li 1999). An Across the Board structure is possible with Hocąk resultatives, as seen in (26), where the verb is moving across conjuncts.

\begin{exe}
\ex
\begin{xlist}

\ex \glll Meredithga mąąsra paras gistak anąga waisgap pereįk. \\
Meredith-ga mąąs-ra paras {$\varnothing$}-gistak anąga waisgap pereįk\\
Meredith-\textsc{prop} metal-\textsc{def} flat \textsc{3s/o}-hit and bread thin\\
\glt `Meredith hit the metal flat and the bread thin.'

\ex \textit{Meredithga \hspace{1.58em} mąąsra \hspace{.9em} gišįnį{s}įnį \hspace{.1em}gižap  \hspace {2.4em} anąga wažątirera}  \newline Meredith-ga \hspace{1.18em} mąąs-ra \hspace {.5em} gišįnį{s}įnį {$\varnothing$}-gižap \hspace{1.3em} anąga wažątire-ra \newline Meredith-\textsc{prop} metal-\textsc{def} shiny \hspace{1.6em} \textsc{3s/o}-polish and \hspace{1em} car-\textsc{def} \newline\textit{sgee}. \newline
sgee \newline
clean \newline
`Meredith polished the metal shiny and the car clean.'

\end{xlist}
\end{exe}
 
The ability of Hocąk resultatives to participate in Across the Board movement is consistent with an analysis that argues for a binary structure (Bowers 1997). I conclude that Hocąk resultatives are straightforwardly analyzable under a binary branching approach. This provides another argument that the result is in a VP-internal position.
 
\section{Syntactic Representation of Hocąk Resultatives}
 
In this section, I propose that resultatives are in a Larsonian VP-shell structure (Larson 1988): a VP structure takes another VP as its complement. This approach follows Li's (1999) structure for English resultatives (cf. Hoekstra 1988; Carrier \& Randall 1992; Levin \& Rappaport Hovav 1995). Larson's (1988) VP-shells are intended to accommodate the double-object construction, where the left-most object is in a higher position than the right-most. If we maintain a binary branching structure, then a resultative has the same structure as the double-object construction. I claim that the structure for Hocąk resultatives is depicted in (27). The result predicate is the complement of the verb, and I assume that the object is base-generated in Spec,VP. The subject is generated in Spec,vP, where ``little v'' is a semi-functional head that licenses external arguments (Chomsky 1995).

\begin{exe}
\ex
\begin{xlist}

\ex \glll Cecilga wažątirera šuuc hogiha. \\
Cecil-ga  wažątire-ra šuuc {$\varnothing$}-hogiha \\
Cecil-\textsc{prop} car-\textsc{def} red \textsc{3s/o}-paint\\
\glt `Cecil painted the car red.'

\ex 
{\hspace{1em}}\newline
\begin{tikzpicture}
\Tree [ .vP [ .NP \edge[roof]; {Cecilga} ] [ .v$'$ [ .VP [ .NP \edge[roof]; {wažątirera\\`the car'} ] [ .V$'$ [ .AP \edge[roof]; {šuuc}\\`red' ] [ .V hogiha\\'`paint' ]] ][ .v ] ] ]
\end{tikzpicture}
\end{xlist}
\end{exe}

The structure in (27b) straightforwardly explains the facts with respect to Hocąk resultatives. First, the result is not a head that forms a compound with the matrix verb since adverbs and intensifiers can intervene. The structure in (27b) shows that the result is an AP and not a CP. This accounts for why the result cannot take complementizer, tense, negation, or declarative suffixes: the result is an AP, which does not contain clause level heads or morphology. This property of result APs is also reflected in the fact that Hocąk resultatives obey the DOR. In (18), only the object `meat' can be modified by the result `red'. This restriction predicts that the result is not a clause. If the result were a clause, the subject in (18) could also be modified by `red' because resultative phrases that project as CPs have a pro subject, which could be linked to the matrix subject. However, this is not the case. To formalize the relationship between the NP object and the adjective, I follow Li's (1999) analysis. The AP can assign its theta-role to the object through mutual m-command.\footnote{I assume that m-command refers to a syntactic relation where X m-commands Y if and only if the first maximal projection that dominates X also dominates Y and X does not dominate Y. In (31b), X is the NP \textit{wažątirera} `the car', and Y is the AP \textit{šuuc `red'.}}In the case of (31), AP and the NP in Spec,VP are both dominated by the same VP node, and they do not dominate each other. Thus, the AP and the object NP mutually m-command each other. On the other hand, the AP does not hold a mutual m-command relationship with the subject in Spec,vP; thus, the AP cannot assign its theta-role to the subject. This results in the DOR effect.

This situation applies to resultatives with unaccusative matrix verbs, as depicted in (28b).

\begin{exe}
\ex
\begin{xlist}

\ex \glll Waisgapra seep {taaxu}.\\
 waisgap-ra seep {$\varnothing$}-taaxu\\
bread-\textsc{def} black \textsc{3s}-burn\\
\glt `The bread burned black.'


\ex
\Tree [ .VP [ .NP \edge[roof]; {waisgapra} ] [ .V$'$ [ .AP \edge[roof]; {seep} ] [ .V taaxu ] ] ]

\end{xlist}
\end{exe}


The AP in (28b) has the same position that it has in (27b); that is, it is the complement of the verb. Thus, the AP maintains the same relationship with the object in Spec,VP whether the verb is transitive or intransitive. Consequently, the AP \textit{seep} `black' and the object \textit{waisgapra} `the bread' are within the same VP, and the DOR effect is preserved. Data from VP-ellipsis has also demonstrated that the result phrase is inside the VP. This is in contrast to depictive phrases, where the depictive can be stranded by VP-ellipsis. Assuming the structure presented above, this contrast falls out naturally. Depictives have been analyzed as VP-adjuncts in English (Levin \& Rappaport Hovav 1995); thus, I suggest that a depictive phrase, such as \textit{wasisik} `energetically' in (25), is adjoined to the upper VP-shell (i.e., vP) in (27b).\footnote{In this paper, I leave it open whether depictives can adjoin to the lower VP-shell.}

To summarize, I have argued that the resultative secondary predicate is the complement to the main verb, and is a phrase. This accounts for a constellation of facts that concern the properties of Hocąk resultatives, including the DOR.

\section{The Result Predicate and Adjectives in Hocąk} 

Thus far I have assumed without comment that the result predicate is an adjective phrase. This section provides evidence that the result is in fact an AP, and thus that Hocąk has adjectives. Traditional grammars (e.g., Lipkind 1945 and Susman 1943) and more recently \citet{Helmbrecht2006} have claimed that Hocąk lacks the lexical class adjectives since there is no distinct inflectional morphology for adjectives and verbs. Instead these works claim that adjectives are a class of stative verbs. For reasons of space, I consider only two of these arguments in detail.

First, \citet{Helmbrecht2006} shows that there is no category establishing morphology with respect to adjectives. Recall that Hocąk has an active-stative split between intransitive verbs. Helmbrecht notes that purported adjectives and stative verbs exhibit parallel agreement morphology, as shown in (29) and (30), respectively.

\begin{exe}
\ex
\begin{xlist}
\ex \gll
hį-xete \hspace{36pt} b. {}  nį-xete \hspace{48pt} c. {} xete-ire\\
1-big {} {} {} 2-big {} {}  {} big-\textsc{3s.pl}\\
\glt `I am big.' \hspace{1.2cm} `You are big.' \hspace{1.1cm} `They are big.'

\end{xlist}
\end{exe}

\begin{exe}
\ex
\begin{xlist}
\ex \gll
hį-šiibre \hspace{30pt}  b. {} nį-šiibre \hspace{43pt} c. {} šiibre-ire\\
1-fall {} {} {} 2-fall {} {} {} fall-\textsc{3s.pl}\\
\glt `I fell.' \hspace{2cm} `You fell.'  \hspace{1.7cm} `They fell.'


\end{xlist}
\end{exe}

Example (29) illustrates that the stative set of agreement markers may be used with adjectives: in (29a,b), the prefixes \textit{hiį-} and \textit{niį-} mark 1st and 2nd person respectively, and in (29c) \textit{-ire} encodes third person plural. The example in (30) with the stative verb \textit{šiibre} `fall' shows that this verb bears the same agreement markers. Since Hocąk is an active-stative language, the similarities between (29) and (30) follow if apparent adjectives are stative verbs. Second, apparent adjectives can be used predicatively without any morphological modification or without the help of auxiliaries, as seen in (31a). \citet{Helmbrecht2006} asserts that the lack of auxiliaries is possible for all adjectives in Hocąk. This possibility extends to verbs as well. (31a) shows an example of the verb \textit{nįį} `swim'.

\begin{exe}
\ex
\begin{xlist}

\ex \glll Wijukra seepšąną. \\
wijuk-ra {$\varnothing$}-seep-šąną\\
cat-\textsc{def} \textsc{3s}-black-\textsc{decl}\\
\glt `The cat is black.'


\ex \glll Hocįcįkra nįį eeja nįįpšąną.\\
hocįcįk-ra nįį eeja {$\varnothing$}-nįįp-šąną\\
boy-\textsc{def} water there \textsc{3s}-swim.\textsc{act}-\textsc{decl}\\
\glt `The boy swam in the lake.'

\end{xlist}
\end{exe}


Thus, since verbs and purported adjectives may also be the main predicate of the clause, there is no structural difference between adjectives and verbs.

In the following subsections, I present two arguments that the resultative phrase projects as an AP in Hocąk resultatives.\footnote{\citet{Baker2003} has previously argued that a main characteristic of adjectives is that they can occur as secondary resultative predicates.} In the first subsection, I argue that the linear ordering of the result and the matrix verb indicates that the result is an AP. In the second subsection, I turn to the fact that (stative) verbs are ungrammatical as a result predicate. I argue that only gradable predicates (i.e., adjectives) can participate in resultatives. 

\subsection{The Temporal Iconicity Condition and Resultatives}

Following \citet{Li1993}, I suggest that the fact that the result precedes the verb in resultative predication provides evidence that the result is an adjective in Hocąk.\footnote{Thanks to Yafei Li (personal communication) for bringing this diagnostic to my attention.} Specifically, I argue that since the result precedes the matrix verb in resultatives, Li's (1993) Temporal Iconicity Constraint would be violated if the result were a verb. Rather, since the result must precede the verb in Hocąk resultatives, the result must not be a verb. Instead, I claim that the result is an adjective.

Li (1993:499) proposes his constraint in order to account for the restrictions on the order of verbs in V-V resultative compounds in Chinese and Japanese. The first V (V-cause) always encodes the event, while the second V (V-result) indicates the result of the event. 

Li shows that V-cause must temporally and morphologically precede V-result. Li formalizes this constraint as in (32).

\begin{exe}

\ex
 \emph{Temporal Iconicity Constraint (TIC)}:\\
 Let A and B be two subevents (activities, states, changes of states, etc.) and let A$'$ and B$'$ be two verbal constituents denoting A and B, respectively; then the temporal relation between A and B must be directly reflected in the surface linear order of A$'$ and B$'$ unless A$'$ is an argument of B$'$ or vice versa.
 
 \end{exe}

For example, Li notes that in both Chinese and Japanese, V-cause is the first verb of the compound. Consider the Chinese example in (33) and the Japanese example in (34).

 \begin{exe}
 
 \ex \gll T\'{a}otao ti\`{a}o-f\'{a}n-le (Y$\overline{\rm{o}}$uyou le).\\
 Taotao jump-bored-\textsc{asp} Youyou \textsc{le}\\
 \glt `Taotao jumped and as a result he/(Youyou) got bored.' (Li 1993:480, (1b))
 
 \ex \gll John-ga Mary-o karakai-akiru-ta.\\
 John-\textsc{nom} Mary-\textsc{acc} tease-bored-\textsc{past}\\
 \glt `John teased Mary and as a result John got bored.'  (Li 1993:481, (2b))
 
 \end{exe}

What is important to note here is that V-cause always precedes V-result. In (33), the V-cause ti\`ao `jump' necessarily precedes V-result f\'an `bored'. Without the parentheses in (33), Taotao's jumping causes Youyou to become bored. With the parentheses in (33), Taotao's jumping makes himself become bored. In (34), the V-cause \textit{karakai} `tease' must appear to the left of the V-result \textit{akiru} `bored'. A further piece of evidence for the TIC comes from serial-verbs in Sranan and \textsubdot{I}j\textsubdot{o}. Sranan is syntactically a head-initial language, whereas \textsubdot{I}j\textsubdot{o} is head-final. Both examples in (35) illustrate that the verb phrase that denotes getting ahold of the instrument linearly precedes the central action. That is, `take the knife' in Sranan comes before `cut the bread', and the same pattern is seen in \textsubdot{I}j\textsubdot{o} with `basket take' preceding `yam cover'.

\begin{exe}
\ex
\begin{xlist}

\ex \gll Mi e teki a nefi koti a brede. \hspace{2cm} (Sranan; SVO)\\
I \textsc{asp} take the knife cut the bread \\
\glt `I cut the bread with the knife.'

\ex \gll \'{a}r\`{a}\'{u} su-ye \'{a}k\`{i} buru teri-m\'{i}. \hspace{2.6cm} (\textsubdot{I}j\textsubdot{o}; SOV)\\
she basket take yam cover-\textsc{past}\\
\glt `She covered a yam with a basket.' (Li 1993:500, (38))

\end{xlist}
\end{exe}

We find similar evidence from manner-of-directed motion serial verbs in Hocąk. These serial verbs consist of a manner-of-motion verb (e.g., \textit{nųųwąk} `run') and a directional motion verb (e.g., \textit{hii} `arrive'). In Hocąk, the order of these two verbs cannot be reversed. Example (36) shows that the linear order of \textit{nųųwąk} `run' and \textit{hii} `arrive' must be \textit{nųųwąk-hii}. The verb \textit{hii} `arrive' must always be the second verb. This directly follows from the TIC: a running event must logically precede the arriving event.

\begin{exe}
\ex
\begin{xlist}

\ex[] {\glll Matejaga Teejop eeja nųųwąk \textbf{hii}.\\
Mateja-ga Teejop eeja nųųwąk {$\varnothing$}-hii\\
Mateja-\textsc{prop} Madison there run \textsc{3s}-arrive\\
\glt `Mateja ran to Madison.'}

\ex[*] {\glll Matejaga Teejop eeja  \textbf{hii} nųųwąk.\\
Mateja-ga Teejop eeja  hii {$\varnothing$}-nųųwąk\\
Mateja-\textsc{prop} Madison there  arrive \textsc{3s}-run\\
\glt Intended: `Mateja ran to Madison.'}

\end{xlist}
\end{exe}

Despite the strong predictions that the TIC makes, it is not intended to account for all resultative constructions. According to Li's proposal, the TIC applies only if two conditions are met: one, the constituents involved are both verbal, and two, the verbal constituents must not be in a predicate-argument relation (e.g., causatives). Here I am only concerned with the first condition, as this second condition does not apply to Hocąk resultatives. Li presents an example from German to illustrate the first constraint, as in (37).

\begin{exe}

\ex \gll Er will das Eisen flachschlagen.\\
he wants the iron flat.pound\\
\glt `He wants to pound the iron flat.' (Li 1993:501, (41))

\end{exe}

The result encoded by \textit{flach} `flat' linearly precedes the activity \textit{schlagen} `pound'. Since \textit{flach} `flat' is an adjective, Li claims that the TIC does not apply. Rather the head-final structure of German determines the order of \textit{flach} `flat' and \textit{schlagen} `pound'. 

In summary, while the TIC applies to verbal constituents, the TIC has nothing to say about when adjectives form similar events with verbs.

Let us return to the Hocąk data. We see that the result precedes the matrix verb, as in (38a). That is, \textit{paras} `flat' linearly precedes \textit{gistak} `hit'. In fact it is ungrammatical for the result to be postverbal, as shown in (38b).

\begin{exe}
\ex
\begin{xlist}

\ex[] {\glll Meredithga mąąsra \textbf{paras} gistakšąną.\\
Meredith-ga mąąs-ra paras {$\varnothing$}-gistak-šąną\\
Meredith-\textsc{prop} metal-\textsc{def} flat \textsc{3s/o}-hit-\textsc{decl}\\
\glt `Meredith hit the metal flat.'}

\ex[*] {\glll Meredithga mąąsra  gistakšąną, \textbf{paras}.\\
Meredith-ga mąąs-ra  {$\varnothing$}-gistak-šąną paras\\
Meredith-\textsc{prop} metal-\textsc{def}  \textsc{3s/o}-hit-\textsc{decl} flat\\
\glt Intended: `Meredith hit the metal flat.'}

\end{xlist}
\end{exe}

Accordingly, if apparent adjectives in Hocąk are stative verbs, then the grammaticality of examples like (38a) is surprising. We expect (38a) to be ungrammatical, given the TIC. Since the TIC does not rule out examples like (38a), we can conclude that the result is not a verb. This is similar to the German example in (37). Moreover, the fact that the order that the TIC predicts, as in (38b), is ungrammatical also leads to the conclusion that the result is not a verb.\footnote{More needs to be said as to why the result cannot be postverbal. Johnson \& \citet{Rosen2014} propose that constituents are moved to a postverbal position via an EPP feature that can only attract DPs. I leave a full explanation of this issue open for now.} I take this as evidence that the result is an AP.

\subsection{Barring Verbs as the Result}

In this section, I show that adjectives can appear in resultative secondary predication, while verbs cannot. In order to account for the contrast, I argue that we need to slightly refine the structure of the result phrase: the result phrase in Hocąk is an AP that contains a degree phrase. Following \citet{Cover1997}, I assume that only gradable adjectives have a degree argument, and that degree heads need to bind such a degree argument. I show that non-gradable adjectives are incompatible with resultatives in Hocąk. Thus, if verbs do not have a degree argument to be discharged, the structure will be ruled out as an instance of vacuous quantification. Compare (39) that has \textit{žiipįk} `short' as the result with (40) that uses the verb \textit{šiibre} `fall'.

\begin{exe}

\ex[] {\glll Matejaga peešjįra \textbf{žiipįk} rucgisšąną. \\
 Mateja-ga peešjį-ra žiipįk {$\varnothing$}-rucgis-šąną\\
Mateja-\textsc{prop} hair-\textsc{def} short \textsc{3s/o}-cut-\textsc{decl}\\
\glt `Mateja cut the hair short.'}

\ex[*] {\glll Matejaga peešjįra \textbf{šiibre} rucgisšąną. \\
 Mateja-ga peešjį-ra šiibre {$\varnothing$}-rucgis-šąną\\
Mateja-\textsc{prop} hair-\textsc{def} fall \textsc{3s/o}-cut-\textsc{decl}\\
\glt Intended: `Mateja cut the hair (so that) it falls.'}

\end{exe}

The ungrammaticality of a verb like \textit{šiibre} `fall' in a resultative construction (40) indicates that this predicate is somehow fundamentally different than the one in (39). If we take a closer look at Hocąk, we notice that verbs are not the only elements that cannot be a secondary resultative predicate. While I argue that only adjectives can be resultative predicates in Hocąk, not all adjectives are available in this position. Crucially, non-gradable adjectives cannot appear as a result predicate. The example in (41) illustrates this for the non-gradable adjective \textit{t'ee} `dead', which is ungrammatical as a result predicate. Note that the English equivalent is grammatical, as indicated by the translation in (41).

\begin{exe}
\ex[*] {\glll Bryanga caara t'ee guucšąną.\\
Bryan-ga caa-ra t'ee {$\varnothing$}-guuc-šąną\\
Bryan-\textsc{prop} deer-\textsc{def} dead \textsc{3s/o}-shoot-\textsc{decl}\\
\glt Intended: `Bryan shot the deer dead.'}
\end{exe}

To account for the restriction seen in (41), I claim that the resultative predicate in Hocąk takes a DegP in its specifier, as shown in (42). I label this degree phrase ``Deg\textsubscript{RES}P.''\footnote{\citet{Corver1997} argues that DegP dominates the AP (as in (i)). Differently than Corver, the structure in (42) follows \citet{Jackendoff1977} and Bhatt \& \citet{Pancheva2004}, among others, and places DegP in Spec,AP. Nothing crucially hinges on the placement of the degree phrase, however. 
\begin{exe} \ex \Tree [ .DegP [ .AP ]  [ .Deg ] ]
\end{exe}}

\begin{exe}
\ex 
{\hspace{1em}}\newline
\begin{tikzpicture}
\Tree [ .VP [ .NP \edge[roof]; {wažątirera\\`the car'} ] [ .V$'$ [ .AP [ .Deg\textsubscript{RES}P \edge[roof]; {{$\varnothing$}} ] [ .A šuuc\\`red' ] ] [  .V hogiha\\`paint' ]] ]
\end{tikzpicture}
\end{exe}

Hocąk resultatives are thus obtained by specifying the eventuality of the result to the highest degree. This is consistent with Wechsler's (2005) proposal on the constraints on the result predicate.Wechsler asserts that the result must express a gradable property with a maximum degree, when the object NP is the argument of the matrix verb. I assume that only gradable adjectives can take DegPs in their specifiers, while non-gradable adjectives lack this ability. The degree head is an operator and thus has to bind a variable. If gradable adjectives have a degree argument (or grade-role) in its argument structure, then the degree head will be able to bind it. On the other hand, if non-gradable adjectives lack this degree argument, then the structure will be ruled out since all operators have to bind a variable. Consider the contrast between the gradable adjective \textit{sgįgre} `heavy' in (43a) and the non-gradable adjective \textit{t'ee} `dead' in (43b) with the degree element \textit{eegišge} `too'.

\begin{exe}
\ex
\begin{xlist}

\ex[] {\glll Henryga eegišge sgįgre.\\ 
Henry-ga eegišge {$\varnothing$}-sgįgre\\
Henry-\textsc{prop} too \textsc{3s}-heavy\\
\glt `Henry is too heavy.'}


\ex[*] {\glll Caara eegišge t'ee\\
Caa-ra eegišge {$\varnothing$}-t'ee\\
deer-\textsc{def} too \textsc{3s}-dead\\
\glt Intended: `The deer is too dead.'}

\end{xlist}
\end{exe}

I propose that \textit{eegišge} realizes a Deg head; thus, an example like in (43a) has the structure in (44).\footnote{As noted in footnote 11, it could also be the case that DegP dominates the AP. I suggest that \textit{eegišge} `too' would be in the specifier of a head-final and phonologically null Deg. See \citet{Rosen2015} for more information.}

\begin{exe}
\ex 
{\hspace{1em}}\newline
\begin{tikzpicture}
\Tree [ .AP [ .DegP \edge[roof]; {eegišge}\\`too' ] [ .A  sgįgre\\`heavy' ] ]   
\end{tikzpicture}
\end{exe}

I attribute the ungrammaticality of non-gradable adjectives with \textit{eegišge} `too' to the hypothesis that the degree head associated with eegišge `too' must bind the degree argument of a lexical item. Since non-gradable adjectives in Hocąk do not have degree arguments as part of their lexical entry (cf. Higginbotham 1985; Corver 1997), the degree head does not have a degree argument to bind. This results in ungrammaticality. Under the present analysis, since the result takes a degree phrase in its specifier, it is expected that a non-gradable adjective is not allowed as a result predicate. In the case of the resultative in (41), \textit{t'ee} `dead' is ill-formed because the degree operator in Deg does not have a variable in its scope that it can bind.\footnote{I assume that color adjectives, such as \textit{šuuc} `red', are gradable (Kennedy \& McNally 2010).} Let us return to the fact that verbs are ungrammatical as resultative predicates. Following \citet{Higginbotham1985} and \citet{Corver1997}, I assume that verbs do not have a grade-role; rather they have an event-role. This is evidenced by the ungrammaticality of the verb \textit{šiibre} `fall' with \textit{eegišge} `too' in (45).\footnote{The contrast between (43) and (45) illustrates another way in which stative verbs and adjectives differ. In this paper, I am only concerned with how they differ with respect to resultatives. Rosen (2014; 2015) presents more diagnostics for the existence of adjectives in Hocąk.}

\begin{exe}

\ex[*] {\glll Hunterga eegišge šiibre.\\
Hunter-ga eegišge {$\varnothing$}-šiibre\\
Hunter-\textsc{prop} too \textsc{3s}-fall.\textsc{stat}\\
\glt Intended: `Hunter fell too much/a lot.'}

\end{exe}

I argue that resultative examples like (40) with verbs are ungrammatical since a degree head can measure the state of the adjective, but it cannot link the event of a verb. In other words, example (40) is ruled out because there is a mismatch between the selectional restrictions of DegP and a verb phrase. This explains why verbs are barred from resultatives in Hocąk. In this subsection, we see that verbs cannot appear as the result predicate in Hocąk. The reason that verbs cannot appear as the result, I claim, is that the result predicate takes a special degree phrase that I labeled ``Deg\textsubscript{RES}P'' in its specifier. A straightforward explanation arises if we assume that degree phrases in Hocąk must bind a degree argument. Since I am assuming that verbs lack a degree argument, verbs are not allowed as a result predicate. Thus, I contend that the result predicate in Hocąk is an AP.

\subsection{Implications: Status of Adjectives}

I have presented evidence that the result predicate in Hocąk resultatives projects as an AP (an adjective). This puts resultatives in Hocąk in line with resultative constructions cross-linguistically that use APs as result predicates (cf. English resultatives). Moreover, these data indicate that Hocąk has the lexical category adjective. This is a significant result since Hocąk has been previously described as only having nouns and verbs (see section 2.3). The previous traditional literature (e.g., Helmbrecht 2006) has focused primarily on the morphological similarities between stative verbs and adjectives. The data from resultatives have shown that these similarities can be misleading. Rather adjectives surface in at least one environment in Hocąk; namely, resultatives (see Rosen 2014; 2015 for further discussion of these issues in Hocąk, and Baker 2003 and Dixon 2004 cross-linguistically).

\section{Conclusion}
This paper has offered a description and an analysis of Hocąk resultatives. I have shown that the result predicate must not be a clause and must be in a VP-internal position. I have argued that Hocąk resultatives project a phrasal AP as the complement of the verb in a Larsonian ``VP shell'' (Larson 1988). This proposal is supported by the fact that resultatives in Hocąk have many of the properties that have been attributed to resultatives cross-linguistically, such as in English. In particular, resultatives in Hocąk obey the DOR, and the resultative phrase is adjectival. I conclude with the hope that this paper will continue to improve our understanding of resultatives and the structure of predication in Hocąk and Siouan languages.

\section* {Acknowledgments}
I am extremely grateful to Cecil Garvin for teaching me about his language. Thanks also to Meredith Johnson, Yafei Li, Becky Shields, Catherine Rudin, Rand Valentine, and an anonymous reviewer for valuable discussion and comments, and to Iren Hartmann for allowing me to use her Lexique Pro dictionary. Lastly, material in this paper was presented at SSILA 2014. I would like to thank the audience there.

\section*{Abbreviations}
1, 2, 3 = first, second, third person; \textsc{comp} = complementizer; \textsc{decl} = declarative; \textsc{def} = definite; \textsc{indef} = indefinite; \textsc{o} = object agreement; \textsc{pl} = plural; \textsc{prop} = proper noun; \textsc{refl} = reflexive; \textsc{s} = subject agreement. 

\section*{References}


\printbibliography[heading=subbibliography,notkeyword=this]

\begin{reflist}

Baker, Mark C. 2003. Lexical categories: Verbs, nouns, and adjectives. Cambridge: Cambridge
University Press.

Bhatt, Rajesh, and Roumyana Pancheva. 2004. Late merger of degree clauses. Linguistic Inquiry 35:1-45.

Bowers, John. 1997. A binary analysis of resultatives. Texas Linguistic Forum 38:43?58.

Carrier, Jill, and Janet Randall. 1992. The argument structure and syntactic structure of resultatives. Linguistic Inquiry 23:173-234.

Chomsky, Noam. 1982. Some concepts and consequences of the theory of Government Binding. Cambridge, MA: MIT Press.

Chomsky, Noam. 1995. The minimalist program. Cambridge, MA: MIT Press.

Corver, Norbert. 1997. The internal syntax of the Dutch extended adjectival projection. Natural Language and Linguistic Theory 15:289-368.

Dixon, R.M.W. 2004. Adjective classes in typological perspective. Adjective classes: A cross-linguistic typology, ed. by R.M.W Dixon and Alexandra Y. Aikhenvald, 1-49. Oxford: Oxford University Press.

Helmbrecht, Johannes. 2006. Are there adjectives in Hocąk (Winnebago)? Lexical categories and root classes in Amerindian languages ed. by Ximena Lois and Valentina Vapnarsky, 289?316. Bern: Peter Lang International Academic Publishers.

Higginbotham, James. 1985. On semantics. Linguistic Inquiry 16:547?94.

Jackendoff, Ray. 1977. X-bar syntax. Cambridge, MA: MIT Press.

Johnson, Meredith. 2013. Verb phrase ellipsis and v: Evidence from Hocąk. Paper read at the 87th Annual Linguistic Society of America, Boston, Massachusetts.

Johnson, Meredith, and Bryan Rosen. 2014. Rightward movement and postverbal constituents New evidence from Hocąk. Paper read at the 45th Chicago Linguistics Society, University of Chicago.

Kennedy, Chris, and Louise McNally. 2010. Color, context and compositionality. Synthese 174:79-98.

Larson, Richard K. 1988. On the double object construction. Linguistic Inquiry 19:335-391.

Levin, Beth, and Malka Rappaport Hovav. 1995. Unaccusativity: At the syntax-lexical semantics interface. Cambridge, MA: MIT Press.

Li, Yafei. 1993. Structural head and aspectuality. Language 69:480-504.

Li, Yafei. 1999. Cross-componential causativity. Natural Language \& Linguistic Theory 17:445-497.

Lipkind, William. 1945. Winnebago grammar. New York: King's Crown Press.

Matthewson, Lisa. 2004. On the methodology of semantic fieldwork. International Journal of American Linguistics 70:369-415.

Perlmutter, David. 1978. Impersonal passives and the unaccusative hypothesis. Proceedings of the 4th Berkeley Linguistics Society, pp. 157-190. University of California, Berkeley.

Rosen, Bryan. 2014. On the existence of adjectives in Hocąk. Paper read at the 88th Annual Linguistic Society of America, Minneapolis, Minnesota.

Rosen, Bryan. 2015. The syntax of adjectives in Hocąk. Doctoral dissertation, University of Wisconsin --- Madison.

Song, Hongkyu. 2005. Causatives and resultatives in Korean. Doctoral dissertation, University of Wisconsin --- Madison.

Susman, Amelia. 1943. The accentual system of Winnebago. Doctoral dissertation, Columbia University.

Washio, Ryuichi. 1997. Resultatives, compositionality and language variation. Journal of East Asian Linguistics 6:1-49.

Wechsler, Stephen. 2005. Resultatives under the ``event-argument homomorphism'' model of telicity. In The syntax of aspect, ed. by Nomi Erteschik-Shir and Tova Rapoport, 255-273. Oxford: Oxford University Press.

Williams, Alexander. 2008. Word order in resultatives. Proceedings of the 26th West Coast Conference on Formal Linguistics, ed. by Charles B. Chang and Hannah J. Haynie, 507-515. Somerville, MA: Cascadilla Press.

Williamson, Janis Shirley. 1984. Studies in Lakhota grammar. Doctoral dissertation, University of California, San Diego.

Woolford, Ellen. 2010. Active-stative agreement in Choctaw and Lakota. Revista Virtual de Estudos da Linguagem 8:6-46.

\end{reflist}
\end{document}