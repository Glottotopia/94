\addchap{Introduction to Part III}
\begin{refsection}

Siouan languages have received relatively little attention in general linguistic theory, and so pose a challenge even to theories based on typological generalizations from broad selections of the world's languages. The head-final, partially polysynthetic nature of Siouan languages raises issues for some claims in syntactic theory, and interesting issues arise in phonology as well. This section of the volume comprises five chapters applying formal linguistic theory to problems in the phonological or syntactic structure of a single Siouan language. (Cross-linguistic studies are in the next section, Part IV.)

David Rood (``The phonology of Lakota voiced stops'') reexamines a longstanding problem, the phonological status of voiced stops in Lakota. He proposes a new analysis drawing on autosegments and feature geometry to account for the sonorant-like behavior of /b/ and /g/ in lenition, nasalization, and cluster contexts, and concludes that Lakota is not a voicing language but an aspiration language. 

John Boyle (``The syntax and semantics of internally headed relative clauses in Hidatsa'') analyzes the internally-headed-relative-clause construction in Hidatsa, from both formal syntactic and formal semantic perspectives. Working within the Minimalist framework, he demonstrates that Hidatsa IHRC are nominalized clauses; using Heim's framework, he then presents a formal semantic explanation for the well-known indefiniteness restriction on the heads of IHRC. 

This section of the volume concludes with three interrelated chapters with overlapping authors, all dealing with Ho-Chunk syntax, especially the existence and structure of verb and adjective phrases (VP and AP) in the language.

Meredith Johnson (``A description of verb-phrase ellipsis in Hoc\k{a}k'') demonstrates that Ho-Chunk does have true verb-phrase ellipsis, with cross-linguistically typical characteristics. This argues strongly for the existence of VP in Ho-Chunk.

Bryan Rosen (``On the structure and constituency of Hoc\k{a}k resultatives'') continues the theme of arguing that Ho-Chunk has a full range of syntactic categories, this time including both VP and AP. The claim that Ho-Chunk has adjectives is controversial: nearly all work on Ho-Chunk and other Siouan languages argues or assumes that these words are stative verbs.

Meredith Johnson, Bryan Rosen and Mateja Schuck (``Evidence for a VP constituent in Hoc\k{a}k'') rounds out this part of the volume by cataloguing the arguments for a configurational analysis of Ho-Chunk (and, by extension, other Siouan languages as well). Subjects and objects are shown to behave differently with respect to a number of tests, including scope as well as the elliptical and resultative constructions discussed in the previous two chapters.

 
\end{refsection}

