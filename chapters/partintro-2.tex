\addchap{Introduction to Part II}
\begin{refsection}

%content goes here
Several Siouan languages are without native speakers -- ``extinct'' or ``sleeping'' -- and the rest are endangered. Because of the situation, documentation, retention, revival, reclamation and revitalization are crucial -- both for Native communities who want to use their languages, and for linguists seeking to record usage patterns, establish details of historical change or investigate the limits of human grammatical structures. Non-Native linguists, community members and descendants have tended to approach and answer language description and retention questions in different ways, with different interests. The chapters in this part of the volume illustrate a variety of approaches to language documentation, pedagogy, and resource enhancement.

Linda Cumberland (``In his own words: Robert Rankin recalls his work with the Kaw people and their language'') interviews Rankin about his long association with the Kaw Language Program, his early fieldwork experiences and the importance of language documentation to ensure that future generations have the ability to reconstruct and revive their language even after intergenerational transmission fails. This chapter highlights Bob Rankin's contributions to applied and descriptive linguistics. Kaw people's appreciation is evident in the inscription on his tombstone: ``W\'iblaha\textsuperscript{n} K\'a\textsuperscript{n}ze \'ie sh\'o\textsuperscript{n}sho\textsuperscript{n} ni.'' (`Thanks to you the Kanza language lives on.')

Jimm Goodtracks, Bryan James Gordon, and Saul Schwartz (``Perspectives on Chiwere revitalization'') present a three-pronged, personal account of their individual and collective involvements with the Chiwere (also known as Báxoje Jíwere \~{N}út\^{}achi, or Ioway and Otoe-Missouria) language. In separate, individually-written sections, each author gives a unique view of past, present, and future prospects of Chiwere, drawn from their various roles as Elder, teacher, dictionary-compiler, language-nest participant, researcher, and language-program director. 

Justin T. McBride (``Reconstructing post-verbal negation in Kansa: A pedagogical problem'') addresses a common issue in the teaching of a language which no longer has an active speech community: that of gaps in the recorded data. As the numbers of fluent speakers decline, Siouan languages are increasingly ``housed'' in the inevitably incomplete descriptions of linguists and in non-fluent tribal and descendant communities. McBride uses a variety of tools, including historical data, syntactic theory, and consultation with speakers of closely related languages to reconstruct an appropriate negative conditional form (`wouldn't') in Kansa (Kaw). The combination of formal syntactic argumentation and pedagogical purpose is unusual but productive. 

Jill D. Greer (``Baxoje-Jiwere grammar Sketch'') provides an overview of the grammar of Baxoje-Jiwere (Chiwere). Although necessarily very short, it covers the basic facts of the language, from phonology through morphology, syntax, word coinage and variation indexing gender and dialect. As is typical for a grammar of a Siouan language, the verb receives the most attention, with its multitudinous prefixes, suffixes and clitics. 

\section*{References}


\printbibliography[heading=subbibliography,notkeyword=this]

\begin{reflist}

Cumberland, Linda A. and Robert L. Rankin. 2012. \textit{Kaa\textsuperscript{n}ze Ie wayaje: An annotated dictionary of Kaw (Kanza)}. Kaw City, Ok: Kaw Nation.

Rankin, Robert L. 1989. Kansa (A. Yamamoto, Trans.). In T. Kamei, R. Kono, and E. Chino (Eds.), \textit{The Sanseido encyclopaedia of linguistics (Vol 2): Languages of the world}, pt. 2 (pp. 301-307). Tokyo: Sanseido Press.

\end{reflist}
%\printbibliography[heading=subbibliography]
\end{refsection}

