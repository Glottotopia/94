\documentclass[output=paper]{LSP/langsci}
\author{Ryan M. Kasak}
\title{A distant genetic relationship between Siouan-Catawban and Yuchi}

\abstract{A lack of ancient written records is no impediment to establishing genetic relationships between languages at great time depths. While scholars like Sapir (1929) have proposed genetic groupings based on particular lexical similarities, other scholars have utilized a multifaceted approach to arguing for relatedness by comparing both lexical items and morphological material, given the fact that the latter is less prone to change over time than the former (Goddard 1975; Vajda 2010). This paper assesses Rankin's (1996; 1998) earlier analysis of the plausibility of a relation from common descent between Siouan-Catawban and Yuchi, which is currently considered by most an isolate. By comparing cognates and establishing possible sound correspondences, and by examining the peculiarity of the verbal template with respect to the placement of the first person plural marker vis-\`a-vis the preverb and verb, and the use of nasal ablaut in Yuchi to mark future tense that is similar to \textit{i\textipa{N}}-ablaut in Dakotan languages, this paper builds upon Rankin's original case for a genetic link between Yuchi and Siouan-Catawban. While more constrained in scope than Chafe's (1976) Macro-Siouan proposal, this paper adds to the body of support for Siouan-Catawban and Yuchi sharing a common ancestor.
KEYWORDS: [Siouan, Yuchi, phylogenetics, templatic morphology, comparative method]}

\maketitle

\begin{document}

\section{Introduction}

The absence of pre-Contact written records and the increasing loss of native speakers are major problems in researching linguistic change in North American indigenous languages. Unlike Hittite and Tocharian, whose written records provided major breakthroughs in our current understanding of the spread and evolution of Indo-European languages, no such breakthrough is likely to be found in the archaeological record. Rather, deeper genetic relations among North American languages must be found by sifting through old hymnals, Jesuit memoirs, and page after page of field notes left behind by past researchers.

The purpose of this paper is to re-visit the idea of a plausible common ancestry for the Siouan-Catawban language family and the language isolate Yuchi (a.k.a. Euchee). This line of inquiry grows out of earlier work done by the late Bob Rankin (1996; 1998). Though Yuchi has been grouped with Siouan-Catawban in the past, there is no consensus on its status as a distant relative or simply a language that may have the occasional similarity here and there.

The overarching goal is twofold. Firstly, I wish to summarize the state of linguistic scholarship up to this point for both Siouan-Catawban and Yuchi. In addition to looking at purely linguistic data, I piece together what is known about these language groups to demonstrate that there are significant non-linguistic factors to support the idea that Siouan-Catawban and Yuchi can be related, notably drawing from historical accounts of the proximity of the Yuchi to Siouan and Catawban peoples from the sixteenth century onwards. Secondly, I wish to support the notion that Yuchi is a distant relative of the Siouan-Catawban languages by providing lexical and morphological evidence.

The task of establishing genetic relationships among languages of the New World is not altogether dissimilar from using methods traditionally reserved for languages with robust written histories. Bloomfield (1925) manages to apply the same methods used by Indo-Europeanists to create proto-forms for Central Algonquian languages. By using the comparative method, he is able to convincingly reconstruct numerous proto-forms for seven different groups of Algonquian languages, further cementing the idea that measuring linguistic change in North America is not a lost cause. His work, however, focuses on languages whose relation is readily apparent and well-accepted. A far trickier task is to connect two languages or language families whose relationship is not well-accepted, particularly when dealing with isolate languages. 

Goddard (1975) demonstrates a similarity in morphology between Proto-Algon-quian and the Californian languages Yurok and Wiyot. In addition to morphology, Goddard finds a small group of lexical items and posits a set of plausible reconstructions to explain how such linguistically and geographically divergent languages can ultimately be shown to be quite similar. Today, Yurok and Wiyot are grouped with Algonquian as part of the Algic family, and their status as related languages is readily accepted. Deeper genetic relationships are more easily accountable through the comparison of morphological features than lexical items due to the fact that lexical change is possible at a faster rate than morphological change. The fact that most Balto-Slavic languages have a robust case system despite thousands of years of separation from Proto-Indo-European, or the widespread use of the inherited Proto-Polynesian passive marker *ia in its daughter languages is testament to the more languid pace at which morphology changes. A more recent example of the power of comparative morphology is that of the Den\'e languages of North America and the Yeniseian languages of central Siberia. Despite a potential time depth of 13,000 to 15,000 years, it is still possible to show cognacy between a variety of inflectional and derivational morphology, as well as individual lexical items (Vajda 2010).

Dunn (2009) has even taken this concept to its next logical step by utilizing Bayesian methods of phylogenetic interference to provide evidence of similarity in the morphology of Austronesian and Papuan languages of Island Melanesia. While Dunn's methods are not employed in this work, the idea behind them is the same: can a convincing argument be made for a relationship between two different language groups using morphological evidence despite a lack of lexical similarity? Ultimately, I argue in favor of the morphological similarities found in both Siouan-Catawban and Yuchi to be more than coincidental. Such similarities are most likely due to genetic inheritance.

This paper is organized into six sections. In \sectref{sec:kasak:2}, I summarize the historical scholarship leading up to the classification of the Catawban family and the Siouan family, along with their eventual classification as related language groups. In addition to pointing out the development of the notion of a Siouan-Catawban language family, I explore the work previously conducted on Yuchi. \sectref{sec:kasak:3} highlights past research into the relationship Siouan-Catawban may have with Yuchi, and the critiques of those groupings made by other linguists. 

In \sectref{sec:kasak:4}, I give a list of lexical items that appear to be cognates and posit a probable set of sound correspondences between Siouan-Catawban and Yuchi. \sectref{sec:kasak:5} is dedicated to building on the lack of cognates with morphological cognacy, while \sectref{sec:kasak:6} offers a conclusive summary of the data found within this paper in addition to adding further commentary on some other possible long-distance genetic relationships between Siouan-Catawban and other indigenous North American languages.

\section{Previous scholarship on Siouan-Catawban and Yuchi}

Siouan-Catawban, which is often called simply Siouan, is a language family whose speakers are predominantly found on the Great Plains, though some languages were once spoken around the Ohio River Valley and along the Atlantic and Gulf coasts (Mithun 1999). Catawban split from Siouan proper some time in the distant past, which Rankin (1996) posits is at least 4,000 years before the present.\footnote{For a more detailed phylogenetic look at the relationship between the Siouan languages, see Rankin (2010).}

\subsection{Catawban language family}

Catawban only has two attested members, Woccon and the eponymous Catawba. Woccon went extinct in the early 1700s, and is only known by a list of 143 words printed in 1709. Carter (1980) identifies the first known attempts to link Woccon and Catawba by Western philologists in the early nineteenth century to Adelung \& Vater (1816), who created a side-by-side wordlist to compare the two languages. Gallatin (1836) builds upon the the notion that Woccon and Catawba share a common descent and expands the comparison between the two languages.

\subsection{Siouan language family}

In addition to examining the relationship between Woccon and Catawba, Gallatin (1836) also is credited for being the first scholar to posit a Siouan language family, named after their most well-known members, the Sioux. 

\subsubsection{Early classification of languages within Siouan}

His first classification breaks the Siouan language family into four clades: 1) Ho-Chunk (also called Winnebago); 2) Sioux, Assiniboine, and Cheyenne; 3) Hidatsa (sometimes called Minetare in older literature), Mandan, and Crow; and 4) Osage, Kaw, Quapaw, Iowa, Otoe-Missouria, and Omaha-Ponca (Parks \& Rankin 2001). His classification of Cheyenne as a Siouan language is now outmoded, as Cheyenne is now uncontroversially classified as an Algonquian language.

Gallatin's (1836) identification and classification of Siouan languages serves as the jumping point for all future research into the internal relationships between the Siouan languages. Mandan started out as something of an issue for early scholars, as it bears a strong lexical affinity to the Missouri Valley languages Hidatsa and Crow. However, it clearly was more distinct from these two languages, both lexically and grammatically. Thus, several groupings done by scholars after Gallatin struggled to place Mandan within the language family, alternatively lumping it with Crow and Hidatsa or giving it a phylum of its own. Regrettably, the status of Mandan is still somewhat suspect even today, as Rankin (2010) advocates recognizing Mandan as its own branch of the Siouan family tree due to its distinct morphology.

\subsubsection{Discovery of an additional branch of the Siouan family tree} Hale (1883) breaks new ground by documenting the Tutelo language of Virginia, which bears a strong lexical and morphological affinity to the Siouan languages of the Great Plains. Thus, for the first time, there is conclusive evidence of a more widespread relationship between the Siouan languages and those farther east of the Mississippi. 

Hale further recounts the recent discovery of documents that place peoples bearing Siouan names around the Appalachian mountains from modern-day Virginia to North Carolina. Swiss-American linguist Albert Gatschet identifies the language Biloxi, spoken on the Gulf of Mexico, as a Siouan language (Dorsey 1894). Earlier, Gatschet had visited the Catawbas of South Carolina in 1881 and remarked on how similar this language is to the Siouan languages (Gatschet 1900). 

\subsubsection{A possible eastern origin} The realization that Tutelo and Biloxi had a clear affinity with the Dakotan languages of the Great Plains was a great discovery. However, the question arose of whether there was a migration of some ancestral group from the east to the west or vice versa. Swanton's (1909) subsequent analysis of Ofo (a.k.a. Mosopelea), a now-extinct language recorded in northern Mississippi, helped to paint a picture of westward migration from around the Ohio River Valley. Records from early explorers to the region strengthened his hypothesis. Before Swanton's field work, Ofo was assumed to be a Muskogean language. The intuition that Ofo belonged with the Muskogean languages is in part due to the fact that /f/ is not found in Siouan languages, but common among languages of the western regions of the Deep South (Rankin 2004, Kaufman 2014). 

Hanna (1911) states that the Ofo lived in eight villages between the Muskingum and Scioto rivers, north of the Ohio River. When French explorer Jean-Baptiste-Louis Franquelin explored the vicinity of these eight villages in 1684, he noted that they had all been destroyed. The Ofo had been attacked and driven from their homeland by the Haudenosaunee (a.k.a. Iroquois),\footnote{Thank you to the annonymous reviewer who suggested that I use the endonym ``Haudenosaunee,'' rather than the exonym Iroquois, whose etymology is typically considered to be derogatory (cf. Day 1968).} whose aggression during the Beaver Wars had caused other groups from the eastern Great Lakes region to flee to safer lands in the West (Swanton 1952). Connecting the Ofo with Siouan languages, combined with the anthropological and historical data on the Mosopelea, created a more complete picture of the time frame of when many Siouan languages shifted westward towards the Great Plains or southward towards the Gulf Coast. 

\subsubsection{Support from missionary texts} Koontz (1995a; 1995b) adds support to an eastern origin for an ancestral homeland for Siouan-Catawban speakers being somewhere in the East by classifying the understudied Michigamea language  as Siouan. The Michigamea were a member of the Illinois Confederacy and were thought to have spoken a dialect of Miami-Illinois, whose range stretched from northern Arkansas to near Lake Michigan, though the northernmost extent of their habitation is somewhat in question. 

A Jesuit who visited the Michigamea in 1673 found himself unable to communicate with them in any of the six other languages he spoke, including Illinois. Curiously, the Michigamea were also regarded as go-tos for dealing with the nearby Quapaw tribe, whose language was clearly Siouan. The recorded evidence of Michigamea is scant, but two complete sentences were enough to clearly show that Michigamea is not an Algonquian language, but a Siouan one. Its status within the Siouan language family is not completely understood, but Michigamea shows very strong affinities to the Dhegihan branch (Koontz 1995a; Koontz 1995b).

\subsubsection{Support from historical toponymy} In addition to searching for information in missionary texts, more modern scholarship by Booker, Hudson \& Rankin (1992) examines toponyms and ethnonyms documented by Spanish explorers during three expeditions in the 1500s to corroborate the idea that there were once Siouan peoples along the Eastern seaboard. Hernando de Soto, Trist\'an de Luna y Arellano, and Juan Pardo all led multi-year exploratory missions into the American mainland from Spanish-held territory near modern-day Tampa, Florida. 

The expeditions took place approximately a decade apart from each other, and together covered territory spanning modern-day Florida, Alabama, Tennessee, Georgia, and the Carolinas. Booker, Hudson \& Rankin (1992) outline each of the place names and give the most likely modern analysis for what language group under which to classify them. The names, written in Spanish orthography, give strong clues that the Spaniards had visited a large number of Catawban villages, and possibly one non-Catawban village that of possible Siouan stock: the Chequini. If these people were Siouan, they were likely speakers of a Virginia Siouan language.

When English-speaking settlers began to settle the Atlantic coast of North America in large numbers, there was often little to no trace of the inhabitants described by the Spanish. No doubt, the spread of disease and conflicts among the indigenous groups played an enormous role in the large-scale demographic shifts of the Southeast (Mann 2006). 

\subsubsection{Siouan tribes in the midwest}

The presence of Siouan tribes in the Great Lakes region during the 1600s makes sense when couched in a historical context. That is, around the time of more aggressive European colonization and expansion, the Haudenosaunee and Al\-gon\-qui\-an peoples of the Eastern seaboard became entangled in the Beaver Wars, in which the aforementioned groups vied for dominance of the fur market and trade with Europeans, pushing refugees west over the Mississippi or into the Southeast, displacing other autochthonous peoples. 

Jennings (1990) also places the Lakota near modern-day Chicago near the southern tip of Lake Michigan in 1648, meaning that they had not yet crossed the Mississippi River until some time in the late seventeenth or early eighteenth century. The Manahoac tribe, whose autonym was identical to that of the Tutelo, were likewise driven from the Piedmont Plateau of Virginia by the Haudenosaunee, who claimed their territory as hunting grounds by right of conquest.

\subsubsection{Summary of the eastern origin question} 

The overlap of linguistic, anthropological, and historical data together support the idea that the majority of all Siouan-Catawban peoples resided in or around the Ohio River Valley by the seventeenth century, only to join numerous other tribes in flight before the aggression of the Haudenosaunee.

\subsection{Yuchi language}

Currently, Yuchi is fluently spoken in Oklahoma by a small group of elders, with Linn (2000) stating that their number was around a dozen, though it is likely less now. There are some middle-aged heritage speakers who passively understand the language, but are mostly unable to engage in conversation in Yuchi. The language is considered an isolate, though that idea is called into question in subsequent sections of this paper.

\subsubsection{Early records} 

The first records of the Yuchi place them in the Southeast near the Upper Tennessee Valley during the middle of the sixteenth century (Gatschet 1885). The Chisca are associated in literature with the Yuchi. Hernando de Soto encountered these people on his expedition, and he sent two men to find their villages, as they were reported to have gold. 

Later, in 1567, the Spanish burned down two of their villages after the rumors of the Chisca having gold turned out to be false. There is no convincing argument as to why the Chisca are to be associated with the Yuchi other than the fact that the Cherokee, Yuchi, and Koasati of the area all seem to share several loanwords to support the idea of cultural contact (Booker, Hudson, \& Rankin 1992).

\subsubsection{Removal from the southeast} 

The Yuchi moved south into what is now Alabama, Georgia, and South Carolina during the seventeenth century due to pressure from the migration of the much stronger Cherokee into their ancestral lands (Jackson 2012). The main bulk of the Yuchi people were known to have resided in northern Georgia during the early nineteenth century. The Yuchi have had a long relationship with the neighboring Creek people, having been members with the Creek Confederacy. The greater part of the Yuchi nation was forced into moving to the newly designated Indian Territories out West between 1836 and 1840. The Yuchi settled in what is now Oklahoma alongside the Creek, though some Yuchi left with other Creek allies to go south into Florida, where they were absorbed into the Seminole nation (Mithun 1999).

\subsubsection{Known linguistic work on Yuchi} 

Gatschet's (1885) work among the Yuchi in Oklahoma was the first major effort in the study of the Yuchi language that goes beyond the creation of simple word lists. This effort was followed up some time later by the German-American anthropologist G\"unter Wagner (1934). Wagner's (1934) grammar of Yuchi was the most comprehensive analysis of the language until Mary Linn's (2000) dissertation.

\section{Previous attempts at a Siouan-Yuchi connection}

Given what is now known about the location of ancestral Siouan-Catawban and Yuchi peoples in the early days of European expansion, there are a few major points worth mentioning explicitly: 1) the Yuchi lived in close proximity to Catawban and some Siouan people during the sixteenth century; 2) the Yuchi continued to live in close proximity to the Catawba and to several other Siouan languages well into the eighteenth century; and 3) it is quite possible that the Yuchi and the Siouan-Catawban peoples had lived in close proximity for longer periods of time before the the early exploration of the Southeast and Northeast by European. It is with these facts in mind that I propose the connection between Yuchi and Siouan-Catawban is more than just geographic.

\subsection{Initial suspicions of common ancestry}

Following Gallatin's (1836) grouping of Siouan with Catawban, other scholars began to posit relationships of other languages to Siouan. Latham (1856) first attempted to link Siouan with Haudenosaunee, saying that they appeared to belong ``to some higher class'' that may even include other languages, such as Catawba, Cherokee, Choctaw, and Caddo. Lewis (1871), after becoming interested in the Haudenosaunee following his law school research into treaties with the Cayuga, believed that the Iroquoian languages were offshoots of the Dakotan languages. Both Latham and Lewis, however, based their assumptions on very small word lists, appealing to the idea that both languages were related based on spurious data and broad claims.

\subsection{Sapir's ``Hokan-Siouan''}

The previous attempts to link Siouan and Catawban with other languages, including Yuchi, never produced a satisfactory connection. The question of how indigenous languages were related to each other greatly interested Edward Sapir, who famously lumped Siouan with many other languages into a family called Hokan-Siouan (Sapir 1929). 

Some of the groupings he made were rather spurious, based on very small or suspect sets of data. He mentioned that ``a certain amount of groping in the dark cannot well be avoided in the pioneer stage of such an attempt at this,'' acknowledging the fact that he still had much to flesh out in his explanation for proving genetic relationships between Hokan and other indigenous languages (Sapir 1920: 289). 

This pioneer stage developed into a massive putative phylogeny of North American languages, where the languages of North America are divided into six ``superstocks:'' 1) Eskimo-Aleut, 2) Algonquian-Wakashan, 3) Na-Den\'e, 4) Penutian, 5) Hokan-Siouan, and 6) Uto-Aztecan. Of these six classifications, Hokan-Siouan, later to be called Macro-Siouan, was the amalgamation of several major language families, including Siouan-Catawban, Iroquois, Caddoan, Muskogean, Natchez, Tunica, Yuchi, and several languages of the Southwest (Sapir 1929). 

This concept of a ``Greater'' Siouan language family has waxed and waned, but of all the possible relationships put forth, it seems that Siouan-Yuchi was one of the more accepted relationships (Campbell \& Mithun 1979).

\subsection{A possible link to Proto-Gulf}

In her attempts to make a case for a Gulf language family, Haas (1951, 1952) references the Proto-Siouan reconstruction by Wolff (1950a, 1950b, 1950c, 1950d) and compares two words in Proto-Siouan and Yuchi with her reconstructions for their analogues in Proto-Gulf, noting an interesting correspondence between them. On the basis of two possible proto-forms, Haas compares Wolff's Proto-Siouan construction *am\textpolhook{\'a} `land, earth' and *min\'i `water' with her Proto-Gulf *(ih)a$\gamma$\textipa{\super w}a\~ni($\gamma$a) `land' and *ak{\textipa{\super w}}ini `water.' 

Haas (1951, 1952) notes that *$\gamma$\textipa{\super w} and *k{\textipa{\super w}} correspond to Proto-Siouan *m in both words, and that in both words, the vowel quality following the labialized velar is the same. That vowel, in turn, is followed by a nasal stop. She supposes that the Yuchi word \emph{tse} `water' is likewise analogous to her Proto-Gulf form, suggesting that Yuchi /ts/ originates from an earlier *k\textipa{\super w}, which would make Yuchi \emph{tse} a cognate of *ak{\textipa{\super w}}ini. While she does not overtly say that there must be a connection between Siouan-Yuchi and her Gulf family, she certainly implies that a link is plausible, though quite difficult to prove.\footnote{An anonymous reviewer points out that this similarity may not be due to common genetic descent, but borrowing due to long-term contact. Kaufman (2014) argues that the languages of the Lower Mississippi Valley form a Sprachbund, explaining certain similarities of Ofo and Biloxi to surrounding languages. However, this region is not thought of as a possible Siouan Urheimat (cf. Parks \& Rankin 2001: 104), which either leaves us with a reduced likelihood of these resemblances being purely due to borrowing through long-term contact or signifying that the language groups found in this region have frequently been in and out of contact with each other multiple times during prehistory.}

\subsection{Chafe's ``Macro-Siouan''}

The Hokan-Siouan language family has seen various incarnations in the literature, most notably in Chafe's (1976) scaled-back version of a super-family that includes Siouan, Caddoan, and Iroquioan. Chafe does not claim to have conclusively proven the existence of Macro-Siouan, though he describes his findings as ``tantalizing, if inconclusive.'' His chief argument comes in the form of lexical resemblances shared between the three language families. Campbell (2000) devotes a sizable amount of space to the idea of Macro-Siouan, stating that it has a twenty percent probability, and a seventy-five percent confidence level, though those percentages are not explicitly given concrete metrics. His appraisal of Chafe's work is largely dominated by personal communication from Robert Rankin, who picks apart several lexical items as being false cognates.

In addition to issues with the choice of lexical items, Chafe's (1976) argument for a possible Macro-Siouan lacks any kind of systematic sound correspondences. Campbell (2000) reports that Rankin disagrees with Chafe's assessment of Caddoan preverbs being related the Siouan instrumentals, as the two morphological phenomena are believed to derive from different sources; Siouan instrumental prefixes derive from verbal roots, while Caddoan preverbs derive from incorporated nouns. Explaining the presence of non-verbal preverbal morphology as being derived from the same source is problematic, and Campbell remarks that Chafe is simply trying to connect two items that could have easily evolved independently or could be part of some areal feature. The fact that they could have arisen as areal features is interesting in of itself, as it would point to the Urheimat of each language family being close to one another at the time each language developed.

\subsection{Siouan-Yuchi}

Carter (1980) lists several Yuchi words in his comparison of Woccon and Catawba, showing that the two languages have some small degree of cognacy. Rankin (1996; 1998) remains agnostic on the connection of Siouan to Caddoan and Iroquioan, but also makes the case that Yuchi belongs to the Siouan family. The case for a Siouan-Yuchi connection originates from Sapir (1929), and Haas (1952) notes that Sapir had viewed Siouan and Yuchi as closely related based on a small set of lexemes.

Rankin's (1998) most recent attempt to show a relationship between Yuchi and Siouan largely skips over lexical data and concentrates on establishing a correspondence between the morphology of Siouan and Yuchi. He notes that the Proto-Siouan-Catawban word *ree `go there' and Yuchi \emph{\textbeltl a} `go' bear a strong resemblance, which Kasak (2012; 2013) builds upon by matching the Proto-Siouan-Catawban motion verbs to cognates in Yuchi, to be explained below.

\section{Phono-lexical evidence}

A classic method for arguing for genetic relationships is the establishment of regular sound correspondences between cognates. This section examines the posited reconstructed phonemic inventories of Yuchi and Proto-Siouan. In addition, I posit several regular correspondences between Proto-Siouan and Yuchi, adding Catawba cognates where available. While a complete reconstruction of what a Proto-Siouan-Catawban-Yuchi would look like is not within the scope of this paper, I make the case that at least some correspondences are possible based on the limited set of cognates discovered so far. The Proto-Siouan forms come from the Comparative Siouan Dictionary (Rankin et al 2015).

\subsection{Proto-Siouan sound inventory}

Rankin, Carter, \& Jones (n.d.) posit the following sound inventory for Proto-Siouan:

\begin{table}[h!]
\caption{Consonant inventory for Proto-Siouan} \label{psiinventory}
    \begin{tabular}{llllll} \lsptoprule
    ~                   & Labial    & Dental     & Palatal & Velar    & Glottal \\
   \midrule
  Plosives   & ~                   & ~                   & ~                & ~                   & ~                \\
    preaspirated        & ʰp & ʰt & ~                & ʰk & ~                \\
    postaspirated       & pʰ & tʰ & ~                & kʰ & ~                \\
    glottalized         & p'                  & t'                  & ~                & k'                  & '                \\
    plain               & p                   & t                   & ~                & k                   & ~                \\
    ~                   & ~                   & ~                   & ~                & ~                   & ~                \\
  Fricatives & ~                   & ~                   & ~                & ~                   & ~                \\
    plain               & ~                   & s                   & \v{s}            & x                   & h                \\
    glottalized         & ~                   & s'                  & \v{s}'           & x'                  & ~                \\
    ~                   & ~                   & ~                   & ~                & ~                   & ~                \\
Resonants  & ~                   & ~                   & ~                & ~                   & ~                \\
    sonorant            & w                   & r                   & y                & ~                   & ~                \\
    obstruent           & W                   & R                   & ~                & ~                   & ~                \\ \lspbottomrule
    \end{tabular}
\end{table}

In addition to the consonants listed above in Table \ref{psiinventory}, Proto-Siouan is assumed to have had five oral vowels /a e i o u/ with two contrasting lengths, as well as three nasal vowels /\k{a} \k{i} \k{u}/, which also had a length distinction. Furthermore, Proto-Siouan likely had a pitch accent, marking high versus non-high pitch, and possibly a falling pitch as well. The obstruent resonants are denoted as *W and *R because it is not entirely certain what sounds they might have been, but they both have distinct reflexes in the modern languages.

\subsection{Yuchi sound inventory}

The modern Yuchi language, as described by Linn (2000), carries a much larger consonant inventory than that of Proto-Siouan; see Table \ref{yuchiinventory}.

\begin{table}[h!]
\caption{Yuchi consonant inventory} \label{yuchiinventory}
    \begin{tabular}{llllll}\lsptoprule
    ~                           & Labial    &  Dental      &  Palatal        &  Velar      & Glotta  \\
    \midrule
     Plosives            & ~                   & ~                    & ~                       & ~                   & ~                \\
    postaspirated               & pʰ & tʰ  & ~                       & kʰ & ~                \\
    glottalized                 & p'                  & t'                   & ~                       & k'                  & ~                \\
    plain                       & p                   & t                    & ~                       & k                   & '                \\
    voiced                      & b                   & d                    & ~                       & g                   & ~                \\
    ~                           & ~                   & ~                    & ~                       & ~                   & ~                \\
    Affricates        & ~                   & ~                    & ~                       & ~                   & ~                \\
    postaspirated               & ~                   & tsʰ & \v{c}ʰ & ~                   & ~                \\
    glottalized                 & ~                   & ts'                  & \v{c}'                  & ~                   & ~                \\
    plain                       & ~                   & ts                   & \v{c}                   & ~                   & ~                \\
    voiced                      & ~                   & dz                   & j                  & ~                   & ~                \\
    ~                           & ~                   & ~                    & ~                       & ~                   & ~                \\
    Fricatives         & ~                   & ~                    & ~                       & ~                   & ~                \\
    glottalized                 & f'                  & s'                   & \v{s}'                  & ~                   & ~                \\
    plain                       & f                   & s                    & \v{s}                   & ~                   & h                \\
    ~                           & ~                   & ~                    & ~                       & ~                   & ~                \\
   Lateral Fricatives & ~                   & ~                    & ~                       & ~                   & ~                \\
    glottalized                 & ~                   & '\textbeltl                & ~                       & ~                   & ~                \\
    plain                       & ~                   & \textbeltl               & ~                       & ~                   & ~                \\
    ~                           & ~                   & ~                   & ~                       & ~                   & ~                \\
 Liquids           & ~                   & ~                    & ~                       & ~                   & ~                \\
glottalized                 & ~                   & 'l                   & ~                       & ~                   & ~                \\
plain                       & ~                   & l                    & ~                       & ~                   & ~                \\
 ~                           & ~                   & ~                    & ~                       & ~                   & ~                \\
Nasals            & ~                   & ~                    & ~                       & ~                   & ~                \\
glottalized                 & ~                   & 'n                   & ~                       & ~                   & ~                \\
plain                       & ~                   & n                    & ~                       & ~                   & ~                \\
~                           & ~                   & ~                    & ~                       & ~                   & ~                \\
 Glides           & ~                   & ~                    & ~                       & ~                   & ~                \\
glottalized                 & 'w                  & ~                    & 'y                      & ~                   & ~                \\
plain                       & w                   & ~                    & y                       & ~                   & ~                \\ \lspbottomrule
\end{tabular}
\end{table}

Yuchi features a large consonant inventory. All glottalized obstruents are ejective consonants, while glottalized sonorants are actually pronounced with creaky voice. 

Yuchi's vowel inventory contains three oral front vowels /i e \ae/ and three oral back vowels /a o u/. Yuchi also has a richer nasal vowel system, with at least four phonemic nasal vowels /\k{a} \k{e} \k{i} \k{o}/. Linn (2000) mentions [\k{\ae}], but notes that it is likely an allophone of /\k{e}/. Wagner (1934) did not record a distinct /\ae/, and wherever /\ae/ is found in modern Yuchi, Wagner (1934) had written down /e/, /\k{e}/ or /a/. Since a small number of minimal pairs can be found, Linn (2000: 44) argues /\ae/ is a phoneme.

The inventories of both Proto-Siouan and Yuchi have much overlap, especially with respect to the abundance of postaspirated stops and glottalized stops and fricatives. However, accounting for the richness of the creaky-voiced sonorants is a daunting challenge. Let us begin by examining some potential cognates, and see if the two sound systems can be reconciled.

\subsection{Some cognates}
 
In looking at motion verbs in Proto-Siouan (PSi), Catawba (Cat) and Yuchi (Yu), a great similarity was found (Kasak 2013); Table \ref{motionverbs}.

\begin{table}[h]
\centering
\caption{Verbs of motion in Proto-Siouan, Catawba, and Yuchi} \label{motionverbs}
    \begin{tabular}{llllll}\lsptoprule
   PSi & ~        & Cat  & ~                    & Yu     & ~                  \\ 
\midrule 
*r\'EEh       & `go there'          & d\'aa      & `go there' 			 & \textbeltl a    & `go' (\textsc{prog}) \\
    *kr\'EEh      & `go back there'     & dukr\'aa & `go back there'      & ---         & ---                  \\
    *h\'ii       & `arrive there'      & ---        & ---                    & ---         & ---                  \\
    *k\'ii       & `arrive back there' & ---        & ---                    & ji   & `go' (\textsc{incept})   \\
    *h\'uu       & `come here'         & h\'uu    & `come here'          & ---         & ---                  \\
    *k\'uu       & `come back here'    & dukh\'uu & `come back here'     & g\k{o} & `come'             \\
   *rh\'ii      & `arrive here'       & ---        & ---                    & \textbeltl i    & `arrive'           \\
    *kr\'ii      & `arrive back here'  & ---        & ---                    & ---         & ---                  \\ \lspbottomrule
    \end{tabular}
\end{table}

Wherever PSi *r or *rh appears in Table \ref{motionverbs}, or wherever Catawba [d]$\sim$/r/ appears, Yuchi /\textbeltl/  is found. In addition, PSi *E\footnote{This ablaut vowel sound was likely pronounced [e] in general but could become [a] or [\k{i}] under certain conditioned circumstances. The \emph{Comparative Siouan Dictionary} reconstructs `to go there' as *r\'ee(he), as the ablaut vowel is not posited as a separate phoneme in Rankin, Carter, \& Jones (2015).} becomes /a/ in Catawba and Yuchi. Likewise, there appears to be a correspondence between PSi *k and Yuchi /g/, as well as PSi *uu and Yu /\k{o}/ in *k\'uu $\sim$ /g\k{o}/.

The form of the inceptive form of `to go' in Yuchi /ji/ could stem from frication of the *k with the *i. The affrication of [k] before a front vowel is typologically well-attested in many different language families. Once frication occurred, the onset could have become voiced and the length distinction lost, giving a possible course of change *k\'ii $>$ {\v{c}}\'ii $>$ {\v{c}}i $>$ \emph{ji} `to go.' Palatalization of a /k/ to /\textipa{tS}/ in the environment of a [+high, +front] segment is a typologically robust diachronic phenomenon: e.g., seen in the change from pre-Old English /dre\textipa{N}k+j+an/ `to cause to drink' > [dren\textipa{tS}an] > drench. Similarly, word-initial voicing was found in earlier forms of English, which is the cause for pairs like \textit{fox::vixen}, and is still a distinctive feature of certain varieties of West Country English. This initial voicing lines up with the potential initial voicing in the change from *k\'uu to /g\k{o}/. In both Proto-Siouan and Yuchi, there is an obvious correspondence between *ii and /i/ in Table \ref{motionverbs}. See \S4.4 for additional *ii and /i/ correspondences.

While not earth-shattering, the fact that all three language groups more or less appear to have retained a set of motion verbs with extremely similar semantics is suggestive of a more than casual relationship. Since these data alone are unlikely to sway anyone, additional correspondences are needed.

\subsection{Correspondence with Proto-Siouan *ii}

As shown earlier, the Proto-Siouan verbs *k\'ii `arrive back there' and *rh\'ii `arrive here' appear to have cognates in Yuchi: \emph{ji} and \emph{\textbeltl i}. A few additional examples of *ii to /i/ correspondence appear below in Table \ref{*ii}:

\begin{table}[h]
\footnotesize
\centering
\caption{Correspondences between PSi *ii and Yuchi /i/}\label{*ii}
	\begin{tabular}{llllll}\lsptoprule
	Proto-Siouan 			&	~	&	Catawba	&	~			&	Yuchi		&	~	\\
\midrule
	*s\'\i i(-re)				&	`yellow'	&	siri		&	`clear (as water)'	&	ti			&	`yellow'\\
	*(wa-)'\'\i i(-re)	&	`blood'		&	iit		&	`blood'					&	we'i		&	`blood'\\ 
	*aʰp\'\i i				&	`liver'		&	hip\'\i iy\k{a}		&	`his liver'					&	y'\k{o}pi\v{c}{ʰ}i		&	`liver'\\ 
	*k\'\i i & `arrive back there' & --- & --- & ji & `go' (\textsc{incept})\\
	*rh\'\i i & `arrive here' & --- & --- & \textbeltl i    & `arrive'  \\
\lspbottomrule	\end{tabular}
\end{table}

With the examples from Table \ref{motionverbs} and Table \ref{*ii}, there are a total of five cognates with *ii to /i/. Seeing as how Yuchi lacks phonemic long vowels, it is unsurprising that any long vowels in Proto-Siouan would correspond to short vowels in Yuchi.  

\subsection{Correspondence with Proto-Siouan *y}

Carter (1980) suggests that there is a relationship between PSi and Proto-Catawban (PCa) *y and /'y/ and /\v{s}/ in Yuchi, along with additional cognates in Lakota (Lak) and Biloxi (Bil), as seen in Table \ref{*y}:\footnote{Carter (1980) does not distinguish Yuchi /y/ from /'y/. All his data are represented as /'y/ under Linn's (2000) analysis.}

\begin{sidewaystable}
\caption{Correspondences between PSi *y and Yuchi /'y/ and /\v{s}/}\label{*y}
\begin{tabular}{p{1.4cm}p{1.8cm}p{.5cm}lllp{2.4cm}p{1.4cm}}\lsptoprule
    Woccon & ~             & Catawba & ~             &  Yuchi & ~             &  Siouan          & ~                              \\
\midrule    
yonne           & `tree'        & yana             & `tree'        & 'ya            & `tree'        & *wi-y\k{\'a}\k{a} (PSi)                                     & `tree'                   \\
    yau             & `fire'        & ya               & 'fire'        & 'yati          & `fire'        & \v{c}ʰ\k{a}ka (Lak)                & `match'             \\
    yah-testea      & `black, blue' & ya\v{c}i         & `ashes'       & 'ya\v{s}e      & `ashes, coal' & \v{c}ʰaxota (Lak)& `ashes'\\
    yau-huk         & `snake'       & ya               & `snake'       & \v{s}a         & `snake'       & *yeka (PSi) & `leg(?)'                    \\
    yauh            & `road'        & y\k{a}        & `road'        & 'yu\v{s}t'\ae  & `road(?)'       & \v{c}ʰ\k{a}ku (Lak)& `road' \\
    ---             & ---           & -yo              & `flesh, meat' & \v{s}o         & `body, waist  & *i-y\'oo (PSi) & `flesh' \\
---&---&\v{c}api&`beaver'&\v{s}apa&`fox'&*wi-y\'aape (PSi)&`beaver'\\
\lspbottomrule
    \end{tabular}
\end{sidewaystable}

PSi *y has a reflex of /\v{c}ʰ/ in Lakota, while having reflexes of either /'y/ or /\v{s}/ in Yuchi. For `flesh' and `beaver' on Table \ref{*y}, it is possible that Yuchi /\v{s}/ occurs instead of /'y/ due to the fact that Proto-Siouan has *y in an intervocalic environment. This could have given rise to frication. However, the word for `tree' likewise has a *y in the environment of two vowels, so if the initial hypothesis about *y > \v{s} when in an intervocalic environment is correct, that would mean that the *wi- prefix was lost for Proto-Yuchi before the *y > \v{s} sound change took place. Otherwise, some other factor could be at work, such as a partial sound change that only affected a certain set of lexical items in Yuchi. A third possibility is that some or all of these items could be due to borrowings, and a fourth is coincidence. The underlying theme of this paper investigates a genetic connection between Yuchi and Siouan-Catawban. It is true that there is the possibility that some of these cognate sets could be due to borrowing rather than genetics, but as seen in some of the data above, and certainly more below, there are some very basic lexical items that one expects to have a low instance of borrowing: e.g., numerals, organs, highly-functional non-lexical verbs like `to be,' etc. 

One additional Proto-Siouan form was added, *yeka `leg, thigh.' While this was not included in Carter's (1980) original list of cognates, it would be consistent with the correspondences seen previously from the motion verbs, where PSi *e can correlate to Catawban *a and Yuchi /a/. Furthermore, the presence of a velar in the second syllable of the Woccon word \emph{yau-huk} `leg' might lend some support to the relatedness of *yeka to Woccon \emph{yau-huk} and Yuchi \emph{\v{s}a}. I have also replaced Carter's original correspondence between Siouan-Catawban *y and Yuchi /\v{s}/ by swapping out \emph{\v{s}ag\k{e}} `beaver' with \emph{\v{s}apa} `fox,' since \emph{\v{s}apa} is a clearer lexical cognate.

\subsection{Correspondence with Proto-Siouan *uu and *\k{u}\k{u}}

Given the possible time depth between Modern Yuchi and the Siouan-Catawban languages, it is understandably difficult to find large sets of data that demonstrate relatedness. As shown in Table \ref{motionverbs}, *k\'uu 'come back here' and Yuchi /g\k{o}/ appear to be cognates. If this is so, then we would expect to see other lexemes where *uu is realized as /\k{o}/. In seeking out additional cognates, another correspondence between Yuchi /\k{o}/ and Proto-Siouan arises. In particular, PSi *\k{u}\k{u} appears to be associated with /\k{o}/, as shown in the data below in Table \ref{*uu}:

\begin{table}[h]
\centering
\footnotesize
\caption{Correspondences between PSi *uu and *\k{u}\k{u} and Yuchi /\k{o}/}\label{*uu}
	\begin{tabular}{llllll}\lsptoprule
	Proto-Siouan 							&	~					&	Catawba	&	~							&	Yuchi						&	~	\\
\midrule	*k\'uu			&	`come back here'			&	dukh\'uu		&	`come back here'					&	g\k{o}	&	`come'\\
	*\k{\'u}\k{u}ke				&	`hand'			&	iksa		&	`hand'					&	(di')\k{o}kʰi	&	`(my) arm'\\
	*'\k{\'u}\k{u}				&	`be, do'		&	---		&	---						&	'\k{o}				&	`be'\\ 
	*r\k{\'u}\k{u}pa			&	`two'				&	n\k{a}pre		&	`two'			&	n\k{o}w\k{e}	&	`two'\\ \lspbottomrule
	\end{tabular}
\end{table}

The Yuchi word \emph{(di)'\k{o}khi} `(my) arm' resembles PSi *\k{\'u}\k{u}ke. This is a likely cognate if the *uu merged with *\k{u}\k{u}, and then lost its length distinction and lowered to /\k{o}/. A common process shown on Tables \ref{*ii} and \ref{*uu} is that long vowels in Proto-Siouan invariably shorten in Yuchi.

While the first two items on Table \ref{*uu} could conceivably come through borrowing due to contact, the bottom two items are less likely. The likelihood of borrowing a verb like `to be' or a numeral are much lower than borrowing a lexical verb like `come.' Body parts are also less likely to be borrowed, but not outside the realm of possibility. Cognates among these kinds of words is evidence in favor of inheritance through common genetic descent instead of borrowing through contact.

\subsection{Miscellaneous cognates}

In the interest of time and space, any other cognates that do not neatly fit into a specific sound correspondence appear below in Table \ref{cognates}. The data that follow combines suggested cognates from the \emph{Comparative Siouan Dictionary} (Carter et al. 2015). I have also added a possible cognate from Mandan \emph{ke'm\k{i}'} `vomit,' which shares a remarkably similar shape to the Yuchi \emph{k'w\k{\ae}} `vomit.' To the best of my knowledge, no other connections have been made in other Siouan languages to this Mandan word, and as such, no Proto-Siouan form is immediately possible. Any false assumptions and leaping to conclusions are my fault alone.

\begin{table}[h]
\footnotesize
\centering
\caption{Miscellaneous Siouan-Yuchi cognates} \label{cognates}
	\begin{tabular}{llllll}\lsptoprule
	Proto-Siouan 						&	~						&	Catawba	&	~							&	Yuchi						&	~	\\
\midrule
	*Wa-										&	`by cutting'		&	---		&	---						&	pʰa	&	`cut'\\
	*ʰp\k{\'a}\k{a}he			&	`bag, sack'		&	p\k{a}'		&	`hold, 		&	p'\k{e}				&	`grip, \\ 
	& & & contain' & & sqeeze' \\
	*p\k{\'a}he						&	`call, shout'		&	w\k{o}\k{o}		&	`cry out'		&	p'\ae			&	`call for'\\ 
	*pʰ\k{u}						&	`nostril'		&	hip\k{\'\i}suu'		&	`his nose'		&	d\k{a}p'i		&	`nose'\\ 
	*k'\k{\'\i}(-re)					&	`carry on back'	&	kida		&	`carry 		&	k'\k{o}				&	`carry'\\ 
	& & & and go' & & \\
	ke'm\k{i}' (Mandan)					&	`vomit'	&	---		&	---		&	k'w\k{\ae}			&	`vomit'\\ 
	*wiʰt\'e					&	`bison, cattle'	&	wid\'ee		&	`bison'		&	wedi				&	`cow'\\ 
	*{\'\i}te					&	`face'	&	neen		&	`face'		&	da				&	`face'\\ 
	*t\'aati					&	`father'	&	nane		&	`father'		&	t'\k{e}				&	`father'\\ 
	*ki-si					&	`good, heal'	&	---		&	---		&	's\k{e}				&	`good'\\ 
	*is\'aapE					&	`black'	&	---		&	---		&	'ispi				&	`black'\\ 
	*i\v{s}\'aapE					&	`dark'	&	---		&	---		&	'i\v{s}pi				&	`dirty'\\ 
	*riih-\v{s}\'\i					&	`dance'	&	bari		&	`dance'		&	\v{s}t\k{i}\v{c}i			&	`dance'\\ 
	*(i-)\v{s}\'\i ipe					&	`intestines'	&	---		&	---		&	\v{c}ʰi			&	`guts'\\ 
 	*waR\'oo					&	`potato, ground nut'	&	witakii		&	`potato'		&	tʰo(bi\textbeltl o)				&	`potato'\\ 
	*\v{s}i-(r)-\k{\'a}te		&	`knee, lap'		&	---		&	---		&	\v{s}'\k{\ae}tʰo		&	`knee, lap'\\ 
	*(wa-)\k{\'\i}\k{i}(-re)			&	`rock'	&	\k{i}\k{i}ti		&	`stone'		&	ti			&	`rock'\\ 
	*r\k{\'\i}\k{i}-ha(-he)			&	`breathe'	&	---		&	---		&	dih\ae 'e		&	`my breath, \\ 
	& & & & & life' \\
	*w\k{i}he							& `female'				&	\k{\'\i}\k{i}ya & `woman'		&	w\k{\ae}	& `woman, \\
	& & & & & female' \\
*h\k{\'a}\k{a}		& `night, darkness'			&	--- & ---		&	f'\k{a}	& `night'\\ \lspbottomrule
	\end{tabular}
\end{table}

Each of the first four items in Table \ref{cognates} contains a bilabial consonant in Proto-Siouan, Catawba, and Yuchi. Three out of four of these correspond to /p'/ in Yuchi with the Siouan counterparts all have word-initial /ʰp/, /p/, and /pʰ/. It is possible that all three of these sounds collapsed into /p'/ in Yuchi, with the plain *p corresponding to Yuchi /p/ intervocalically and in consonant clusters. Additional investigation into other cognates is needed before any kind of regular sound change can be posited.

\subsection{Summary of phono-lexical evidence}

There are several items that clearly look to be common to all three language groups found within the data, such as *wiʰt\'e$\sim$\emph{wid\'ee}$\sim$\emph{wedi} `bison, cattle.' A small set of sound correspondences can be argued. Namely, *r and *rh can map to /\textbeltl/, *y can map to /'y/ and /\v{s}/, *ii maps to /i/, and *uu and *\k{u}\k{u} map to /\k{o}/. 

Further study of older vocabulary from Wagner's (1931) texts may shed additional light on additional cognates. The lexical items used thus far are from Linn (2000) and Carter et al. (n.d.), it is extremely difficult to find an adequate list of cognates, much less postulate sound correspondences that might lead to the recreation of hypothetical Proto-Siouan-Yuchi forms. This difficulty arises from Linn's pervasive use of paradigms to illustrate Yuchi morphophonology, and as such, it is not in itself as rich a source of lexical data as a dictionary might be. Additional work is certainly needed in this area. However, the fact that even some regular sound changes can be discerned, as well as the sharing of certain high-frequency lexical items like body parts, numerals, copulas, and verbs of motion, provide evidence of connection that is not inherently one of language contact.

\section{Morphological evidence}

Lexicostatistics can only carry one so far before running into a dead end, in addition to being controversial in itself. Regardless of whatever half life a word may have in a language, a language's morphology is much more resistant to decay. The fact that the humble English word \textit{am} can trace its roots directly back to Proto-Indo-European *esmi- shows that morphology has the kind of staying power that open category words simply cannot match.

To date, the most prolific and explicit analysis of Sapir's (1929) idea of a special relationship between Siouan and Yuchi has been undertaken by Robert Rankin (1996, 1998). Through side-by-side comparisons of particular affixes and verbal paradigms, Rankin carefully argues that while certain lexical similarities may be chalked up to borrowings, it is difficult for certain morphological idiosyncrasies to be wholesale borrowed as well. Much of his treatment of the issue assumes the possibility of cognacy through genetic inheritance rather than borrowing, a treatment I have adopted throughout this paper.\footnote{There is the possibility for language contact to have influenced the presence of cognates across these language families. Given that this paper does not purport to be an exhaustive treatment of all scenarios under which a particular lexical item can have passed into one language to or from another, I propose cognacy through common descent as my main hypothesis. Fleshing out individual borrowings and determining the directionality of their borrowings is outside the scope of this paper and remains the subject of further study.}

His argument rests on four key points. He shows that there is a strong similarity in the classificatory systems of Proto-Siouan, Catawba, and Yuchi, as well as in their pronominal prefix morphology. That prefix morphology also has a marked interaction with preverbs for first person plural forms. A somewhat lesser but noteworthy point is that Siouan and Yuchi both feature fricative sound symbolism to add gradience to a verb.

\subsection{Siouan, Catawban, and Yuchi classifiers}

One particularly productive affix in Siouan languages comes from PSi *ko-, which is found on kinship terms that are possessive. The examples in Table \ref{siouanclassifiers} are from Mandan (MA), Tutelo (TU), Dakota (DA), Ofo (OF), Dhegiha (DH), and Ho-Chunk (HC).

\begin{table}[h!]
\centering
\caption{Siouan personal classifiers} \label{siouanclassifiers}
    \begin{tabular}{llll}\lsptoprule
        \textbf{ko}\v{s}\k{u}ka & `his younger brother' (MA) & \textbf{koo}mih\k{a}\k{a} & `girl' (TU) \\ 
        s\k{u}ka\textbf{ku} & `younger brother' (DA) & h\k{\'u}\textbf{ku} & `his mother' (DA) \\ 
         \textbf{hk\'o}ra & `friend' (DH) & hi\v{c}a-\textbf{k\'o}ro & `friend' (HC) \\ 
		\textbf{ko}s\'\i ke & `woman's brother-in-law' (MA) & \textipa{@kifh}\k{u}t\textbf{ku} & `little brother' (OF)\\
		& & ee-\textbf{ko}weei & `their chief' (TU)\\
\lspbottomrule
    \end{tabular}
\end{table}

Mandan appears to have the most robust use of *ko-, using it as a third-person possesion marker for family members. In several other languages, as seen above, *ko- seems to have simply melded onto the noun. Compare PSi *ko- with \emph{ku-}, a prefix in Catawba that has a similar distribution. Like PSi *ko-, Catawba \emph{ku}- appears as part of words denoting people. Table \ref{siouanclassifiers}, as well as Tables \ref{catawbaclassifiers} through \ref{yuchinonpersonal} on the following pages are adapted from Rankin (1998).

\begin{table}[h!]
\centering
\caption{Catawba personal classifiers} \label{catawbaclassifiers}
    \begin{tabular}{llll}\lsptoprule
        \textbf{ku}rii & `son' & \textbf{ku}koo & `girl' \\ 
        \textbf{ko}t\'one & `host' & \textbf{k\textipa{@}}neyana & `his father' \\ 
			{ya \textbf{ku}re nan\'ewa} & `her father' & {\textbf{ka}tiy\'\i ise} & `youngest son' \\
\lspbottomrule
    \end{tabular}
\end{table}

\begin{table}[h!]
\centering
\caption{Yuchi personal classifiers}
    \begin{tabular}{llll}\lsptoprule
    
        \textbf{go}laha & `grandmother' & \textbf{go}ji\textipa{\textbeltl}\k{\ae} & `giant' \\ 
        \textbf{go}tan\'e & `brother' & \textbf{go}k'al\'a & `relatives' \\ 
        \textbf{go}'\k{e} & `baby' & \textbf{go}t'e & `husband, man' \\ \lspbottomrule
    \end{tabular}
\end{table}

Yuchi \textit{go-} is described as being a human-specific prefix, often used for people who are Yuchi, with a different marker used for non-Yuchi. Proto-Siouan similarly has analogous prefixes, as shown above. The \emph{go}- prefix is extremely productive in Yuchi. There is a different third-person singular marker on verbs that is homophonous that marks an impersonal subject, or a subject who is not Yuchi. It is quite possible that the forms are related.

A second classifier is PSi *wi-, which marks animals, food, and items in nature, as shown on Table 11. An identical distribution can be found in Catawba, as seen in Table 12, where morphology bearing the shape /wi-$\sim$w\k{i}-/ appears on animate nouns, certain foods, and natural phenomena. The overlap in both the phonological shape and semantics of PSi *wi- and Catawba /wi-$\sim$w\k{i}-/ strongly suggest that they are inherited from a common ancestor.

\begin{table}[h!]
\centering
\caption{Siouan non-personal classifiers}
    \begin{tabular}{llll}\lsptoprule
    
        *\textbf{wi}y\'aape & `beaver' & *\textbf{wi}ʰt\'e & `bison' \\ 
        *\textbf{wi}\v{s}\k{\'u}ke & `dog' & *\textbf{wi}ʰt\'oxka & `fox' \\ 
        *\textbf{wi}r\'aa & `fire' & *\textbf{wi}r\k{\'\i} & `water' \\\lspbottomrule
    \end{tabular}
\end{table}

\begin{table}[h]
\centering
\caption{Catawba non-personal classifiers}
    \begin{tabular}{llll}\lsptoprule
  
        *\textbf{wi}d\'ee & `bison' & *\textbf{wi}mba & `barred owl' \\ 
        *\textbf{wi}tka & `owl' & *\textbf{w\'\i}dyu & `meat' \\ 
        *\textbf{w\k{i}}t\k{a}' & `rat' & *\textbf{wi}\v{c}awa & `night' \\\lspbottomrule
    \end{tabular}
\end{table}

Both Proto-Siouan and Catawba have some reflex of /wi-/ as their non-personal classifier. Catawba and Proto-Siouan both use *wi- not only to mark animals, but also common foods and weather- and nature-related words. Yuchi, on the other hand, has \emph{we-}:

\begin{table}[h!]
\centering
\caption{Yuchi non-personal classifiers} \label{yuchinonpersonal}
    \begin{tabular}{llll}\lsptoprule
    
        \textbf{we}di & `cow' & \textbf{we}'ya & `deer' \\ 
        \textbf{we}\textbeltl a & `hawk' & \textbf{we}\v{s}i & `sofki, soup' \\ 
        \textbf{we}\v{c}ʰ\k{\ae}'l\k{\ae} & `lightening' & \textbf{we}t'\ae & `rainbow' \\\lspbottomrule
    \end{tabular}
\end{table}

Rankin (1998) notes that while the overlap of these two classificatory affixes is striking, it is conceivable that a classifier system that is two-pronged in nature could have been borrowed. Additional support that this bit of morphology is likely via common genetic descent appears below.

While these similarities in nominal affixation may be chalked up to coincidence, further investigation is needed into verbal and other morphology to determine if any deeper connection between Siouan and Yuchi is plausible.

\subsection{Pronominal morphology}

All three languages at hand are active-stative languages with SOV word order. Often, overt subjects are omitted if clear in the discourse or from the verb. This is easier to do in Yuchi, as Yuchi has a rather robust system of third-person marking, which stands in stark contrast to the zero-marking of third-person that Siouan and Catawban languages generally have.

Rankin (1996, 1998) posits that any non-Siouan language whose pronominal morphology is similar and has comparable idiosyncrasies is a likely candidate to be related to Siouan-Catawban. Unlike individual words, morphology is not so easily borrowed. In Proto-Siouan, there are four reconstructed person markers: first person singular, second person, inclusive first person plural, and exclusive first person plural. Rankin (1998) offers up the following set of correspondences, providing two different pronominal series for Yuchi: a \emph{di-} series for verbs referring to activities, processes, or motion (Linn 2000: 130), and a \emph{do}- series for transitive verbs with specific objects (Linn 2000: 178). Table \ref{allprefixes} below lists subject prefixes in Proto-Siouan, Catawba, and Yuchi to illustrate the similarity between all three languages, particularly between Catawba and Yuchi.


\begin{table}[h]
\centering
\caption{Proto-Siouan, Catawba, and Yuchi subject prefixes}\label{allprefixes}
    \begin{tabular}{lllll}\lsptoprule
    ~                  & PSi   & Catawba &  \multicolumn{2}{c}{Yuchi}  ~              \\
   \hline
    {\textsc1sg}         & *wa-           & \textbf{dV-}              & \textbf{di-}            & \textbf{do-}            \\
    {\textsc2sg}         & \textbf{*ya-}           & \textbf{ya-}              & ne-        & \textbf{yo-}            \\
    {\textsc3sg}          & *$\varnothing$- & \textbf{hi-}   & \textbf{h\k{e}-}/se-/we- & h\k{o}-/syo-/y\k{o}- \\
    {\textsc1pl.incl} & \textbf{*'\k{u}-}       & ha-              & \textbf{'\k{o}-}        & \textbf{'\k{o}-}        \\
    {\textsc1pl.excl} & \textbf{*r\k{u}-}       & ha-              & \textbf{n\k{o}-}        & \textbf{n\k{o}-}        \\
    {\textsc2pl} & *ya-       & wa-              & 'ane-        & '\k{a}yo-         \\
    {\textsc3pl} & *$\varnothing$-       & a-/\textbf{i-}              & h\k{o}-/'o-/\textbf{'i-}/we-        & 'h\k{o}-/'o-/'y\k{o}-         \\\lspbottomrule
    \end{tabular}
\end{table}

At first glance, the similarity between all three languages is not close enough to conclusively demonstrate genetic affiliation. The second person singular seems to be the most promising, with all three languages beginning their second person with a [y] sound. Rankin (1998) points out in the endnotes that there is some variation between *r and *y in second person verbal paradigms in some languages, which could be the reasoning behind the variation between /n/ and /y/ in the second person Yuchi pronominals.

The Proto-Siouan and Yuchi first person plural markers are extremely similar, with *r usually being realized as [n] when followed by a nasal. In addition, the correspondence between *\k{u} and /k{o}/ goes along with the correspondence previously done for *uu and *\k{u}\k{u} and /\k{o}/ in Yuchi. Catawba also has an independent first person pronoun \emph{inu}, which seems analogous to the first person plural, as well as having the first person plural object marker \emph{n\k{u}}- (Voorhis 1984). The only other evidence for an exclusive pronoun in Siouan comes from Mandan, which has \emph{r\k{u}}- as its sole first person plural marker. 

The first person singular for Catawba and Yuchi is extremely similar. Often, [d] in Catawba is the word-initial allophone for /r/, with [r] surfacing intervocalically (Rudes 2007). The environments in which Yuchi [d] surfaces are unclear; Gatschet (1885), Wagner (1934), and Linn (2000) do not spend substantial portions of their works dealing with variation and allophony, preferring to dive right into the meat of morphology. If Catawba /d/ does indeed stem from an underlying /r/, then it would mean some opaque path to explain how Proto-Siouan wound up with *w- for first person, while the other two languages had [d-]. 

Rankin (1998) wonders if some ancient system of allophony could be at work here, but it is entirely possible that all three languages started out with the same segment, and there has been a radical change in one or more groups. One needs only to look at the famous example of PIE *dw- $>$ Armenian \emph{erk}- to be reminded that sound changes can move very far from their original source, given enough time and innovations. It is also possible that all three languages display a reflex of some earlier sound no longer reflected. I am tempted to think about Proto-Algonquian *n- for first person, and how few steps one would need to take to turn [n] into [d], or [n] into [m] and then [w]. While I am not suggesting that is what happened, I am noting that such a particular series of sound changes would not be completely beyond the realm of possibility.

Rankin (1998) does not address third person marking or second person plural marking.\footnote{Yuchi has a wider array of third person markers, depending on whether referring to the gender of the referent and whether the referent is a Yuchi or not. The three-way split for third person on Table \ref{allprefixes} refers to male Yuchi subjects, female Yuchi subjects, and non-Yuchi subjects.} Using Rudes' (2007) description of the Catawba verbal template along with Linn's (2000) Yuchi grammar, we can add to previous analyses, which does add further evidence of a closer affinity between Catawba and Yuchi than to Proto-Siouan, namely in the third person singular, where Catawba has \emph{hi-}, while masculine third person subjects in Yuchi are marked with \emph{h\k{e}}. Additional evidence of similarity in the third person marking of Catawba and Yuchi is in Linn's (2000: 133) description of an alternate third person plural marker \emph{'i}-, which is identical to one of the Catawba allomorphs for third person plural subjects. There seems to be no clear candidate for a shared second person plural marker among the languages above. This point notwithstanding, Catawba and Yuchi appear to have cognate first person singular and third person plural marking, while Yuchi and Proto-Siouan share first person plural marking for both inclusive and exclusive. Second person singular marking is shared across all three groups, and the Yuchi \emph{ne-} may also be shared with Proto-Siouan second person patient marker *y\k{i}-, given that there could have been nasal assimilation onto the *y where *y > n /\underline{~~~} [+nasal]. We see this process yield Mandan and Ho-Chunk \emph{n\k{i}}- and Lakota \emph{ni-}, which are quite similar in shape to Yuchi \emph{ne-}.

\subsection{Templatic morphology}

Both Siouan-Catawban and Yuchi follow a templatic system of morphology. Most notably, Siouan and Catawban first person and second person pronominals tend to get trapped between the verb and preverbal morpohology, while the first person plural is to the left of any other morphology, as the data below on Tables \ref{say} and \ref{yuchipreverbs} from Rankin (1998) show:

\begin{table}[h]
\centering
\caption{Pronominals within the verb `to say' in Siouan} \label{say}
    \begin{tabular}{llll}\lsptoprule
    ~          & PSi          & Dakota  & Mandan     \\
  \hline  
    1.\textsc{sg} & '\'ee\textbf{p}he      & \'e\textbf{p}ha       & \'ee\textbf{p}e'\v{s}    \\
    2.\textsc{sg} & '\'ee\textbf{\v{s}}e   & \'eha                 & \'ee\textbf{t}e'\v{s}    \\
    3.\textsc{sg} & \'eehe                 & \'eya                 & \'eehero'\v{s}           \\
    1.\textsc{pl}     & \textbf{'\k{u}}'\'eehe & \textbf{'\k{u}}'\'eya & \textbf{r}\'eehero'\v{s} \\\lspbottomrule
    \end{tabular}
\end{table}

The verb above demonstrates the morphologically marked behavior of the preverb *\'ee with respect to pronominals. Only the first-person plural can appear before the preverb. Every other pronominal is trapped between the preverb and the verb, as demonstrated clearly in Proto-Siouan and Mandan. Mandan, in particular, is prone to creating portmanteaux with the first person plural marker and the preverb (Hollow 1970). This same process of portmanteau creation can be seen at work in Yuchi, as seen below.

\begin{table}[h]
\centering
\caption{Pronominals within the verb with Yuchi Preverbs}\label{yuchipreverbs}
    \begin{tabular}{lll}\lsptoprule
    {\textsc1sg}     & k'\k{a}\textbf{d}a           & hi\textbf{\v{c}}a  \\
    ~              & `I carry something'          & `I find something'      \\
    {\textsc2sg}     & k'\k{a}\textbf{\v{s}}a       & hi\textbf{\v{s}}a  \\
    ~              & `you carry something'        & `you find something'    \\
   {\textsc1pl.incl} & \textbf{'\k{o}}k'\k{a}\textbeltl a & \textbf{'\k{\'e}}\textbeltl a \\
    ~              & `we carry something'         & `we find something'     \\
   {\textsc1pl.excl} & \textbf{'\k{no}}k'\k{a}\textbeltl a & \textbf{n\k{\'e}}\textbeltl a \\
    ~              & `we carry something'         & `we find something'     \\\lspbottomrule
    \end{tabular}
\end{table}

Siouan has four main preverbs: *\'\i i-, *\'aa-, *\'oo-, and *\'ee-. The preverbs all attract primary stress or high tone, depending on the language. This prominence attraction is also found in the Yuchi instrumental preverb \emph{hi}-. Linn (2000) argues that \emph{hi}- is simply a third-person non-agent marker, but her analysis does not explain how it is that such a preverb could have the same phonetic idiosyncrasy of high tone attraction in addition to causing the first-person plural marker to move out of the expected spot next to the verb and migrate all the way to the left of the verb.\footnote{The presence of high tone on these preverb-like elements in Yuchi is reminiscent of the placement of high tone in Mandan when a word contains a preverb. If \emph{hi-} truly is a third person singular marker of some sort, then it bears even more resemblance to Catawba, which also has \emph{hi-}, but for third person singular subjects. If \emph{hi-} is related to the Proto-Siouan preverbs, perhaps it is related to *\'ii-.} 
The exact same scenario appears for the preverbal element \emph{k'\k{a}-} `something.' For both \emph{k'\k{a}}- and \emph{hi-} to yield this marked position for the first person plural prefix would be an extraordinarily rare occurrence, but for this construction to also be found in Siouan and Catawban languages would be highly improbable.

Another idiosyncrasy shared by Yuchi and Siouan is the tendency to create portmanteaux out of first person plural markers and preverbs. On Table \ref{yuchipreverbs}, both \emph{hi}- and \emph{'\k{o}}- and \emph{hi}- and \emph{n\k{o}}- combine their vowels with vowel in \emph{hi}- to create new, more complex affixes: \emph{hi-} plus \emph{'\k{o}-} become \emph{'\k{e}-}, and \emph{hi-} plus \emph{n\k{o}-} become \emph{n\k{e}}. The prefix \emph{k'a}- `something' itself resembles a portmanteau of the Siouan *\'aa- preverb, which denotes a comitative action and the Siouan reflexive marker *ki-. The prefix \emph{k'a}- is also the reciprocal marker in Yuchi, which affixes immediately to the left of the verb before any pronominals are added on. Siouan reciprocals and reflexives have the same distribution.

\ea \label{yuchimandan}
	\ea
	\glll  \textbf{N\k{o}k'a}tʰetʰe.\\ 
	\textbf{n\k{o}-k'a}-tʰe$\sim$tʰe\\
			\textbf{\textsc{1pl.excl.agt}}-\textbf{\textsc{recp}}-hit$\sim$\textsc{iter}\\ \jambox[0in]{ Yuchi }
	\glt `We beat each other up/We hit each other repeatedly.' (Linn 2000: 250)
	
	\ex  
	\glll \textbf{N\k{\'u}ki}ru\v{s}kapo'\v{s}. \\ 
	\textbf{r\k{u}}-\textbf{ki}-ru-\v{s}kap-o'\v{s}\\
			\textbf{\textsc{1pl.agt}}-\textbf{\textsc{recp}}-by.hand-pinch-\textsc{ind.masc}\\ \jambox[0in]{ Mandan }
	\glt `We pinch each other.' (Hollow 1970: 440)
\z
\z

The similarity between Yuchi and Siouan---represented here with Mandan---is that both languages have a set of inner and outer pronominals: inner pronominals are first person singular and second person markers, which appear closer to the verb than preverbal elements like the Proto-Siouan applicative preverbs *\'aa-, *\'ii, etc. and morphology in Yuchi that intrinsically bears high tone and appears closer to the left edge of the word than these inner pronominals, such as \emph{hi-}, \emph{k'a}-, and \emph{k'\k{a}}-. First person plurals in both Mandan and Yuchi are treated as outer pronominals, meaning they appear further to the left than a preverb.

The data above in (\ref{yuchimandan}) demonstrate the similarity in not only the phonetic realization of the first person plural for both Mandan and Yuchi, but also the similarity in sound and semantics for both of their reciprocal markers. An additional potential cognate with Siouan is the inceptive marker in Yuchi, which is also \emph{k'a}, while the inceptive marker in Mandan is \emph{ka}. This similarity could be coincidental, but Siouan non-pronominal affixes have a somewhat large degree of polysemy.  This strong similarity could be due to ancient borrowing or some old areal feature, but when taken as a whole, the summation of these similarities begins to beg the question of whether we are looking at ancient borrowings or ancient features inherited from an ancestral language. 

Further analysis of non-pronominal morphology in Yuchi is needed, but the preliminary look taken by Rankin (1998) plus the amount added here points to the fact that Sapir (1929) might not have been far off the mark in declaring a genetic relationship between Siouan and Yuchi.

\subsection{Sound symbolism and ablaut}

The final component in Rankin's (1998) analysis of the relationship between Siouan and Yuchi is that both languages share a fricative sound symbolism. A sound symbolism is a relationship between the place of articulation and some kind of scalar contrast. For example, in Mandan, the sounds [s \v{s} x] are in a sound symbolism relationship. In Mandan, two examples of such words are \emph{seroo} `to jingle' and \emph{xeroo} `to rattle,' or \emph{s\k{a}si} `slick' and \emph{\v{s}\k{a}\v{s}i} `smooth.' In Yuchi, such examples are \emph{'ispi} `black' and \emph{'i\v{s}pi} `dirty,' and \emph{\v{c}ʰa\textbeltl a} `pink' and \emph{tstextipa{\super h}ya\textbeltl a} `red.'

In addition to this sound symbolism, Yuchi appears to have some vestiges of a Siouan-type ablaut system. In Siouan languages, there are certain vowels, usually marked with a capital letter in dictionaries, that will change their qualities under the influence of following morphology. In Mandan, there is a class of vowels that typically are realized as [e], but when followed by certain affixes are realized as [a]. Such vowels are marked as /E/ in underlying representation. One suffix that triggers ablaut is the second person pluralizer /-r\k{i}t/:

\ea
	\ea 
	\glll rareeho'\v{s} \\
	ra-rEEh-o'\v{s} \\	
	\textsc{2.agt}-go.there-\textsc{ind.masc} \\
	\glt	`you (\textsc{sg}) went'
	\ex 
	\glll raraahin\k{i}to'\v{s}\\
	ra-rEEh-r\k{i}t-o'\v{s}\\	
	\textsc{2.agt}-go.there-\textsc{2pl-ind.masc}\\
	\glt	`you (\textsc{pl}) went'
	\z
\z

In the Dakotan languages, one such element that triggers ablaut is the future marker \emph{kte}. In Lakota, there are three ablaut grades: a-grade, e-grade, and \k{i}-grade. A-grade is the default form, while e-grade is triggered when the word is the last in a sentence, or various other morphology is present. The last grade is what is most interesting here, as it causes [a] to become [\k{i}] (Ullrich 2008).

In Yuchi, Linn (2000) states that nasalizing the final vowel in the stem denotes future tense. A similar process occurs in Lakota.
\ea
	\ea \textit{Weda.} \hfill Yuchi\\
		`I'm going (now).' (Linn 2000: 279)
	\ex \textit{Wed\k{a}.} \hfill Yuchi\\
		`I'm going to go (soon).' (Linn 2000: 279)
	\ex \textit{Bl\'e.} \hfill Lakota\\
		`I'm going.' (Ullrich 2008: 755)
	\ex \textit{Mn\k{\'\i} kte.} \hfill Lakota\\
		`I'm going to go.' (Ullrich 2008: 755)
	\z
\z
Though there is no overt future marker in Yuchi, the nasality added to the final vowel is very reminiscent of \k{i}-ablaut in Siouan. To date, there has been no mention of a connection between future marking in Yuchi and Siouan ablaut, so this potential morphological cognate is deserving of further study. While not conclusive, this nasalization does raise further questions as to what other morphologically conditioned sound changes are taking place, and how they tie into Siouan-Catawban as a whole.

\section{Conclusion}

Proving an ancient genetic relationship is no easy task. If Parks \& Rankin (2001) are hitting near the mark in estimating the split of Siouan and Catawba from each other around four thousand years ago, how much deeper would we have to go in order to account for the massive lexical and grammatical difference from Yuchi? Though my labors have just scratched at the surface of similarities between Yuchi and Siouan, some small signs of hope can yet be found for proponents of a Siouan-Yuchi family. There are indeed lexical cognates that can be used to create sound correspondences, and there are ample records showing that Catawbans and Siouans were close to the Yuchi during the early days of colonization in North America. Thus, a relationship is plausible, given the close proximity in which these peoples lived and their sharing several very elementary words. With the location of Catawba and the varieties of Ohio Valley Siouan and Yuchi firmly recorded in the Southeast, the center of gravity would suggest a homeland somewhere near the Ohio River Valley.

While Rankin (1996) expresses tempered pessimism regarding the depth at which linguists are capable of probing for relatedness, the morphological data gathered by Rankin (1998), in addition to a few affixes of interest, only cause more questions to be raised. It is quite possible that there are many more correspondences to be found, but that is a task best saved for a different paper.

My principal goal was to investigate the idea of a Siouan-Catawaban-Yuchi family, given Rankin's (1996; 1998) previous efforts to investigate deeper genetic relationships between Siouan and other languages of North America. While these results are not definitive, there is room for some optimism on this front. I believe that there is substance to the idea that Siouan and Yuchi are distant cousins, and I encourage further study into the topic of Siouan's relationship with neighboring languages and groups with a particular eye towards Yuchi. Since Rankin's attempts to connect Yuchi with Siouan-Catawban, similar connections have been made by \citet{Vajda2010}, who puts forth a strong case for a distant genetic relationship between the Na-Den\'e languages of the Americas and the Yeniseian languages of Siberia. While lexical cognates are not overly common, they do exist, but the most compelling evidence is in the similarity of both the inflectional morphology and the sequencing of affixes within the verbal template. A similar argument occurs in \S5 of this paper, by which I conclude that there is some validity to Rankin's claim that Yuchi is distantly related to Siouan-Catawban.

One particular avenue for study is to go through the lexical information from the few extant Yuchi language grammars, field notes, and attempted dictionaries housed at the American Philosophical Society to put together a Yuchi database. With a larger repository of Yuchi vocabulary, the attempt to make lexical connections between Yuchi and Siouan might be more fruitful. Another important task is to continue the work on Catawba started by Rudes (2007) before his passing to investigate whether there are additional similarities to be found between Yuchi and Catawba, since they share more pronominal morphology with each other than they do with Proto-Siouan.

\section*{Acknowledgment}

I would like to thank Steve Anderson for his comments on a earlier draft of this paper, as well as my two anonymous reviewers for their insightful comments. Ultimately, I am grateful to the late Bob Rankin for all his insight on this topic over the years and for sharing his ``tempered'' enthusiasm about the deeper genetic relationships between Siouan and other languages of North America.

\section*{Abbreviations}
1, 2, 3 = first, second, third person; \textsc{agt} = agent; Cat = Catawban; DA = Dakota; DH = Dhegiha; \textsc{excl} = exclusive; HC = Ho-Chunk; \textsc{incept} = inceptive; \textsc{incl} = inclusive; \textsc{ind} = indicative; \textsc{iter} = iterative; Lak = Lakota; MA = Mandan; \textsc{masc} = masculine; OF = Ofo; PCa = Proto-Catawban; PIE = Proto-Indo-European; \textsc{pl} = plural; \textsc{prog} = progressive; PSi = Proto-Siouan; \textsc{recp} = recipient; \textsc{sg} = singular; TU = Tutelo; Yu = Yuchi.

\section*{References}

\printbibliography 

 

 
 

 


Campbell, Lyle \& Marianne Mithun. 1979. Introduction. In Campbell, Lyle \& Marianne Mithun (eds.), The languages of native America: Historical and comparative assessment.  Austin: University of Texas Press.

 
 



Crawford, Jame M. 1970-1973. Field notebooks. Philadelphia, PA: American Philosophical Society Library.



Crawford, James M. 1979. Timucua and Yuchi: Two language isolates of the Southeast. In Campbell, Lyle \& Marianne Mithun. (eds.) The languages of native America: Historical and comparative assessment.  Austin: University of Texas Press.


 


Dorsey, James O. 1894. The Biloxi indians of Louisiana. In Putman, Frederic W. (ed.), Proceedings of the Association for the Advancement of Science, Vol. 43.  Salem: The Salem Press.

 

 


Gatschet, Albert S. 1900. Grammatic sketch of the Catawba language. American Anthropologist, New Series 2(3): 527-549.


 



Goddard, Ives. 1975. Algonquian, Wiyot, and Yurok: Proving a distant genetic relationship. In Kinkade, Marvin D., Kenneth L. Hale, \& Oswald Werner, (eds.) Linguistics and anthropology in honor of C. F. Voegelin. Lisse: Peter de Ridder Press.


 


Haas, Mary. 1952. The Proto-Gulf word for ``land" (with notes on Proto-Siouan). International Journal of American Linguistics 18(4): 238-240.


 

 

 
 


Rankin, Robert L, Richard T. Carter, \& A. Wesley Jones. n.d. Proto-Siouan phonology and grammar. ms. Lawrence: University of Kansas.



Rankin, Robert L.; Richard T. Carter; A. Wesley Jones; John E. Koontz; David S. Rood \& Iren Hartmann, eds. 2015. Comparative Siouan dictionary. Leipzig: Max Planck Institute for Evolutionary Anthropology. (Available online at http://csd.clld.org, Accessed on 2015-09-25.)

 


Sapir, Edward. 1929. Central and North American languages. In Encyclopaedia Britanica 5: 138-141.



 
 

 

Wagner, Günter. 1931. Yuchi tales. In Boas, Franz (ed.) Publications of the American Ethnological Society. New York: G. E. Stechert.



Wagner, Günter. 1934. Yuchi. In Boas, Franz (ed.) Handbook of American Indian languages 3.  New York: Columbia University Press.

 


Wolff, Hans. 1950c. Comparative Siouan III. International Journal of American Linguistics 16(4): 168-178.



Wolff, Hans. 1950d. Comparative Siouan IV. In: International Journal of American Linguistics 17(4): 197-204.
\end{document}