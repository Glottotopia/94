\documentclass[output=paper]{LSP/langsci}
\author{Meredith Johnson}
\title{A description of verb-phrase ellipsis in Hocąk}

\abstract{In this paper, I argue that Hocąk displays verb phrase ellipsis (VPE) and provide the first thorough description of this phenomenon. VPE in Hocąk displays the two defining characteristics of VPE cross-linguistically: it targets all VP-internal material, and it is licensed by a functional head. In the case of Hocąk, I propose that the licensing head is active \emph{v}. Furthermore, Hocąk VPE also shows many other traits of VPE in other languages: the ellipsis site can be found in coordinated and adjacent clauses in addition to embedded and adjunct clauses, VPE is insensitive to the the contents of the VP, and VPE gives rise to both strict and sloppy readings. Lastly, I argue that VPE in Hocąk is derived by deletion of a full-fledged VP, and that the ellipsis site cannot be analyzed as a null \emph{pro} form. KEYWORDS: [Hoocąk, verb phrase, ellipsis, VPE, light verb]}

\maketitle

\begin{document}

\section{Introduction}\label{sec:johnson:1}

This purpose of this paper is to both argue that Hocąk displays verb phrase ellipsis (VPE) and to provide the first thorough description of this phenomenon. In VPE constructions, a VP goes unpronounced when there is an appropriate antecedent VP and a licensing head that identifies the gap. Both of these properties can be seen in the examples of VPE in Hocąk in \REF{ex:johnson:1} below. In each example, the VP in the second conjunct is interpreted as identical to the VP in the first conjunct, even though the former has no phonological realization. Instead, the light verb \emph{ųų} takes the place of the VP.\footnote{}


\ea\label{ex:johnson:1}
\ea
\glll Cecilga {\ob}{vp} wažątirehižą ruwį{\cb} kjane anąga nee šge {\ob}ha'ųų{\cb} kjane.\\
Cecil-ga {} wažątire-hižą {$\emptyset$}-ruwį kjane anąga nee šge ha-ųų kjane\\
Cecil-\textsc{prop} {} car-\textsc{indef} \textsc{3s/o}-buy \textsc{fut} and I also \textsc{1s}-do \textsc{fut}\\
\trans `Cecil will buy a car, and I will too.' 
\ex
\glll Cecilga {\ob}{vp} xjanąre waši{\cb} anąga Bryanga šge {\ob}ųų{\cb}.\\
Cecil-ga {} xjanąre {$\emptyset$}-waši anąga Bryan-ga šge {$\emptyset$}-ųų\\
Cecil-\textsc{prop} {} yesterday \textsc{3s}-dance and Bryan-\textsc{prop} also \textsc{3s}-do\\
\trans `Cecil danced yesterday, and Bryan did too.'
\z
\z

Throughout this paper, I rely on the set of diagnostics of VPE established by \citet{Goldberg2005}, and subsequently used for Indonesian by \citet{Fortin2007}. \citet{Goldberg2005} uses characteristics of English VPE to establish a typology of VPE crosslinguistically, noting that ``English VP Ellipsis has a characteristic set of behavioral traits, the confluence of which is not found in other types of null anaphora'' \citep[27]{Goldberg2005}. 

Goldberg developed this set of traits in order to diagnose verb-stranding verb phrase ellipsis (VVPE) in a variety of languages, including Hebrew, Irish and Swahili. In VVPE, the verb has undergone raising before the remainder of the VP is elided. On the surface, VVPE can be ambiguous between a null object analysis or VPE analysis; thus, some of her diagnostics serve to distinguish these two approaches. An example of VVPE in Hebrew is provided in \REF{ex:johnson:2}:

\begin{singlespacing}
\begin{exe}
\ex\label{ex:johnson:2}
\gll Tazmini {et Dvora} la-misiba? Kvar hizmanti.\\
invite.{\textsc 2fut} Dvora to.the-party already invite.{\textsc 1pst}\\
\trans `Will you invite Dvora to the party? I already invited (Dvora to the party).' \\\citep[14]{Goldberg2005}\\
\vspace{-12pt}
\end{exe}
\end{singlespacing}

Hocąk VPE does not face this problem: there is an overt light verb standing in for the VP, much like English VPE. Nonetheless, the data from Hocąk are consistent with all of the characteristics that Goldberg argues are diagnostic of VPE crosslinguistically. Furthermore, I show that these traits also distinguish VPE from other elliptical phenomena found in Hocąk, including gapping, stripping and null complement anaphora. 

This paper is structured as follows. In \sectref{sec:johnson:2}, I establish that putative VPE in Hocąk displays the two most important characteristics of VPE: it targets all VP-internal material, and it is licensed by a functional head. In the case of Hocąk, I propose that the licensing head for VPE is active \emph{v}. In \sectref{sec:johnson:3}, I show that Hocąk VPE displays other traits that have been attributed to VPE crosslinguistically. \sectref{sec:johnson:4} demonstrates that VPE in Hocąk must be analyzed as a deletion process, rather than a null \emph{pro} form. \sectref{sec:johnson:5} concludes the paper.

\section{Establishing the presence of VPE in Hocąk}\label{sec:johnson:2}

In this section, I show that the construction that I argue instantiates VPE in Hocąk displays the two defining characteristics of VPE. In 2.1, I demonstrate that the ellipsis site includes all VP-internal material. In 2.2, I show that VPE is subject to the presence of an appropriate licensing head.

\subsection{\emph{ųų} targets the VP}

VPE is possible with both intransitive and transitive verbs, as seen in (3-4) below. (3a) and (3b) show that \emph{ųų} can target intransitive VPs. In the examples in \REF{ex:johnson:4} with transitive verbs, the direct object is also included in the ellipsis site.

\begin{exe}
\ex\label{ex:johnson:3}
\begin{xlist}
\ex 
\glll Cecilga {\ob}{vp} kere{\cb} anąga Matejaga šge {\ob}ųų{\cb} .\\
Cecil-ga {} $\emptyset$-kere, anąga Mateja-ga šge $\emptyset$-ųų\\
Cecil-{\textsc prop} {} {\textsc 3s}-leave and Mateja-{\textsc prop} also {\textsc 3s}-do\\
\trans `Cecil left, and Mateja did too.'
\ex 
\glll Meredithga {\ob}{vp} nįįp{\cb} anąga Sarahga šge {\ob}ųų{\cb} .\\
Meredith-ga {} $\emptyset$-nįįp anąga Sarah-ga šge $\emptyset$-ųų\\
Meredith-{\textsc prop} {} {\textsc 3s}-swim and Sarah-{\textsc prop} also {\textsc 3s}-do\\
\trans `Meredith swam, and Sarah did too.'
\end{xlist}
\end{exe}

\begin{exe}
\ex\label{ex:johnson:4}
\begin{xlist}
\ex 
\glll Matejaga {\ob}{vp} {waisgap sguuhižą} rook'į{\cb} anąga Sarahga šge {\ob}ųų{\cb} .\\
Mateja-ga {} {waisgap sguu-hižą} $\emptyset$-rook'į anąga Sarah-ga šge $\emptyset$-ųų\\
Mateja-{\textsc prop} {} cake-{\textsc indef} {\textsc 3s/o}-bake and Sarah-{\textsc prop} also {\textsc 3s}-do\\
\trans `Mateja baked a cake, and Sarah did too.'
\ex 
\glll Meredithga {\ob}{vp} waaruchižą hogiha{\cb} anąga Bryanga šge {\ob}ųų{\cb} .\\
Meredith-ga {} waaruc-hižą $\emptyset$-hogiha anąga Bryan-ga šge $\emptyset$-ųų\\
Meredith-{\textsc prop} {} table-{\textsc indef} {\textsc 3s/o}-paint and Bryan-\textsc{prop} also {\textsc 3s}-do\\
\trans `Meredith painted a table, and Bryan did too.'
\end{xlist}
\end{exe}

VPE can also target other internal arguments. Both indirect objects and resultative phrases are typically analyzed as VP-internal (see e.g., \citealt{Larson1988} and \citealt{LevinEtAl1995}), and they are also subject to VPE. The ditransitive example in \REF{ex:johnson:5} shows that both a direct object and indirect object can be contained in the ellipsis site.


\begin{exe}
\ex\label{ex:johnson:5}
\glll Cecilga {\ob}{vp} Meredithga wiiwagaxhižą hok'ų{\cb} anąga Matejaga šge {\ob}ųų{\cb} .\\
Cecil-ga {} Meredith-ga wiiwagax-hižą $\emptyset$-hok'ų anąga Mateja-ga šge $\emptyset$-ųų\\
Cecil-{\textsc prop} {} Meredith-{\textsc prop} pencil-{\textsc indef} {\textsc 3s/o}-give and Mateja-{\textsc prop} also {\textsc 3s}-do\\
\trans `Cecil gave Meredith a pencil, and Mateja did too.'
\end{exe}

In \REF{ex:johnson:6}, we see examples of VPE with resultative constructions in which the direct object and result have both been elided. 

\begin{exe}
\ex\label{ex:johnson:6}
\begin{xlist}
\ex 
\glll Cecilga  {\ob}{vp} wažątirehiža šuuc hogiha{\cb} anąga Bryanga šge {\ob}ųų{\cb} .\\
Cecil-ga {} wažątire-hiža šuuc $\emptyset$-hogiha anąga Bryan-ga šge $\emptyset$-ųų\\
Cecil-{\textsc prop} {} car-{\textsc indef} red {\textsc 3s/o}-paint and Bryan-{\textsc prop} also {\textsc 3s}-do\\
\trans `Cecil painted a car red, and Bryan did too.'
\ex 
\glll Meredithga  {\ob}{vp} mąąshiža paras gistak{\cb} anąga Matejaga šge {\ob}ųų{\cb} .\\
Meredith-ga {} mąąs-hiža paras $\emptyset$-gistak anąga Mateja-ga šge $\emptyset$-ųų\\
Meredith-{\textsc prop} {} metal-{\textsc indef} flat {\textsc 3s/o}-hit and Mateja-{\textsc prop} also {\textsc 3s}-do\\
\trans `Meredith hit metal flat, and Mateja did too.'
\end{xlist}
\end{exe}

VPE also targets various adjuncts. (7a)-(b) shows that VPE targets VPs containing temporal adjuncts. 

\begin{exe}
\ex\label{ex:johnson:7}
\begin{xlist}
\ex 
\glll Cecilga {\ob}{vp} xjanąre waši{\cb} anąga Bryanga šge {\ob}ųų{\cb} .\\
Cecil-ga {} xjanąre $\emptyset$-waši anąga Bryan-ga šge $\emptyset$-ųų\\
Cecil-\textsc{prop} {} yesterday \textsc{3s}-dance and Bryan-\textsc{prop} also \begin{sc}3s-\end{sc}do\\
\trans `Cecil danced yesterday, and Bryan did too.'

\ex 
\glll Meredithga  {\ob}{vp} hąąpte'e kšeehižą ruuc{\cb} anąga Matejaga šge {\ob}ųų{\cb} .\\
Meredith-ga {} hąąpte'e kšee-hižą $\emptyset$-ruuc anąga Mateja-ga šge $\emptyset$-ųų\\
Meredith-{\textsc prop} {} today apple-{\textsc indef} {\textsc 3s/o}-eat and Mateja-{\textsc prop} also {\textsc 3s}-do\\
\trans `Meredith ate an apple today, and Mateja did too.'
\end{xlist}
\end{exe}

In (8a) and (b), locative adjuncts are included in the ellipsis site. 

\begin{exe}
\ex\label{ex:johnson:8}
\begin{xlist}
\ex 
\glll Cecilga {\ob}{vp} hosto eja waši{\cb} kjane anąga Bryanga šge {\ob}ųų{\cb} kjane.\\
Cecil-ga {} hosto eja $\emptyset$-waši kjane anąga Bryan-ga šge $\emptyset$-ųų kjane\\
Cecil-\textsc{prop} {} gathering there {\textsc 3s}-dance \textsc{fut} and Bryan-\textsc{prop} also {\textsc 3s}-do \textsc{fut}\\
\trans `Cecil will dance at the gathering, and Bryan will too.'
\ex 
\glll Cecilga {\ob}{vp} ciinąk eja wažątirehižą ruwį{\cb} anąga Bryanga šge {\ob}ųų{\cb} .\\
Cecil-ga {} ciinąk eja wažątire-hižą $\emptyset$-ruwį anąga Bryan-ga šge $\emptyset$-ųų\\
Cecil-\textsc{prop} {} city there car-\textsc{indef} {\textsc 3s/o}-buy and Bryan-\textsc{prop} also {\textsc 3s}-do\\
\trans `Cecil bought a car in the city, and Bryan did too.'
\end{xlist}
\end{exe}

\REF{ex:johnson:9} exemplifies VPE with a comitative.

\begin{exe}
\ex\label{ex:johnson:9}
\glll Cecilga {\ob}{vp} hinųkra hakižu waši{\cb} anąga Bryanga šge {\ob}ųų{\cb} .\\
Cecil-ga {} hinųk-ra hakižu $\emptyset$-waši anąga Bryan-ga šge $\emptyset$-ųų\\
Cecil-\textsc{prop} {} woman-\textsc{def} be.with {\textsc 3s}-dance and Bryan-\textsc{prop} also {\textsc 3s}-do\\
\trans `Cecil danced with the woman, and Bryan did too.'
\end{exe}

(10a)-(c) demonstrate that various manner adverbs can also be subject to VPE.

\begin{exe}
\ex\label{ex:johnson:10}
\begin{xlist}
\ex 
\glll Bryanga  {\ob}{vp} teejąki {nįįtašjak taaxu} racgą{\cb} anąga Sarahga šge {\ob}ųų{\cb} .\\ 
Bryan-ga {} teejąki {nįįtašjak taaxu} $\emptyset$-racgą anąga Sarah-ga šge $\emptyset$-ųų\\
Bryan-{\textsc prop} {} often coffee {\textsc 3s}-drink and Sarah-{\textsc prop} also {\textsc 3s}-do\\
\trans `Bryan often drinks coffee, and Sarah does too.'
\ex 
\glll Cecilga  {\ob}{vp} pįįhį mąąnį{\cb} anąga Bryanga šge {\ob}ųų{\cb} .\\
Cecil-ga {} pįįhį $\emptyset$-mąąnį anąga Bryan-ga šge $\emptyset$-ųų\\
Cecil-{\textsc prop} {} quietly {\textsc 3s}-walk and Bryan-{\textsc prop} also {\textsc 3s}-do\\
\trans `Cecil walked carefully/quietly, and Bryan did too.'
\ex 
\glll Meredithga {\ob}{vp} hikųhe {nįįtašjak taaxu} racgą{\cb} anąga Bryanga šge {\ob}ųų{\cb} .\\
Meredith-ga {} hikųhe {nįįtašjak taaxu} $\emptyset$-racgą anąga Bryan-ga šge $\emptyset$-ųų\\
Meredith-{\textsc prop} {} quickly coffee {\textsc 3s}-drink and Bryan also {\textsc 3s}-do\\
\trans `Meredith drank coffee quickly, and Bryan did too.'
\end{xlist}
\end{exe}

In all of the examples in \REF{ex:johnson:7}-\REF{ex:johnson:10}, the adjunct in the antecedent VP is interpreted as being present in the ellipsis site, indicating that \emph{ųų} targets the entire VP rather than just the object(s).

Lastly, complement clauses can also be included in VPE. The example in \REF{ex:johnson:11} has two possible interpretations: either that Meredith also bought a car, or that Meredith also said that Cecil bought a car. Under the second reading, VPE targets the matrix clause, eliding the verb and its complement clause.

\begin{exe}
\ex\label{ex:johnson:11}
\glll Bryanga {\ob}{vp} Cecilga wažątirehižą ruwįže ee{\cb} anąga Meredithga šge {\ob}ųų{\cb} .\\
Bryan-ga {} Cecil-ga wažątire-hižą $\emptyset$-ruwį-že $\emptyset$-ee anąga Meredith-ga šge $\emptyset$-ųų\\
Bryan-{\textsc prop} {} Cecil-{\textsc prop} car-{\textsc indef} {\textsc 3s/o}-buy-{\textsc comp} {\textsc 3s}-say and Meredith-{\textsc prop} also {\textsc 3s}-do\\
\trans `Bryan said that Cecil bought a car, and Meredith did too.'
\end{exe}


\subsection{Licensing of VPE}

The main characteristic that distinguishes VPE from other elliptical processes is the presence of an overt licensing head located in the inflectional domain above the VP. VPE in English can be licensed by a variety of functional elements, such as \emph{do} in (12a), \emph{be} in (12b), \emph{have} in (12c), \emph{can} in (12d) and \emph{will} in (12e). The obligatory presence of an inflectional head has led previous researchers to argue that VPE is licensed by T/Infl. (\citealt{Bresnan1976}, \citealt{Sag1976}, \citealt{Zagona1988}, \citealt{Lobeck1995})

\begin{exe}
\ex\label{ex:johnson:12}
\begin{xlist}
\ex 
Lily wore a skirt, and Molly {\textbf did} too.
\ex 
Lily is reading a book, and Molly {\textbf is} too.
\ex 
Lily hasn't finished the book, but Molly {\textbf has}.
\ex 
Lily can ride a bike, and Molly {\textbf can} too.
\ex 
Lily will leave, and Molly {\textbf will} too.
\end{xlist}
\end{exe}


In contrast, there is no such inflectional head found with stripping or gapping. Stripping is an elliptical phenomenon in which an entire clause is elided except for a single element that is stranded. This is illustrated in (13a). In gapping constructions, the verb (and other potential material) is left unpronounced, while there are two elements that are stranded. An example of gapping can be seen in (13b).

\begin{exe}
\ex\label{ex:johnson:13}
\begin{xlist}
\ex 
Lily came over, and Molly too.
\ex 
Lily brought bagels, and Molly danishes.
\end{xlist}
\end{exe}

In Hocąk, the licensing requirement on VPE is different: VPE is conditioned solely by the presence of the light verb \emph{ųų}. We have seen this in all of the instances of VPE given above. The examples in \REF{ex:johnson:14}-\REF{ex:johnson:18} illustrate that \emph{ųų} is indeed a light verb: it productively combines with both nouns and verbs to create complex predicates. Based on its distribution, I assume that \emph{ųų} realizes the functional head \emph{v}. (Examples 14--18:\citealt{Hartmann2012})

\begin{exe}
\ex\label{ex:johnson:14}
\begin{multicols}{2}
\begin{xlist}
\ex 
mąąnąąpeja\\
`warrior'
\ex 
mąąnąąpeja ųų\\
`be in the military'
\end{xlist}
\end{multicols}
\end{exe}

\begin{exe}
\ex\label{ex:johnson:15}
\begin{multicols}{2}
\begin{xlist}
\ex 
nąąwą\v{g}o\v{g}o\\
`fiddle'
\ex 
nąąwą\v{g}o\v{g}o ųų\\
`play the fiddle'
\end{xlist}
\end{multicols}
\end{exe}

\begin{exe}
\ex\label{ex:johnson:16}
\begin{multicols}{2}
\begin{xlist}
\ex 
waruc\\
`food'
\ex 
waruc ųų\\
`cook, prepare food'
\end{xlist}
\end{multicols}
\end{exe}

\begin{exe}
\ex\label{ex:johnson:17}
\begin{multicols}{2}
\begin{xlist}
\ex 
waagax\\
`paper, letter'
\ex 
waagax ųų\\
`write (a letter)'
\end{xlist}
\end{multicols}
\end{exe}

\begin{exe}
\ex\label{ex:johnson:18}
\begin{multicols}{2}
\begin{xlist}
\ex 
hooxiwi\\
`cough' (verb)\\

\ex 
hooxiwi ųų\\
`have a cold'
\end{xlist}
\end{multicols}
\end{exe}

Tense and modals can be present in VPE constructions; however, they are never obligatory. When present, tense and modals always co-occur with the light verb \emph{ųų}. (19a) shows that the future tense marker \emph{kjane} can follow \emph{ųų}, while (19b) and (c) demonstrate that the modals \emph{ną} and \emph{s'aare} can also appear after \emph{ųų}. When \emph{ųų} is omitted, the result is ungrammatical.

\begin{exe}
\ex\label{ex:johnson:19}
\begin{xlist}
\ex 
\glll Cecilga wažątirehižą ruwį kjane anąga nee šge *(ha'ųų) {\textbf kjane}.\\
Cecil-ga wažątire-hižą $\emptyset$-ruwį kjane anąga nee šge ha-ųų {\textbf kjane}\\
Cecil-\textsc{prop} car-\textsc{indef} \textsc{3s/o}-buy \textsc{fut} and I also \textsc{1s}-do \textsc{fut}\\
\trans `Cecil will buy a car, and I will too.' 
\ex 
\glll Meredithga hąąke wažątirera {pįį'ų ruxuruknį} nųnįge Matejaga *(ųų) {\textbf ną}.\\
Meredith-ga hąąke wažątire-ra {$\emptyset$-pįį'ų ruxuruk-nį} nųnįge Mateja-ga $\emptyset$-ųų {\textbf ną}\\
Meredith-\textsc{prop} {\textsc neg} car-\textsc{def} {\textsc 3s/o}-fix-\textsc{neg} but Mateja-\textsc{prop} {\textsc 3s}-do can\\
\trans `Meredith can't fix the car, but Mateja can.'
\ex 
\glll Meredithga hąąke {nįįtašjak taaxu} ruwįnį nųnįge Matejaga *(ųų) {\textbf s'aare}.\\ 
Meredith-ga hąąke {nįįtašjak taaxu} $\emptyset$-ruwį-nį nųnįge Mateja-ga $\emptyset$-ųų {\textbf s'aare}\\
Meredith-\textsc{prop} {\textsc neg} coffee {\textsc 3s/o}-buy-{\textsc neg} but Mateja-\textsc{prop} {\textsc 3s}-do must\\
\trans `Meredith didn't buy coffee but Mateja must have.' 
\end{xlist}
\end{exe}

Thus, we see that T/Infl does not play the same role in VPE licensing in Hocąk as it does in other languages. However, VPE in Hocąk is constrained by the type of predicate. As the examples in \REF{ex:johnson:20} show, VPE is not licensed with non-agentive verbs:


\begin{exe}
\ex\label{ex:johnson:20}
\begin{xlist}
\ex[*] {
\glll Meredithga kšee gipį anąga Bryanga šge ųų.\\
Meredith-ga kšee $\emptyset$-gipį anąga Bryan-ga šge $\emptyset$-ųų\\
Meredith-\textsc{prop} apple {\textsc 3s}-like and Bryan-\textsc{prop} also {\textsc 3s}-do\\
\trans Intended: `Meredith likes apples, and Bryan does too.'}
\ex[*] {
\glll Cecilga wįįxra waaja anąga Meredithga šge ųų.\\
Cecil-ga wįįx-ra wa-$\emptyset$-haja anąga Meredith-ga šge $\emptyset$-ųų\\
Cecil-\textsc{prop} duck-\textsc{def} {\textsc 3o.pl-3s}-see and Meredith-\textsc{prop} also {\textsc 3s}-do\\
\trans Intended: `Cecil saw the ducks, and Meredith did too.'}
\ex[*] {
\glll Meredithga hoišą anąga Bryanga šge ųų.\\
Meredith-ga $\emptyset$-hoišą anąga Bryan-ga šge $\emptyset$-ųų\\
Meredith-\textsc{prop} {\textsc 3s}-busy.\textsc{stat} and Bryan-\textsc{prop} also {\textsc 3s}-do\\
\trans Intended: `Meredith is busy, and Bryan is too.'}
\ex[*] {
\glll Cecilga hįįcge nųnįge Bryanga hąąke ųųnį.\\
Cecil-ga $\emptyset$-hįįcge nųnįge Bryan-ga hąąke $\emptyset$-ųų-nį\\
Cecil-\textsc{prop} {\textsc 3s}-tired.\textsc{stat} but Bryan-\textsc{prop} {\textsc neg} {\textsc 3s}-do-\textsc{neg}\\
\trans `Cecil is tired, but Bryan isn't.'}
\end{xlist}
\end{exe}

Like other Siouan languages, Hocąk exhibits an active-stative alignment pattern: the active set of verbal person markers is used to index the subject of transitive verbs and active intransitive verbs, while the stative set is used to index the object of transitive verbs and the subject of stative intransitive verbs. This alignment pattern interacts with VPE in revealing ways. While VPE is banned with most stative intransitive verbs, such as those in (20c) and (20d), VPE is possible with certain stative intransitives when they have an agentive reading. \emph{Hokąre} `to fall in' is normally a stative intransitive verb, but it is possible to use it in VPE contexts if the subject falls \emph{deliberately}, as in \REF{ex:johnson:21}. In this context, \emph{ųų} takes the \emph{active} person marker set. In (21b), the verb takes the second person active marker \emph{š-}; the stative marker \emph{nį-} is not permitted. The marker \emph{nį-} is the one that would typically be found on the verb \emph{hokąre}, as shown in \REF{ex:johnson:22}.

\begin{exe}
\ex\label{ex:johnson:21}
\begin{xlist}
\ex 
\glll Meredithga nįį eeja hokąre anąga Bryanga šge ųų.\\
Meredith-ga nįį eeja $\emptyset$-hokąre anąga Bryan-ga šge $\emptyset$-ųų\\
Meredith-\textsc{prop} water there {\textsc 3s}-fall.in and Bryan-\textsc{prop} also {\textsc 3s}-do\\
\trans `Meredith fell into the water (deliberately), and Bryan did too.'
\ex 
\glll Meredithga nįį eeja hokąre anąga (nee) šge š'ųų/*nį'ųų.\\
Meredith-ga nįį eeja $\emptyset$-hokąre anąga nee šge š-'ųų/nį-'ųų\\
Meredith-\textsc{prop} water there {\textsc 3s}-fall.in and you also {\textsc 2s}-do\\
\trans `Meredith fell into the water (deliberately), and you did too.'
\end{xlist}
\end{exe}

\begin{exe}
\ex\label{ex:johnson:22}
\glll Honįkąre.\\
<nį>hokąre\\
<{\textsc 2s}>fall.in.{\textsc stat}\\
\trans `You fell in(to something).' \citep{Hartmann2012}
\end{exe}
	
This restriction on VPE is not due to lexical properties of \emph{ųų}: when \emph{ųų} functions as a light verb, it can form non-agentive verbs, as in \REF{ex:johnson:23}:


\begin{singlespacing}
\begin{exe}
\ex\label{ex:johnson:23}
\begin{xlist}
\ex 
hooxiwi ųų `have a cold' (stative intransitive)
\vspace{12pt}
\ex 
roo taakac ųų `have a fever' (stative intransitive)
\vspace{12pt}
\ex 
paaxšišik ųų `have an upset stomach' (stative intransitive)\\
\citep{Hartmann2012}
\end{xlist}
\end{exe}
\end{singlespacing}

To formalize this restriction on VPE in Hocąk, I adopt \citegen{Merchant2001} proposal that ellipsis takes place when a so-called `[E]-feature' is present on the relevant licensing head. In the case of Hocąk, I propose that an [E]-feature is present only on the agentive \emph{v} head.\footnote{This agentivity requirement on a process that affects the VP is not completely unique to Hocąk. For example, \citet{Hallman2004} notes that English \emph{do so} replacement is restricted to agentive VPs, even though other uses of \emph{do} are not subject to this constraint (e.g., \emph{Max loves studying French, and Mary does (*so) too.}) \citet{Rouveret2012} also shows that VPE in Welsh is licensed uniquely by the light verb \emph{gweund}, and furthermore that VPE is not permitted with stative predicates. The only possibility with stative VPs is VVPE. However, Rouveret also shows that \emph{gweund} is also incompatible with stative predicates in its non-elliptical uses. This contrasts with the behavior of \emph{ųų} in Hocąk and \emph{do} in \emph{do so} in English.}  This accounts for the fact that VPE is solely conditioned by the presence of the light verb (or \emph{v}) \emph{ųų}, and furthermore that only agentive verbs can be elided: if there is a non-agentive \emph{v} present, then ellipsis will not be licensed. 

This conclusion is in line with other research that argues that \emph{v} is responsible for licensing VPE crosslinguistically. Many recent approaches to ellipsis have argued for a link between phases and elliptical phenomena (\citealt{Holmberg2001}, \citealt{vanCraenenbroeck2004}, \citealt{Gengel2007},  \citealt{YoshidaGallego2008}, \citealt{Gallego2009}, among others). Specifically, they propose that ellipsis results when a phasal head (e.g. \emph{v}, C, D) licenses deletion of its complement. These theories are a natural development of Chomsky's (\citeyear{Chomsky2000,Chomsky2001,Chomsky2004}) theory of phases: if ellipsis is PF-deletion, it follows that the units that are sent cyclically to the PF interface are precisely the ones that can be targeted for deletion. More concretely, \citet{Rouveret2012} adopts the phasal analysis of ellipsis, and puts forward a theory to predict which languages permit VPE. He argues that \emph{v} always has an uninterpretable [tense] feature, and that, in languages with VPE, the [tense] feature is valued on \emph{v} phase-internally. Rouveret proposes that the elements that license VPE are all merged in \emph{v}, and subsequently move to Infl. All of these approaches are compatible with the Hocąk data, with the caveat that VPE is more restricted in Hocąk: it is only licensed by active \emph{v}.


\section{Crosslinguistic characteristics of VPE}\label{sec:johnson:3}

In the previous section, I demonstrated that Hocąk displays the two defining characteristics of VPE: the elliptical process in question targets the entire VP, and is conditioned by the presence of a licensing head. \citet{Goldberg2005} discusses five other characteristics of VPE that are not shared by other elliptical phenomena, which are listed in \REF{ex:johnson:24}: 

\begin{exe}
\ex\label{ex:johnson:24}
{\textbf Characteristics of VPE}
\begin{xlist}
\ex 
Possible in both coordinated and adjacent CPs
\ex 
Insensitive to contents of elided VP
\ex 
Ellipsis site can be in a syntactic island
\ex 
Ellipsis site can be embedded
\ex 
Presence of strict and sloppy readings
\end{xlist}
\end{exe}


In the subsections that follow, I show that Hocąk VPE also generally conforms to this typology. In the areas where Hocąk appears to differ from English, I demonstrate that this is due to other differences between the two languages that are independent of ellipsis.

\subsection{Ellipsis licensed in both coordinated and adjacent CPs}

\citet{Goldberg2005} notes that English VPE is possible with a variety of sentence types. VPE is licit when the antecedent VP and elided VP are found in conjoined CPs (25a), in adjacent CPs uttered by the same speaker (25b), and when the antecedent is in a question and the ellipsis site in the answer (25c). In this section, I show that the same is true in Hocąk.

\begin{exe}
\ex\label{ex:johnson:25}
\begin{xlist}
\ex 
Lily hates beets, but Molly doesn't.
\ex 
Lily hates beets. Molly does too.
\ex 
Who hates beets? Molly does.
\end{xlist}
\end{exe}

All of the examples of VPE we saw in section 2 involved two clauses joined by the coordinator \emph{anąga} `and'. VPE is also possible with disjunction, as seen in \REF{ex:johnson:26} with \emph{nųnįge} `but'.

\begin{exe}
\ex\label{ex:johnson:26}
\begin{xlist}
\ex 
\glll Cecilga wažątirehižą ruwį nųnįge nee hąąke ha'ųųnį.\\
Cecil-ga wažątire-hižą $\emptyset$-ruwį nųnįge nee hąąke ha-ųų-nį\\
Cecil-{\textsc prop} car-{\textsc indef} {\textsc 3s/o}-buy but I {\textsc neg} {\textsc 1s}-do-{\textsc neg}\\
\trans `Cecil bought a car, but I didn't.'

\ex 
\glll Sarahga hąąke haas gihinį nųnįge Matejaga ųų.\\
Sarah-ga hąąke haas $\emptyset$-gihi-nį nųnįge Mateja-ga $\emptyset$-ųų\\
Sarah-{\textsc prop} {\textsc neg} berry {\textsc 3s}-pick-{\textsc neg} but Mateja-{\textsc prop} {\textsc 3s}-do\\
\trans `Sarah didn't pick berries, but Mateja did.'
\end{xlist}
\end{exe}


\REF{ex:johnson:27} shows that VPE is also licit in adjacent CPs. In each example, the antecedent VP is found in the first sentence while the ellipsis site is in the second sentence.

\begin{exe}
\ex\label{ex:johnson:27}
\begin{xlist}
\ex 
\glll Meredithga waaruchižą hogiha. Bryanga šge ųų.\\
Meredith-ga waaruc-hižą $\emptyset$-hogiha Bryan-ga šge $\emptyset$-ųų\\
Meredith-{\textsc prop} table-{\textsc indef} {\textsc 3s/o}-paint Bryan-\textsc{prop} also {\textsc 3s}-do\\
\trans `Meredith painted a table. Bryan did too.'
\ex 
\glll Meredithga hąąke {waisgap sguu xuwuxuwuhižą} ruucnį. Bryanga ųų.\\
Meredith-ga hąąke {waisgap sguu xuwuxuwu-hižą} $\emptyset$-ruuc-nį Bryan-ga $\emptyset$-ųų\\
Meredith-{\textsc prop} {\textsc neg} cookie-{\textsc indef} {\textsc 3s/o}-eat-{\textsc neg} Bryan-{\textsc prop} {\textsc 3s}-do\\
\trans `Meredith didn't eat a cookie. Bryan did.'
\end{xlist}
\end{exe}

Lastly, VPE also occurs in question-answer pairs in Hocąk. In (28a), a yes-no question contains the antecedent VP and the answer contains the gap. (28b) demonstrate that the same holds of \emph{wh}-questions.

\begin{exe}
\ex\label{ex:johnson:28}
\begin{xlist}
\ex 
\glll Question: {Nįįtašjak taaxu} šuruwį? Answer: Ha'ųų.\\
{} {nįįtašjak taaxu} šu-ruwį {} ha-ųų\\
{} coffee {\textsc 2s}-buy {} {\textsc 1s}-do\\
 \trans Q: `Did you buy coffee?' A: `I did.'
\ex 
\glll Question: Peežega Cecilga {gišja hii}? Answer: Bryanga ųų.\\
{} peežega Cecil-ga $\emptyset$-{gišja hii} {} Bryan-ga $\emptyset$-ųų.\\
{} who Cecil-{\textsc prop} {\textsc 3s/o}-visit {} Bryan-{\textsc prop} {\textsc 3s}-do\\
\trans Q: `Who visited Cecil?' A: `Bryan did.'
\end{xlist}
\end{exe}


\subsection{Ellipsis and the contents of the VP}

\citet{Goldberg2005} distinguishes VPE from null complement anaphora (NCA) based on the type of constituent that is elided. In NCA, a matrix verb is stranded and its complement is elided. However, NCA is constrained by the contents of the VP: only propositions can be elided. This is illustrated by the contrast between the grammatical NCA examples in (29a) and (c) and the ungrammatical examples in (29b) and (d):

\begin{exe}
\ex\label{ex:johnson:29}
\begin{xlist}
\ex 
Pat doesn't know {\textbf that Terry is moving to Japan}, but Robin knows.
\ex[*] {
Pat doesn't know {\textbf how to speak Inuktitut}, but Robin knows.}
\ex
Pat forgot {\textbf to close the door}, but Robin remembered.
\ex[*] {
Pat forgot {\textbf the answer}, but Robin remembered. \citep[245]{Fortin2007}}
\end{xlist}
\end{exe}

In contrast, the grammaticality of VPE is not dependent on the contents of the VP. The examples in \REF{ex:johnson:30} show that VPE is possible regardless of whether the complement of the VP expresses a proposition or not.

\begin{exe}
\ex\label{ex:johnson:30}
\begin{xlist}
\ex 
Pat doesn't know {\textbf that Terry is moving to Japan}, but Robin does.
\ex 
Pat doesn't know {\textbf how to speak Inuktitut}, but Robin does.
\ex 
Pat forgot {\textbf to close the door}, but Robin didn't.
\ex 
Pat forgot {\textbf the answer}, but Robin didn't.
\end{xlist}
\end{exe}

As \citet{Fortin2007} points out, this diagnostic does not serve to distinguish VPE from NCA in languages with null objects. Hocąk allows both null subjects and objects, as seen in (31b):

\begin{exe}
\ex\label{ex:johnson:31}
\begin{xlist}
\ex 
\glll Wijųkra šųųkra hoxataprookeeja haja.\\
Wijųk-ra šųųk-ra hoxatap-rook-eeja $\emptyset$-haja\\
cat-{\textsc def} dog-{\textsc def} woods-inside-there {\textsc 3s/o}-see\\
\trans `The cat saw the dog in the woods.
\ex 
\glll Hoxataprookeeja haja.\\
hoxatap-rook-eeja $\emptyset$-haja\\
woods-inside-there {\textsc 3s/o}-see\\
\trans `[The cat] saw [the dog] in the woods.' (\citealt[7]{JohnsonEtAl2013})
\end{xlist}
\end{exe}


Thus, it is not surprising that both propositional and non-propositional verbal complements can be null in Hocąk. In \REF{ex:johnson:32}, the complement of the verb \emph{hiperes} `know' can be null both when it is a proposition (32a) or an embedded question (32b). Likewise, both propositional (33a) and DP object (33b) complements of \emph{wakikųnųnį} `forget' surface as null.

\begin{exe}
\ex\label{ex:johnson:32}
\begin{xlist}
\ex 
\glll Sarahga Meredithga rookhožura ruucra hiperes, anąga Matejaga šge hireperesšąną.\\
Sarah-ga Meredith-ga rookhožu-ra $\emptyset$-ruuc-ra $\emptyset$-hiperes anąga Mateja-ga šge $\emptyset$-hiperes-šąną\\
Sarah-{\textsc prop} Meredith-{\textsc prop} pie-{\textsc def} {\textsc 3s/o}-eat-{\textsc comp} {\textsc 3s}-know and Mateja-{\textsc prop} also {\textsc 3s}-know-{\textsc decl}\\
\trans `Sarah knows that Meredith ate the pie, and Mateja knows too.'
\ex 
\glll Sarahga jaagu'ų Meredithga kerera hiperes, anąga Matejaga šge hiperesšąną.\\
Sarah-ga jaagu'ų Meredith-ga $\emptyset$-kere-ra $\emptyset$-hiperes anąga Mateja-ga šge $\emptyset$-hiperes-šąną\\
Sarah-{\textsc prop} why Meredith-{\textsc prop} {\textsc 3s}-leave-{\textsc comp} {\textsc 3s}-know and Mateja-{\textsc prop} also {\textsc 3s}-know-{\textsc decl}\\
\trans `Sarah knows why Meredith left, and Mateja knows (why Meredith left) too.'
\end{xlist}
\end{exe}

\begin{exe}
\ex\label{ex:johnson:33}
\begin{xlist}
\ex 
\glll Bryanga {nįįtašjak taaxu} ruwįra wakikųnųnį, nųnįge Meredithga hąąke wakikųnųnįnį.\\
Bryan-ga {nįįtašjak taaxu} $\emptyset$-ruwį-ra $\emptyset$-wakikųnųnį nųnįge Meredith-ga hąąke $\emptyset$-wakikųnųnį-nį\\
Bryan-{\textsc prop} coffee {\textsc 3s}-buy-{\textsc comp} {\textsc 3s}-forget but Meredith-{\textsc prop} {\textsc neg} {\textsc 3s}-forget-{\textsc neg}\\
\trans `Bryan forgot to buy coffee, but Meredith didn't forget.'
\ex 
\glll Bryanga {waisgap sguura} wakikųnųnį, nųnįge Meredithga hąąke wakikųnųnįnį.\\
Bryan-ga {waisgap sguu-ra} $\emptyset$-wakikųnųnį nųnįge Meredith-ga hąąke $\emptyset$-wakikųnųnį-nį\\
Bryan-{\textsc prop} cake-{\textsc def} {\textsc 3s/o}-forget but Meredith-{\textsc prop} {\textsc neg} {\textsc 3s/o}-forget-{\textsc neg}\\
\trans `Bryan forgot the cake, but Meredith didn't forget (the cake).'
\end{xlist}
\end{exe}


Thus, this particular diagnostic does not work for Hocąk due to independent factors. The complement of verbs like `know' and `forget' can always be null, presumably due to the availability of object \emph{pro} drop.\footnote{A full comparison of NCA and VPE is not possible in Hocąk. VPE with verbs like `know' and `forget' is ungrammatical (examples omitted for space purposes) since these verbs are non-agentive.}

\subsection{Ellipsis in syntactic islands}

\citet{Goldberg2005} notes that the ellipsis site in VPE constructions can be inside an adjunct island, while gapping is not permitted in adjuncts. This is shown by the contrast between (34a) and (b) below:

\begin{exe}
\ex\label{ex:johnson:34}
\begin{xlist}
\ex 
Lily finished her sandwich before Molly did.
\ex[*] {
Lily finished the sandwich before Molly the pizza.}
\end{xlist}
\end{exe}

The same contrast is found in Hocąk. The examples in \REF{ex:johnson:35} show that the gap in VPE constructions can be found inside adjunct clauses (which precede the main clause in these examples). In (35a), the ellipsis site is in the clause headed by `if', in (35b) the ellipsis site is in the clause headed by `because', and in (35c) it is in the clause headed by `before.'

\begin{exe}
\ex\label{ex:johnson:35}
\begin{xlist}
\ex 
\glll Bryanga ųų kjanegi Meredithga Hunterga (nišge) {gišja hii} kjane.\\
Bryan-ga $\emptyset$-ųų kjane-gi Meredith-ga Hunter-ga (nišge) $\emptyset$-{gišja hii} kjane\\
Bryan-\textsc{prop} {\textsc 3s}-do {\textsc fut}-if Meredith-\textsc{prop} Hunter-\textsc{prop} also {\textsc 3s/o}-visit {\textsc fut}\\
\trans `Meredith will visit Hunter if Bryan will.'
\ex 
\glll Bryanga hąąke ųųnįge Meredithga (nišge) hąąke Hunterga {gišja hiinį}.\\
Bryan-ga hąąke $\emptyset$-ųų-nį-ge Meredith-ga (nišge) hąąke Hunter-ga $\emptyset$-{gišja hii-nį}\\
Bryan-\textsc{prop} {\textsc neg} {\textsc 3s}-do-{\textsc neg}-because Meredith-\textsc{prop} also {\textsc neg} Hunter-\textsc{prop} {\textsc 3s/o}-visit-{\textsc neg}\\
\trans `Meredith didn't visit Hunter because Bryan didn't.'
\ex 
\glll Keenį Sarahga ųųnį Matejaga {waisgap sguu xuwuxuwuhižą} ruucšąną.\\
keenį Sarah-ga $\emptyset$-ųų-nį Mateja-ga {waisgap sguu xuwuxuwu-hižą} $\emptyset$-ruuc-šąną\\
before Sarah-{\textsc prop} {\textsc 3s}-do-{\textsc neg} Mateja-{\textsc prop} cookie-{\textsc indef} {\textsc 3s/o}-eat-{\textsc decl}\\
\trans `Mateja ate a cookie before Sarah did.'
\end{xlist}
\end{exe}

In contrast, gapping is ungrammatical in adjuncts. \REF{ex:johnson:36} illustrates that the the gap cannot be located in an adjunct clause headed by `if' (36a), `because' (36b) or `before' (36c).


\begin{exe}
\ex\label{ex:johnson:36}
\begin{xlist}
\ex[*] {
\glll Matejaga rookhožuhižągi Meredithga {waisgap sguuhižą} rook'į kjane.\\
Mateja-ga rookhožu-hižą-gi Meredith-ga {waisgap sguu-hižą} $\emptyset$-rook'į kjane\\
Mateja-{\textsc prop} pie-{\textsc indef}-if Meredith-{\textsc prop} cake-{\textsc indef} {\textsc 3s/o}-bake {\textsc fut}\\
\trans Intended: `Meredith will bake a cake if Mateja will bake a pie.'}	
\ex[*] {
\glll Sarahga {wažą honąkipįnįhižąge} Matejaga wažątirehižą ruwį.\\
Sarah-ga {wažą honąkipįnį-hižą-ge} Mateja-ga wažątire-hižą $\emptyset$-ruwį\\
Sarah-{\textsc prop} bicycle-{\textsc indef}-because Mateja-{\textsc prop} car-{\textsc indef} {\textsc 3s/o}-buy\\
\trans Intended: `Mateja bought a car because Sarah bought a bicycle.'}
\ex[*] {
\glll Keenį Bryanga {waisgap sguu xuwuxuwuhižąnį} Meredithga kšeehižą ruucšąną.\\
keenį Bryan-ga {waisgap sguu xuwuxuwu-hižą-nį} Meredith-ga kšee-hižą $\emptyset$-ruuc-šąną\\
before Bryan-{\textsc prop} cookie-{\textsc indef-neg} Meredith-{\textsc prop} apple-{\textsc indef} {\textsc 3s/o}-eat-{\textsc decl}\\
\trans Intended: `Meredith ate an apple before Bryan ate a cookie.'}
\end{xlist}
\end{exe}

\subsection{Ellipsis in embedded clauses}

\citet{Goldberg2005} also shows that the ellipsis site in VPE constructions can be inside an embedded clause, while this is not true of other types of ellipsis. (37a) demonstrates that VPE is licit in an embedded clause, while (37b)-(c) illustrate that neither gapping nor stripping are possible in an embedded clause.

\begin{exe}
\ex\label{ex:johnson:37}
\begin{xlist}
\ex
Lily went to the zoo, and I think (that) Molly did too.
\ex[*] {
Lily went to the zoo, and I think (that) Molly the aquarium.}
\ex[*] {
Lily went to the zoo, and I think (that) Molly too.}
\end{xlist}
\end{exe}

In Hocąk, VPE is licit in the complement clause of various matrix verbs, including `know' (38a), `want' (38b), `think' (38c) and `say' (38d).


\begin{exe}
\ex\label{ex:johnson:38}
\begin{xlist}
\ex
\glll Bryanga hąąke {nįįtašjak taaxu} ruwįnį, nųnįge Meredithga ųųra yaaperesšąną.\\
Bryan-ga hąąke {nįįtašjak taaxu} $\emptyset$-ruwį-nį nųnįge Meredith-ga $\emptyset$-ųų-ra <ha>hiperes-šąną\\
Bryan-\textsc{prop} {\textsc neg} coffee {\textsc 3s}-buy-\textsc{neg} but Meredith-\textsc{prop} {\textsc 3s}-do-{\textsc comp} <{\textsc 1s}>know-{\textsc decl}\\
\trans `Bryan didn't buy coffee, but I know Meredith did.'
\ex
\glll Meredithga hąąke Hunterga {gišja hiinį} nųnįge Bryanga ųų roogų.\\
Meredith-ga hąąke Hunter-ga $\emptyset$-{gišja hii-nį} nųnįge Bryan-ga $\emptyset$-ųų $\emptyset$-roogų\\
Meredith-{\textsc prop} {\textsc neg} Hunter-{\textsc prop} {\textsc 3s/o}-visit-{\textsc neg} but Bryan-{\textsc prop} {\textsc 3s}-do {\textsc 3s}-want\\
\trans `Meredith didn't visit Hunter, but Bryan wants to.'
\ex
\glll Matejaga hąąke {wažą honąkipįnįhižą} ruwįnį, nųnįge Cecilga ųųže yaare.\\
Mateja-ga hąąke {wažą honąkipįnį-hižą} $\emptyset$-ruwį-nį nųnįge Cecil-ga $\emptyset$-ųų-že <ha>hire\\
Mateja-{\textsc prop} {\textsc neg} bicycle-{\textsc indef} {\textsc 3s/o}-buy-{\textsc neg} but Cecil-{\textsc prop} {\textsc 3s}-do-{\textsc comp} <{\textsc 1s}>think\\
\trans `Mateja didn't buy a bicycle, but I think Cecil did.'
\ex
\glll Sarahga hąąke waarucra hogihanį, nųnįge Meredithga ųųže ee.\\
Sarah-ga hąąke waaruc-ra $\emptyset$-hogiha-nį nųnįge Meredith-ga $\emptyset$-ųų-že $\emptyset$-ee\\
Sarah-{\textsc prop} {\textsc neg} table-{\textsc def} {\textsc 3s/o}-paint-{\textsc neg} but Meredith-{\textsc prop} {\textsc 3s}-do-{\textsc comp} {\textsc 3s}-say\\
\trans `Sarah didn't paint the table, but Meredith said she did.'
\end{xlist}
\end{exe}

Unlike English, Hocąk does not exhibit any constraint on gapping in embedded contexts. (39a) and (b) show that the gap can be embedded under the verbs \emph{hire} `think' and \emph{ee} `say', respectively.

\begin{exe}
\ex\label{ex:johnson:39}
\begin{xlist}
\ex
\glll Meredithga {wažą honąkipįnįhižą} ruwį anąga Bryanga wažątirehižą yaare.\\
Meredith-ga {wažą honąkipįnį-hižą} $\emptyset$-ruwį anąga Bryan-ga wažątire-hižą <ha>hire\\
Meredith-{\textsc prop} bicycle-{\textsc indef} {\textsc 3s/o}-buy and Bryan-{\textsc prop} car-{\textsc indef} {\textsc <1s>}think\\
\trans `Meredith bought a bicycle, and I think that Bryan bought a car.'
\ex
\glll Meredithga kšeehižą ruuc anąga Matejaga wažązihižą hihe.\\
Meredith-ga kšee-hižą $\emptyset$-ruuc anąga Mateja-ga wažązi-hižą <ha>ee\\
Meredith-{\textsc prop} apple-{\textsc indef} {\textsc 3s/o}-eat and Mateja-{\textsc prop} orange-{\textsc indef} <{\textsc 1s}>say\\
\trans `Meredith ate an apple and I said that Mateja ate an orange.'
\end{xlist}
\end{exe}

The examples in \REF{ex:johnson:40} show that Hocąk exhibits stripping. (40a) illustrates stripping with an object remnant after the coordinator `and', while the example in (40b) has an object remnant with disjunction. (40c) shows that stripping is also possible with a subject remnant after the coordinator.

\ea\label{ex:johnson:40}
\ea
\glll Sarahga šųųkhižą haja, anąga wijukhižą šge.\\
Sarah-ga šųųk-hižą $\emptyset$-haja anąga wijuk-hižą šge\\
Sarah-{\textsc prop} dog-{\textsc indef} {\textsc 3s/o}-see and cat-{\textsc indef} also\\
\trans `Sarah saw a dog, and a cat too.'
 \ex
\glll Meredithga hąąke kšeehižą ruucnį, nųnįge {waisgap sguu xuwuxuwuhižą}.\\
Meredith-ga hąąke kšee-hižą $\emptyset$-ruuc-nį nųnįge {waisgap sguu xuwuxuwu-hižą}\\
Meredith-{\textsc prop} {\textsc neg} apple-{\textsc indef} {\textsc 3s/o}-eat-{\textsc neg} but cookie-{\textsc indef}\\
\trans `Meredith didn't eat an apple, but a cookie.'
 \ex
\glll Bryanga {nįįtašjak taaxu} racgą, anąga Matejaga šge.\\
Bryan-ga {nįįtašjak taaxu} $\emptyset$-racgą anąga Mateja-ga šge\\
Bryan-{\textsc prop} coffee {\textsc 3s}-drink and Mateja-{\textsc prop} also\\
\trans `Bryan drank coffee, and Mateja too.'
\z
\z


As is the case in English and other languages, stripping is ungrammatical in embedded clauses in Hocąk. This is shown in (41a) for an object remnant with conjunction, (41b) for an object remnant with disjunction and (41c) for a subject remnant with conjunction.

\ea\label{ex:johnson:41}
\ea[*] {
\glll Matejaga wažahe gipį, anąga kšeexete šge yaare.\\
Mateja-ga wažahe $\emptyset$-gipį anąga kšeexete šge <ha>hire\\
Mateja-{\textsc prop} banana {\textsc 3s}-like and pineapple also <{\textsc 1s}>think\\
\trans Intended: `Mateja likes bananas, and I think (she likes) pineapple too.'}
\ex[*] {
\glll Bryanga hąąke wažątirehižą ruwįnį, nųnįge Cecilga {wažą honąkipįnįhižą} ee.\\
Bryan-ga hąąke wažątire-hižą $\emptyset$-ruwį-nį nųnįge Cecil-ga {wažą honąkipįnį-hižą} $\emptyset$-ee\\
Bryan-{\textsc prop} {\textsc neg} car-{\textsc indef} {\textsc 3s/o}-buy-{\textsc neg} but Cecil-{\textsc prop} bicycle-{\textsc indef} {\textsc 3s}-say\\
\trans Intended: `Bryan didn't buy a car, but Cecil said (he bought) a bicycle.'}
\ex[*] {
\glll Sarahga {waisgap sguuhižą} rook'į, anąga Bryanga Meredithga šge ee.\\
Sarah-ga {waisgap sguu-hižą} $\emptyset$-rook'į anąga Bryan-ga Meredith-ga šge $\emptyset$-ee\\
Sarah-{\textsc prop} cake-{\textsc indef} {\textsc 3s/o}-bake and Bryan-{\textsc prop} Meredith-{\textsc prop} also {\textsc 3s}-say\\
\trans Intended: `Sarah baked a cake, and Bryan said Meredith (baked a cake) too.'}
\z
\z

To conclude, the possibilities of having an ellipsis site in embedded contexts differ between English and Hocąk: VPE and gapping are not differentiated by embedding, but VPE and stripping are. However, gapping and VPE are still distinguished in adjunct clauses: as we saw in 3.3, VPE is grammatical in adjunct clauses \REF{ex:johnson:35} while gapping is not \REF{ex:johnson:36}.

\subsection{Presence of strict and sloppy readings}

Another characteristic of VPE is the fact that elided pronouns and anaphors give rise to two different identity readings. The English example in \REF{ex:johnson:42} is ambiguous. Under the so-called ``strict'' reading, the referent of the pronoun is identical in both the antecedent and elided VP. Under the ``sloppy'' reading, the pronoun behaves like a variable, and the referent of the anaphor is different for each conjunct.

\begin{exe}
\ex\label{ex:johnson:42}
Lily saw herself in the mirror, and Molly did too.\\
\textit{Strict reading:} Molly saw Lily in the mirror.\\
\textit{Sloppy reading:} Molly saw herself in the mirror.
\end{exe}

\citealt{Fortin2007} shows that stripping also gives rise to both strict and sloppy readings, as in the example in \REF{ex:johnson:43}. However, there is another possible interpretation for the second conjunct: the remnant can be interpreted as the object of the stripped clause. Fortin terms this additional reading the ``object reading.'' This third reading is unique to stripping constructions, as the remnant DP in VPE is always interpreted as the subject of the elided constituent. 

\begin{exe}
\ex\label{ex:johnson:43}
Lily saw herself in the mirror, and Molly too.\\
\textit{Strict reading:} Molly saw Lily in the mirror.\\
\textit{Sloppy reading:} Molly saw herself in the mirror.\\
\textit{Object reading:} Lily saw Molly in the mirror.
\end{exe}

In Hocąk, strict and sloppy readings are available with both VPE and stripping, while the additional ``object reading'' is possible only with stripping. In the examples in \REF{ex:johnson:44}, the antecedent VP contains a possessed object. (44a) is an instance of VPE, and the second conjunct has two possible interpretations: either Hunter visited Bryan's mother (strict reading) or Hunter visited his own mother (sloppy reading). In (44b), the second conjunct contains a stripping ellipsis site. Both the strict and sloppy readings are available, but the object reading is also possible: the sentence could mean that Bryan visited Hunter.

\begin{exe}
\ex\label{ex:johnson:44}
\begin{xlist}
\ex
\glll Bryanga hi'ųnį hiira homąkįnį anąga Hunterga šge ųų.\\
Bryan-ga hi'ųnį $\emptyset$-hii-ra $\emptyset$-homąkįnį anąga Hunter-ga šge $\emptyset$-ųų\\
Bryan-{\textsc prop} mother {\textsc 3s-poss-def} {\textsc 3s/o}-visit and Hunter-{\textsc prop} also {\textsc 3s}-do\\
\trans `Bryan visited his mother, and Hunter did too.'
\ex
\glll Bryanga hi'ųnį hiira homąkįnį anąga Hunterga šge.\\
Bryan-ga hi'ųnį $\emptyset$-hii-ra $\emptyset$-homąkįnį anąga Hunter-ga šge\\
Bryan-{\textsc prop} mother {\textsc 3s-poss-def} {\textsc 3s/o}-visit and Hunter-{\textsc prop} also\\
\trans `Bryan visited his mother, and Hunter too.'
\end{xlist}
\end{exe}

The examples in \REF{ex:johnson:45} show that the same readings are possible with reflexives. The second conjunct of (45a) contains a VPE gap, and it has two interpretations: either Meredith hit Mateja (sloppy) or Meredith hit herself (strict). In the stripping example in (45b), both strict and sloppy readings are possible, but so is the ``object reading'' under which Mateja hit Meredith.

\begin{exe}
\ex\label{ex:johnson:45}
\begin{xlist}
\ex
\glll Matejaga hokijį anąga Meredithga šge ųų.\\
Mateja-ga $\emptyset$<kii>hojį anąga Meredith-ga šge $\emptyset$-ųų\\
Mateja-{\textsc prop} {\textsc 3s}<{\textsc refl}>hit and Meredith-{\textsc prop} also {\textsc 3s}-do\\
\trans `Mateja hit herself, and Meredith did too.'

\ex
\glll Matejaga hokijį anąga Meredithga šge.\\
Mateja-ga $\emptyset$<kii>hojį anąga Meredith-ga šge\\ 
Mateja-{\textsc prop} {\textsc 3s}<{\textsc refl}>hit and Meredith-{\textsc prop} also\\ 
\trans `Mateja hit herself, and Meredith too.'
\end{xlist}
\end{exe}

Thus, while strict and sloppy readings are available with both VPE and stripping, stripping constructions have the additional reading that \citealt{Fortin2007} calls the ``object reading''.


\section{Deletion vs. pro-form analysis}\label{sec:johnson:4}

In the previous two sections, I presented arguments that Hocąk exhibits VPE. In this section, I further argue that VPE in Hocąk is derived by a deletion process. There are two main approaches to any given elliptical phenomena: the ellipsis site is either a deleted phrase or a null \emph{pro}-form. Here, I extend two arguments in favor of a deletion approach of English VPE to Hocąk. First, I show that extraction from the ellipsis site is possible. Second, I demonstrate that ellipsis sites can contain the antecedent to a pronoun outside of the gap.

\citet{FiengoMay1994} argue that English VPE is best analyzed as VP deletion. Their argument is based on cases of object extraction from the ellipsis site. In (46a), we see that the object of the second clause has undergone \emph{wh}-movement out of the ellipsis site. (46b) illustrates the phenomenon known as antecedent-contained deletion (ACD). In ACD constructions, the ellipsis site is found inside of a relative clause and is licensed under identity with the matrix VP. The head of the relative clause (here, \emph{everyone}) is the object of the elided VP. In both (46a) and (46b), movement of the object in the elided VP has taken place. This is not expected under a \emph{pro}-form analysis of VPE: a \emph{pro}-form has no internal structure, and thus there should be no object position inside the ellipsis site that the extracted object could have originated in. In contrast, a deletion analysis posits a full-fledged VP in the ellipsis site which undergoes deletion at a later stage in the derivation. In the examples in \REF{ex:johnson:46}, the object originated inside the elided VP, and underwent movement before deletion took place.

\begin{exe}
\ex\label{ex:johnson:46}
\begin{xlist}
\ex 
I know which book Max read, and {\textbf which book} Oscar didn't.
\ex 
Dulles suspected everyone who Angleton did. (\citealt[229]{FiengoMay1994}, 257)
\end{xlist}
\end{exe}

Likewise, Hocąk constructions with \emph{ųų} cannot be analyzed as a \emph{pro}-form, as object extraction is permitted. (47a) shows that focused elements can be extracted from the ellipsis site, and (47b) exemplifies the movement of \emph{wh}-words from the ellipsis site.\footnote{Like other Siouan languages, Hocąk is a \emph{wh}-in-situ language. However, \emph{wh}-words can undergo focus driven movement.}

\begin{exe}
\ex\label{ex:johnson:47}
\begin{xlist}
\ex 
\glll Meredithga waagaxra ruw\k{\i}, nųnįge {\textbf wiiwagaxra} hąąke ųųnį.\\
Meredith-ga waagax-ra $\emptyset$-ruwį, nųnįge wiiwagax-ra hąąke $\emptyset$-ųų-nį\\
Meredith-\textsc{prop} paper-\textsc{def} {\textsc 3s/o}-buy but pencil-\textsc{def} {\textsc neg} {\textsc 3s}-do-\textsc{neg}\\
\trans `Meredith bought the paper, but the pencil, she didn't.'
\ex 
\glll Jaagu Bryanga ruwįra yaaperesšąną, nųnįge {\textbf jaagu} Hunterga ųųra hąąke yaaperesnį.\\
Jaagu Bryan-ga $\emptyset$-ruwį-ra <ha>hiperes-šąną nųnįge jaagu Hunter-ga $\emptyset$-ųų-ra hąąke <ha>hiperes-nį\\
what Bryan-\textsc{prop} {\textsc 3s/o}-buy-\textsc{comp} <{\textsc 1s}>know-\textsc{decl} but what Hunter-\textsc{prop} {\textsc 3s}-do-\textsc{comp} {\textsc neg} <{\textsc 1s}>know-\textsc{neg}\\
\trans `I know what Bryan bought, but I don't know what Hunter did.'
\end{xlist}
\end{exe}

As the example in \REF{ex:johnson:48} shows, ACD is also grammatical in Hocąk. ACD would not be possible if \emph{ųų} were a \emph{pro}-form, since the head of the relative clause is the object of the elided VP.

\begin{exe}
\ex\label{ex:johnson:48}
\glll Bryanga ruwį, jaagu Meredithga ųųra.\\
Bryan-ga $\emptyset$-ruwį jaagu Meredith-ga $\emptyset$-ųų-ra\\
Bryan-\textsc{prop} \textsc{3s/o}-buy what Meredith-\textsc{prop} \textsc{3s}-do-\textsc{comp}\\
\trans `Bryan bought what(ever) Meredith did.'
\end{exe}

The second argument in favor of a deletion analysis of VPE in Hocąk comes from so-called ``missing antecedents.'' \citet{HankamerSag1976} demonstrate that the gap in English VPE constructions can contain the antecedent to a pronoun. In the non-elliptical example in (49a), the DP \emph{a camel} in the second conjunct serves as the antecedent for the pronoun \emph{it} in the third conjunct. In (49b), the VP in the second conjunct is elided, resulting in a missing antecedent for the pronoun \emph{it}. Nonetheless, the sentence is still grammatical. It is important to note that the instance of \emph{a camel} in the first conjunct cannot be the antecedent for the pronoun \emph{it}: as \REF{ex:johnson:50} shows, DPs under the scope of negation cannot serve as antecedents for pronouns.


\begin{singlespacing}
\begin{exe}
\ex\label{ex:johnson:49}
\begin{xlist}
\ex 
I've never ridden a camel, but Ivan's ridden a camel$_i$, and he says it$_i$ stank horribly.
\vspace{12pt}
\ex 
I've never ridden a camel, but Ivan has, and he says it$_i$ stank horribly. (\citealt[403]{HankamerSag1976})
\end{xlist}
\end{exe}
\end{singlespacing}

\begin{exe}
\ex[*] {\label{ex:johnson:50}
I've never ridden a camel$_i$, and it$_i$ stank horribly. (\citealt[404]{HankamerSag1976})}
\end{exe}

\citet{HankamerSag1976} argue that the grammaticality of the example in (49b) points to a deletion analysis of VPE. These facts are not readily explained under a \emph{pro}-form analysis: since the ellipsis site would not have internal structure at any point in the derivation, the elided VP in (49b) would never contain the antecedent for the following pronoun. 

Examples of VPE with missing antecedents are also grammatical in Hocąk. In (51a), the DP \emph{kšeexetehižą} `a pineapple' in the second conjunct is the antecedent for the null pronominal subject of the verb \emph{sguu} `sweet'. In (51b), the VP containing the antecedent is elided, and the resulting sentence is grammatical. Like English, a pronoun cannot find its antecedent in a negated clause \REF{ex:johnson:52}.

\begin{exe}
\ex\label{ex:johnson:51}
\begin{xlist}
\ex 
\glll Hąkaga kšeexetehižą haacnį, nųnįge Matejaga kšeexetehižą ruuc, anąga sguu ee.\\
hąkaga kšeexete-hižą $\emptyset$<ha>ruuc-nį nųnįge Mateja-ga kšeexete-hižą $\emptyset$-ruuc anąga $\emptyset$-sguu $\emptyset$-ee\\
never pineapple-{\textsc indef} {\textsc 3s<1s>}eat-{\textsc neg} but Mateja-{\textsc prop} pineapple-{\textsc indef} {\textsc 3s/o}-eat and {\textsc 3s}-sweet {\textsc 3s}-say\\
\trans `I never ate a pineapple, but Mateja ate a pineapple, and she said it was sweet.'
\ex 
\glll Hąkaga kšeexetehižą haacnį, nųnįge Matejaga ųų, anąga sguu ee.\\
hąkaga kšeexete-hižą $\emptyset$<ha>ruuc-nį nųnįge Mateja-ga $\emptyset$-ųų anąga $\emptyset$-sguu $\emptyset$-ee\\
never pineapple-{\textsc indef} {\textsc 3s<1s>}eat-{\textsc neg} but Mateja-{\textsc prop} {\textsc 3s}-do and {\textsc 3s}-sweet {\textsc 3s}-say\\
\trans `I never ate a pineapple, but Mateja did, and she said it was sweet.'
\end{xlist}
\end{exe}

\begin{exe}
\ex[*] {\label{ex:johnson:52}
\glll Hąkaga kšeexetehižą haacnį anąga sguu.\\
hąkaga kšeexete-hižą $\emptyset$<ha>ruuc-nį anąga $\emptyset$-sguu\\
never pineapple-{\textsc indef} {\textsc 3s<1s>}eat-{\textsc neg} and {\textsc 3s}-sweet\\
\trans `I never ate a pineapple, and it was sweet.'}
\end{exe}


Both the extraction facts and pronoun antecedent facts point to an analysis in which the contents of elided VPs in Hocąk are present syntactically, and that the omission of elided VPs is due to a deletion process.

\section{Conclusion}\label{sec:johnson:5}

In this paper, I examined an elliptical phenomenon that I argue instantiates VPE in Hocąk. This process targets all VP-internal material, including direct objects, indirect objects, result phrases, temporal adjuncts, locative adjuncts, comitatives, manner adverbs and complement clauses. VPE is conditioned by the presence of a licensing head, which I showed is the light verb \emph{ųų} in Hocąk. However, Hocąk VPE is constrained in that the antecedent verb must be active. I propose that this restriction is due to the fact that active \emph{v} is the licenser. This elliptical process displays many other traits that \citet{Goldberg2005} and \citet{Fortin2007} demonstrate are characteristic of VPE crosslinguistically. I also briefly discussed that Hocąk VPE should be analyzed as VP deletion, rather than a VP \emph{pro}-form. This paper constitutes the first in depth description of VPE in Hocąk, and contributes to the literature on the properties of VPE crosslinguistically. 

\section*{Acknowledgment}

First and foremost, I would like to thank Cecil Garvin for sharing his language with me and kindly answering my endless questions. Thanks also to Yafei Li, Bryan Rosen and Rand Valentine for discussion and comments at various stages of this project. A portion of this work was presented at the 87th Annual Meeting of the Linguistic Society of America. I thank the audience there for questions and comments. I would also like to thank an anonymous reviewer and Catherine Rudin for comments on an earlier draft of this paper. Lastly, thanks to Iren Hartmann for generously providing me access to her Lexique Pro Hocąk dictionary to assist with my fieldwork.

\section*{Abbreviations}
1, 2, 3 = first, second, third person; {\textsc comp} = complementizer; {\textsc decl} = declarative; {\textsc def} = definite; {\textsc fut} = future; {\textsc indef} = indefinite; {\textsc neg} = negative; {\textsc o} = object agreement; {\textsc pl} = plural; {\textsc poss} = possessive; {\textsc prop} = proper name; {\textsc pst} = past tense; {\textsc refl} = reflexive; {\textsc s} = subject agreement {\textsc stat} = stative verb.
 
 \printbibliography[heading=subbibliography,notkeyword=this]

\end{document}






