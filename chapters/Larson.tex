\documentclass[output=paper]{LSP/langsci}
\author{Rory Larson}
\title{Regular sound shifts in the history of Siouan}

\abstract{The team of contributors to the Comparative Siouan Dictionary (CSD) reconstructed a phonemic set for Proto-Siouan, together with the necessary reflexes to produce the actual speech sounds found in the various daughter languages.  Until recently, this system was common knowledge within the Siouanist community, since the participants in the CSD project were active as the leaders of that community, and were available to explain the predicted sound shifts.  With the passing, retirement, or disappearance of most of the CSD team, however, it seems that it might be useful to document the reconstructed system and its most important regular reflexes, as an aid to comparative studies.  This paper will rely primarily on the CSD, edited until his passing by Dr. Robert Rankin, to summarize the regular sound shifts known to have occurred in Siouan.  It will prioritize sound shifts in which separate phonemes or clusters have collapsed together to become indistinguishable in the daughter languages, since this is where interesting confusion is most likely to occur. KEYWORDS: [Siouan, proto-Siouan, sound shifts, regular reflexes, historical phonology]} 

\maketitle

\begin{document}

\section{Introduction}  
In 1984, a group of linguists studying Siouan languages began a  project under NEH and NSF sponsorship to assemble a comparative dictionary of the Siouan language family.  The principal investigator was David S. Rood. The team included Richard T. Carter, A. Wesley Jones and Robert L. Rankin as senior editors, along with Rood and John E. Koontz.  Together with Willem de Reuse, Randolph Graczyk, Patricia A. Shaw and Paul Voorhis, the dictionary team began their project at the Comparative Siouan Workshop held at the University of Colorado in 1984.  A number of other scholars, including Louanna Furbee, Jimm Goodtracks, Jill Hopkins Greer, Kenneth Miner, Carolyn Quintero, Kathleen Shea and Mark Swetland also contributed information.

This undertaking was huge.\footnote{For an enlightening and wryly humorous history of the project, see ``The Comparative Siouan Dictionary Project'' by David S. Rood and John E. Koontz, 2002, written by two of its principal participants.} The present writer regrettably turned down an offer by Richard Carter in about 1997 of an advance copy of the dictionary manuscript, on the assurance that it would be published within a year.  In fact, it was never completed to the editors' satisfaction or published in book form. Carter himself retired from active work in the Siouan field after around 2002, and Robert Rankin became the principal steward of the project.  In 2006, Rankin distributed a .pdf file of the manuscript as it stood so far to interested members of the Siouanist community, on condition that any further requests be submitted to himself or David Rood.  The file runs to nearly a thousand pages, and is full of working notes and comments about the various words and their relationships, mostly by Rankin.

Rankin, Carter, and their colleagues developed a sophisticated understanding of the phonological and phylogenetic relationships among the various groups of Siouan.  Until recently, they formed a body of respected linguistic ``elders'' who freely shared this lore on request with more junior scholars.  With the untimely passing of Robert Rankin in February of 2014, however, and the retirement, disappearance, or focus shift of most of the other leading members of the team, the framework they developed seems in danger of being forgotten by the Siouanist community.  This paper is intended to address that concern.  Drawing on working notes found throughout the CSD, as well as years of discussions on the Siouan List, it will attempt to summarize the model of Siouan phonology and its standard sound shifts built by Rankin and the CSD team, with occasional comments and additions from the writer.  All references to the CSD are to the 2006 version. (In May 2015, after this paper was completed, the most recent version was made available online at csd.clld.org (Rankin et al, 2015) where most of the original notes and comments can now be found.)


\section{The Siouan family tree}

The CSD recognizes four major branches of Siouan. In the far northwest is Missouri Valley, or Crow-Hidatsa, consisting of the Crow and Hidatsa languages. Next is Mandan, an isolate within Siouan. Third is Mississippi Valley Siouan, or ``MVS'', which itself has three branches. Fourth is Southeastern, or Ohio Valley Siouan, at the southeastern end of the Siouan span.

MVS branches into Dakotan, which includes the five ``Sioux'' dialects of Santee-Sisseton (Dakota), Yankton-Yanktonai, Teton (Lakota), Assiniboine and Stoney;\footnote{Parks and \citet{DeMallie1992} presents the results of a major dialect survey undertaken to clarify the relations among the various Dakotan dialects, a field in which considerable confusion had prevailed in the literature prior to their work.  The CSD itself is rather deficient in Dakotan material other than Lakota and Dakota, although it contains a few words from four other categories: Stoney, Assiniboine, Yankton, and ``Sioux Valley''.  The latter presumably refers to the Sioux Valley reservation in southwestern Manitoba, which Parks and DeMallie classify along with the Minnesota ``Dakota'' within the broad Santee-Sisseton dialect group.  The present paper will follow Parks and DeMallie in sub-classification of Dakotan.}  Winnebago-Chiwere, composed of Hooc\k{a}k and the Iowa, Oto and Missouria languages; and Dhegiha, comprising Omaha-Ponca, Kaw, Osage and Quapaw.  Southeastern Siouan contains Biloxi and Ofo as one branch, and Tutelo and Saponi as another.

Catawba is the language most closely related to Siouan.  Though sound-historical relationships are not very clear, Catawba examples are often included in an entry's word list.  The language probably next most closely related is Yuchi (cf. Kasak 2016 (in this volume)), but few examples of Yuchi are given.


\section{The Reconstructed Proto-Siouan phoneme set}

The CSD team recognizes eight vowels for Proto-Siouan, five oral and three nasal, which were distinguished also by length:\footnote{Rankin, et al., 1998. ``Proto Siouan Phonology and Grammar''.}
\begin{center}
\begin{tabular}[t]{c c c c c c c}
\lsptoprule
i & & u & & \k{i} & & \k{u}\\

e & & o \\

 & a & & & & \k{a} \\
\lspbottomrule
\end{tabular}
\end{center}		

Basic stops are *p, *t, *k, and the glottal stop.  Proto-Siouan had a series of alveolar, palatal and velar fricatives: *s, *\v{s}, and *x, as well as *h.  It also had three resonants, *w, *r and *y.  Minimally, its consonant structure was as follows:
\begin{center}
\begin{tabular}[t]{c c c c c c c}
\lsptoprule
& Labial & Alveolar & Palatal & Velar & Glottal \\
\hline
Stops: & p & t & & k & \textipa{P} \\

Fricatives:  & & s & \v{s} & x & h \\

Resonants: & w & r & y \\
\lspbottomrule
\end{tabular}
\end{center}

Many of the consonants of Siouan have occurred in clusters, however, so the actual historical picture is more complex than this.  Stops can be adjoined to other stops in almost any order, all non-glottal stops and fricatives can be glottalized, aspiration (h) can occur either before or after stops, and combinations can occur involving fricatives, stops and resonants.  In particular, there exist two historical phonemes that manifest as either stops or resonants in the daughter languages, called ``funny w'' and ``funny r''.  We symbolize these sounds as *W and *R.  *R was found in Proto-Siouan, and *W in MVS only.  Rankin believed that *R was originally a combination of *r with a laryngeal, either *h or the glottal stop.

Notably, Siouan had no distinct nasal consonant series.  When *w or *r occurred in the environment of a nasal vowel, they usually manifested as |m| or |n|, respectively.

Accent in Proto-Siouan was normally on the second syllable of a word.

\section{Historical Siouan sound shifts}

One of the first sound shifts affecting Siouan was a process called ``Carter's Law'.  Wherever a simple stop, *p, *t or *k, occurred before the vowel of an accented syllable, the stop itself was more prominently ``marked'', either by lengthening it or by preaspirating it.  In the CSD, these are considered to be preaspirated.  Thus, *p, *t and *k become *hp, *ht and *hk before an accented syllable.  Since accent normally was on the second syllable of a word, these preaspirated stops and their derivatives are usually found inside the word rather than at the beginning.  When they are found at the beginning of a word, it may be an indication of a lost initial syllable.
\begin{center}
\begin{tabular}[t]{c c c c c }

Carter's Law: & *p\'V & > & *hp\'V  \\

& *t\'V & >  & *ht\'V \\

&  *k\'V & >  & *hk\'V\\
\end{tabular}
\end{center}

\subsection{Missouri Valley (Crow-Hidatsa) reflexes}

In Missouri Valley Siouan, loss of historical aspiration, loss of nasal vowels and the merger of *y with *r are the most sweeping transformations of the Proto-Siouan phonemic inventory.  Several other changes also occur.

\begin{itemize}
\item As in Mandan, Proto-Siouan aspiration is lost.\footnote{CSD 2006, p. 50, 85.} This notably includes the preaspirate series produced by the operation of Carter's Law.
\begin{center}				
\begin{tabular}[t]{c c c c c c }
Loss of aspiration: & *hp & > & *|p|	& < & 	*p \\
& *ht	 & > & *|t|	& <	 & *t \\
& *hk & > & *|k|	 & < & *k \\
\end{tabular}
\end{center}

\item Phonemic nasalization is completely lost.  The three Proto-Siouan nasal vowels merge with their oral counterparts, and neither vowels nor consonants are distinguished by nasality.\footnote{CSD 2006, p. 109.}
\begin{center}
\begin{tabular}[t]{c c c c c c }
Loss of nasal vowels: &	*\k{a} & >	& *|a| & < & *a \\
& *\k{i} & > & *|i| & < & *i \\
& *\k{u} & > & *|u| & < & *u \\
\end{tabular}
\end{center}

\item As in Mandan and Hooc\k{a}k, Proto-Siouan *y merges with *r.

\begin{center}
\begin{tabular}[t]{c c c c c c c}
*y/*r merger: & & *y	 & >  & *|r| &  <	 & *r 
\end{tabular}
\end{center}

\item Between vowels at the end of a word, *h is lost .

\begin{center}
\begin{tabular}[t]{c c c c c c c}
Loss of intervocalic *h: & & *V\textsubscript{1}hV\textsubscript{2} & > & *|V\textsubscript{1}V\textsubscript{2}|	
\end{tabular}
\end{center}

\item Rightward vowel exchange, in which the first two vowels of a word are swapped.\footnote{CSD 2006, p. 193, 788.}  Both Crow and Hidatsa show this feature, but not necessarily in the same words, which suggests that this change was spreading at the time Crow and Hidatsa separated.

\begin{center}
\begin{tabular}[t]{c c c c c c c}
Rightward vowel exchange:	 & & *CV\textsubscript{1}CV\textsubscript{2} & > & |CV\textsubscript{2}CV\textsubscript{1}|	
\end{tabular}
\end{center}
\end{itemize}

\subsubsection{Hidatsa reflexes}

Few changes are specific to Hidatsa.  There may be a few vowel shifts and cluster changes.  Proto-Siouan *w generally manifests as |m|.

\begin{itemize}
\item Short *o is raised to |u|:\footnote{CSD 2006, p. 137, 922.}	 \hspace{1em} *o	>	|u|
\item *xk becomes |hk|:\footnote{CSD 2006, p. 193.} \hspace{3.2em}  *xk	>	|hk|
\item *w becomes |m|: \hspace{4.1em} *w	>	|m|
\end{itemize}

\subsubsection{Crow reflexes}

Crow is more innovative.  The biggest change is complete loss of glottals, usually with lengthening of the following vowel.  Proto-Siouan *x manifests as |x\v{s}|.  Proto-Siouan *t becomes |s|.

\begin{itemize}
\item Glottalization is lost, but is reflected in the lengthening the following vowel, usually with rising pitch.\footnote{CSD 2006, p. 232.}

\begin{center}
\begin{tabular}[t]{c c c c c c c}
Loss of glottals:	 & & *(C)\textsuperscript{\textipa{P}}V & > & |(C)V\'V|
\end{tabular}
\end{center}

\item *x becomes |x\v{s}|:\footnote{	CSD 2006, p. 124.}	\hspace{1em} *x	>	|x\v{s}|
\item *t becomes |s|: \hspace{2em}  *t	>	|s|
\end{itemize}

\subsection{Mandan reflexes}

In Mandan, loss of historical aspiration\footnote{CSD 2006, p. 50.} and the merger of *y with *r are the most notable sound shifts, as well as a peculiar reversal of sibilants.\footnote{CSD 2006, p. 126.}

\begin{itemize}
\item As in Crow and Hidatsa, historical aspiration is lost, including the preaspirate series.

\begin{center}
\begin{tabular}[t]{c c c c c c }
 Loss of aspiration: & *hp & > & |p| & < & *p \\
& *ht & >	 & |t| & < & *t \\
& *hk & > & |k| & < & *k \\
\end{tabular}
\end{center}

\item As in Crow, Hidatsa, and Hooc\k{a}k, Proto-Siouan *y merges with *r.
\begin{center}
\begin{tabular}[t]{c c c c c c c}
*y/*r merger: & & *y	 & > & |r| & < & *r
\end{tabular}
\end{center}
\item Proto-Siouan *s and *\v{s} swap phonetic value.  *s becomes |\v{s}| and *\v{s} becomes |s|.

\begin{center}
\begin{tabular}[t]{c c c c c c }
*s/*\v{s} reversal: & & *s & > & |\v{s}| \\
& & *\v{s} & > & |s| \\
\end{tabular}
\end{center}

\item The cluster *sp metathesizes to become |ps|. More generally, there seems to be a usual, but not quite complete, constraint against having |p| as the second element of a cluster.\footnote{CSD 2006, p. 275. }
\begin{center}
\begin{tabular}[t]{c c c c c}
*sp metathesis:	 & & *sp & > & |ps|
\end{tabular} 
\end{center}

\item Before a consonant, the absolutizing or generalizing *wa- prefix loses its vowel through syncopation, and the *w becomes |p|.\footnote{CSD 2006, p. 793.}
\begin{center}
\begin{tabular}[t]{c c c c c }
*wa- syncopation: & & *waC & > & |pC|
\end{tabular}
\end{center}
\end{itemize} 

\subsection{MVS reflexes}

In Mississippi Valley Siouan (MVS), the fricatives are divided between a voiceless series and a voiced series.  This is also the only branch of Siouan in which the preaspirates are clearly distinguishable.  Another major transformation is the loss of short, unaccented vowels in the initial syllable,\footnote{CSD 2006, p. 10.} and the production of clusters that result from this syncopation.  This frequently involves the absolutizing or generalizing *wa- prefix, as well as the first person subject pronoun *wa\textsuperscript{1}- prefix.  Also, the *hr cluster becomes *ht,\footnote{CSD 2006, p. 199.} merging with the original preaspirate *ht.  For this group, we may restate the basic consonant set as follows:
\vspace{1em}
\begin{center}
\begin{tabular}[t]{c c c c c c c}
\hline \hline
& & Labial & Alveolar & Palatal & Velar & Glottal \\
\hline
Stops: & \\
& Simple:	& p	 & t & & k & \textipa{P} \\
& Preaspirate: & hp & ht	 & & hk \\
& Postaspirate: & ph & th & & kh \\
& Glottalized:	& p\textsuperscript{\textipa{P}}	& t\textsuperscript{\textipa{P}} & & k\textsuperscript{\textipa{P}} \\

Fricatives: & \\
& Voiceless: & & s	& \v{s}	& x	 & h \\
& Voiced:	& & z & \v{z} & \textipa{G} \\
& Glottalized:	& & s\textsuperscript{\textipa{P}} & \v{s}\textsuperscript{\textipa{P}}	 & x\textsuperscript{\textipa{P}} \\
Resonants: & \\
& Normal:	& w & r & y \\
& ``Funny'': & W & R \\
\hline \hline
\end{tabular}
\end{center}
\vspace{1em}

\begin{itemize}
\item The Proto-Siouan fricatives are divided between a voiced and a voiceless set, possibly according to phonological conditions.
\begin{center}
\begin{tabular}[t]{c c c c c c }
Voiced/voiceless fricative split: & *s	 &  >  & *|s| and *|z| \\
& *\v{s} & > & *|\v{s}| and *|\v{z}| \\
& *x & > & *|x| and *|\textipa{G}| \\
\end{tabular}
\end{center}

\item Proto-Siouan *pr merges with syncopated *w-r to become MVS *br.
\begin{center}
\begin{tabular}[t]{c c c c c c c}
*pr/*w-r syncopation: & & *w-r & > & *|br| & < & *pr
\end{tabular}
\end{center}

\item Syncopated Proto-Siouan *w-w usually becomes MVS *W.\footnote{CSD 2006, p. 164, 193, 213}
\begin{center}
\begin{tabular}[t]{c c c c c }
*w-w syncopation: & & *w-w & > & *W
\end{tabular}
\end{center}

\item Syncopated Proto-Siouan *\textit{wa}\textsuperscript{1}- used as the first person affixed pronoun `I', however, becomes MVS *m when it precedes *w or the glottal stop.\footnote{CSD 2006, p. 10.}
\begin{center}
\begin{tabular}[t]{c c c c c c c}
I-*\textit{wa}\textsuperscript{1}-w syncopation: & & *\textit{wa}\textsuperscript{1}-w & > & *m & < & *\textit{wa}\textsuperscript{1}-\textipa{P} 
\end{tabular}
\end{center}
\item Syncopated Proto-Siouan *w-C, where C is a voiceless contoid, becomes MVS *pC.\footnote{CSD 2006, p. 793.} 

\begin{center}
\begin{tabular}[t]{c c c c c c }
*w-C syncopation: & & *w-h & > & *ph \\
& & *w-t & > & *pt \\
\end{tabular}
\end{center}

\item Proto-Siouan *hr merges with preaspirate *ht to become *|ht|.\footnote{CSD 2006, p. 199.}
\begin{center}
\begin{tabular}[t]{c c c c c c c }
*hr/ht merger: & &*hr & > & *|ht| & < & *ht
\end{tabular}
\end{center}
\end{itemize}

\subsubsection{Dakotan reflexes}

In Dakotan, vowel length is lost.  Proto-Siouan *y manifests as aspirated |\v{c}h|.  So too do cases in which *r is preceded by *i.  Many inalienably owned nouns beginning with |\v{c}h| in Dakotan are explained as *r-initial stems preceded by the *i- of inalienable possession.  When *r stands alone without an adjacent consonant, it manifests as |y|.  When *k is preceded by a front vowel, it palatalizes to |\v{c}| in Dakotan.  Otherwise, the main sound shifts involve clusters.  In particular, Proto-Siouan or MVS preaspirates become postaspirates, merging with that series.\footnote{CSD 2006, p. 199, 269, 818.} The cluster *rh, which is important in a few words, becomes plain |h|.\footnote{CSD 2006, p. 165.} In Dakotan, clusters of two stops are frequent, and the cluster *wR becomes *|br|, merging this with the MVS *br series. In all Dakotan languages, *br manifests as |mn| before a nasal vowel.  Stoney and Assiniboine manifest *br and *kr the same regardless of environment, but these sounds alternate in the other three dialects according to whether the following vowel is oral or nasal.

\begin{itemize}
\item vowel length is lost: \hspace{4.2em} *VV	>	|V|	<	*V

\item *y- and *ir- merge as |\v{c}h-|: \hspace{2em} *y-	>	|\v{c}h-|	<	*ir-

\item *rV becomes |yV|:	 \hspace{5.3em} *rV	>	|yV|

\item *k after front vowel becomes |\v{c}|: \hspace{1em} *ik	>	|i\v{c}|

\hspace{14.2em} *ek	>	|e\v{c}|

\item preaspirates merge with postaspirates:

\begin{center}
\begin{tabular}[t]{c c c c c c }
*hp & > & |ph| & < & *ph \\
*ht	& > & |th| & < & *th \\
*hk & > & |kh| & < & *kh \\
\end{tabular}
\end{center}

\item *rh becomes |h|: \hspace{4em} *rh	>	|h|	<	*h

\item *wR merges with MVS *br: \hspace{1em}*wR	>	*|br|	<	*br
\item  Dakotan *br then manifests as |mn| before a nasal vowel:              *br\k{V}    >          *|mn\k{V}|
\end{itemize}

\subsubsubsection{Santee-Sisseton reflexes}

\begin{itemize}
\item  *br alternates by nasality: \hspace{1em} *brV		>	|md| or |bd|\footnote{In the CSD, ``Sioux Valley'' seems to agree with ``Dakota'' in practically everything except that the one case recorded of a Sioux Valley word with a *br reflex before an oral vowel shows this as |bd|, rather than the usual |md| for Dakota.}

\hspace{12em} *br\k{V}	>	|mn|

\item  *kr alternates by nasality:  \hspace{1.2em} *krV       >          |hd|

\hspace{12em} *kr\k{V}	>	|hn|

\item *R manifests as |d|: \hspace{4.5em} *R	>	|d|
\end{itemize}

\subsubsubsection{Yankton-Yanktonai reflexes}

\begin{itemize}
\item *br alternates by nasality: \hspace{1em} *brV	>	|bd| or |md|

\hspace{12em} *br\k{V} 	>	|mn|
\item  *kr alternates by nasality:  \hspace{1.2em} *krV       >          |kd| or |gd|

\hspace{12em} *kr\k{V} 	>	|kn| or |gn|

\item *R manifests as |d|: \hspace{4.5em} *R	>	|d|\footnote{The CSD records very few words of Yankton, none of which are useful here.  Parks and DeMallie, 1992, stress that the long-repeated claim that Yankton-Yanktonai is ``Nakota'', is false; their self-designation, when not misled by confused linguists, is ``Dakota'', which means that *R manifests as |d|, not |n|, in their dialect.  The true Nakotas are the Assiniboines and the Stoneys.  The authors clearly illustrate the *kr clusters for this group on pages 245-6, but do not include any *br clusters.  Words listed on the Yankton Reservation pedagogical website, Hubbeling, Cook, et al., show that *br before an oral vowel normaly manifests as |bd|, or perhaps sometimes as |md| or |mbd|.}
\end{itemize}

\subsubsubsection{Teton (Lakota) reflexes}

\begin{itemize}
\item  *br alternates by nasality:  \hspace{1em} *brV	>	|bl|

\hspace{12em} *br\k{V}	>	|mn|

\item  *kr alternates by nasality: \hspace{1em}   *krV       >          |gl|

\hspace{12em} *kr\k{V}	>	|gn|
\item *R manifests as |l|: \hspace{4.5em} *R	>	|l|
\item *tp becomes |kp|:\footnote{CSD 2006, p. 253, 265, 865.} \hspace{4.5em} *tp	>	|kp|	 <	*kp
 \end{itemize}
 
 \subsubsubsection{Assiniboine reflexes}
 
 \begin{itemize}
 \item  *br always manifests as |mn|:     \hspace{1.5em}  *br       >          |mn|\footnote{The CSD contains only a few words of Assiniboine, or Nakota.  Parks and DeMallie, 1992, demonstrate that *R becomes |n|, and that *kr manifests as |kn| before both oral and nasal vowels.  The preliminary Assiniboine text developed by Parks, DeMallie and Cumberland, 2012, contains words with *br clusters showing that these manifest as |mn| regardless of the nasality of the following vowel.}
 \item *kr always manifests as |kn|:       \hspace{1.6em}        *kr       >          |kn|
 \item *R manifests as |n|: \hspace{5.3em} *R	>	|n|
 \end{itemize}
  
 \subsubsubsection{Stoney reflexes}
 
 \begin{itemize}
 \item Fricatives tend to shift forward:  \hspace{.7em}   *s         >          |\textipa{T}|

\hspace{13.6em}      *\v{s}         >          |s|

\item Free simple stops are voiced:  \hspace{1.6em}   *p        >          |b|

\hspace{13.6em}      *t         >          |d|

\hspace{13.6em}       *k        >          |g|

\item *br always manifests as |mn|: \hspace{1.6em}    *br       >          |mn|

\item *kr always manifests as |hn|:  \hspace{1.7em}   *kr       >          |hn|

\item *R manifests as |n|:  \hspace{6em}     *R       >          |n|

 \item *tk becomes |kt|:  \hspace{7em}      *tk       >          |kt|       <          *kt
 \end{itemize}
 
\subsubsection{Winnebago-Chiwere reflexes}

Hooc\k{a}k and IOM share a number of innovations.  The cluster *pt merges with preaspirate *ht.  Proto-Siouan simple stop *p before vowels becomes |w|.  Generally, it appears that the postaspirate stop series merges with the simple stop series.  The *rh cluster also merges with the simple stop *t.  As in Dhegiha, the presumed cluster *wR always seems to reduce to simple *|R|.  

Both languages show a sporadic tendency to nasalize vowels that are not nasal in other MVS languages.\footnote{CSD 2006, p. 50.} Both of them also sometimes replace a glottal stop with a glottalized |t\textsuperscript{\textipa{P}}| following *i.  This could be interpreted as an epenthetic |y| being naturalized as *|r|, and then converted to |t| before the glottal stop.  The problem is that the glottal stop itself would seem to be in the way of obtaining the epenthetic |y| in the first place.  Rankin suggests that in verb paradigms, the glottal stop is lost in conjugated forms, and that the conjugated form was recast back into the main verb.

\begin{itemize}
\item *pt becomes |ht|: \hspace{7em} *pt	>	*|ht|	<	*ht
\item *rh becomes *|d|: \hspace{7em} *rh	>	*|d|	<	*t
\item *wR merges with MVS *R: \hspace{3em} *wR	>	*|R|	<	*R
\item *p becomes |w| before a vowel: \hspace{1em} *pV	>	*|wV|
\item *i\textsuperscript{\textipa{P}}V verbs become |it\textsuperscript{\textipa{P}}V|:	\hspace{4em} *i\textsuperscript{\textipa{P}}V	>	*|it\textsuperscript{\textipa{P}}V|
\end{itemize}
 
\subsubsubsection{Hooc\k{a}k reflexes}

Hooc\k{a}k shows quite a number of sound shifts of its own.  One of its biggest is that it levels vowel length on monosyllables: the vowel of all monosyllabic words is long.\footnote{CSD 2006, p. 303, 797.} Further, it creates many new monosyllabic words by dropping the trailing final vowel, especially *-e.  On top of this, it creates an extra syllable within an obstruent-sonorant cluster, by inserting the vowel that follows the cluster into the spot between the two consonants as well.\footnote{Helmbrecht, Johannes. ``The Accentual System of Hoc\k{a}k'', p. 123-4. This Hooc\k{a}k pattern of back-filling an obstruent-sonorant cluster with the following vowel is known as `Dorsey's Law'.}  As in Mandan, Crow and Hidatsa, Proto-Siouan *y merges with *r.  The *t series, except for glottalized t *t\textsuperscript{\textipa{P}}, is affricated into a |\v{c}| series.  An *r\textsuperscript{\textipa{P}} cluster may become either |t\textsuperscript{\textipa{P}}| or |k\textsuperscript{\textipa{P}}|.\footnote{CSD 2006, p. 816-817.}

\begin{itemize}
\item The vowels in monosyllables are always long.
\begin{center}
Long monosyllables: \hspace{1em} *CV(C)    >	*|CVV(C)|     <	*CVV(C)
\end{center}
\item Trailing final vowels are often dropped, making even more monosyllables.
\begin{center}
Trailing vowels dropped: \hspace{1em} 	*CVCe    >	*|CVVC|        <	*CVVCe
\end{center}
\item Obstruent plus sonorant clusters are broken up by insertion of the following vowel between the obstruent and the sonorant.
\begin{center}
Back insertion of vowel:	 \hspace{1em}  *C\textsubscript{obst}C\textsubscript{son}V\textsubscript{1}    >	*|C\textsubscript{obst}V1C\textsubscript{son}V\textsubscript{1}|
\end{center}
\item As in Crow, Hidatsa, and Mandan, Proto-Siouan *y merges with *r.
\begin{center}
*y/*r merger: \hspace{1em} *y	>	|r|	<	*r
\end{center}
\item *t series affricatizes: \hspace{1em} *t  >  |\v{j}|  

\hspace{9.2em} *ht	 >  |\v{c}|  

\item *r\textsuperscript{\textipa{P}} becomes |t\textsuperscript{\textipa{P}}| or |k\textsuperscript{\textipa{P}}|: \hspace{1em} *r\textsuperscript{\textipa{P}}	>	|t\textsuperscript{\textipa{P}}| or |k\textsuperscript{\textipa{P}}|
\item *R manifests as |d|:\footnote{Helmbrecht, Johannes. Phonetics and Phonology, p. 1, and personal communication. This sound is written `t' in the CSD and in modern Wisconsin Hoo\k{a}k orthography. But the `t' is voiced in prevocalic and intervocalic position, where it is the reflex of *R.} \hspace{2em} *R	>	|d|
\end{itemize}

\subsubsubsection{IOM reflexes}

A distinctive features of IOM is its forward shifting of the fricatives.  Siouan *s becomes |\textipa{T}|, and *\v{s} becomes |s|.\footnote{CSD 2006, p. 245. } In clusters of *k before a fricative, the |k| is replaced by a glottal stop.\footnote{CSD 2006, p. 857.}  As in Kaw and Osage, the *t-series, including *t\textsuperscript{\textipa{P}}, is affricatized before a front vowel *i or *e.  Initial *o- regularly becomes |u-|.\footnote{CSD 2006, p. 893.}

One of the most interesting features of IOM is its treatment of the Proto-Siouan *y phoneme.  As in several other Siouan languages, Proto-Siouan *y merges with another phoneme.  Uniquely to IOM, however, the *y words are split about evenly between which other phoneme they merge with.  Some of them merge with Siouan *r, as in Hooc\k{a}k, Mandan, Crow and Hidatsa.  Others remain |y|, but these are joined by MVS *\v{z}, which itself becomes |y| in IOM.  The fact that many IOM *y fail to merge with *r is mentioned in the CSD, but the significance of the counter-merger of these *y with MVS *\v{z} seems not to have been noticed.  For IOM only, we must consider the *y phoneme to be two distinct phonemes, *y\textsubscript{1} and *y\textsubscript{2}.

\begin{itemize}
\item Fricatives shift forward:
\begin{center}
\begin{tabular}[t]{c c c c c c }
*s & > & |\textipa{T}| \\
*\v{s} & > & |s| \\
\end{tabular}
\end{center}

\item *k before fricative becomes |\textsuperscript{\textipa{P}}|: \hspace{1em}*kS > |\textsuperscript{\textipa{P}}S|
\item Initial *o- becomes |u-|: \hspace{4em} *o- 	>	|u-|
\item *R manifests as |d|: \hspace{6em} *R	>	|d|
\item *y\textsubscript{1} merges with *r as |r|: \hspace{ 4em} *y\textsubscript{1}	>	*|r|	<	*r
\item *y\textsubscript{2} merges with *\v{z} as |y|: \hspace{4em} *y\textsubscript{2}	>	|y|	<	\v{z}

\item *t-series affricates before *i/*e: \hspace{1em} *ti  > *|\v{c}i| 

\hspace{14em} *te  >  *|\v{c}e| 

\hspace{14em} *hte  >  *|h\v{c}e|

\hspace{14em} *t\textsuperscript{\textipa{P}}e >  *|\v{c}\textsuperscript{\textipa{P}}e| 

\hspace{14em} etc. 

\end{itemize}

\subsubsection{Dhegiha Reflexes}

Dhegiha is characterized by substantial shifts and mergers in its vowel structure.  The nasal Proto-Siouan vowel *\k{u} merges with *\k{a}, producing a variably pronounced low back vowel with minimal rounding.  The oral vowel *u also shifts forward to become |\"u|.  In Dhegiha, Siouan *y merges completely with MVS *\v{z}.  Unlike the other MVS languages, the preaspirate stops do not merge with another stop series.  In most Dhegiha languages, these manifest as `tense', or double-long unaspirated stops, but in Osage they manifest as preaspirates.  Proto-Siouan *rh becomes |th|.\footnote{CSD 2006, p. 165.} MVS stop clusters collapse into a single stop, of the preaspirate series.  The clusters *ks and *ps become |s|, and the clusters *k\v{s} and p\v{s} become |\v{s}|.\footnote{CSD 2006, p. 64, 123, 222, 849.} Siouan *xw becomes |ph|.\footnote{CSD 2006, p. 180.} As in Winnebago-Chiwere, the presumed cluster *wR always seems to reduce to simple *|R|.

\begin{itemize}
\item *\k{u} merges with *\k{a}: \hspace{3.1em} *\k{u}	>	|\k{a}|	<	*\k{a}
\item *u becomes *|\"u|:	 \hspace{4.1em} *u	>	|\"u|
\item *y merges with MVS *|\v{z}|: \hspace{1em} *y	>	|\v{z}|	<	*\v{z}
\item *rh merges with *th: \hspace{3em} *rh	>	|th|	<	*th
\item *xw merges with *ph:	\hspace{ 3em} *xw	>	|ph|	<	*ph
\item *ps and *ks merge with *s: \hspace{1em} *ps	>	|s|	<	*s

\hspace{12em} *ks	>	|s|	<	*s
\item *p\v{s} and *k\v{s} merge with *\v{s}: \hspace{1em} *p\v{s}	>	|\v{s}|	<	*\v{s}

\hspace{12em} *k\v{s}	>	|\v{s}|	<	*\v{s}
\item *wR merges with MVS *R: \hspace{1em} *wR	>	*|R|	<	*R
\item Stop clusters merge with preaffricate stops (general pattern):	
\begin{center}
\begin{tabular}[t]{c c c c c c }
*pt & > & *|ht| & < & *ht \\
*pk	& > & *|hk| & < & *hk \\
*tp & > & *|ht| & < & *ht \\
*tk & > & *|ht| & < & *ht \\
*kp	& >	 & *|hp|	& <	& *hp \\
*kt	& >	& *|ht| & < & *ht \\
\end{tabular}
\end{center} 
\end{itemize} 
 
\subsubsubsection{Omaha-Ponca reflexes}

Omaha and Ponca carry the vowel reorganization begun in Dhegiha even further.  Dhegiha *\"u, from Siouan *u, now loses its rounding and merges completely with Siouan *i.  Behind it, the Siouan *o vowel is raised to |u|.  Siouan *R manifests as |n|, thereby merging with the |n| from Siouan *r before a nasal vowel.  The plain Siouan glottal stop disappears, while the glottalized velar clusters *k\textsuperscript{\textipa{P}} and *x\textsuperscript{\textipa{P}} both reduce to |\textsuperscript{\textipa{P}}| as a neo-glottal stop.  The preaspirate stop series manifest as tense, while simple stops are voiced.  The postaspirate *ph usually, but not always, reduces to |h|.  The Siouan *r phoneme manifests as what I call `ledh', a quick, smooth, flip of the tongue from an apical |l| to edh and off the back of the front teeth.  Linguists generally indicate it with the edh symbol, \textipa{D}, though l and r would be equally reasonable choices.  Additionally, an entire series of new stops is being generated from a custom of affricating the t-series stops as a ``baby talk'' method of suggesting smallness or cuteness.

\begin{itemize}
\item Dhegiha *\"u merges with *i: \hspace{1em} *\"u	>	|i|	<	*i
\item *o becomes |u|: \hspace{6em} *o	>	|u|
\item *R manifests as |n|: \hspace{4.2em} *R	>	|n|	<	*n < *r
\item *\textsuperscript{\textipa{P}} disappears: \hspace{6.2em} *V\textsuperscript{\textipa{P}}V	>	|VV|
\item *k\textsuperscript{\textipa{P}} and x\textsuperscript{\textipa{P}} become |\textsuperscript{\textipa{P}}|:	\hspace{3.2em} *k\textsuperscript{\textipa{P}}	>	|\textsuperscript{\textipa{P}}|	<	*x\textsuperscript{\textipa{P}}
\item *ph usually becomes |h|: \hspace{2.2em} *ph	>	|h|

\item Free simple stops are voiced:	\hspace{1em}*p	>	|b|

\hspace{13em} *t	>	|d|

\hspace{13em}*k	>	|g|

\item Preaspirate stops are tense: \hspace{1em} *hp	>	|pp|

\hspace{12.5em} *ht	>	|tt|
					
\hspace{12.5em} *hk	>	|kk|

\item Diminutive t-series transform:	\hspace{1.1em} |d|	dim.>	|\v{j}|

\hspace{14em} |t|	dim. >	|\v{c}|
					
\hspace{14em} |tt|	dim. >	|\v{c}\v{c}|

\hspace{14em} |th|	dim. >	|\v{c}h|

\hspace{14em} |t\textsuperscript{\textipa{P}}|	dim.>	|\v{c}\textsuperscript{\textipa{P}}|
\end{itemize}

\subsubsubsection{Kaw-Osage reflexes}

Kaw and Osage share a characteristic of dropping the velar stop from the *kr cluster and replacing the cluster with |l|.  It seems that both of them also merge the glottalized fricatives *s\textsuperscript{\textipa{P}} and *\v{s}\textsuperscript{\textipa{P}} into a glottalized dental/alveolar affricate |c\textsuperscript{\textipa{P}}| (|ts\textsuperscript{\textipa{P}}|).\footnote{CSD 2006, p. 856.} As in IOM, the *t-series, including *t\textsuperscript{\textipa{P}}, is affricatized before a front vowel *i or *e.  

\begin{itemize}
\item *kr drops the velar stop: \hspace{3em} *kr	>	|l|
\item *s\textsuperscript{\textipa{P}} and *\v{s}\textsuperscript{\textipa{P}} merge as |c\textsuperscript{\textipa{P}}|: \hspace{3em} *s\textsuperscript{\textipa{P}}	>	|c\textsuperscript{\textipa{P}}|	<	*\v{s}\textsuperscript{\textipa{P}}

\item *t-series affricates before *i/*e: \hspace{1em} *ti	>	*|\v{c}i|

\hspace{14em} *te	>	*|\v{c}e|

\hspace{14em} *hte	>	*|h\v{c}e|

\hspace{14em} *t\textsuperscript{\textipa{P}}e	>	*|\v{c}\textsuperscript{\textipa{P}}e|

\hspace{14em} etc.
\end{itemize}

4.3.3.2.1  Kaw reflexes
\vspace{1em}

Kaw agrees with Omaha and Ponca in voicing the free simple stops and in pronouncing the aspirated stops as tense.  In Kaw, Siouan free *r manifests as |y|.

\begin{itemize}
\item Free *r manifests as |y|: \hspace{1em} *r	>	|y|
\item *R manifests as |d|: \hspace{3em} *R	>	|d|
\end{itemize}

4.3.3.2.2  Osage reflexes
\vspace{1em}

In Osage, the preaspirate series is pronounced with preaspiration, and the free simple stops are voiceless.  Siouan free *r manifests as edh or ledh (\textipa{D}).  *ph manifests as |p\v{s}|.\footnote{ CSD 2006, p. 64.} 

\begin{itemize}
\item Free *r manifests as |\textipa{D}|: \hspace{1em} *r	>	|\textipa{D}|
\item *R manifests as |t|: \hspace{3em} *R	>	|t|
\item *ph manifests as |p\v{s}|: \hspace{2em} *ph	>	|p\v{s}|
\end{itemize}

\subsubsubsection{Quapaw reflexes}

In Quapaw, free Siouan *r manifests as |d|.  It seems that simple stops sometimes become tense.\footnote{CSD 2006, p. 833.} The Siouan cluster *p\textsuperscript{\textipa{P}} is reduced to plain glottal stop.\footnote{CSD 2006, p. 831.} 

\begin{itemize}
\item Free *r manifests as |d|: \hspace{5em} *r	>	|d|
\item Simple stops may become tense:	\hspace{1em} *t	>	|tt|
\item *p\textsuperscript{\textipa{P}} becomes |\textsuperscript{\textipa{P}}|: \hspace{8em} *p\textsuperscript{\textipa{P}}	>	|\textsuperscript{\textipa{P}}|
\end{itemize}

\subsection{Southeastern Siouan reflexes}

Very few systematic sound shifts characterize Southeastern Siouan as a whole.  One mentioned in the CSD is the loss of glottalized fricatives.  Also, it seems that *\v{s} usually affricatizes to |\v{c}|.

\begin{itemize}
\item Fricatives lose glottalization and merge with the corresponding plain form.  Thus, *S\textsuperscript{\textipa{P}} > *|S|.\footnote{CSD 2006, p. 856.}

\begin{center}
\begin{tabular}[t]{c c c c c c c c c}
Fricatives deglottalize: & & *s\textsuperscript{\textipa{P}}	& >	 & *|s| & < & *s \\
& & *\v{s}\textsuperscript{\textipa{P}}	& > & *|\v{s}| & < & *\v{s} \\
& & *x\textsuperscript{\textipa{P}}	 & > & *|x| & < & *x \\
\end{tabular}
\end{center}

\item *\v{s} then usually becomes |\v{c}|:\footnote{CSD 2006, p. 99, 126, 167, 827, 931.} \hspace{1em} *\v{s}	>	*|\v{c}|	
\end{itemize}

\subsubsection{Tutelo reflexes}

Tutelo seems conservative.  The only significant change noted involves the Proto-Siouan *\v{s} and *s phonemes.

\begin{itemize}
\item *\v{s} normally becomes |\v{c}|: \hspace{4.2em} *\v{s}	>	|\v{c}|
\item Sometimes, *\v{s} becomes |s|:\footnote{CSD 2006, p. 912.} \hspace{2.9em} *\v{s}	>	|s|	<	*s
\item *s is indifferently pronounced:\footnote{CSD 2006, p. 54, 931.} \hspace{1.2em} *s	>	|s| or |\v{s}|
\end{itemize}

\subsubsection{Ofo-Biloxi reflexes}

In Ofo and Biloxi, initial Proto-Siouan *w or *h before a vowel is lost.\footnote{CSD 2006, p. 7, 223, 817, 929.} 

\begin{itemize}
\item *wV becomes plain |V|: \hspace{1em} *wV	>	|V|
\item *hV becomes plain |V|: \hspace{1em} *hV	>	|V|
\end{itemize}

\subsubsubsection{Biloxi reflexes}

Biloxi is fairly conservative.  Final *-i and *-e merge as |-i|,\footnote{CSD 2006, p. 901.}  and the glottal stop often appears as |h|.\footnote{CSD 2006, p. 103.} 

\begin{itemize}
\item Final *-e merges with *-i: \hspace{2.1em} *-e	>	|-i|	<	*-i
\item The glottal stop becomes |h|: \hspace{1em} *\textsuperscript{\textipa{P}}	>	|h|
\end{itemize}

\subsubsubsection{Ofo reflexes}

Ofo is much more innovative.  Proto-Siouan *y becomes aspirated |\v{c}h|,\footnote{CSD 2006, p. 85, 242.} as in Dakotan.  The CSD suggests that Proto-Siouan *\v{s} before an accented syllable may have become aspirated |\v{c}h| as well.\footnote{CSD 2006, p. 827.} Notably, the *s fricative changes to |f|, while Proto-Siouan *x shifts forward to become a neo-|\v{s}|.\footnote{CSD 2006, p. 174, 299.} Several of the Proto-Siouan clusters do interesting things as well.  In the case of a glottalized stop consonant, the glottal stop seems to shift forward so that it releases prior to the stop.  This phenomenon is suggested in Ofo transcriptions as a neutral vowel appearing epenthetically in front of the stop that in other languages is known to be glottalized.  The stop consonant is then aspirated as well.

\begin{itemize}
\item *y becomes |\v{c}h|: \hspace{5em} *y	>	|\v{c}h|
\item Accented *|\v{s}| becomes |\v{c}h|: \hspace{1em} *\v{s}\'V	>	*|\v{c}\'V|	>	|\v{c}h\'V|
\item *s becomes |f|: \hspace{6em} *s	>	|f|
\item *x becomes |\v{s}|: \hspace{6em} 	*x	>	|\v{s}|
\item *hs becomes |fh|:\footnote{CSD 2006, p. 174, 299.}  \hspace{4.8em} 			*hs	>	|fh|
\item *Cr becomes |l|:\footnote{CSD 2006, p. 90.}  \hspace{5em} 			*Cr	>	|l|
\item *C\textsuperscript{\textipa{P}} becomes |\textipa{@}Ch|:\footnote{CSD 2006, p. 229, 232.}  \hspace{3.9em} *C\textsuperscript{\textipa{P}}	>	|\textipa{@}Ch|
\end{itemize}

\section*{Abbreviations}

CSD = Comparative Siouan \citet{Dictionary2006}; IOM = Iowa-Otoe-Missouria; MVS = Mississippi Valley Siouan.


\section*{References} 

\newenvironment{reflist} {\begin{list} {} {\listparindent -.25in
\leftmargin .3in} \item \ \vspace{-.3in} } {\end{list} }

\begin{reflist}

Helmbrecht, Johannes.  2011. The Accentual System of Hoc\k{a}k.  In Jan Wohlgemuth and Michael Cysouw (eds.), \textit{Rara and Rarissima}, 117-143. Berlin: De Gruyter Mouton.

Helmbrecht, Johannes.  Phonetics and Phonology.  Ms, privately shared with this writer.

Hubbeling, Gail, LaVena Cook, et. al.  2015.  Dakoteyah Wogdaka!.  URL: https://www.nativeshop.org/learn-dakota.html.  Ihanktowan (Yankton) Sioux Reservation, South Dakota.

 Kasak, Ryan. 2016. A Distant Genetic Relationship Between Siouan-Catawban and Yuchi. In Rudin, Catherine and Bryan Gordon (eds.), \textit{Advances in the study of Siouan languages and linguistics}. Berlin: Language Science Press. 5-39.

Parks, Douglas R. and Raymond J. DeMallie. 1992. Sioux, Assiniboine, and Stoney Dialects: A Classification. \textit{Anthropological Linguistics} 34(1). 233-255.

Parks, Douglas R., Raymond J. DeMallie, and Linda A. Cumberland.  2012.  Stories told by George Shields, Sr. In Assiniboine Narratives from Fort Belknap, Montana.  Preliminary Edition.  American Indian Studies Research Institute: Indiana University.

Rankin, Robert L., Richard T. Carter, A. Wesley Jones, and other contributors.  2006.  Comparative Siouan Dictionary.  Unpublished textbase, shared as a .pdf file.

Rankin, Robert L., Richart T. Carter, and A. Wesley Jones.  1998.  Proto Siouan Phonology and Grammar.  In Li, Xingzhong, Luis Lopez and Tom Stroik (eds.), \textit{Proceedings of the 1997 Mid-America Linguistics Conference}.   Columbia: Linguistics Area Program of the University of Missouri-Columbia.  366-375.

Rankin, Robert L.; Richard T. Carter; A. Wesley Jones; John E. Koontz; David S. Rood and Iren Hartmann, eds. 2015. \textit{Comparative Siouan Dictionary}. Leipzig: Max Planck Institute for Evolutionary Anthropology. (Available online at http://csd.clld.org, Accessed on 2015-09-25.) 

Rood, David S. and John E. Koontz.  2002.  The Comparative Siouan Dictionary Project. In Frawley, William, Kenneth C. Hill and Pamela Monroe (eds.),  \textit{Making dictionaries: Preserving indigenous languages of the Americas}. Berkeley: University of California Press. 259-281.  

\end{reflist}
\end{document}