\addchap{Introduction to Part I}
\begin{refsection}

%content goes here
The relative degree of ``genetic'' relatedness of the major branches of the Siouan language family is quite well-established: the Catawban languages split off first, then the Missouri Valley Siouan languages, followed by the Southeastern Siouan and Mississippi Valley Siouan languages. Among the latter branch, the Dakotan languages split off first, followed by the Dhegiha and Jiwere-Ho-Chunk sub-branches. Rankin (\citeyear{Rankin1988}; \citeyear{1998}; \citealt{RankinEtAl1998}; etc.) contributed much to developing and supporting this understanding, alongside advances in rigorous application of the comparative method. Open questions include the possibility of relationships with Yuchi, Iroquoian languages and Caddoan languages, and areal connections. Chapters in Part 1 of this volume address some of these issues, as well as considering what we can learn from early attempts to write Siouan languages.

Ryan Kasak (``A distant genetic relationship between Siouan-Catawban and Yuchi'') argues that evidence exists to link Yuchi to the Siouan family. Though scarce and not fully conclusive, the evidence includes phonological and morphological correspondences strong enough to make a reasonable case for a genetic relationship between Yuchi and Siouan-Catawban. 

David Kaufman (``Two Siouan languages walk into a Sprachbund'') details the effects on Ofo and Biloxi of their participation in the Lower Mississippi Valley (LMV) language area; these languages share many lexical, phonetic and grammatical traits with genetically unrelated languages across the southeastern present-day United States. 

Rory Larson (``Regular sound shifts in the history of Siouan'') summarizes the current state of knowledge of the sound changes and correspondences distinguishing each branch and sub-branch of the Siouan family. These phonetic correspondences were worked out as part of the Comparative Siouan Dictionary project (recently made available on line as Rankin et al 2015), of which Bob Rankin was a central member. This concise catalog of all the known sound changes will be invaluable to anyone working with Siouan etymologies or cognates in the future.

Kathleen Danker (``Ba-be-bi-bo-ra: Refinement of the Ho-Chunk syllabary in the 19\textsuperscript{th} and 20\textsuperscript{th} centuries'') presents a glimpse into the process of formation of a Native writing system. This syllabary was inspired by one used by neighboring Algonquian peoples, then progressively changed to better represent the phonologically quite different Ho-Chunk language before being supplanted by writing in English, and by an alphabetic Ho-Chunk orthography.

Anthony Grant (``A forgotten figure in Siouan and Caddoan linguistics: Samuel Stehman Haldeman (1812--1880)'') is another study of writing systems, in this case an early attempt to write Kanza and Osage. Haldeman developed a universal phonetic orthography, one of several precursors to the modern IPA, which he tried out on a variety of languages including these two Siouan ones. While not entirely successful at representing all the sounds of Kanza and Osage, Haldeman's word lists do provide some insights into the pronunciation of these languages at a time earlier than other available information on them. 


\section*{References}

\printbibliography[heading=subbibliography]

\begin{reflist}



Rankin, Robert L., Richard T. Carter, and Wesley Jones. 1998. Proto-Siouan phonology and grammar. In Xingzhong Li, Luis Lopez and Tom Stroik, eds., \textit{Papers from the 1997 Mid-America linguistics conference}, 366-375. Columbia: University of Missouri-Columbia.



Rankin, Robert L.; Richard T. Carter; A. Wesley Jones; John E. Koontz; David S. Rood and Iren Hartmann, eds. 2015. \textit{Comparative Siouan Dictionary}. Leipzig: Max Planck Institute for Evolutionary Anthropology. (Available online at http://csd.clld.org, Accessed on 2015-09-25.)
\end{reflist}
\end{refsection}

