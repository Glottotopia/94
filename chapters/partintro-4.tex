\addchap{Introduction to Part IV}
\begin{refsection}

%content goes here
Part IV consists of three chapters which compare some area of grammar across the Siouan family. The phenomena vary -- coordination constructions, intonational and morphosyntactic variations as indices of information structure, and the expression of possession -- but the approach of all three authors is similar in that they compare the facts of a number of different Siouan languages, from different branches of the Siouan family, within the framework of a broader, cross-linguistic typology.

Catherine Rudin (``Coordination and related constructions in Omaha-Ponca and in Siouan languages''), rather than demonstrating a shared characteristic of the Siouan family, shows that there is no common or typical Siouan coordination pattern. Words meaning `and' or `or' are not cognate among the various Siouan languages, sometimes even within one branch of Siouan. The syntactic structure of conjoining constructions varies from language to language as well, sometimes involving structures other than true coordination. This suggests that syntactic coordination was not found in Proto-Siouan, but developed later in the individual languages.

Bryan James Gordon (``Information-Structural Variations in Siouan Languages'') looks at the frequently ignored area of information structure, using a corpus study of variations in constituent order, reduction, intonational contour and other markings corresponding to specific types of topic, focus and linking relations across a range of Siouan languages. The chapter includes discussion of the methodology the author developed for coding intonational and information structures.


Johannes Helmbrecht (``NP-internal possessive constructions in Hooc\k{a}k and other Siouan languages'') surveys at least one language from each branch of the Siouan family and examines how they express different types of possession. Besides Ho-Chunk, the languages treated are Crow, Hidatsa, Mandan, Lakota, Osage and Biloxi. These languages have many similarities, but differ in how four basic morphosyntactic possessive types match up with semantic categories of possession (ownership, attribution of property, kinship, etc.).


%\section*{References}

%\newenvironment{reflist} {\begin{list} {} {\listparindent -.25in
%\leftmargin .3in} \item \ \vspace{-.3in} } {\end{list} }

%\begin{reflist}

%\end{reflist}
%\printbibliography[heading=subbibliography]

\end{refsection}

