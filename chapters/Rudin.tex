\documentclass[output=paper]{LSP/langsci}
\author{Catherine Rudin}
\title{Coordination and related constructions in Omaha-Ponca and in Siouan languages}

\abstract{Syntactic constructions expressing semantic coordination vary widely across the Siouan language family. A case study of possible coordinating conjunctions in Omaha-Ponca demonstrates that distinguishing coordination from other means of expressing `and' relations is a non-trivial problem. A survey of words translated as `and,' `or,' or `but' in Siouan languages leads to the conclusion that neither coordinating conjunctions nor the syntactic structures containing them are reconstructable across the Siouan family. It is likely that Proto-Siouan lacked syntactic coordination. KEYWORDS: [Siouan, coordination, subordination, conjunction, comitative]}

\maketitle

\begin{document}

\section{Introduction}

All languages have ways of expressing additive, disjunctive, and adversative relations among entities or propositions. In European languages these relations are expressed by two distinct syntactic means: coordination and subordination. In Siouan languages these two types of conjunction construction are also present, but the distinction between them is less robust, less clear, and coordination may not have existed at all historically. Neither coordinating conjunctions (`and,' `or,' `but') nor the syntactic structures containing them are reconstructable across the Siouan family.

I begin this examination of coordination in Siouan by defining coordination and discussing some of the issues involved in distinguishing coordinate from subordinate conjunction (\sectref{sec:rudin:2}). This is followed in \sectref{sec:rudin:3} by a case study of additive coordination and coordinate-like constructions in Omaha-Ponca, the Siouan language with which I am most familiar. \sectref{sec:rudin:4} is a survey of available data on coordination across all branches and most of the languages in the Siouan language family, with a summary table. \sectref{sec:rudin:5} concludes the chapter with a discussion of the (non)universality of coordination constructions and some speculations on the history and origins of coordination in Siouan.

\section{Issues in defining and identifying coordination}
\subsection{The syntax of coordination}

Traditionally, coordination is a structure of the type shown in \REF{ex:rudin:1}:

\begin{exe}
\ex 	
\Tree [ .X [ .X ] [ .X ] ] 
\end{exe}	

In this structure two or more conjuncts of identical grammatical category together constitute a larger syntactic unit of the same category. These conjuncts might for instance be noun phrases, verbs, or clauses:

\begin{exe}
\ex 		
\begin{minipage}[b]{0.2\textwidth}
\Tree
[ .NP [ .NP ] [ .NP ] ]
\end{minipage}
\begin{minipage}[b]{0.2\textwidth}
\Tree
[ .V [ .V ] [ .V ] ]
\end{minipage}
\begin{minipage}[b]{0.2\textwidth}
\Tree
[ .CP [ .CP ] [ .CP ] ]
\end{minipage}
\end{exe}

The conjuncts are sisters, of equal syntactic status, in a symmetrical constituent. Neither coordinate is subordinate to or included in the other. Equality of status is seen by coordinate NPs bearing the same case and triggering plural agreement in languages where those categories are overtly marked. In addition coordinate phrases resist extraction (\citealt{Ross1967}, Coordinate Structure Constraint), and any movement out of them must be ``across the board'' movement out of all the conjuncts. Thus in English when two pronouns are coordinated, as in \REF{ex:rudin:3}; they must both be nominative in subject position, they require a plural verb, and they cannot be separated.

\begin{exe}
\ex \begin{xlist}
\ex  {[She and I] were chosen.}
\ex[ * ] {[She and me]}
\ex[ * ] {[She and I] was chosen.}
\ex[ * ] {[She] was chosen [and I].}
\end{xlist} 
\end{exe}

This contrasts with a non-coordinate construction like that in \REF{ex:rudin:4}, in which the two pronouns are different cases, the verb is singular, agreeing with only the first pronoun, and the subordinate portion of the construction can be moved.

\begin{exe} 
\ex \begin{xlist}
\ex {[She] [along with me] was chosen.}
\ex  {[She] was chosen [along with me].}
\end{xlist}
\end{exe}

Coordinate constructions may or may not contain an overt coordinating conjunction, a word translating as \textit{and, or, but}, etc. If there is one, it may occur between the conjuncts, or after the last one, or may be repeated (before or after each conjunct).

\begin{exe}
\ex			
\begin{minipage}[b]{0.2\textwidth}
\Tree
[ .X [ .X ] [ .conj ] [ .X ] ]
\end{minipage}
\begin{minipage}[b]{0.2\textwidth}
\Tree
[ .X [ .X ] [ .X ] [ .conj ] ]
\end{minipage}
\begin{minipage}[b]{0.2\textwidth}
\Tree
[ .X [ .X ] [ .conj ] [ .X ] [ .conj ] ]
\end{minipage}
\end{exe}

In recent theories of syntax (i.e. Minimalism), coordinate structures are instead treated as asymmetric constructions headed by the coordinator: ``CoordP'' or ``\&P,'' or the similar ``Boolean Phrase'' structure argued for by \citet{Munn1993}. This type of structure is adopted partly for theory-internal reasons such as Kayne's Linear Correspondence Axiom (\citeyear{Kayne1994}), but also for reasons having to do with intonation, ellipsis, and other phenomena which often suggest that the conjunction is more closely associated with one conjunct than with the other. See \citet{Citko2011} for detailed discussion. Under this view coordinate structures look something like those in \REF{ex:rudin:6}; presumably Siouan languages, being strongly head-final, would tend to have the left-branching variant shown on the right:

\begin{exe}
\ex	
\begin{minipage}[b]{0.3\textwidth}
\Tree
[ .\&P [ .XP ] [ .\&$'$ [ .\& ] [ .XP ] ] ]
\end{minipage}
\begin{minipage}[b]{0.3\textwidth}
\Tree
[ .\&P [ .\&$'$ [ .XP ] [ .\& ] ] [ .XP ] ]
\end{minipage}
\end{exe}

Issues of whether the conjunction forms a constituent with either the preceding or following X, and whether there is such a thing as a Coordination Phrase, are obviously important if one is concerned with distinguishing ``true'' coordination from other constructions such as comitatives which have similar meanings. Under the ``\&P'' analysis coordination has a syntactic configuration much like comitative or subordination structures, with one conjunct higher than the other, making it less straightforward to explain the distinctive behavior of coordinate structures, as well as less clear what criteria distinguish coordinate from subordinate structures. Numerous works have wrestled with these issues theoretically and across languages, e.g. \citet{Wesche1995} and  \citet{FabriciusHansenRamm2008}. I lack data to deal with such questions in most of the Siouan languages, so the exact structure of apparently coordinate phrases is left vague in what follows. Detailed research within each language will be needed to sort it out.

It is likely that many of the structures which translate ``and/or/but'' in various Siouan languages are actually not coordinate. Several other types of syntactic constructions often express semantic coordination. These include at least the following: \REF{ex:rudin:1} comitatives (prepositional phrases or subordinate clauses expressing ``accompaniment'' or a ``with'' relation); \REF{ex:rudin:2} adverbial clauses with temporal or other subordinate relations to a matrix clause (``when,'' ``although,'' ``having done X,'' etc.); \REF{ex:rudin:3} simple listing of nouns, verbs, or clauses (that is, concatenation of separate items which do not form a larger constituent of any kind, sometimes with elements meaning ``too,'' ``also,'' ``furthermore,'' ``however'' or a phrase which sums them up (``both,'' ``all''); \REF{ex:rudin:4} co-subordinate or clause-chaining constructions, (see e.g. \citealt{Graczyk2007}; \citealt{Boyle2007}).

There are a number of problematic coordination constructions in languages of the world, for instance a coordinator analyzed as a transitive verb in a Papua New Guinean language (\citealt{BrownDryer2008}),  partial/covert coordination of the \textit{nie s Ivan} 'we with Ivan' = `Ivan and I ' type in Slavic (e.g. \citealt{McNally1993}; \citealt{Larson2014}), special treatment of commonly linked items \citep{Waelchli1995}, and overlaps with serial constructions (\citealt{Carstens2002}). I do not deal with these specifically, but mention them just as a further reminder that the syntax of coordination is not necessarily a simple issue. For a useful typological overview of coordination, see \citet{Haspelmath2007}; other general treatments include \citet{Johannessen1998} and  \citet{VanOirsouw1987}. 

\subsection{The semantics of coordination}

Coordinators join constituents with diverse semantic relations, though the semantic aspects of coordination have received less attention than its syntax. Different authors use widely varying terminology for the meanings coordination can express; see for instance Citko's \REF{ex:rudin:2011} discussion of Andrej Malchukov's system of classification of coordination constructions into semantic types. Among the terms used in the literature are \textit{Additive, Adversative, Comitative, Consecutive, Concessive, Contrastive, Correction, Disjunctive, Mirative} and others.

In the cursory survey of the Siouan data below I will for the most part ignore issues of semantics beyond the gross level of meaning indicated by being translated in a grammar or dictionary as `and' versus `but' or `or' --- roughly Additive, Adversative, and Disjunctive. From the data available it is often not clear precisely what range of meanings are covered by a given conjunction. Semantic classification of the conjunctions will require detailed investigation of usage in each individual language, and will surely interact with numerous factors, including modality, adverbial modifiers, same or different subject of conjoined clauses, and so on. I leave this entire area for future research. For the present I simply list all elements which seem to translate `and,' `but,' or `or' in any of their meanings.

\subsection{Identifying lexical coordinators}

Another issue is that some of these lexical items, although they translate English coordinators, may in fact not be coordinators. This is yet another area which provides fertile ground for future, deeper research into each individual Siouan language. Coordinating conjunctions can be difficult to distinguish from sentence-initial or sentence-final elements (complementizers, discourse particles, switch-reference markers, and other clause-linking morphemes), and from comitative or adverbial words. Coordinators often develop historically into sentence-initial or -final elements, presumably by way of a stage involving elided conjuncts. Historical change can go the other way too: as \citet{Mithun1988} and \citet{Stassen2000} both point out, many languages have coordinating conjunctions which are recently and transparently derived from various sources, including comitative prepositions, adverbial particles, aspect markers, and clausal (subordinating) conjunctions. This leads to situations in which the same word is sometimes a coordinator, sometimes not, and teasing apart the two usages is tricky; such is the case for example with Bulgarian \textit{no}, \textit{ama}, \textit{ami} (\citealt{Fielder2008}) and Australian English \textit{but} (\citealt{MulderThompson2008}). Given the slipperiness of this issue in well-studied European languages, it should be no surprise that identifying coordinators can be problematic when dealing with spoken or inconsistently written data in a language with no tradition of written prose or punctuation conventions.

\section{Additive Coordination in Omaha-Ponca}

My interest in coordination in Siouan was sparked not by theoretical considerations but by a practical problem of language teaching. In an Omaha language class in 2002, a student's question of how to say `and' turned out to be unexpectedly hard to answer, with no one word corresponding to English \textit{and}. There are several clause connectors which are at least plausible candidates for coordinators in Omaha-Ponca, but nothing which syntactically coordinates nominal or other non-clausal phrases. To say things like `I have a cat and two dogs' or `That dress is black and white' our Omaha-speaking consultants rephrased with non-coordinate constructions, to the sometimes frustrated bewilderment of the English-dominant students.  In this section I examine various options for expressing additive coordination (`and') in Omaha-Ponca and consider whether they are true coordination or involve some other strategy such as adverbial modifiers or subordination. This case study illustrates both the richness and complexity of the data and the difficulty of conclusively distinguishing coordination from non-coordinate structures in a Siouan language.

\subsection{Coordination of clauses: \textit{shi}  and similar words}  

    The word most commonly offered by Omaha consultants as a translation for `and' is \textit{shi}, which often occurs as an apparent sentence conjoiner, or at least a discourse link between sentences. Koontz (1984, p. 52) lists \textit{shi} along with \textit{ki}, \textit{goⁿ}, \textit{goⁿki}, \textit{oⁿska}, and \textit{egithe} in a table of ``sentence introducers'' culled from James Owen Dorsey's 19th century Omaha and Ponca materials; the same words are found in my field recordings from 100 years later.  It is an open question whether these words start a new sentence or not; i.e. whether the structure is [S \textit{shi} S] or [S][\textit{shi} S], with [\textit{shi} S] constituting a separate sentence.\footnote{David Rood (pc) points out that [S shi] [S] might be a more expected split into two sentences in a verb final language, but shi is not sentence final, in written texts or spoken prosodic contours.}  Dorsey apparently considered them to be the start of a new sentence, but it is unclear why. Presumably he heard a preceding pause, or speakers when dictating to him tended to pronounce \textit{shi} with the following sentence. But there is often a pause or break before a conjunction in English as well, sentences do begin with coordinating conjunctions (in spite of prescriptive prohibitions), and in the more recent view of coordination, the conjunction does form a tighter unit with one of the joined clauses. Even if we assume the entire string [S \textit{shi} S] is a single sentence, it is unclear whether the two smaller sentences so joined are syntactically coordinated or one subordinate to the other. Omaha has no clear markers of subordination that I know of (e.g. no nonfinite verb forms).  

The precise meaning of \textit{shi} is another issue:  Koontz states that \textit{shi} differs from the other ``introducers'' in that it has a meaning of `again' or `marks repetition', but this meaning is not always apparent to me. \textit{Shi} sometimes seems to indicate repetition, but not always.  In the examples below,\footnote{These examples are from my field tapes, recorded in the late 1980s and 1990s, in Macy Nebraska. I am grateful to the National Science Foundation and Wenner-Gren Foundation for support, and to the speakers quoted here, Clifford Wolfe Sr., Bertha Wolfe, and Mary Clay, for sharing their language with me. The orthography used in this paper is the ``Macy Standard'' spelling used at Umoⁿhoⁿ Nation School and the University of Nebraska.}   \textit{shi} (boldfaced) seems to mark not so much repetition as simple additive coordination semantics --- `and also' --- or even contrast, as in \REF{ex:rudin:8} or \REF{ex:rudin:10}.  In some discourses \textit{shi} strings together several sentences or clauses in a row, as in \REF{ex:rudin:10} and \REF{ex:rudin:11}.  Example \REF{ex:rudin:11} in particular is a fairly extended discourse in which nearly every sentence after the first starts with \textit{shi}, and the discourse is a list of items, with no sense of repetition except the continued idea of praying for something. Note that \textit{shi} cooccurs with other ``sentence introducers,'' for example, \textit{goⁿki} (in \REF{ex:rudin:10} and \REF{ex:rudin:11}), and with arguably subordinating adverbial \textit{ki} (in \REF{ex:rudin:11}).\footnote{`?' in examples marks words which were unclear when transcribing field tapes or whose meaning is unknown. Since the morphological breakdown of most words is immaterial for the purposes of the paper, glosses are not necessarily morpheme-by-morpheme. Clitics are separated with an equal sign.}

\ea
\gll Th\'ishti  xt\'awithe.  \textbf{Shi} th\'ishti xt\'oⁿthathe.  \'Eshti  xt\'oⁿtha=i  ge \textbf{shi}  w\'ishti xt\'aathe.\\
 	you   	\textsc{1sgA}.like.2P  and 	you     	2A.like.{1sgA}  	s/he   	like.1P=\textsc{prox}  	?  	and  	I.too  	\textsc{sgA}.like.3P\\
\trans `I like you.  And you like me.  She likes me and I like her too.'

\ex 
\gll  Zhiⁿg\'a  ama \'agudishti \'udon  w\'anoⁿ\textipa{P}oⁿ=noⁿ, \textbf{shi}  \'agudishti  w\'anoⁿ\textipa{P}oⁿ=bazhi=noⁿ.\\
children 	the 	some        	good  	listen.to.\textsc{1plP=hab} 	and some listen.to.\textsc{1plP}=\textsc{neg=hab}\\
\trans `Some of the children are good; they listen to us, but some of them don't listen to us.'

\ex 
\gll  \textbf{Shi}  g\'oⁿki 	sha\'oⁿ 	ama ... sha\'oⁿ  	x\'e=ta=i  \'a=bi=ama.\\          
    	and  	then 	Sioux 	the 	 ... Sioux  	bury=\textsc{fut=prox} 	say=\textsc{prox=quot}\\
\trans `And the Sioux ...  His wish was for the Sioux to bury him.'

\ex 
\gll  G\'oⁿki 	\textbf{shi} 	g\'a=tʰe  oⁿg\'ahi  ki    \textbf{shi} 	wachʰ\'igagha ama  sh\'oⁿ-gagha=i=tʰe 	ki 	\textbf{shi} 	sh\'oⁿshoⁿ    \textbf{shi} 	zhu\'awagthe 	agth\'e=tʰa=ama.\\
    	then 	and that=the \textsc{1plA}.go.there when 	and dancers the end-do=\textsc{prox=evid} when 	and right.away and 	together 	took.\textsc{1plP}.home=\textsc{evid=aux}\\
\trans `We would go there but as soon as the dancers quit they took us right home.'
\z 

\ea
\ea
\gll  Wak\'oⁿda 	thiⁿkʰe 	shti  	bth\'aha=ta=miⁿkʰe\\
	god          	the       	too  	\textsc{1sgA}.pray=\textsc{fut=1sg.aux}\\
\trans `I'm going to pray to God.'
\ex
\gll	\textbf{Shi} g\'age 	iⁿd\'adoⁿ th\'e 	am\'a 	n\'ikashiⁿga 	am\'a 	shti ew\'ewaha=tʰe\\
and 	that   what     	this 	the 	person    	the 	too 	\textsc{1sgA}.pray.for.it=\textsc{evid}\\
\trans `I (will) pray for the people who had these things.'

\ex
\gll	\textbf{Shi}  um\'oⁿhoⁿ 	ti    	thoⁿ 	shti 	ag\'iwahoⁿ.\\
and 	Omaha  	house 	the 	too 	\textsc{1sgA}.pray.for.it.\textsc{refl}\\
\trans `And I (will) pray for my Omaha camp/village.'  (i.e. for the present-day reservation)
\ex
\gll	\textbf{Shi} tʰ\'oⁿwoⁿgtha d\'uba  \'edi moⁿth\'iⁿ um\'oⁿhoⁿ shti ew\'ewaha. \\         
and town several there 3A.walk Omaha  	too 	\textsc{1sgA}.pray.for.3P\\
\trans `And I (will) pray for the Omaha who are in various cities.'  (i.e. off reservation)
\ex
\gll	G\'age \textbf{shi} 	gah\'i 	nikashiⁿga.\\
	this 	and 	chief  	person\\
\trans `And for the council.'
\ex
\gll	\textbf{Shi} uzh\'oⁿge  	oⁿg\'athe 	dshtoⁿ.\\
	and  	road/path 	\textsc{lA}.go  	maybe\\
\trans `And for the path we will take.'  (i.e. for our lives)
\ex
\gll	Aw\'oⁿhoⁿ 	eg\'oⁿ  	\'e=ta=miⁿkʰe\\
	\textsc{1sgA}.pray 	thus 	that=\textsc{fut=1sg.aux}\\
\trans `I will pray for those things.' 
\z
\z

The other ``sentence introducers'' listed by Dorsey and Koontz include \textit{ki}, \textit{goⁿ}, \textit{goⁿki}, and \textit{kigoⁿki}, all meaning `and, and then'. Their distribution is similar to that of \textit{shi}; both in Dorsey's texts and in mine, they occur written at the beginning of sentences as well as joining two sentences or clauses, and they indicate a range of connections between those clauses, sometimes temporal and sometimes not. 

\subsection{Coordination of non-sentential categories: does it exist?}

\textit{Shi} and the other sentence conjoiner/introducers generally do not occur in conjoining contexts other than linking sentences. That is, they appear not to coordinate nominals or other non-clausal categories (though see \REF{ex:rudin:35} below).  In the case of nominals, several patterns occur, generally consisting of a string of NPs with a word meaning something like `also' at the end, sometimes with some element between the individual NPs as well.  

Koontz (1984, p. 201) gives the formula \underline{NP, NP \textit{ethoⁿba}} for conjoined nominals in Dorsey.  This pattern is found in modern materials as well.  Example \REF{ex:rudin:12} is a sentence from the story ``Jimmy and Blackie,'' translated into Omaha as a school booklet in the 1980s, and \REF{ex:rudin:13} is an example from a conversation I recorded in 1990. \textit{Ethoⁿba} is etymologically related to the number two (\textit{noⁿba}) and probably best treated as an element meaning `both' or `the two of them' instead of as a conjunction.

\begin{exe}
\ex 
\gll  Iⁿnoⁿha  	akʰa, iⁿdadi    \textbf{ethoⁿba} 	thethudi 	gthiⁿ 	e=shti.  \\
	my.mother 	the  	my.father 	also        	here       	live   	they=too\\
\trans `My mom and also my dad, they live here too.'

\ex 
\gll  Ivan 	akʰ\'a Silas 	\textbf{\'ethoⁿba} uk\'ikizhi. \\         
Ivan the   Silas	 also       	brothers\\
\trans`Ivan and Silas, those two were brothers.'
\end{exe}

Ardis Eschenberg (pc) reports that the elders/language teachers at Umoⁿhoⁿ Nation school in the early 2000s generally used \underline{NP, NP \textit{shti}} for conjoined nominals.  I have found some examples of this too, but actually very few with this exact pattern.  Example \REF{ex:rudin:14} is one.  Most sentences with \textit{shti} in my data have variations on the pattern such as \textit{shti} after a single NP \REF{ex:rudin:15}, or repeated \textit{shti} \REF{ex:rudin:16}, \REF{ex:rudin:17}.   Note that \textit{shti} cooccurs with \textit{shi} in \REF{ex:rudin:16} to coordinate three NPs:  \underline{NP \textit{shti}, NP \textit{shti, shi} NP}.  In \REF{ex:rudin:17} the second conjunct looks like a postverbal afterthought. The word \textit{shti} `too, also' could perhaps be analyzed as a coordinator, but seems more likely to be an adverbial element, perhaps related to \textit{xti} `very'.

\ea
\gll  Ith\'adi, 	ih\'oⁿ  	akʰa 	\textbf{shti} 	g\'inita  	ezh\'e 	goⁿk\'i  	ith\'adi 	ama, 	 ih\'oⁿ 	akʰ\'a 	zh\'ugigtha=bazh\'i.\\
	his.father 	his.mother 	the 	too 	living 	?  	and  	his.father 	the 	his.mother 	the 	together=\textsc{neg}\\
\trans `His father and his mother are both alive, but his father and mother do not live 	together.'

\ex 
\gll Tim 	akʰ\'a 	iw\'ikoⁿ=ta=akʰa     Clifford \textbf{shti} uth\'aha 	uw\'ikoⁿ=ta-akʰa\\
Tim the	 \textsc{3A}.help.\textsc{1sgP}=\textsc{fut=3aux}  Clifford too 	? \textsc{3A}.help.\textsc{3P}=\textsc{fut=3aux}\\
\trans `Tim will help me and Clifford.'
  
\ex 
\gll Shi  n\'ikashiⁿga h\'utoⁿga  wa'\'u  \textbf{shti} sha\'oⁿ \textbf{shti} \textbf{shi} w\'axe d\'uba 	ed\'i 	atʰ\'i-ama.\\
and 	person  winnebago 	woman too sioux  too  and white 	some there \textsc{3A}.arrive-\textsc{pl.aux} \\
\trans `And a Winnebago woman, some Sioux, and some whites were also there.'

\ex	
\gll Shi 	w\'ondoⁿ 	ith\'adi  \textbf{shti } h\'oⁿdi 	ug\'ikitha  	ih\'oⁿ  	akʰ\'a  	\textbf{shti}.\\
	and 	both 	his.father 	too 	last.night 	\textsc{3A}.was.talking.to.\textsc{3P} 	his.mother 	the 	too\\
\trans`And last night he was talking to both his father and his mother.'
\z

In my elicited data conjoined nominals most often take the form \underline{NP (\textit{egoⁿ}), NP} \underline{\textit{shenoⁿ}}, with degree elements literally meaning `so much' or `that much' as in examples \REF{ex:rudin:18} through \REF{ex:rudin:24}. The awkward literal gloss with `as ... that extent' could perhaps be better rendered `as well as ... all of those'. In any case, this seems unlikely to be a coordinate construction.

\begin{exe}
\ex
\gll T\'eska 	tan\'uka 	\textbf{\'egoⁿ} 	wazh\'iⁿga 	\textbf{\'egoⁿ}	n\'u \textbf{sh\'enoⁿ} thatʰ\'e  xt\'aathe.\\
	cow   	meat    	as     	chicken   	as     	potato 	that.extent 	eat      	\textsc{1sgA}.like\\
\trans `I like to eat beef and chicken and potatoes.'    
 
 \ex
\gll Watʰ\'e 	zh\'ide 	\textbf{\'egoⁿ} 	hiⁿb\'e 	sk\'a    \textbf{	sh\'enoⁿ}  	bth\'iwiⁿ.\\
	dress   	red  	as   	shoe 	white 	that.extent 	\textsc{1sgA}.buy\\
\trans I bought a red dress and white shoes.'  

\ex
\gll Watʰ\'e  zh\'ide,  hiⁿb\'e	sk\'a,  wath\'ade  p\'ezhitu \textbf{sh\'enoⁿ} abthiⁿ.\\ 
dress   	red   	shoe 	white 	hat  	green 	that.extent 	\textsc{1sgA}.have \\
\trans `I have a red dress, white shoes, and a green hat.'   

\ex
\gll S\'ezi  	tʰe 	sh\'e  	\textbf{shenoⁿ} 	\'ahige 	oⁿg\'athiⁿ. \\
	orange 	the 	apple 	that.extent 	much  	\textsc{1plA}.have \\
\trans `We have plenty of (both) oranges and apples.'

\ex
\gll Mary akʰ\'a  	\textbf{\'egoⁿ} wi \textbf{sh\'enoⁿ} Macy ata oⁿg\'atha. \\
 Mary	the  as  	I 	that.extent  	Macy to 	\textsc{1plA}.go.there \\
\trans `Mary and I went to Macy.'

\ex
\gll John akʰ\'a \textbf{\'egoⁿ} Mary akʰ\'a \textbf{sh\'enoⁿ} Macy ata ah\'i=tʰe. \\
 John the 	as  Mary the 	that.extent  Macy to \textsc{3plA}.arrive.there=\textsc{evid} \\
\trans `John and Mary went to Macy.'
 
\ex 
\gll Tim akʰ\'a Cliffford \textbf{\'egoⁿ} wi \textbf{sh\'enoⁿ} iw\'ikoⁿ=ta=akʰa. \\
Tim the Clifford as 	I  that.extent  help.\textsc{1sgP}=\textsc{fut=3aux} \\
\trans `Tim will help Clifford and me.'
\end{exe}

This \underline{NP \textit{egoⁿ}, NP \textit{shenoⁿ}} pattern also occurs in bilingual booklets produced by the Umoⁿhoⁿ Nation school; the translations are from the booklets as well:
	
\begin{exe}	
\ex
\gll Jimmy 	akʰa \textbf{egoⁿ} Sabe akʰa \textbf{shenoⁿ} \\
Jimmy the  	as 	black 	the  	that.extent \\
\trans `Jimmy and Blackie' (title of booklet)

\ex 
\gll Nuzhiⁿga ga tʰoⁿ e=\textbf{egoⁿ} mizhiⁿga ga  tʰoⁿ e=shti 	\textbf{shenoⁿ}  \\
boy 	this 	the 	he=as   	girl   this the she=too  that.extent 	 \\

\textit{uwawakizhi}.
my.younger.siblings
\trans `This is my little brother and sister.' 	
\end{exe}	 
	
A more literal translation of \REF{ex:rudin:26} would be `Like this boy, this girl also, as a group they are my little siblings.' Another pattern combines the previous two:   \underline{NP \textit{egoⁿ}, NP \textit{shti}}; \REF{ex:rudin:27} is an elicited example from my field tapes, \REF{ex:rudin:28} a spontaneously produced sentence. 

\begin{exe}	
\ex
\gll Mary akʰ\'a \textbf{\'egoⁿ}  w\'i=\textbf{shti} Macy 	ata 	oⁿg\'atha. \\
Mary the as I=too Macy to 	\textsc{1plA}.go.there \\
\trans`Mary and  I went to Macy.'

\ex
\gll Ih\'oⁿ  wi\'axchi  \textbf{\'egoⁿ} ith\'adi  \textbf{shti} wi\'axchi. \\
their.mother  just.one so their.father too just.one \\
\trans `They have the same mother and the same father too.'
\end{exe}	

Simply juxtaposing a string of nominals is another coordination strategy, and quite a common one, though I will not give any examples. In fact, all of the nominal coordination patterns we have seen so far could be interpreted as simple listing of noun phrases, with some kind of focus element following one or more of the nominals and/or a summing-up element at the end of the nominal string. Given the lack of case marking and near-absence of number agreement  in Omaha-Ponca,\footnote{Third person plural is not audibly marked in many verbs, and in those where it is, it is homophonous with
proximate singular marking.} as well as the likely status of most if not all lexical noun phrases as adjuncts in this language, the usual tests for coordinate as opposed to other structures tend not to apply, and it is difficult to distinguish for example coordinate from comitative constructions.

 A final, very common way of expressing English `and' in situations involving two participants acting together is with the verb \textit{zhugthe} `be with, accompany, be together'. This verb sometimes occurs following two nouns which could be seen as coordinated but are probably just listed; \REF{ex:rudin:29} is more literally `Mary, John, being together they went to Macy.'

\begin{exe}	
\ex
\gll Mary akʰ\'a  John	Macy ata 	\textbf{zh\'ugthe} 	ah\'i. \\
Mary the John Macy to  together arrive.there \\
\trans `Mary went to Macy with John. /Mary and John went to Macy.'
\end{exe}

In non-elicited examples, there is almost never more than one lexical noun phrase with \textit{zhugthe}; instead one nominal is given and the other is understood as accompanying it. In \REF{ex:rudin:30} only the woman is mentioned; the other participant is already present in the discourse. In \REF{ex:rudin:31} the unmentioned participant is the speaker, and interestingly the verb is first person singular, not plural, indicating that the construction is definitely comitative and not coordination of an overt with a null NP.\footnote{In playback speakers commented that the second verb, \textit{atʰ\'i}, could have been \textit{oⁿg\'atʰi} (first person plural), like the verb of the next clause; \textit{zhu\'agithe} however would still be first person singular.}   

\begin{exe}	
\ex
\gll Agth\'i (i)tʰediki shi wa'\'u shtewiⁿ \textbf{zh\'ugthe} agth\'i=itʰe. \\
\textsc{3A}.came.home 	when and woman whatsoever together came.home=\textsc{evid} \\
\vspace{-2.5em}\trans `When he came home, he came home with a woman.' (He and some woman came home.)

\ex
\gll Wa'\'u  wiw\'ita T\'es\'oⁿwiⁿ \textbf{zhu\'agithe}  atʰ\'i, she=kʰe oⁿg\'atʰi. \\
woman 	my  White.Buffalo together.1\textsc{suus}  \textsc{1sgA}.arrive	this=the 	\textsc{1plA}.arrive \\
\trans `My wife White Buffalo and I are both here; we came here.' (more literally, `My wife White Buffalo, together with my own, I came here...')
\end{exe}

There are thus several ways of expressing semantic coordination of nominals in Omaha-Ponca, but none for which a strong case can be made that it is a syntactic coordinate construction or any clear candidate for a coordinating construction. The nominal ``coordination'' patterns above are all basically lists of NPs with the option of adding a word or words stressing repetition or accompaniment. The picture is even more dubious for adverbs, nominal modifiers, and other non-clausal constituent types. \citet{Koontz1984} does not mention conjunction of categories other than nominals. I did not think to elicit them in field work, and have not found naturally produced examples.  The kind of sentences my Omaha language class students wanted to say, like `I'm wearing a red and yellow shirt,' seem impossible to express without resorting to multiple clauses (`My shirt is red and it is also yellow.')

\subsection{Discussion: Once more on \textit{shi}}

Having concluded that Omaha-Ponca has no clear coordinating conjunction or coordination construction for non-clausal coordination, I return briefly to my best candidate for clausal coordinating conjunction, \textit{shi}. In section 2.1 I presented a number of examples of \textit{shi} apparently linking clauses together; however, it may actually be an adverbial of some sort, not a conjunction, in which case Omaha-Ponca would not have any true coordination, even of clauses. It often appears in positions other than clause-initial, most often preverbal, as in the following examples. Here it is clearly not conjoining anything, but does have an `again' sense:  

\begin{exe}	
\ex
\gll \'Oⁿba 	w\'ethabthiⁿ 	ki 	\textbf{shi}  	wat'\'exe=ta=ama.  \\
day  	third          	at 	and 	funeral=\textsc{fut=aux}  \\
\trans `There'll be another funeral Wednesday.'   (Wednesday again will be a funeral.)

\ex 
\gll Oⁿw\'oⁿthatʰoⁿ th\'ishtʰoⁿ=i 	tʰedi 	t\'apuska 	ta 	\textbf{shi}  h\'athe 	oⁿg\'akʰi. \\
\textsc{1plA}.eat  	finish=\textsc{prox} 	when 	school 	to 	and ?  	\textsc{1plA}.arrive.back \\
\trans `After dinner we went back (again) to the school.'

\ex
\gll \'Oⁿba 	wiⁿ 	Isht\'iⁿthiⁿkhe 	akʰ\'a \textbf{shi} ed\'i=bi=ama. \\
day	one 	Monkey 	the 	and 	there=\textsc{pl=quot} \\
\trans `One day Monkey was there (again), they say.'  (traditional story opening)
\end{exe}

However, it is possible that this is a different \textit{shi} from the sentence-coordinating one. Further research is obviously needed. My data contain a few examples in which \textit{shi} might be interpreted as conjoining nominal phrases, following the last in a string of NPs:  \underline{NP, NP \textit{shi}}. None are very convincing, however, and \textit{shi} in them can plausibly be taken as an adverbial expressing repetition. In \REF{ex:rudin:36}, for instance, fighting was a regular occurrence.

\begin{exe}
\ex
\gll Um\'oⁿhoⁿ kʰe sha\'oⁿ kʰe \textbf{shi}  w\'oⁿdoⁿ 	kik\'ina=noⁿ=i \\
Omaha    	the 	Sioux  	the 	and both \textsc{3A}.\textsc{refl}.fight=\textsc{hab}=\textsc{prox} \\
\trans `The Omaha and the Sioux tribes used to fight each other.'    
\end{exe}

This section thus concludes rather inconclusively: Omaha-Ponca apparently has no non-clausal coordination, and may or may not have coordination of clauses. 

\section{Siouan languages: An overview}

At this point we leave the details of Omaha-Ponca and turn to a shallow but broad survey of the Siouan family. In spite of limited data on many members of the family and the challenges of interpretation and analysis, there is quite a lot we can say about coordination in Siouan languages. In several of the languages coordination has been described in some detail. Nearly all of the languages have recorded equivalents of `and,' and many have equivalents for `or' or `but,' though their morpho-syntactic status is often unclear. In many of the languages coordination of clauses is different than coordination of noun phrases or other categories, as we saw in Omaha-Ponca. Perhaps the most interesting result of a survey of Siouan coordination is the lack of unity within the family. No coordinators are reconstructable, there are no widespread cognates, and strategies for expressing coordination differ from language to language. It appears likely that Proto-Siouan had no true coordination. In this section I briefly describe the data from each sub-branch of Siouan (starting with Dhegiha because it is most familiar to me; information on Omaha-Ponca is repeated in brief form for completeness). No examples are given in this section and no attempt is made to justify the lexical items given as (possible) coordinators; instead, anything mentioned in sources is listed.

\subsection{Dhegiha}
 
\textbf{Omaha-Ponca} (data from \citealt{DorseyND}, \citealt{Koontz1984}, \citealt{Rudin2003} and my own fieldwork)\footnote{These sources use several different orthographies. In the interest of consistency I have spelled all Omaha-Ponca words in the modern ``Macy Standard'' spelling.} has several ways of expressing `and'. As discussed above, different conjunctions are used to coordinate clauses and NPs. Clauses may be conjoined with \textit{ki}, \textit{goⁿ}, \textit{shi} `again, and then,' \textit{goⁿki}, \textit{kigoⁿki} `and then'. Dorsey considers \textit{ki} to be Ponca and \textit{goⁿ} to be Omaha; both of these are said to join ``substantive clauses''. \textit{Goⁿ} is likely the same as subordinating \textit{(e)goⁿ} `having (done),' related to postposition \textit{egoⁿ} `like, as'. \textit{Goⁿki} and \textit{kigoⁿki} are pretty clearly combinations of these two conjunctions. \textit{Shi} is perhaps the best candidate for a true coordinator, although it, like the others listed here, occurs most often sentence initially (conjoining the sentence to the preceding discourse semantically if not syntactically). NPs are occasionally joined by \textit{goⁿ}; this may actually be the postposition mentioned above. More commonly two NPs are followed by \textit{edoⁿba/\'ethoⁿba} `also, both;' literally `the two of them'. A string of three or more NPs may be followed by \textit{edabe} `also'. Two or more NPs can be followed by \textit{shti} `too'. Although Dorsey does not list it, one of the most common strategies for coordinating NPs in my data is \textit{egoⁿ ... shenoⁿ} `both ... and;' literally `as ... that-extent', Ardis Eschenberg (p.c.) finds \textit{egoⁿ ... thoⁿzhoⁿ} used in the same way. The most common translation of `and' with NPs is clearly not syntactic coordination: a comitative construction with the verb \textit{zhugthe} `be with'. Simple juxtaposition (listing) of conjuncts with no conjunction is common for both S and NP coordination. `Or' and `but' in Omaha-Ponca are formed with the `and' conjunctions for joining clauses, and to the best of my knowledge do not exist at all for NPs. Dorsey lists \textit{goⁿ ... ite ki} `either ... or', \textit{shoⁿ doⁿste} `either-or, perhaps', and \textit{doⁿste} at end of clause `or' (the latter two in Dorsey's slip file). `But' is commonly expressed by \textit{shi} `and' connecting two clauses, the second of which is negative or contrasts in some way.

\textbf{Osage} (data from \citealt{Quintero2004}) coordinates NPs using \textit{\'ee\textipa{D}\k{o}\k{o}pa} `the two of them' following two or more NPs. (Compare Omaha-Ponca \textit{ethoⁿba}.) Verb agreement suggests that this is a true coordination structure; however, it is possible that the two NPs are appositive and the plural verb actually agrees with pronominal \textit{\'ee-}. Another possible NP coordinator is \textit{\v{s}ki} `also'. Clauses are coordinated by juxtaposition without a conjunction: ``There is no Osage equivalent to the English use of \textit{and} to conjoin sentences; rather, the elements are strung together with no intervening forms of any kind'' \REF{ex:rudin:455}. Quintero gives no information on `or' or `but'.

\textbf{Kaw (Kansa)} (data from   \citealt{CumberlandRankin2012}; Justin McBride p.c.; Robert Rankin p.c.) has an `and' coordinator, \textit{\v{s}i}, which is used in a variety of syntactic environments (postverbal, preverbal, postnominal, clause initial) and apparently can conjoin both clauses and nominals. McBride states that it usually seems to be used adverbially (`again') or adjectivally (`another'), but can also symmetrically coordinate clauses. Numerous conjunctions with meanings like `and, then, so' exist, but all seem to be subordinating rather than coordinating. The conjunction \textit{d\k{a}} `and, then, so' occurs between clauses and in other coordinating situations; Rankin, in a 2012 email, states that ``Kaw ... seems to allow the conjunction \textit{d\k{a}} (often reduced to d-schwa ...) in exactly the same places English would allow `and'''; he suggest this is a result of adopting Spanish coordination structures. Further evidence of Spanish influence is the clearly borrowed coordinator \textit{pero} `but'. I have no information on `or' in Kaw.

\textbf{Quapaw} (data from \citealt{Rankin2002,Rankin2005}) probably has conjunctions similar to those in the other Dhegiha languages, but I have very little information. Rankin's grammar and dictionary list \textit{\c{s}i} `and' (cf. Omaha-Ponca \textit{shi}, Kaw \textit{\v{s}i} , but give no indication of how it is used.

\subsection{Winnebago-Chiwere}
 
\textbf{Hooc\k{a}k} (data from \citealt{Helmbrecht2004}; confirmed by Iren Hartmann p.c.) has three apparently straightforward coordinating conjunctions, which Helmbrecht labels as follows: \textit{\'an\k{a}ga} `and' (coordinate); \textit{n\k{i}\k{i}g\'e\v{s}ge} (\textit{n\k{i}gee\v{s}ge}) `or' (disjunction); \textit{n\k{u}n\k{i}ge} `but' (adversative). The `and' and `or' words are used to conjoin all types of syntactic constituents: NP, VP, S, ``obliques'' (adjunct phrases), and AdvP. The conjunctions are placed between the coordinated phrases, or in the case of three coordinated NPs, preceding the last NP (X Y \textit{\'an\k{a}ga} Z `X, Y and Z'). Helmbrecht argues that \textit{\'an\k{a}ga} conjunction is true coordination: the resulting constituent requires plural agreement, and an overt pronoun is needed to conjoin a 1st or 2nd person. Hooc\k{a}k also has a comitative construction with the verb \textit{haki\v{z}u} `to be together,' as well as some other, presumably subordinating conjunctions: \textit{n\k{a}ga, hirean\k{a}ga} `along with' conjoins animate subjects or objects, and clauses can be conjoined with \textit{`eegi} `and then' or \textit{\v{s}ge/hi\v{s}ge} `also, even' (placed after 2nd conjunct). Helmbrecht also discusses negation of one or both conjuncts; a special conjunction \textit{h\k{a}k\'e}, used at the beginning of S or NP, expresses `and not/but not'.

\textbf{Chiwere} (data from \citealt{Goodtracks1992}; \citealt{Greer2016} (this volume); Bryan Gordon, p.c.) has several ways of expressing `and'. These include words meaning `with' (\textit{t\'ogre, ins\'uⁿ, in\'uⁿki}), `also' (\textit{hedaⁿ, -daⁿ, na, -ku}), `again' (\textit{\v{s}ige}), and a set of discourse connectives in the form of clefts, with copula \textit{ar\'e: ar\'eda, ed\'a, ar\'edare, \'edare, h\'edare}. In addition, a string of nominals can be followed by \textit{inuⁿki} or \textit{br\'oge}. Gordon also lists `bracketing' conjunctions: \textit{\v{s}uⁿ, gas\'uⁿ, nah\'e\v{s}uⁿ}, and a number of subordinating connectives. `But' is \textit{n\'una}.

\subsection{Dakotan}
 
There is information available on several of the Dakotan languages and dialects; some sources include data from more than one dialect. I have found no information on Stoney.

\textbf{Assiniboine} (data from \citealt{West2003}; \citealt{Cumberland2005}; \citealt{Levin1964}) has two main `and' coordinators, \textit{h\~ik} and \textit{h\~ikn\'a}, but sources differ somewhat in their descriptions of how these are used. West argues explicitly that \textit{h\~ikn\'a} conjoins VP or V, not clauses; i.e. it occurs in the context VP \textit{h\~ikn\'a} VP or V \textit{h\~ikn\'a} V. She analyzes it as head of a CoP with the first conjunct VP/V as complement and the second one as specifier (pp. 32-38). Clauses are joined by \textit{h\~ik} repeated after each clause: S \textit{h\~ik} S \textit{h\~ik}. Cumberland, on the other hand, shows all categories joined by non-repeating \textit{h\~ik}: NP \textit{h\~ik} NP, V \textit{h\~ik} V, VP \textit{h\~ik} VP. (Levin, cited in \citealt[36]{Stassen2000}) discusses a third coordinator, \textit{ka}, which conjoins NP. There is also a comitative construction with \textit{kici} `with' at the end of a string of NPs. I have no information on `but' or `or' in Assiniboine.

\textbf{Lakota}\footnote{In general I have used the orthography of the source in this paper. However, in the case of Lakota, I have
standardized all the disparate orthographies of the various sources to the modern standard spelling system used by the Lakota Language Consortium.} (data from  \citealt{RoodTaylor1996}; \citealt{Ingham2002}; \citealt{Ulrich2016};  \citealt{BoasDeloria1941}) has several `and' conjunctions: \textit{na, nah\'a\textipa{N}} `and also'; \textit{\v{c}ha, \v{c}ha\textipa{N}kh\'e} `and so'; \textit{yu\textipa{N}k\v{h}\'a\textipa{N}} `and then'; \textit{na} can coordinate nouns or clauses, while the others appear to coordinate only clauses. Lakota also has a word meaning `or': \textit{na\'i\textipa{N}\v{s}} and several expressing contrastive coordination `but': \textit{\'eya\v{s}, k'\'eya\v{s}, tk\v{h}\'a, kh\'e\v{s}, \v{s}k\v{h}\'a. \'Eya\v{s}} is also listed as an interjection meaning `well, but'. Numerous other conjunctions are listed, including \textit{ho, hon\'a} `furthermore', \textit{nak\'u\textipa{N}} `also', \textit{h\'e u\textipa{N}} `therefore', \textit{tk\v{h}\'a\v{s}} `but indeed', and others. Ulrich gives examples of an apparent comitative, \textit{ki\v{c}h\'i}, as well. It is not entirely clear to me whether the `and/or/but' conjunctions are all coordinators or whether some (or all) are subordinating conjunctions. Rood and Taylor define ``conjunction'' as connecting two sentences, but at least the `and' and `or' words can also conjoin ``parts of a sentence, such as nominals or verbs''. The position of all the conjunctions is between conjuncts in their examples, but they state there are ``two possible positions: in the second slot from the beginning or in the last slot in the sentence.'' David Rood (p.c.) points out that obligatory ablaut before \textit{na} and \textit{na\'i\textipa{N}\v{s}} suggests a strong bond between the conjunction and the preceding verb.

\textbf{Dakota} (data from \citealt{Riggs1851};  \citealt{BoasDeloria1941}) has unsurprisingly some similar conjunctions to Lakota, though some also differ. Several words translate `and': \textit{k'a, \v{c}ha, u\textipa{N}kh\'a\textipa{N}, nak\'u\textipa{N}. U\textipa{N}kh\'a\textipa{N}} conjoins clauses with different subjects, while \textit{k'a} conjoins nouns and clauses with same subject; no details are given of the usage of the other `and' words. `But' is \textit{tukh\'a}, and `or' is \textit{k'a i\v{s}}. \citeauthor{BoasDeloria1941} give forms from several dialects; alongside the Lakota forms in the previous paragraph they also list Dakota forms, usually labelled as ``Yankton'' and/or ``Santee'' dialect, including \textit{k'a} 'and, \textit{u\textipa{N}kh\'a\textipa{N}} `and then'.     

\subsection{Missouri Valley}
 
\textbf{Crow} (data from \citealt{Graczyk2007}) has very different strategies for conjoining clauses and
nominals. For coordinate nominals, the conjunctions are \textit{-dak} `and' and \textit{-xxo} `or'. Both are suffixes (or enclitics), but at different levels: \textit{-dak} suffixes to NP, while \textit{-xxo} suffixes to N$'$. Both conjunctions are repeated after each conjunct; \textit{-dak} may and \textit{-xxo} must be omitted after the final conjunct. There is also a comitative construction involving the transitive verb \textit{\'axpa} `be with' (also `marry') with same-subject marking or an incorporation structure. Clauses in Crow are linked by switch-reference marking rather than conjunction. Graczyk analyzes apparently coordinate clauses as `co-subordination' or clause-chaining: a string of clauses with switch-reference markers but no sentence final clitic, except for the last clause, which determines the speech-act type of the entire string (eg. declarative). The adversative `but' relation between clauses is marked with \textit{-htaa} (suffix on clause) or \textit{hehtaa} (sentence connector).

\textbf{Hidatsa} (data from \citealt{Boyle2005,Boyle2007,Boyle2011}) has significantly changed its coordination constructions in quite recent times. Boyle points out that Crow and Hidatsa share some cognate morphology in the area of conjunctions (e.g. Hidatsa \textit{-k} is cognate with Crow \textit{-dak}), but Hidatsa has innovated a semantic distinction involving specificity and inclusiveness of NPs. In the area of clausal/verbal coordination, Hidatsa's former switch-reference markers have evolved into English-like coordinators (\citealt{Boyle2011}). At present, the following morphemes express `and': \textit{hii} coordinates S's; \textit{-k} coordinates NP (with a nonspecific reading when suffixed to both NPs and a specific reading when suffixed only to the first NP); \textit{-\v{s}ek} coordinates NPs with a non-specific reading; \textit{-a} coordinates V in serial verb construction; \textit{-ak} (the old Same Subject marker) coordinates V or VP. There is apparently no `but' coordinator; adversative meaning ``is shown with juxtaposition with one element being negated'' (John Boyle p.c.).

\textbf{Mandan} (data from \citealt{Clarkson2012}; Randolph Graczyk p.c.) links clauses via a switch reference system similar to that of Crow. The morpheme \textit{ni} is used both as a same-subject marker for clauses and as a NP coordinator. NP coordination is accomplished with a coordinator following each NP; coordinating conjunctions used in this way include \textit{eheni, -kini, -hini, -kiri}, all meaning `and'. In modern usage two new coordinators appear, not found in older texts: \textit{hi(i)} with NPs and \textit{ush} with clauses. Both occur between conjuncts rather than after each conjunct. Clarkson claims that coordination is much more common in recent texts than in those from the early 20th century, suggesting that Mandan syntax, like that of Hidatsa, has been restructured under pressure from English. I have no information about alternative or adversative coordination in Mandan.  

\subsection{Southeastern Siouan}
 
\textbf{Biloxi} (data from \citealt{Zenes2009}; based on   \citealt{DorseySwanton1912}) has an NP coordinator \textit{y\k{a}} `and' which suffixes either to each NP or just the last one; it is also possible for NPs simply to be listed. Clauses are coordinated by simple juxtaposition. Zenes treats the latter two constructions (concatenated NPs and S's) as CoordP with a zero coordinator. Coordination of a series of object NPs is expressed by coordinating clauses with the same verb repeated (`I planted onions, I planted potatoes, I planted turnips'). Disjunction of NPs is expressed by \textit{ha} `or' following the second NP. Zenes gives no information about `or' with sentences or clauses. Biloxi also has a comitative construction with \textit{n\k{o}pa} following the second NP.

\textbf{Ofo} (\citealt{DorseySwanton1912}; Robert Rankin, p.c.) apparently coordinates clauses only by juxtaposition with no conjunction. I have no further information about Ofo coordination, and none at all about Tutelo.

\subsection{Summary}

The known possibly-coordinating conjunctions of the Siouan languages are summarized in Tables \ref{coord} and \ref{morecoord} To give some sense of their syntax, the conjunctions are shown with the type of constituents they conjoin when this is known; for instance S \textit{\textbf{ki}} S means \textit{ki} can occur between two clauses; NP NP \textit{\textbf {shti}} means \textit {shti} occurs at the end of a string of NPs.

\begin{table}
\caption{Coordinating(?) conjunctions} \label{coord}
\small
\begin{tabular}{ l  l  l  l  }
\lsptoprule
Language & Additive \textbf{\textit{and}} & Disjunctive \textbf{\textit{or}} & Adversative \textbf{\textit{but}} \\
\midrule
Omaha- & S \textbf{\textit{ki}} S; NP \textbf{\textit{ki}} NP & \textbf{\textit{goⁿ}}  S \textbf{\textit{ite ki}} &  S \textbf{\textit{shi}} S-NEG \\
Ponca & S \textbf{\textit{goⁿ}} S; NP \textbf{\textit{goⁿ}} NP & `either ... or' & \\
& S \textbf{\textit{shi}} S & S  \textbf{\textit{dshtoⁿ shi}}  S \textbf{\textit{dshtoⁿ}} & \\
& NP NP \textbf{\textit{edoⁿba}}/ \textbf{\textit{\'ethoⁿba}} & `maybe and maybe' & \\
& NP NP NP \textbf{\textit{edabe}} & S \textbf{\textit{shoⁿ}} S \textbf{\textit{doⁿste}} & \\
& NP \textbf{\textit{egoⁿ}} NP \textbf{\textit{shenoⁿ}} & `either-or, perhaps' & \\ \vspace{1em}

& NP NP \textbf{\textit{shti}} & S \textbf{\textit{doⁿste}} `or' &  \\  \vspace{1em}
Osage	& NP NP \textbf{\textit{\'ee\textipa{D}\k{o}\k{o}pa}} &   &   \\ 

Kaw & S \textbf{\textit{\v{s}i}} S &   & \textbf{\textit{pero}} \\  \vspace{1em}
& S \textbf{\textit{d\k{a}}} S & & \\  \vspace{1em}

Quapaw & \textbf{\textit{\c{c}i}} &   &   \\

Hooc\k{a}k & S \textbf{\textit{\'an\k{a}ga}} S & S \textbf{\textit{n\k{i}\k{i}g\'e\v{s}ge}} S & \textbf{\textit{n\k{u}n\k{i}ge}} \\
& also conjoins NP, VP,  & also conjoins NP, VP, & \\  \vspace{1em}
& AdvP, oblique & AdvP, oblique & \\

Chiwere & \textbf{\textit{\v{s}ige}} & & \textbf{\textit{n\'una}} \\
& \textbf{\textit{hedaⁿ}}, -\textbf{\textit{daⁿ}} & & \\
& NP NP \textbf{\textit{inuⁿki}} & & \\  \vspace{1em}
& NP NP \textbf{\textit{br\'oge}} & & \\

Assiniboine & V \textbf{\textit{h\~ikn\'a}} V, VP \textbf{\textit{h\~ikn\'a}} VP &   &   \\
& S \textbf{\textit{h\~ik}} S; also with NP, & & \\
&  V, VP, etc. & & \\
& S \textbf{\textit{h\~ik}} S \textbf{\textit{h\~ik}}; also with & & \\
& NP, V, VP, etc. & & \\ \vspace{1em}
& NP \textbf{\textit{ka}} NP & & \\

Lakota	& S \textbf{\textit{na}} S, NP \textbf{\textit{na}} NP,  & S \textbf{\textit{na\'i\textipa{N}\v{s}}} S & S \textbf{\textit{\'eya\v{s}}} S, NP \textbf{\textit{\'eya\v{s}}} \\
& V \textbf{\textit{na}} V & & NP, V \textbf{\textit{\'eya\v{s}}} V \\
& S \textbf{\textit{yu\textipa{N}k\v{h}\'a\textipa{N}}} S & & \textbf{\textit{k'\'eya\v{s}}}  \\
& S \textbf{\textit{\v{c}ha}} S & & \textbf{\textit{tk\v{h}\'a}} \\
& S \textbf{\textit{\v{c}ha\textipa{N}kh\'e}} S & & \textbf{\textit{kh\'e\v{s}}} \\ \vspace{1em}
& & & \textbf{\textit{\v{s}k\v{h}\'a}} \\

Dakota & S \textbf{\textit{k'a}} S, NP \textbf{\textit{k'a}} NP & NP \textbf{\textit{na\'i\textipa{N}\v{s}}} NP & \textbf{\textit{tukh\'a}} \\
& S \textbf{\textit{u\textipa{N}k\v{h}\'a\textipa{N}}} S  & \textbf{\textit{k'a i\v{s}}} & \\
& \textbf{\textit{nak\'u\textipa{N}}} & & \\
& \textbf{\textit{\v{c}ha}} & & \\
\lspbottomrule
\end{tabular}
\end{table}

\begin{table}
\caption{Coordinating(?) conjunctions continued} \label{morecoord}
\small
\begin{tabular}{ l  l  l  l  }
\lsptoprule
Language & Additive \textbf{\textit{and}} & Disjunctive \textbf{\textit{or}} & Adversative \textbf{\textit{but}} \\
\midrule  \vspace{1em}
Crow & NP \textbf{\textit{dak}} NP \textbf{\textit{dak}} & N$'$ \textbf{\textit{xxo}} N$'$ \textbf{\textit{xxo}} &  \\

Hidatsa & S \textbf{\textit{hii}} S &   & juxtaposition with  \\
& NP-\textbf{\textit{k}}; NP-\textbf{\textit{k}} NP-\textbf{\textit{k}} & & negation \\
& NP -\textbf{\textit{\v{s}ek}} NP & & \\
& V-a V (serial verb) & & \\ \vspace{1em}
& V-\textbf{\textit{ak}} V; VP-\textbf{\textit{ak}} VP & & \\

Mandan & S-\textbf{\textit{ni}} S &   & \\
& \textbf{\textit{ush}} S \textbf{\textit{ush}} S & & \\
& NP \textbf{\textit{eheni}} NP (\textbf{\textit{eheni}}) & & \\
& NP-\textbf{\textit{kini}} NP-\textbf{\textit{hini}} & & \\
& NP-\textbf{\textit{kiri}} NP(-\textbf{\textit{kiri}}) & & \\ \vspace{1em}
& NP \textbf{\textit{hii}} NP & & \\
 \vspace{1em}
Biloxi & NP NP \textbf{\textit{y\k{a}}}; NP \textbf{\textit{y\k{a}}} NP \textbf{\textit{y\k{a}}} & NP NP \textbf{\textit{ha}} & \\
 \vspace{1em}
Ofo	 & --- & & \\

Tutelo & ---  & & \\
\lspbottomrule
\end{tabular}  
\end{table}

A partial list of comitative (`with') subordinators is given in \tabref{comitative}.  Presumably the other Siouan languages also have comitative constructions; I list here only those which were mentioned in one of my sources as a common way to express `and' coordination.
 
\begin{table}
\caption{Comitative words} \label{comitative} 

\begin{tabular} [t]{ l  l  }
\lsptoprule
Omaha-Ponca	& \textit{zhugthe} \\
Chiwere &  \textit{t\'ogre},  \textit{in\'uⁿ}, \textit{in\'uⁿki} \\
Assiniboine &  \textit{kici} \\
Lakota & \textit{ki\v{c}hi} \\
Crow & \textit{\'axpa} \\
Biloxi & \textit{n\k{o}pa} \\
\lspbottomrule
\end{tabular}
\end{table}

\section{Conclusion}
 
What can we learn from the array of facts above? The most striking conclusion that emerges from the data is the lack of unity among the Siouan languages. Even within subfamilies, the Siouan languages are quite diverse in their treatment of coordination. We can identify several areas of disagreement: \REF{ex:rudin:1} The languages differ in the types of constituents that can be coordinated, some having only clausal coordination, while others can coordinate NPs and other types of constituents as well, and some may have no true
coordination at all, but use various types of subordination, co-subordination, or simple concatenation to express the relations English expresses with `and'/`or'/`but'. \REF{ex:rudin:2} They differ in the constituent order within coordination constructions, with the conjunction following the first conjunct (XP \& XP), the second conjunct (XP XP \&) or each of the conjuncts (XP \& XP \&), and may also differ in whether the conjunction forms a constituent with a following or preceding conjunct ((XP \&) XP); (XP (\&XP)). The hierarchical structure of each of these configurations has not been studied in most of the languages. Given the generally head-final nature of phrase structure in Siouan languages, if the conjunction heads a coordination phrase it is expected that the complement of the ``\&'' head would be to its left; an XP occurring to the right could be a specifier, which we would expect to be less closely associated with the conjunction than the complement. \REF{ex:rudin:3} They differ in the lexical items expressing additive, disjunctive, and adversative coordination. Some of the words or suffixes for `and'/`or'/`but' are cognate among subfamilies --- for instance, most of the Dhegiha branch have [\v{s}i] or something similar, and the Dakotan branch share something like [na]. But no coordinators appear to be cognate across the family. \REF{ex:rudin:4} Finally, the languages differ also in the expression of comitative and other ``semantically coordinated'' phrases.

In short, there does not seem to be a ``typical Siouan'' coordination pattern, nor does it look like we can reconstruct proto-Siouan coordinators. Clearly there has been innovation in at least some of the languages -- perhaps all -- and at least in one or two cases there has been borrowing of coordinators and/or coordination patterns from European languages, suggesting quite recent change in this semantic field. In at least some languages the most common way to conjoin NPs is with a comitative, not a coordinate construction. (This is my impression in Omaha-Ponca, and Cumberland (p.c.) has the same impression in Assiniboine, for example.) Is it possible there was no morpho-syntactic coordination in proto-Siouan?

In fact, this is not as unlikely as it might first appear. \citet{Mithun1988} suggests overt coordination tends to come with literacy: in spoken language simple concatenation tends to be common, while in writing, where intonational cues are lacking and one cannot assume the same degree of common knowledge with one's audience, explicit morphosyntactic coordination is more useful. It is certainly not the case that unwritten languages never have true coordination, but as a statistical tendency it makes some sense. Many languages, Mithun says, seem to have developed coordinating conjunctions after exposure to written languages or after developing an indigenous tradition of writing. Since Siouan languages were, until recently, not written, perhaps lack of an inherited coordination construction and associated morphology is not surprising. The borrowing or innovation of coordinators as speakers became literate in English or other European languages (as well as perhaps in the Native languages) seems logical under this view.

In spite of the lack of overt morphological or lexical coordinators in some languages, Mithun considers coordination as a syntactic and semantic structure to be universal. \citet{Stassen2000}, on the other hand, claims coordination, or at least nominal coordination, is not universal. He divides languages into two types: ``WITH-languages,'' which have only a comitative (NP with NP) or subordinating strategy for conjoining NPs , and ``AND-Languages,'' which also have a coordinate strategy. Stassen acknowledges that Native American languages tend to be problematic and difficult to classify into his two categories. This preliminary study of the Siouan family certainly bears out the elusiveness of coordination constructions in these languages.

\section*{Acknowledgment}

Earlier versions of much of this material were presented at the Comparative Siouan Grammar Workshop/Siouan and Caddoan Languages Conference, Lincoln, NE 2009, and at the 2015 SSILA meeting in Portland. I would like to thank the participants at both events, as well as the two reviewers for this volume, for helpful comments and data. As was so often true, at so many meetings, Bob Rankin's insightful comments at and after the 2009 workshop were especially valuable, and data he gathered was a major source of information on Dhegiha languages. I am grateful to have known Professor Rankin and hope that this paper in some small way contributes to his legacy.

\section*{Abbreviations}

1, 2, 3 = first, second, third person; A = agent; \textsc{aux} = auxiliary; \textsc{evid} = evidential; \textsc{fut} = future; \textsc{hab} = habitual; \textsc{neg} = negative; P = patient; \textsc{pl} = plural; \textsc{prox} = proximate; \textsc{quot} = quotative; \textsc{refl} = reflexive; \textsc{suus} = suus (reflexive possessive). 

\section*{References}
\todo{Dorsey n.d. has two different references attached to it. It is not clear, which one is cited}

\printbibliography[heading=subbibliography,notkeyword=this]

\begin{reflist}
  

Dorsey, James Owen. ms; no date. A grammar and dictionary of the Ponka language. 4800 Dorsey Papers. Washington: Smithsonian Institution.

Dorsey, James Owen. ms; no date. Slip file for a dictionary of Omaha-Ponca. Online at http://omahalanguage.unl.edu/images.php
 
\end{reflist}
\end{document}

