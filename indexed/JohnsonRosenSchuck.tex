% 14
\documentclass[output=paper]{LSP/langsci}
\author{Meredith Johnson\and Bryan Rosen\lastand Mateja Schuck}
\title{Evidence for a VP constituent in Hocąk}

\abstract{Since at least \citealt{Williamson1984}, there has been a debate over the configurationality of Siouan languages \citep{Boyle2007,Graczyk1991a,West2003,VanValin1985,VanValin1987}. In this paper, we argue that a nonconfigurational approach does not account for the asymmetries between subjects and objects in Hocąk. We propose that Hocąk is a configurational language in that the language has a verb phrase (VP): the object and the verb form a constituent to the exclusion of the subject. This structure captures the differences between subjects and objects with respect to locative scope, quantifier scope, verb phrase ellipsis, and resultatives. 
% KEYWORDS: [Ho-Chunk, configurationality, verb phrase, subject-object asymmetries]
}
\ChapterDOI{10.17169/langsci.b94.177}

\maketitle

\begin{document}

\section{Introduction}

Since at least \citealt{Williamson1984}, there has been a debate over the \isi{configurationality} of Siouan languages \citep{Boyle2007,Graczyk1991a,West2003,VanValin1985,VanValin1987}.  The purpose of this paper is to weigh in on this issue with evidence (based on original fieldwork) from \ili{Hocąk}. By providing novel data from \isi{locative scope}, \isi{quantifier scope}, verb phrase \isi{ellipsis}, and resultatives, we argue that \ili{Hocąk} has a verb phrase (\isi{VP}). This adds empirical support for previous studies that have argued that Siouan languages have a verb phrase (e.g., \citealt{Boyle2007}; \citealt{Graczyk1991a}; \citealt{West2003}).

	The crucial observation that we make is in this paper is that there exist a number of subject-\isi{object} asymmetries. To account for these data, we propose a \isi{syntax} for \ili{Hocąk} that consists minimally of the structure shown in \REF{ex:jrs:1}.

\ea\label{ex:jrs:1}
\Tree [ .\isi{XP} [ .Subject ] [ .\isi{VP} [ .Object ] [ .Verb ] ] ]
\z

By contrast, we argue that a flat, nonconfigurational structure such as the one in \REF{ex:jrs:2} cannot adequately account for the data (cf.  \citealt{VanValin1985,VanValin1987,Williamson1984}).

\ea\label{ex:jrs:2}
\Tree [ .\isi{XP} [ .Subject ] [ .Object ] [ .Verb ] ]
\z
	
This paper is organized as follows. In \sectref{sec:jrs:2}, we outline previous analyses that argue in favor of a \isi{flat structure} for various Siouan languages, and then discuss how the \ili{Hocąk} data compare. \sectref{sec:jrs:3} reviews arguments for a \isi{VP} in other Siouan languages, and shows that similar arguments can be made for \ili{Hocąk}. In \sectref{sec:jrs:4}, we provide four new arguments in favor a of \isi{VP} analysis of \ili{Hocąk}. \sectref{sec:jrs:5} concludes the paper.


\section{Arguments in favor of a flat structure}\label{sec:jrs:2}

In this section, we provide background on the nature of \isi{configurationality} in the context of \ili{Hocąk} (and other Siouan languages). \sectref{sec:jrs:2.1} outlines the previous nonconfigurational accounts (\citealt{Hale1983} and \citealt{Jelinek1984}) that stand in contrast to the configurational account that we propose in this paper. In \sectref{sec:jrs:2.2}, we review the previous arguments for a flat \isi{VP} structure in Siouan languages. Then in \sectref{sec:jrs:2.3}, we show that \ili{Hocąk} displays all three of the prototypical characteristics of being a nonconfigurational language.

\subsection{Non\isi{configurationality} and pronominal arguments:  Hale (1983) and Jelinek (1984)}\label{sec:jrs:2.1}

Since \citet{Hale1983}, nonconfigurational languages have been typologically characterized by the three traits given in \REF{ex:jrs:3}:

\begin{exe}
\ex\label{ex:jrs:3} Properties of nonconfigurational languages

	i.	Free \isi{word order}

	ii.	Extensive null anaphora

	iii.	Presence of discontinuous constituents
\end{exe}

Hale's approach makes use of two levels of representation: \textit{lexical structure} (LS) and \textit{phrase structure} (PS). Hale argues that all languages are configurational at LS; that is, the \isi{subject} asymmetrically c-commands the \isi{object}. However, this asymmetry is not realized at the level of PS in nonconfigurational languages: the phrase structure is flat. This is the definition of \isi{configurationality} that is most adopted by Siouanists. For example, \citet{Boyle2007} claims that \ili{Hidatsa} is a configurational language on the grounds that there are subject-\isi{object} asymmetries that are indicative of a \isi{VP} constituent. (See also  \citealt{VanValin1985,VanValin1987,Williamson1984}, and \citealt{West2003}.) 

	Another formal account of non\isi{configurationality} is \citegen{Jelinek1984} \textit{Pronominal Argument Hypothesis} (\isi{PAH}). According to the \isi{PAH}, person markers are the actual arguments of the verb, while the overt NPs are adjuncts adjoined high in the clause, as in \REF{ex:jrs:4}. We use ``TP'' (Tense Phrase) for the phrase that represents the sentence level.

\begin{exe}
\ex\label{ex:jrs:4} 
\Tree [ .TP [ .NP\textsubscript{i} \edge[roof]; {\textit{subject}} ] [ .TP [ .NP\textsubscript{j} \edge[roof]; {\textit{object}} ] [ .Verb [ . Agreement ] [ .Verb ] ] ] ]
\end{exe}

The overt NPs, when present, are coindexed with the person markers. Since adjuncts are known to have freer distribution of \isi{word order} than arguments, the ``free'' \isi{word order} in nonconfigurational languages is accounted for. Adjuncts are also never obligatory, explaining the possibility of \textit{pro}-drop of all NPs in nonconfigurational langugages. Lastly, this proposal accounts for the presence of apparent discontinuous constituents in nonconfigurational languages. Jelinek proposes that more than one \isi{adjunct} NP can be coindexed with a given person marker. Thus, what appear to be discontinuous NPs are actually two separate NPs that correspond to the same argument.

 In contrast, a configurational language is one that that does show subject-\isi{object} asymmetries and has a \isi{VP} constituent, as depicted in \REF{ex:jrs:5} below.

\begin{exe}
\ex\label{ex:jrs:5} 
\Tree [ .TP [ .T ] [ .vP [ .NP \edge[roof]; {\textit{subject}} ] [ .\isi{VP} [ .NP \edge[roof]; {\textit{object}} ] [ .Verb ] ] ] ]
\end{exe}

Example \REF{ex:jrs:5} shows that the \isi{subject} and \isi{object} are not in \isi{adjunct} positions: they do not adjoin to the TP (or Sentence). Following \citet{Chomsky1995}, we assume that the \isi{subject} is base-generated in a position outside of the \isi{VP}, which we label ``vP.'' The \isi{object} merges as an argument of the verb inside the \isi{VP}. Thus, by ``\isi{VP}'' we refer to the constituent that contains the \isi{object}, the verb, and perhaps other modifier material. Crucially, the \isi{subject} is not considered part of the \isi{VP}.
 
\subsection{Previous analyses: Williamson (1984), Van Valin (1985, 1987)}\label{sec:jrs:2.2} 

In this section, we discuss arguments in favor of a nonconfigurational analysis of Siouan languages that have been put forth in previous works.
	
\citet{Williamson1984} argues that \ili{Lakota} is nonconfigurational because it lacks the subject-\isi{object} asymmetries traditionally associated with the Empty Category Principle (ECP).  Long distance \textit{wh}-extraction of the \isi{subject} over an overt complementizer is possible in \ili{Lakota}; that is, the language does not display \textit{that}-trace effects. Long distance extraction out of \textit{wh}-islands from \isi{subject} position is also allowed in \ili{Lakota}.  Examples \REF{ex:jrs:6}--\REF{ex:jrs:8} below illustrate these facts:

\begin{exe}
\ex\label{ex:jrs:6} \gll Mary		tuwa		wąyąke	\textbf{ki}			ilukcha 		he \\
Mary 	who		see 			\textsc{comp}		you.think	\textsc{q} \\
\trans `Who do you think that Mary saw?' (\citealt[281]{Williamson1984}, (64a))
\ex\label{ex:jrs:7} \gll Tuwa		hel			na\v{z}\k{i} 		he		\textbf{ki}			ilukcha 		he? \\
who		there		stand		\textsc{dur} 	\textsc{comp}		you-think	\textsc{q} \\
\trans `Who do you think that was standing there?' (\citealt[281]{Williamson1984}, (65a))
\ex\label{ex:jrs:8} \gll	Tohą		tuwa	u				pi	\textbf{ki} 			slolyaya		he? \\
when		who	come		\textsc{pl} 	\textsc{comp} 	you.know \textsc{q} \\
\trans `Who do you know when is coming?' (\citealt[281]{Williamson1984}, (66a))
\end{exe} 

In a language with subject-\isi{object} asymmetries, long-distance \textit{wh}-extraction of the \isi{subject} should not be possible, as doing so would constitute a violation of the ECP (as evidenced by the ungrammaticality of the \ili{English} translations in \REF{ex:jrs:7}-\REF{ex:jrs:8}).  Because \ili{Lakota} appears to allow long-distance \textit{wh}-extraction from \isi{subject} position, Williamson argues that the language has no subject-\isi{object} asymmetry and thus lacks a \isi{VP} constituent.

\citet{VanValin1985,VanValin1987} also argues for a nonconfigurational analysis of \ili{Lakota} on the basis of the lack of Weak Crossover and Binding Condition C effects.  First, let us consider the diagnostic from Weak Crossover (WCO). A WCO violation occurs when a pronoun is coreferential with the \textit{wh}-trace in \isi{subject} or \isi{object} position and neither one c-commands the other (\citealt{Sportiche1985}).  \REF{ex:jrs:9} illustrates an \ili{English} example of WCO: the \textit{wh}-word \textit{who} undergoes movement from an \isi{object} position (represented by a trace, ``t'') to the left edge of the clause. \textit{Who} must ``cross over'' the co-indexed pronoun \textit{his}. Since \textit{who} and \textit{his} cannot refer to the same person, the sentence is ungrammatical. 

\begin{exe}
\ex\label{ex:jrs:9}  *Who\textsubscript{i} does his\textsubscript{i} mother love t\textsubscript{i}?
\end{exe}

Thus in a language with a \isi{VP} node, a coreferential reading between the \textit{wh}-word and \isi{possessive pronoun} in the sentence in \REF{ex:jrs:10} below would be expected to be unavailable.

\begin{exe}
\ex\label{ex:jrs:10} \gll $\varnothing$-tha-kh\'ola-ku 	ki	tuw\'a	wąy\k{\'a}ka	he? \\
3-\textsc{poss}-friend-\textsc{poss} the who		\textsc{3sg}.see.\textsc{3sg} 	\textsc{q} \\
\trans `Who\textsubscript{i} did his\textsubscript{i}  friend see?' (\citealt[379]{VanValin1987})
\end{exe}

Because the construction in \REF{ex:jrs:9} does not cause a WCO violation in \ili{Lakota}, Van Valin argues that no subject-\isi{object} asymmetry exists in the language, and thus it does not possess a \isi{VP}.
	
Van Valin additionally cites the lack of Binding Condition C (BCC) violations in \ili{Lakota} as evidence that the language lacks a subject-\isi{object} asymmetry.  This is due to the fact that \isi{binding} conditions crucially rely on a c-command relationship between anaphors, pronouns and r-expressions.  Van Valin argues that since there appear to be no BCC violations in \ili{Lakota}, the \isi{subject} must not c-command the \isi{object}. This falls out of an analysis where both NPs are attached at the TP (or sentence) level. We return to BCC violations in the next section.

\subsection{\ili{Hocąk} data}\label{sec:jrs:2.3}

\citet{Hale1983} and \citet{Jelinek1984} identify three properties that they claim are common to all nonconfigurational languages: free \isi{word order}, extensive null anaphora, and discontinuous constituents. Below, we show that \ili{Hocąk} does display each of the three classic signs of non\isi{configurationality} put forth by Hale and Jelinek, as well as a number of additional characteristics of nonconfigurational languages proposed by \citet{Baker1996}.  

First, NP arguments may appear in a variety of orders.  This is expected in an analysis under which there is a \isi{flat structure} and all NPs are adjuncts adjoined at the TP (or sentence) level. SOV \isi{word order} is the most common in \ili{Hocąk}, as in \REF{ex:jrs:11}. Any variation in \isi{word order} has discourse-informational effects, as hinted at by the \ili{English} translations given in the examples below. As shown in \REF{ex:jrs:12a}, a participant displaced to the left serves a \isi{topic} or focus function, whereas participants displaced to the right are interpreted as anti-topics (e.g., ``backgrounded'' or discourse-old), as shown in \REF{ex:jrs:12b}--\REF{ex:jrs:12e}.

\ea
\label{ex:jrs:11} 
\glll Hin\k{u}kra wa\v{z}ątirera ruw\k{i}  \\
hin\k{u}k-ra	wa\v{z}ątire-ra 	$\varnothing$-ruw\k{i} \\
lady-\textsc{def}	car-\textsc{def}	 \textsc{3s/o}-buy \\
\trans `The lady bought the car.'
\z

\ea\label{ex:jrs:12}
\ea\label{ex:jrs:12a}
\glll Wa\v{z}ątirera,	hin\k{u}kra		ruw\k{i} \\
wa\v{z}ątire-ra 	hin\k{u}k-ra		$\varnothing$-ruw\k{i} \\
car-\textsc{def}			lady-\textsc{def}		\textsc{3s/o}-buy \\
\trans `The car, the lady bought it.' 

\ex\label{ex:jrs:12b}
\glll Wa\v{z}ątirera		ruw\k{i},			hin\k{u}kra\\
wa\v{z}ątire-ra	$\varnothing$-ruw\k{i}	hin\k{u}k-ra \\
car-\textsc{def} 			\textsc{3s/o}-buy 	lady-\textsc{def} \\
\trans `Someone bought the car, (it was) the lady.' 

\ex\label{ex:jrs:12c}
\glll Hin\k{u}kra 		ruw\k{i}, 		wa\v{z}ątirera \\
hin\k{u}kra 	$\varnothing$-ruw\k{i}, 		wa\v{z}ątire-ra \\
lady-\textsc{def} \textsc{3s/o}-buy car-\textsc{def} \\
\trans `the lady bought something, (it was) the car.' 

\ex\label{ex:jrs:12d}
\glll Ruw\k{i}, 		wa\v{z}ątirera,	hin\k{u}kra\\
$\varnothing$-ruw\k{i}	wa\v{z}ątire-ra	hin\k{u}k-ra \\
\textsc{3s/o}-buy car-\textsc{def} lady-\textsc{def} \\
\trans `Someone bought something, (it was) the car, the lady.' 

\ex\label{ex:jrs:12e}
\glll Ruw\k{i}, 		hin\k{u}kra, 	wa\v{z}ątirera \\
$\varnothing$-ruw\k{i} 	hin\k{u}k-ra 	wa\v{z}ątire-ra \\ 
\textsc{3s/o}-buy lady-\textsc{def} car-\textsc{def} \\
\trans `Someone bought something, (it was) the lady, the car.'
\z
\z

It is also possible for NP arguments to have freedom of placement among each other. The default order of arguments in a ditransitive construction is Agent > Indirect Object > Direct Object; however, their order can vary.  This is shown below in \REF{ex:jrs:13}, where the \isi{subject} NP \textit{hin\k{u}khi\v{z}ą} `a woman' can appear in several different positions.

\ea\label{ex:jrs:13}
\glll (Hin\k{u}khi\v{z}ą,) 	hoc\k{i}c\k{i}hi\v{z}ą 	(hin\k{u}khi\v{z}ą,) wiiwagaxhi\v{z}ą 	(hin\k{u}khi\v{z}ą,) hok'\k{u}.\\
hin\k{u}k-hi\v{z}ą 		hoc\k{i}c\k{i}-hi\v{z}ą 	hin\k{u}k-hi\v{z}ą 		wiiwagax-hi\v{z}ą 	hin\k{u}k-hi\v{z}ą 	$\varnothing$-hok\k{u}\\
	woman-\textsc{indef}	boy-\textsc{indef} 	woman-\textsc{indef} 	pencil-\textsc{indef} 	woman-\textsc{indef} 	\textsc{3s/o}-give\\
\trans `A woman gave a boy a pencil.'
\z
	
Second, NPs corresponding to arguments can be freely omitted in \ili{Hocąk}. Examples of this are shown below in \REF{ex:jrs:14}, where the agent and patient/theme arguments are omitted:

\ea\label{ex:jrs:14}
\ea \label{ex:jrs:14a} 
\glll Wij\k{u}kra	\v{s}\k{u}\k{u}kra		hoxataprookeeja 		haja.\\
wij\k{u}k-ra	\v{s}\k{u}\k{u}k-ra	hoxatap-rook-eeja 	$\varnothing$-haja \\
	cat-\textsc{def} 	dog-\textsc{def} 	woods-inside-there 	\textsc{3s/o}-see \\
\trans `The cat saw the dog in the woods.'

\ex \label{ex:jrs:14b}
\glll Hoxataprookeeja haja. \\
hoxatap-rook-eeja 	$\varnothing$-haja \\
woods-inside-there \textsc{3s/o}-see \\
\trans `It (the cat) saw it (the dog) in the woods.'
\z 
\z 

Sentence \REF{ex:jrs:14b} is grammatical and can (under the right \isi{discourse} context) have the equivalent meaning to \REF{ex:jrs:14a}; however, it is missing the agent and patient/theme NPs \textit{wij\k{u}kra} and \textit{\v{s}\k{u}\k{u}kra}. This is also expected under \citegen{Hale1983}, \citegen{Jelinek1984}, and \citegen{Baker1996} analyses: NPs have \isi{adjunct} status and thus are not obligatory.

\ili{Hocąk} also displays discontinuous constituents. Demonstratives and quantifiers may be separated from the head noun, as shown in \REF{ex:jrs:15} with \textit{\v{z}e'e} `that': 

\ea\label{ex:jrs:15}
\ea \label{ex:jrs:15a}
\glll Wijukra	\v{s}\k{u}\k{u}k	\v{z}e'e	haja \\
wijuk-ra	\v{s}\k{u}\k{u}k	\v{z}e'e	$\varnothing$-haja \\
	cat-\textsc{def} 	dog	that		\textsc{3s/o}-see\\
\trans `The cat saw that dog.' 

\ex \label{ex:jrs:15b}
\glll \v{Z}e'e	wijukra	 \v{s}\k{u}\k{u}k	haja \\
\v{z}e'e	wijuk-ra	\v{s}\k{u}\k{u}k	$\varnothing$-haja \\
that cat-D\textsc{def} dog \textsc{3s/o}-see \\
\trans `The cat saw that dog.'
\z 
\z 

Discontinuous constituents are expected under \citegen{Hale1983} and \citegen{Jelinek1984} analyses of non\isi{configurationality}, due to the fact that NPs have the status of adjuncts. Hale and Jelinek propose that multiple adjuncts can be associated with the same argument in a given sentence. Thus, the demonstrative and head noun in \REF{ex:jrs:15b} are actually two separate NPs that both correspond to the \isi{object}.

In addition to \citegen{Hale1983} classic characteristics of non\isi{configurationality}, \ili{Hocąk} displays four additional traits of nonconfigurational languages discussed by \citet{Baker1996}. First, \ili{Hocąk} does not display BCC effects within \isi{clauses}.  As discussed in the previous section, this lack of BCC effects is expected when there is no asymmetry between the \isi{subject} and the \isi{object} In \REF{ex:jrs:16} below, coreference between the \isi{subject} `he' and the \isi{possessor} `Bryan' is grammatical.

\begin{exe}
\ex\label{ex:jrs:16}	
\glll (Ee) 	Bryanga 			hi'\k{u}ni 	hiira 				homąk\k{i}n\k{i}. \\
Ee 		Bryan-ga 		hi'\k{u}ni 	hii-ra 			$\varnothing$-homąk\k{i}n\k{i} \\
he 		Bryan-\textsc{prop} mother 	\textsc{poss-def} 	\textsc{3s/o}-visit  \\
\trans `He\textsubscript{i} visited Bryan\textsubscript{i}'s mom.'
\end{exe}
However, as Baker shows to be true in other nonconfigurational languages, \ili{Hocąk} does display BCC effects across \isi{clauses}. In \REF{ex:jrs:17}, coreference between the matrix \isi{subject} `she' and the embedded \isi{object} `Meredith' is impossible.

\begin{exe}
\ex\label{ex:jrs:17}	
\glll (Ee) 	Hunterga 	Meredithga		hajara 						hiraperes\v{s}ąną. \\
 Ee 		Hunter-ga 	Meredith-ga 		$\varnothing$-haja-ra 				$\varnothing$-hiraperes-\v{s}ąną \\
she 		Hunter-\textsc{prop} 	Meredith-\textsc{prop} 	\textsc{3s/o}-see-\textsc{comp} 	\textsc{3s}-know-\textsc{decl}  \\
\trans `She\textsubscript{*i/j} knows that Hunter saw Meredith\textsubscript{i}.'
\end{exe}	

Second, \ili{Hocąk} lacks NP anaphors, which are also argued by \citet{Baker1996} to be nonexistent in nonconfigurational languages. Instead, reflexive and reciprocal meanings are expressed morphologically on the verb, as seen in \REF{ex:jrs:18}:

\begin{exe}
\ex\label{ex:jrs:18} 
\glll Meredithga		anąga 	Hunterga 			hokikij\k{i}ire.\\
Meredith-ga	 	anąga 	Hunter-ga 			<kiki>hoj\k{i}-ire \\
Meredith-\textsc{prop} and 		Hunter-\textsc{prop} 	<\textsc{refl}>hit-\textsc{3s.pl} \\
\trans `Meredith and Hunter hit each other'
\end{exe}
	
Third, according to \citet{Baker1996}, nonconfigurational languages should lack both universal quantifiers that are grammatically singular and negative quantifiers. \ili{Hocąk} does not have a universal quantifier that is grammatically singular.  In \REF{ex:jrs:19} below, both \textit{hanąąc} `all/every' and \textit{hi\v{z}ąki\v{s}ąną} `each' trigger plural agreement on the verb.

\begin{exe}
\ex\label{ex:jrs:19}	
\begin{xlist}	
\ex
\glll Bryanga {waisgap sguu xuwuxuwura} 	hanąą 	waruuc\v{s}ąną.\\
Bryan-ga 	{waisgap sguu xuwuxuwu-ra} 	hanąą 	wa-$\varnothing$-ruuc-\v{s}ąną \\
	Bryan-\textsc{prop} 	cookie-\textsc{def}	all 		\textsc{3o.pl-3s}-eat-\textsc{decl} \\
\trans `Bryan ate every cookie/all of the cookies.'

\ex 
\glll Hoc\k{i}c\k{i}ra 	hi\v{z}ąki\v{s}ąną 	{waisgap sguu xuwuxuwuhi\v{z}ą }		ruucire.\\
hoc\k{i}c\k{i}-ra 	hi\v{z}ąki\v{s}ąną 	{waisgap sguu xuwuxuwu-hi\v{z}ą }		ruuc-ire \\
	boy-\textsc{def} 		each 					cookie-\textsc{indef}		eat-\textsc{3s.pl}\\
\trans `Each boy ate a cookie.'
\end{xlist}
\end{exe}

\ili{Hocąk} also does not possess negative quantifiers: instead, the equivalents to `nothing' and `nobody' are expressed through a combination of clausal \isi{negation} and indefinite pronouns. This is shown in \REF{ex:jrs:20a)}and \REF{ex:jrs:20b)}, respectively.

\ea\label{ex:jrs:20}
\ea\label{ex:jrs:20a)}
\glll Wawaahiwira 	hąąke 	wa\v{z}ą 	hiiran\k{i}. \\
wa<ha>hohi-wi-ra 						hąąke 	wa\v{z}ą 	hii-ire-n\k{i} \\
	\textsc{3o.pl<1s>}beat-\textsc{1/2pl}-\textsc{comp} \textsc{neg}		thing 	do-\textsc{3s.pl-neg} \\
\trans `When we beat them, they didn't score at all.' (\citealt{Hartmann2012}) 

\ex \label{ex:jrs:20b}
\glll Hąąki\v{z}ą 	{n\k{i}\k{i}ta\v{s}jak taaxura} 	karasgepn\k{i}. \\
hąąke-hi\v{z}ą  {n\k{i}\k{i}ta\v{s}jak taaxu-ra }	$\varnothing$-kara-rasgep-n\k{i} \\
	\textsc{neg-indef} 	coffee-\textsc{def}	\textsc{3s}-own-drink.up-\textsc{neg} \\
\trans `Nobody finished his coffee.'
\z 
\z 
	
Finally, \ili{Hocąk} lacks WCO effects. In \REF{ex:jrs:21} below, a coreferential reading between the \isi{possessive pronoun} and the \isi{object} \textit{wh}-word is grammatical.

\begin{exe}
\ex\label{ex:jrs:21}
\begin{xlist} 	
\ex
\glll Hi'\k{u}ni		hiira					pee\v{z}ega		haja? \\
hi'\k{u}ni		hii-ra 			pee\v{z}ega		$\varnothing$-haja \\
	mother		3\textsc{poss-def}		who	\textsc{3s/o}-see \\ 
    
\ex
\glll Pee\v{z}ega 	hi'\k{u}ni 		hiira		haja?\\
pee\v{z}ega 	hi'\k{u}ni 		hii-ra 		$\varnothing$-haja \\
	who 	mother 	3\textsc{poss-def} 	\textsc{3s/o}-see \\
\trans `Who\textsubscript{i} did his\textsubscript{i} mother see?'
\end{xlist}
\end{exe}

Recall from the previous subsection that \citet{VanValin1985,VanValin1987} uses the lack of BCC and WCO effects in \ili{Lakota} to argue for a nonconfigurational \isi{syntax}. While \ili{Hocąk} also lacks BCC and WCO effects, we argue that this does not constitute conclusive evidence of the lack of a \isi{VP} constituent in the language. In the remainder of the paper, we provide other arguments that strongly favor a \isi{VP} analysis for \ili{Hocąk}. We leave an explanation for the lack of BCC and WCO effects in \ili{Hocąk} for future research.

\section{Arguments in favor of a VP}\label{sec:jrs:3}

\subsection{Previous analyses: Boyle (2007), Graczyk (1991), West (2003)}

In the previous section, we presented arguments in favor of a nonconfigurational, VP-less analysis in several Siouan languages.  In this section, we present arguments in favor of a configurational analysis of Siouan languages (that is, arguments in favor of a \isi{VP} analysis).  The first piece of evidence comes from \isi{word order} restrictions. Recall that one of \citegen{Hale1983} and \citegen{Jelinek1984} typifying characteristics of nonconfigurational languages is free \isi{word order}.  Across Siouan languages, neutral \isi{word order} is SOV.  Several Siouanists have argued that other word orders have discourse-informational effects, and thus that \isi{word order} is not actually free in these languages. For example, \citet{West2003} shows that in \ili{Assiniboine} sentences with OSV \isi{word order}, the fronted \isi{object} has a preferred focus reading; otherwise, the first argument is interpreted as the \isi{subject}. This is shown below in \REF{ex:jrs:22}. 

\begin{exe}
\ex\label{ex:jrs:22}	\gll  \v{s}k\'o\v{s}obena w\~a\v{z}\'i 	hok\v{s}\'ina 	\v{z}e 		y\'uda \\
banana 		a 			boy 	\textsc{det} 	ate \\
\trans `The boy ate a banana (not the apple).' (preferred translation) or 

`A banana ate the boy'  \citep[49]{West2003} 
\end{exe}

The same is true of \ili{Hidatsa}.  \citet{Boyle2007} shows that unmarked \isi{word order} is SOV, with exceptions occurring in topicalization or focus constructions.  This is shown below in \REF{ex:jrs:23} with neutral SOV \isi{word order} and \REF{ex:jrs:24} OSV order:

\begin{exe}
\ex\label{ex:jrs:23} 
\glll buush\'igesh wash\'ugash \'eegaac\\
puu\v{s}\'ihke-\v{s}  ma\v{s}\'uka-\v{s} \'eekaa-c \\
cat-\textsc{det.d} dog-\textsc{det.d} see-\textsc{decl}  \\
\trans `The cat sees the dog.'  \citep[214]{Boyle2007}

\ex\label{ex:jrs:24} 
\glll mas\'ugash 		buush\'igesh 	\'eegaac\\
mas\'uka-\v{s} 		puus\'ihke-\v{s} 	\'ekaa-c \\
dog-\textsc{det.d} 		cat-\textsc{det.d}\textsc{det}	see-\textsc{decl} \\
\trans `The cat sees the dog.' \citep[214]{Boyle2007}

\end{exe}
	
\citet{Graczyk1991a} observes that SOV is neutral \isi{word order} for \ili{Crow} as well, and that other word orders have discourse-informational effects.  This is shown below, where \REF{ex:jrs:25} has neutral \isi{word order}, and \REF{ex:jrs:26} has OVS \isi{word order}:

\begin{exe}
\ex\label{ex:jrs:25} \gll shik\'aak-kaatee-sh ash\'e 		hii-\'ak \\
boy-\textsc{dimin-det} 		home 	reach-\textsc{ss} \\
\trans `The little boy reached home' \citep[101]{Graczyk1991a}

\ex\label{ex:jrs:26} \gll iaxp-\'uua 		\'itchi-kiss-uua-sh			koot\'aa 	h\'ii-k 	hinne		tal\'ee-sh \\
their.feather-\textsc{pl} good-sport-\textsc{pl}-\textsc{det}		entirely 	reach-\textsc{decl}	this 	oil-\textsc{det} \\
\trans `It entirely covered their beautiful feathers, this oil' \citep[103]{Graczyk1991a}
\end{exe}

In \REF{ex:jrs:26}, OSV \isi{word order} is used to deemphasize the discourse-old \isi{subject} \textit{talee} `oil', and emphasize the \isi{object} \textit{iaxp} `their feather'.  Based on these \isi{word order} restrictions, West, Boyle and Graczyk all argue that \ili{Assiniboine}, \ili{Hidatsa} and \ili{Crow} are configurational. 
	
The second piece of evidence that has been previously used to show the presence of a \isi{VP} in Siouan languages comes from enclitics. \citet{West2003} and \citet{Boyle2007} use the scope of enclitics to argue for a \isi{VP} constituent. \citet{Boyle2007} demonstrates that the \ili{Hidatsa} habitual enclitic \textit{-\textipa{P}ii} takes scope over both verbs in the example in \REF{ex:jrs:27} below:

\begin{exe}
\ex\label{ex:jrs:27} 
\glll Doosha	wiri\textipa{P}\'eeraga 	ad\'a\textipa{P}a 	 k\textsuperscript{h}\'uuiidoog?\\
too\v{s}\textsuperscript{h}a 	wiri-\'eeraka 	at\'a-a k\textsuperscript{h}\'uu-\textipa{P}ii-took \\
how 	sun-\textsc{dem} 	appear-\textsc{cont}  come.up-\textsc{hab.sg.spec} \\
\trans `How does the Sun always appear and come up? (he wondered)' \citep[223]{Boyle2007}
\end{exe}

The situation is the same in \ili{Assiniboine}.  In \REF{ex:jrs:28} below, the aspectual clitic \textit{s'a} scopes over both verbs, not just to the one to which it is attached:

\begin{exe}
\ex\label{ex:jrs:28} \gll Wiy\'{\~a}-bi 		\v{z}\'e-na 	woy\'uta 	sp\~ay\'{\~a}-bi 	hikn\'a 	hay\'abi 		ga\v{g}\'e\v{g}e-bi 	s'a. \\
woman-\textsc{pl} the-\textsc{pl} 	food 	cook-\textsc{pl} \textsc{conj} 	clothes 	sew-\textsc{pl} 	\textsc{hab} \\
\trans `The women usually cooked the food and sewed the clothes.' \citep[39]{West2003}
\end{exe}

The sentence in \REF{ex:jrs:28} cannot mean `the women cooked the food and usually sewed the clothes' (\citealt{West2003}).  If \ili{Assiniboine} had no \isi{VP}, this reading should not be possible: the clitic should only be able to scope over the verb it is attached to. Both \citet{Boyle2007} and \citet{West2003} argue that the clitics head a functional projection that c-commands the coordinated elements, which are VPs.  Thus, enclitic scope provides evidence in support of the existence of a \isi{VP} in \ili{Hidatsa} and \ili{Assiniboine}.
	
It has been argued for other Siouan languages (\citealt{Boyle2007}, \citealt{West2003}) that \isi{coordination} itself targets VPs, since \isi{coordination} can target a constituent that includes the \isi{object} and verb. In contrast, \isi{coordination} can never target the \isi{subject} and verb to the exclusion of the \isi{object}.  \citet{Boyle2007} shows that in \ili{Hidatsa}, the \isi{subject} of the second clause must be the same as the \isi{subject} of the first clause in \REF{ex:jrs:29}:

\begin{exe}
\ex\label{ex:jrs:29} 
\glll Alex w\'ia ik\'aaa r\'eec.\\
Alex w\'ia ik\'aa-a r\'ee-c \\
Alex woman see-\textsc{cont} leave-\textsc{decl} \\
\trans `Alex saw the woman and (Alex/*the woman) left.' \citep[217]{Boyle2007} 
\end{exe}

\citet{West2003} provides similar data from \ili{Assiniboine} to support a configurational analysis, as shown in \REF{ex:jrs:30} below:

\begin{exe}
\ex\label{ex:jrs:30} \gll W\'iy\~a 	 \v{z}e 		[wic\'a 	\v{s}e 	way\'aga] h\~ikn\'a 	[c\'eya]. \\
woman 	\textsc{det} 	man 		the see 				conj 		cry \\
\trans `The woman saw the man and cried.'

*`The woman saw the man and he cried' \citep[34]{West2003}
\end{exe}

As in \ili{Hidatsa}, the \isi{subject} of the second conjoined verb \textit{c\'eya} `cry' in \REF{ex:jrs:30} can only be \textit{w\'iy\~a} `the woman'.  In a nonconfigurational language, either NP should be able to be the \isi{subject} of the second verb; thus Boyle and West argue that \ili{Hidatsa} and \ili{Assiniboine} are configurational and have a \isi{VP} constituent.

\subsection{\ili{Hocąk} data}

In the previous subsection, we presented previous arguments for a configurational analysis of several Siouan languages.  In this section, we show that the tests used by \citet{Boyle2007} for \ili{Hidatsa}, \citet{Graczyk1991a} for \ili{Crow}, and \citet{West2003} for \ili{Assiniboine} yield the same results when applied to \ili{Hocąk}.
	
First, \isi{word order} is crucial to disambiguate subjects and objects in \ili{Hocąk}.  In \REF{ex:jrs:31} below, the first argument is interpreted as the \isi{subject}:

\begin{exe}
\ex\label{ex:jrs:31} 
\glll Wijukra	\v{s}\k{u}\k{u}kra 		haja.\\
wijuk-ra 		\v{s}\k{u}\k{u}k-ra 		$\varnothing$-haja \\
cat-\textsc{def} 		dog-\textsc{def} 	\textsc{3s/o}-see \\
\trans `The cat saw the dog.'

$\neq$ `The dog saw the cat'
\end{exe}

A reading in which the dog saw the cat is also possible for \REF{ex:jrs:31}, but only when the first argument is followed by an intonational pause.
	
As shown in the previous section, \citet{Boyle2007} and \citet{West2003} provided evidence from enclitic scope to show that \ili{Hidatsa} and \ili{Assiniboine} have a \isi{VP} constituent.  The same proves true in \ili{Hocąk}.  In \REF{ex:jrs:32}-\REF{ex:jrs:34} below, the enclitics \textit{g\k{i}n\k{i}} `already', \textit{ege} `might' and \textit{\v{z}ee\v{z}i} `hopefully' take scope over both coordinated verbs in the (b) examples, even though they are only attached to the second verb.

\begin{exe}
\ex\label{ex:jrs:32}
\begin{xlist}  
\ex
\glll Hunterga 	toora 	tuuc 	wahiig\k{i}n\k{i}. \\
Hunter-ga 	too-ra 	tuuc 	wa-$\varnothing$-hii=g\k{i}n\k{i} \\
	Hunter-\textsc{prop} potato-\textsc{def} 	be.cooked 	\textsc{3o.pl-3s-caus}=already \\
\trans `Hunter already cooked the potatoes.' 

\ex
\glll Hunterga  	toora 		tuuc 		wahii 	anąga 	warucg\k{i}n\k{i}. \\
Hunter-ga  too-ra	tuuc 	wa-$\varnothing$-hii 	 anąga wa-$\varnothing$-ruuc=g\k{i}n\k{i} \\
Hunter-\textsc{prop} potato-\textsc{def} 	be.cooked	 \textsc{3o.pl-3s-caus} and 3O.PL-3S-eat=already\\
\trans`Hunter already cooked the potatoes and ate them.'
\end{xlist}
\ex\label{ex:jrs:33}
\begin{xlist} 
\ex
\glll Matejaga 	tookewehiege. \\
Mateja-ga 	$\varnothing$-tookewehi=ege \\
	Mateja-\textsc{prop} 	\textsc{3s/o}-be.hungry=might \\
\trans `Mateja might (very well) get hungry.' 
\ex
\glll Matejaga 			tookewehi 		anąga 	kerege. \\
Mateja-ga 			$\varnothing$-tookewehi 	anąga 	$\varnothing$-kere=ege  \\
Mateja-\textsc{prop} 	\textsc{3s}-be.hungry and 			\textsc{3s}-leave=might \\
\trans `Mateja might (very well) get hungry and leave.'
\end{xlist}

\ex\label{ex:jrs:34}	
\begin{xlist} 
\ex
\glll Bryanga 			{n\k{i}\k{i}ta\v{s}jak taaxu }	ruw\k{i}\v{z}ee\v{z}i. \\
Bryan-ga 		{n\k{i}\k{i}ta\v{s}jak 	taaxu }	$\varnothing$-ruw\k{i}=\v{z}ee\v{z}i \\
	Bryan-\textsc{prop} 	coffee	\textsc{3s/o}-buy=wish \\
\trans `Hopefully Bryan will buy coffee.' 
\ex
\glll Bryanga 	{n\k{i}\k{i}ta\v{s}jak taaxu }	ruw\k{i} 		anąga 	h\k{u}\k{u}k'\k{u}\v{z}ee\v{z}i. \\
Bryan-ga 	{n\k{i}\k{i}ta\v{s}jak taaxu }	$\varnothing$-ruw\k{i}  	anąga 	<h\k{i}>$\varnothing$-hok'\k{u}=\v{z}ee\v{z}i \\
	Bryan-\textsc{prop} 	coffee 		\textsc{3s/o}-buy		 and 		\textsc{<1o>3s}-give=wish \\
\trans `Hopefully Bryan will buy coffee and give it to me.'
\end{xlist}
\end{exe}	
	
If \ili{Hocąk} lacked a \isi{VP}, this pattern would be unexpected: the clitics should only be able to scope over the verb to which they are attached. Instead, the clitics in the (b) examples above take scope over both coordinated verb phrases. This indicates that the constituent that clitics scope over is a \isi{VP}, and that these enclitics attach at the \isi{VP} level. 
	
Lastly, \citet{Boyle2007} and \citet{West2003} showed that \isi{coordination} targets VPs in \ili{Hidatsa} and \ili{Assiniboine}, providing further evidence for a configurational analysis of these languages.  Coordination also targets VPs in \ili{Hocąk}, as shown in \REF{ex:jrs:35} and \REF{ex:jrs:36} below. In these examples, the \isi{subject} of the first conjunct, \textit{wąąkwa\v{z}oon\k{i}ra} `the hunter', must also be the \isi{subject} of the second conjunct.  Example \REF{ex:jrs:36} is especially revealing, as the only possible meaning is not as pragmatically plausible: it would (arguably) be more likely for the bear to die in that scenario.

\begin{exe}
\ex\label{ex:jrs:35} 
\glll Wąąkwa\v{z}oon\k{i}ra 		hk{u}k{u}ra 			ruxe 				ank{a}ga 	t'eehii. \\
wąąkwa\v{z}oonk{i}-ra 	hk{u}k{u}c-ra 		$\varnothing$-ruxe  		ank{a}ga 	$\varnothing$-t'ee-hii \\
hunter-\textsc{def} 					bear-\textsc{def} 	\textsc{3s/o}-chase and 		\textsc{3s}-die-\textsc{caus} \\
\trans`The hunter chased and killed the bear.'
  \ex\label{ex:jrs:36} 
\glll Wąąkwa\v{z}oon\k{i}ra 		h\k{u}\k{u}cra 		guuc 				anąga 	t'ee. \\
wąąkwa\v{z}oon\k{i}-ra 	h\k{u}\k{u}c-ra 		$\varnothing$-guuc 			anąga 	$\varnothing$-t'ee \\
hunter-\textsc{def} 		bear-\textsc{def} 	\textsc{3s/o}-shoot 		and 		\textsc{3s}-die \\
\trans `The hunter shot the bear and [the hunter] died.'
\end{exe}
	
If there was no subject-\isi{object} asymmetry, either `hunter' or `bear' should be a possible \isi{subject} for the second conjuncts in \REF{ex:jrs:35} and \REF{ex:jrs:36}. Thus, these examples show that \isi{coordination} in \ili{Hocąk} targets a constituent that excludes the \isi{subject}; namely, the \isi{VP}.

\section{New Evidence for a \isi{VP} in Hocąk}\label{sec:jrs:4}

\subsection{Scope of Locatives}

The first piece of new evidence for a \isi{VP} involves the interpretation of locative adjuncts. The neutral position of locative adjuncts is shown in \REF{ex:jrs:37} with \textit{hoxataprookeeja} `in the woods' appearing between the \isi{object} and the verb.

\begin{exe}
\ex\label{ex:jrs:37} 
\glll Wijukra	\v{s}uukra		hoxataprookeeja		haja.\\
wijuk-ra	\v{s}uuk-ra		hoxatap-rook-eeja		$\varnothing$-haja \\
cat-\textsc{def}		dog-\textsc{def} 	woods-inside-there	\textsc{3s/o}-see \\
\trans `The cat saw the dog in the woods.' 
\end{exe}

The translation in \REF{ex:jrs:37} is ambiguous. The \ili{English} sentence has three possible interpretations, as outlined in \REF{ex:jrs:38} below.

\begin{exe}
\ex\label{ex:jrs:38} 
\begin{xlist}
\ex \label{ex:jrs:38a} The cat is in the woods, and it saw the dog. The dog is not in the woods. 
\ex \label{ex:jrs:38b} The dog is in the woods, and the cat saw the dog. The cat is not in the woods.
\ex \label{ex:jrs:38c}Both the cat and the dog are in the woods, and the cat saw the dog.
\end{xlist}
\end{exe}

In \ili{Hocąk}, however, only the interpretations in \REF{ex:jrs:38b} and \REF{ex:jrs:38c} are available for \REF{ex:jrs:37}; that is, the locative \isi{adjunct} must describe the location of the \isi{object}. This is true even if the locative \textit{hoxataprookeeja} `in the woods' is clause-initial or clause-final, as in \REF{ex:jrs:39a} and \REF{ex:jrs:39b} , respectively. These sentences cannot have the reading in \REF{ex:jrs:38a}, where only the dog can be in the woods.

\begin{exe}
\ex\label{ex:jrs:39}
\begin{xlist}
\ex 
\glll Hoxataprookeeja, 		wijukra	\v{s}uukra			haja.\\
hoxatap-rook-eeja		wijuk-ra	\v{s}uuk-ra		$\varnothing$-haja \\
	woods-inside-there 	cat-\textsc{def}		dog-\textsc{def}		\textsc{3s/o}-see \\
\trans `In the woods, the cat saw the dog.'
\ex 
\glll Wijukra	\v{s}uukra			haja,				hoxataprookeeja.\\
wijuk-ra	\v{s}uuk-ra		$\varnothing$-haja			hoxatap-rook-eeja \\
	cat-\textsc{def}		dog-\textsc{def}		\textsc{3s/o}-see	woods-inside-there \\
\trans `The cat saw the dog in the woods.'
\end{xlist}
\end{exe}
	
A nonconfigurational analysis cannot readily account for this subject-\isi{object} asymmetry: if \ili{Hocąk} had a \isi{flat structure}, we would not expect the locative to be able to modify only the \isi{object}.
	
Alternatively, we argue that the \isi{object} NP is the unique \isi{complement} to the verb. We account for the scope facts by suggesting that the locative phrase can merge in two locations. If the locative adjoins to the \isi{VP} (that is, the constituent that contains the \isi{object} and the verb) then the reading in (38b) is available: the locative only has scope over the \isi{object}. On the other hand, if the locative adjoins to a position above the \isi{VP}, then the reading in (38c) is obtained: the locative then scopes over both arguments.

\subsection{\isi{Verb Phrase} Ellipsis (\isi{VPE})}

As first discussed by \citet{Johnson2013}, \ili{Hocąk} displays a process of \isi{VPE} in which the \isi{light verb} \textit{\k{u}\k{u}} replaces the verb and the \isi{object}, to the exclusion of the \isi{subject} \REF{ex:jrs:40}:

\begin{exe}
\ex\label{ex:jrs:40} 
\glll Cecilga	wa\v{z}ątirehi\v{z}ą		ruw\k{i}	kjane		anąga	nee		\v{s}ge		ha\k{u}\k{u}			kjane.\\
Cecil-ga	wa\v{z}ątire-hi\v{z}ą	$\varnothing$-ruw\k{i}		kjane		anąga	nee	\v{s}ge ha-\k{u}\k{u}		kjane \\
Cecil-\textsc{prop}		car-\textsc{indef}			\textsc{3s/o}-buy		\textsc{fut}	and	I	also	\textsc{1s}-do	\textsc{fut} \\
\trans `Cecil will buy a car, and I will too.'
\end{exe}

The examples in \REF{ex:jrs:41} show that \isi{VPE} also targets certain adjuncts. (41a) shows that \isi{VPE} targets VPs containing temporal adjuncts. In (41b), a locative \isi{adjunct} is included in the \isi{ellipsis} site. (41c) exemplifies \isi{VPE} with a \isi{comitative}. In all of these examples, the \isi{adjunct} in the antecedent \isi{VP} is interpreted as being present in the \isi{ellipsis} site, indicating that \textit{\k{u}\k{u}} targets the entire \isi{VP} rather than just the \isi{object}.

\begin{exe}
\ex\label{ex:jrs:41}
\begin{xlist}
\ex 
\glll Cecilga 			xjanąre		wa\v{s}i anąga	Bryanga			\v{s}ge  	\k{u}\k{u}.\\
Cecil-ga			xjanąre		$\varnothing$-wa\v{s}i		anąga	Bryan-ga			\v{s}ge		$\varnothing$-\k{u}\k{u} \\
	Cecil-\textsc{prop}		yesterday	\textsc{3s}-dance	and			Bryan-\textsc{prop}	also	\textsc{3s}-do \\
\trans `Cecil danced yesterday, and Bryan did too.'
\ex 
\glll Cecilga 			ciinąk	eja		wa\v{z}ątirehi\v{z}ą		ruw\k{i}	anąga	Bryanga			\v{s}ge  \k{u}\k{u}. \\
Cecil-ga			ciinąk	eja		wa\v{z}ątire-hi\v{z}ą	$\varnothing$-ruw\k{i}			anąga	Bryan-ga	\v{s}ge \k{u}\k{u}. \\
Cecil-\textsc{prop}	city there	car-\textsc{indef}	\textsc{3s/o}-buy and Bryan-\textsc{prop}	also $\varnothing$-\k{u}\k{u} \textsc{3s}-do \\
\trans `Cecil bought a car in the city, and Bryan did too.'

\ex 
\glll Cecilga 		hin\k{u}kra	haki\v{z}u		wa\v{s}i		anąga	Bryanga			\v{s}ge		\k{u}\k{u}.\\
Cecil-ga		hin\k{u}k-ra	haki\v{z}u		$\varnothing$-wa\v{s}i		anąga	Bryan-ga	\v{s}ge		$\varnothing$-\k{u}\k{u} \\	Cecil-\textsc{prop}		woman-\textsc{def} 	be.with		3S-dance	and			Bryan-\textsc{prop}	also	\textsc{3s}-do \\
\trans `Cecil danced with the woman, and Bryan did too.'
\end{xlist}
\end{exe}

Constructions with \textit{\k{u}\k{u}} cannot be analyzed as a \textit{pro}-form, as \isi{object} extraction is permitted. \REF{ex:jrs:42a} shows that focused elements can be extracted from the \isi{ellipsis} site. Furthermore, antecedent-contained deletion (ACD) is also possible {ex:jrs:42b}. ACD would not be possible if \textit{\k{u}\k{u}} were a \textit{pro}-form, since the head of the \isi{relative clause} is the \isi{object} of the elided verb phrase.

\ea\label{ex:jrs:42}
\ea \label{ex:jrs:42a}
\glll Meredithga		waagaxra	ruw\k{i},		n\k{u}n\k{i}ge		\textbf{wiiwagaxra}	hąąke	\k{u}\k{u}n\k{i}.\\
Meredith-ga  waagax-ra	 $\varnothing$-ruw\k{i} 	n\k{u}n\k{i}ge		wiiwagax-ra	hąąke $\varnothing$-\k{u}\k{u}-n\k{i}\\
Meredith-\textsc{prop}	paper-\textsc{def}	\textsc{3s/o}-buy	but pencil-\textsc{def} \textsc{neg}	 \textsc{3s}-do-\textsc{neg}\\
\trans `Meredith bought paper but didn't (buy) pencils.'
\ex \label{ex:jrs:42b}
\glll  Bryanga			ruw\k{i},				\k{i}aagu		Meredithga			\k{u}\k{u}ra.\\
Bryan-ga			$\varnothing$-ruw\k{i}	\k{i}aagu		Meredith-ga	$\varnothing$-\k{u}\k{u}-ra\\
	Bryan-\textsc{prop}	\textsc{3s/o}-buy		what		Meredith-\textsc{prop}		3S-do-\textsc{comp}\\
\trans `Bryan bought what(ever) Meredith did.'
\z
\z

\isi{VPE} is also permitted in embedded \isi{clauses} and adjuncts, which is also inconsistent with a \textit{pro}-form analysis. \REF{ex:jrs:43a} exemplifies \isi{VPE} in an embedded clause, and \REF{ex:jrs:43b}--\REF{ex:jrs:43c} show that \isi{ellipsis} sites are licit inside \isi{adjunct} \isi{clauses}.

\ea\label{ex:jrs:43}
\ea\label{ex:jrs:43a}
\glll Bryanga	hąąke	{n\k{i}\k{i}ta\v{s}jak taaxu}		ruw\k{i}n\k{i},		n\k{u}n\k{i}ge		Meredithga \k{u}\k{u}ra 	yaaperes\v{s}ąną. \\
Bryan-ga hąąke	{n\k{i}\k{i}ta\v{s}jak taaxu} $\varnothing$-ruw\k{i}-n\k{i} n\k{u}n\k{i}g	Meredith-ga $\varnothing$-\k{u}\k{u}-ra	<ha>hiperes-\v{s}ąną \\
Bryan-\textsc{prop}	\textsc{neg}	coffee	 \textsc{3s/o}-buy-\textsc{neg} but	 Meredith-\textsc{prop} \textsc{3s}-do-\textsc{comp}	\textsc{<1s>}know-\textsc{decl} \\
\trans `Bryan didn't buy coffee, but I know Meredith did.'
\ex \label{ex:jrs:43b}
\glll Bryanga	\k{u}\k{u}	kjanegi	Meredithga	Hunterga	(ni\v{s}ge)	 {gi\v{s}ja hii} kjane.\\
Bryan-ga	$\varnothing$-\k{u}\k{u}  kjane-gi	 Meredith-ga	Hunter-ga	ni\v{s}ge  {$\varnothing$-gi\v{s}ja hii} kjane. \\
Bryan-\textsc{prop} \textsc{3s}-do	\textsc{fut}-if	 Meredith-\textsc{prop} Hunter-\textsc{prop}	 also	 \textsc{3s/o}-visit	 \textsc{fut} \\
\trans `Meredith will visit Hunter if Bryan will.'
\ex \label{ex:jrs:43c}
\glll Bryanga			hąąke	\k{u}\k{u}n\k{i}ge	Meredithga		(ni\v{s}ge)		hąąke Hunterga	{gi\v{s}ja hiin\k{i}.}\\
Bryan-ga	hąąke	$\varnothing$-\k{u}\k{u}-n\k{i}-ge	Meredith-ga	 ni\v{s}ge	hąąke  Hunter-ga  {gi\v{s}ja hii-n\k{i}} \\
Bryan-\textsc{prop}	\textsc{neg}	\textsc{3s}-do-\textsc{neg}-because Meredith-\textsc{prop} also	\textsc{neg} Hunter-\textsc{prop} \textsc{3s/o}-visit-\textsc{neg} \\
\trans `Meredith didn't visit Hunter because Bryan didn't.'
\z
\z

The presence of \isi{VPE} constitutes strong evidence for a configurational analysis of \ili{Hocąk}: in a \isi{flat structure}, there is no \isi{VP} constituent that can be targeted by \isi{ellipsis}. Since at least \citet{Ross1969}, the presence of \isi{VPE} in \ili{English} has been used as an argument in favor of a \isi{VP} constituent that contains the verb and \isi{object} to the exclusion of the \isi{subject}. \ili{Hocąk} also displays \isi{VPE}, which leads us to conclude that \ili{Hocąk} must have a \isi{VP} constituent.

\subsection{Quantifier scope}

Another piece of evidence in favor of a configurational analysis of \ili{Hocąk} comes from \isi{quantifier scope}. As discussed in \citet{Johnson2014} and \citet{JohnsonRosen2014}, linear order determines the scope of quantified phrases in \ili{Hocąk}. In a sentence with SOV \isi{word order}, the \isi{subject} obligatorily distributes over the \isi{object}. This is shown below in (44a), where the sentence can only describe a situation in which each man caught a different fish. However, the interpretation changes with SVO \isi{word order}: (44b) can only describe a situation in which each man caught the same fish. Lastly, in a sentence with OVS \isi{word order}, the \isi{subject} scopes over the \isi{object}, as shown in (44c).

\begin{exe}
\ex\label{ex:jrs:44}
\begin{xlist}
\ex 
\glll Wąąkra	hi\v{z}ąki\v{s}ąną		hoohi\v{z}ą	gisikire.\\
wąąk-ra		hi\v{z}ąki\v{s}ąną		hoo-hi\v{z}ą	$\varnothing$-gisik-ire. \\
		man-\textsc{def}		each					fish-\textsc{indef}		\textsc{3o}-catch-\textsc{3s.pl} \\
\trans `Each man caught a fish.' (each > a; *a > each)
\ex 
\glll Wąąkra		hi\v{z}ąki\v{s}ąną		gisikire,			hoohi\v{z}ą.\\
wąąk-ra		hi\v{z}ąki\v{s}ąną		$\varnothing$-gisik-ire,	hoo-hi\v{z}ą. \\
		man-\textsc{def}		each					\textsc{3o}-catch-\textsc{3s.pl}		fish-\textsc{indef} \\
\trans `Each man caught a fish.'  (a > each; *each > a)
\ex 
\glll Hoohi\v{z}ą	gisikire,	wąąkra	 hi\v{z}ąki\v{s}ąną.\\
hoo-hi\v{z}ą	$\varnothing$-gisik-ire,		wąąk-ra	hi\v{z}ąki\v{s}ąną. \\
		fish-\textsc{indef}		\textsc{3o}-catch-\textsc{3s.pl}		man-\textsc{def}		each \\
\trans `Each man caught a fish.' (each > a; *a > each)
\end{xlist}
\end{exe}

These facts cannot be adequately accounted for if the \isi{subject} and \isi{object} are in a \isi{flat structure} in \ili{Hocąk}: there is no principled way that linear order could account for the interpretation of the sentences in \REF{ex:jrs:44}. In contrast, the interpretation of basic SOV \isi{word order} in (44a) is straightforwardly explained under a \isi{VP} analysis: the \isi{subject} is higher than the \isi{object} and thus scopes over it. Furthermore, we follow \citet{Johnson2014} and  \citet{JohnsonRosen2014} and propose that postverbal objects (44b) and subjects (44c) obligatorily take wide scope because they undergo movement that targets a position high in the clause. 

\subsection{Resultatives and the Direct Object Restriction}

We now turn to an argument from resultatives in \ili{Hocąk}. Resultatives are complex predicates that put together a means \isi{predicate} (i.e., a verb) and a result \isi{predicate}, where neither is licensed by a \isi{conjunction} or an adposition \citep[507]{Williams2008}. As seen in \REF{ex:jrs:45}, \ili{Hocąk} exhibits resultatives: (45a) shows that the result \textit{paras} `flat' is immediately to the left the verb \textit{gistak} `hit', and a similar example is shown in (45b) with the result \textit{\v{s}uuc} `red and the verb \textit{hogiha} `paint'.

\begin{exe}
\ex\label{ex:jrs:45}
\begin{xlist}
\ex 
\glll Meredithga	mąąsra		paras	gistak\v{s}ąną.\\
Meredith-ga			mąąs-ra	paras	$\varnothing$-gistak-\v{s}ąną \\
	Meredith-\textsc{prop}		metal-\textsc{def}		flat	 \textsc{3s/o}-hit-\textsc{decl} \\
\trans `Meredith hit the metal flat.'
\ex 
\glll Cecilga	wa\v{z}ątirera	 \v{s}uuc	hogiha.\\
Cecil-ga	wa\v{z}ątire-ra \v{s}uuc	$\varnothing$-hogiha \\
	Cecil-\textsc{prop}	car-\textsc{def}	red		\textsc{3s/o}-paint \\
\trans `Cecil painted the car red.'
\end{xlist}
\end{exe}

Subjects and objects behave differently in the \isi{resultative} construction. First, only the \isi{object} can be modified by the result. Second, only prototypical unaccusative verbs can be used in the \isi{resultative} construction. We use both of these pieces of evidence to support our claim that there is a \isi{VP} constituent in \ili{Hocąk}.

It has previously been observed for other languages, such as \ili{English}, that the \isi{resultative} \isi{predicate} must be linked to the ``deep'' \isi{object} of the verb.  \citet{LevinRappaportHovav1995} refer to this constraint as the \textit{Direct Object Restriction} (henceforth, DOR). In particular, the restriction states that only the \isi{object} of a transitive verb or the \isi{subject} of an unaccusative verb can be modified by the result \isi{predicate}. In contrast, a result \isi{predicate} cannot be linked to the \isi{subject} of an unergative verb. Consider the representative \ili{English} examples below in \REF{ex:jrs:46}.

\begin{exe}
\ex\label{ex:jrs:46}
\begin{xlist}
\ex John hammered the metal flat.	(transitive)
\ex The water froze solid.						(unaccusative)
\ex *The dog barked hoarse.	(unergative; ungrammatical as \isi{resultative})
\end{xlist}
\end{exe}
	
\ili{Hocąk} resultatives obey the DOR. This is restriction is shown in \REF{ex:jrs:47} with the transitive verb \textit{gistak} `hit'.

\begin{exe}
\ex\label{ex:jrs:47} 
\glll Rockyga			wan\k{i}ra			\v{s}uuc 		gistak\v{s}ąną.\\
Rocky-ga		wan\k{i}-ra		\v{s}uuc		$\varnothing$-gistak-\v{s}ąną \\
Rocky-\textsc{prop}	meat-\textsc{def}	red			\textsc{3s/o}-hit-\textsc{decl} \\
\trans = `Rocky hit the meat red.'

$\neq$`Rocky hit the meat red and he was red as a result.'
\end{exe}

Since \textit{wan\k{i}ra} `the meat' is in \isi{object} position, it can be modified by the result, while the \isi{subject} of matrix verb Rocky cannot. Thus, \REF{ex:jrs:47} establishes a clear subject-\isi{object} asymmetry. If \ili{Hocąk} had a \isi{flat structure}, we would not expect the result to only be able to modify the \isi{object}. In other words, the asymmetry would be difficult to explain without the presence of a \isi{VP} constituent.
	
Furthermore, only unaccusative (as opposed to unergative; cf. \citealt{Perlmutter1978}) verbs are compatible with resultatives in \ili{Hocąk}. This is demonstrated by the contrast between \REF{ex:jrs:48} and \REF{ex:jrs:49}.

\begin{exe}
\ex\label{ex:jrs:48}
\begin{xlist}
\ex     
\glll Xaigirara			sgaasgap		ziibre.\\
xaigira-ra			sgaasgap		$\varnothing$-ziibre \\
	chocolate-\textsc{def}	sticky			\textsc{3s}-melt \\
\trans `The chocolate melted sticky.'
\ex  
\glll Waisgapra		seep		taaxu.\\
waisgap-ra		seep		$\varnothing$-taaxu \\
	bread-\textsc{def}		black		\textsc{3s}-burn \\
\trans `The bread burned black.'
\end{xlist}
\end{exe}
\begin{exe}
\ex\label{ex:jrs:49}
\begin{xlist}
\ex   [*]{ 
\glll Hinukra			n\k{i}\k{i}ra				teek	nąąwą.\\
hinuk-ra			n\k{i}\k{i}-ra				teek	$\varnothing$-nąąwą \\
		woman-\textsc{def}	 throat-\textsc{def}	sore	\textsc{3s/o}-sing \\
\trans (Intended: `The woman sang her throat sore.')}
\ex[*]{\label{ex:jrs:} 
\glll Henryga			wagu\k{i}irera		paras		nąąk\v{s}ąną.\\
Henry-ga		wagu\k{i}ire-ra	paras		$\varnothing$-nąąk-\v{s}ąną \\
		Henry-\textsc{prop}	shoe-\textsc{def}	flat	\textsc{3s/o}-run-\textsc{decl} \\
\trans (Intended: `Henry ran the shoe(s) flat.')}
\end{xlist}
\end{exe}

Prototypical unaccusatives, such as \textit{ziibre} `melt' and \textit{taaxu} `burn', can serve as the matrix verb of resultatives in \REF{ex:jrs:48}. On the other hand, prototypical unergative verbs, such as \textit{nąąwą} `sing' and \textit{nąąk} `run', cannot, as in \REF{ex:jrs:49}. Compare the \ili{Hocąk} examples in \REF{ex:jrs:49} to the \ili{English} example in (46c). (46c) is ungrammatical because there was no \isi{object} present for the result \isi{predicate} to modify. In contrast, while the \ili{Hocąk} examples in \REF{ex:jrs:49} have an \isi{object}, they are still ungrammatical.
	
Assuming \citegen{Perlmutter1978} unaccusative hypothesis, the single argument of an unaccusative verb is internal to the \isi{VP}, whereas the argument of an unergative verb is VP-external. The contrast between \REF{ex:jrs:48} and \REF{ex:jrs:49} provides evidence that \ili{Hocąk} has an unaccusative-unergative split:\footnote{To the best of our knowledge, such a split has not been previously observed in \ili{Hocąk}. However, see \citet{Williamson1984} and \citet{West2003}, among others, for possible unaccusative-unergative splits in \ili{Lakota} and \ili{Assiniboine}, respectively.}  if there were no such distinction between unaccusative and unergative verbs, \REF{ex:jrs:49} would be expected to be grammatical, contrary to fact. If the \ili{Hocąk} \isi{VP} were flat, we would not expect unergative verbs with resultatives to be ungrammatical. As a result, this shows that the \isi{VP} in \ili{Hocąk} is not flat: we conclude that the data in this section provides further evidence for a \isi{VP} in \ili{Hocąk}.

\subsection{Structure of the \ili{Hocąk} VP}

In the sections above, we have seen that \ili{Hocąk} shows subject-\isi{object} asymmetries with respect to \isi{word order}, the enclitic scope, and \isi{coordination}. These same subject-\isi{object} asymmetries have been previously documented in other Siouan languages. We also demonstrated that the facts from \isi{VPE}, resultatives and the scope of adjuncts and arguments constitute additional subject-\isi{object} asymmetries. The fact that we find so many asymmetries between the \isi{subject} and \isi{object} indicates that the \isi{subject} and the \isi{object} do not both form a constituent with the verb. Instead, we argue that these facts can be accounted for if the \isi{object} is the \isi{complement} of the verb in a \isi{VP} constituent. The \isi{subject} is base generated in a phrase that is external to the \isi{VP}, which we tentatively label ``\isi{XP}.'' A basic transitive verb phrase is represented in \REF{ex:jrs:50}.

\begin{exe}
\ex\label{ex:jrs:50} 
\Tree [ .\isi{XP} [ .Subject ] [ .\isi{VP} [ .Object ] [ .Verb ] ] ] 
\end{exe}

\section{Conclusion}\label{sec:jrs:5}

The question of whether Siouan languages are configurational or nonconfigurational has been under debate for the past three decades. In this paper, we have presented new evidence to support a configurational analysis of \ili{Hocąk}.  We first showed that the tests previously used by \citet{Boyle2007} for \ili{Hidatsa}, \citet{Graczyk1991a} for \ili{Crow} and \citet{West2003} for \ili{Assiniboine} to argue in favor of a \isi{VP} constituent are also applicable in \ili{Hocąk}.  Next we presented novel evidence from \isi{locative scope} verb phrase \isi{ellipsis}, \isi{quantifier scope}, and \isi{resultative} constructions which further support our claim that a \isi{VP} constituent exists in \ili{Hocąk}. 

\section*{Acknowledgments}
We would like to extend our deepest thanks to Cecil Garvin\ia{Garvin, Cecil}, our language consultant, without whom this research would never have come to be. Thanks also to Iren Hartmann\ia{Hartmann, Iren} for access to her Lexique Pro \isi{dictionary}.

\section*{Abbreviations}
The abbreviations used in the \ili{Hocąk} examples are: 1, 2, 3 = first, second, third person; \textsc{comp} = complementizer; \textsc{decl} = declarative; \textsc{def} = definite; \textsc{dur} = durative; \textsc{fut} = future; \textsc{indef} = indefinite; \textsc{neg} = negative; \textsc{o} = \isi{object} agreement; \textsc{poss} = possessive; \textsc{q} = question; \textsc{prop} = proper noun; \textsc{pst} = past tense; \textsc{pl} = plural; \textsc{refl} = reflexive; \textsc{s} = \isi{subject} agreement; \textsc{sg} = singular. The glosses for data from other languages follow the conventions of the works they are drawn from.
 
 

\printbibliography[heading=subbibliography,notkeyword=this]

\end{document} 