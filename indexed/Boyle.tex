% 11
\documentclass[output=paper]{LSP/langsci}
\author{John P. Boyle}
\title{The syntax and semantics of internally headed relative clauses in {Hidatsa}}

\abstract{Hidatsa is a highly endangered member of the Missouri Valley branch of the Siouan language family. Like all Siouan languages, Hidatsa is a left-branching SOV language with many polysynthetic characteristics. Hidatsa specifically, and Siouan in general, is typologically unusual in that it only allows relative clauses to be internally headed. This paper examines the structure of internally headed relative clauses (IHRCs) in Hidatsa. I provide a syntactic analysis within the Minimalist Program as proposed by \citet{Chomsky1995, Chomsky1998, Chomsky2001a} and show that IHRCs are nominalized sentences that serve as DP in larger superordinate clauses. This extends previous analyzes of IHRC for Native American languages, most notably \citet{Williamson1987}, \citet{Cole1987}, \citet{Culy1990}, and \citet{Basilico1996} among others. I then show that Hidatsa, like other languages with IHRCs, obeys the indefiniteness restriction first proposed by \citet{Williamson1987} for Lakota.  Employing the general framework first proposed by \citet{Heim1982} and developed by \citet{Williamson1987} and \citet{Basilico1996}, I provide a semantic explanation for the fact that the head noun of an IHRC cannot be marked as definite. Following \citet{Williamson1987} and \citet{Culy1990}, I treat the indefinite determiner that follows the head of the IHRC as a variable and not as a quantifier. I argue that the head noun must move covertly to a [SPEC, CP] position to escape VP-level existential closure. 
% KEYWORDS: [Hidatsa, internally headed relative clause, Indefiniteness Restriction, nominalization]
}
\ChapterDOI{10.17169/langsci.b94.174}

\maketitle

\begin{document}

\section{Introduction}  

\ili{Hidatsa} is a highly endangered member of the Missouri Valley branch of the Siouan language family and is spoken by approximately 175 speakers, primarily on the Ft. Berthold Indian Reservation in North \ili{Dakota}. Typologically, \ili{Hidatsa} is left branching with a complement-head SOV \isi{word order} and, like most other Siouan languages, it has an active-stative pronominal system. Along with \ili{Crow}, the other member of the Missouri Valley branch of Siouan, it has many polysynthetic characteristics, most notably productive \isi{noun incorporation} \citep{Rankinetal2003, Boyle2007}. One aspect of \ili{Hidatsa} that is of particular interest is the structure of its relative \isi{clauses}.
	
Like many other Siouan languages (see \citealt{Drummond1976} and \citealt{Cumberland2005} (\ili{Assiniboine}); \citealt{Williamson1987} and \citealt{RoodTaylor1996} (\ili{Lakota}); \citealt{Quintero2004} (\ili{Osage}), and \citealt{Graczyk1991b}, \citeyear{Graczyk2007} (\ili{Crow}), among others), relative \isi{clauses} in \ili{Hidatsa} are internally headed. This means that the noun that is modified by the \isi{relative clause} is positioned within the \isi{relative clause} and is not external to it, as happens in languages like \ili{English}.
	
In \sectref{sec:boyle:2}, I will provide data that shows that \ili{Hidatsa} RCs are nominalized \isi{clauses} that act as DPs for superordinate predicates. I will also detail how IHRCs in \ili{Hidatsa} differ with regard to \isi{specificity}. In \sectref{sec:boyle:3}, I will discuss the notion of \textsc{head} and show that the head of the IHRC in \ili{Hidatsa} does indeed stay low in the subordinate construction (counter to claims made by \citet{Kayne1994}, \citet{Bianchi1999} and \citet{DiSciullo2005}). In \sectref{sec:boyle:4}, I will examine several previous works on IHRC, most notably \citet{Williamson1987} and \citet{Culy1990}. In \sectref{sec:boyle:5}, I will provide a syntactic analysis within a minimalist framework showing how the derivation of the IHRC is built in \ili{Hidatsa}. In \sectref{sec:boyle:6}, I will argue that the head of the clause must move covertly in order to escape the \isi{existential closure} of the \isi{VP}. This is motivated by a semantic necessity for full interpretation. This analysis explains Williamson's \isi{indefiniteness restriction} for IHRCs. \sectref{sec:boyle:7} will provide a brief conclusion as well as avenues for further research.

\section{The status of \ili{Hidatsa} RCs}\label{sec:boyle:2}

\subsection{\ili{Hidatsa} RCs as DPs}\label{sec:boyle:2.1}

In \ili{Hidatsa}, RCs are nominalized constructions that act as DP for superordinate \isi{clauses}. As stated above, \ili{Hidatsa} has an SOV \isi{word order} as shown below in \REF{boyle1}.

\ea \textbf{Common S-O-V sentence with simple DP arguments} \label{boyle1}

\glll w\'ia\v{s} wac\'ee\v{s} \'ikaac\\
w\'ia-\v{s}  wac\'ee-\v{s} \'ikaa-c\\
woman-\textsc{det.d} man-\textsc{det.d} see-\textsc{decl}\\
\trans`The woman saw the man.' 
\z

In \ili{Hidatsa}, RCs can serve in any syntactic role that can be filled by an ordinary DP. In example \REF{boyle2}, the RC (in brackets) is \isi{subject} of the main clause.  

\ea \textbf{RC serving as subject} \label{boyle2}

\glll {\ob}wac\'ee akuwaap\'aahi\v{s}{\cb}  w\'ia \'ikaac\\	
[wac\'ee aku-waap\'aahi-\v{s}] w\'ia  \'ikaa-c\\
[man \textsc{rel.s}-sing-\textsc{det.d}] woman see-\textsc{decl}\\
\trans`The man that sang saw the woman.' 
\z

In \REF{boyle3}, the RC is the \isi{object} of the main clause.

\ea \textbf{RC serving as object} \label{boyle3}

\glll w\'ia\v{s}\'e\textipa{P}eri  {\ob}wac\'ee akuwaap\'aahi\v{s}{\cb}  \'ikaac\\
w\'ia-\v{s}\'e\textipa{P}e-ri  [wac\'ee aku-waap\'aahi-\v{s}] \'ikaa-c\\
woman-\textsc{dem-top} [man \textsc{rel.s}-sing-\textsc{det.d}] see-\textsc{decl}\\
\trans `The woman saw the man that sang.' 
\z

In both examples \REF{boyle2} and \REF{boyle3} the \isi{word order} of the main clause is SOV. RCs can also be \isi{possessed} as shown in \REF{boyle4}. In these types of constructions, the \isi{possessed} RC takes a possessive prefix, in this case the 3rd person possessive prefix \textit{ita}-.

\ea \textbf{Possessed RC} \label{boyle4}

\glll r\'aaruwa  {\ob}ita\textipa{P}aru\textipa{P}ap\'axihe{\cb}   h\'iiwareec\\
r\'ee-a-ruwi-a  [ita-aru-ap\'axi-hee] h\'ii-wareec\\
go-\textsc{cont}-along-\textsc{cont} [\textsc{3.poss.a-rel.n}-stop.rest-\textsc{3.caus.d.sg}] arrive-\textsc{ne}\\
\trans `Going along, he (First Worker) arrived at the place where Sun stops to rest.' (\citealt[I: 21]{Lowie1939})
\z

RCs in \ili{Hidatsa} can also function as objects of postpositions, shown in example \REF{boyle5}. In these types of constructions, the head of the RC takes a postpositional suffix. In this example, the postpositional suffix is \textit{-wahu} `inside' (shown in bold). 

\ea \textbf{Object of a postpositional phrase} \label{boyle5}

\glll r\'aaruwaa   {\ob}wiraw\'ahu  aru\v{s}iipik\'aatikua{\cb}   h\'irawa w\'aakiruk u\textipa{P}\'u\v{s}iawareec\\
r\'aa-ruwa-a [wira-\textbf{w\'ahu} aru-\v{s}iipi-k\'aati-kua] h\'irawi-a w\'aaki-ruk u\textipa{P}\'u\v{s}ia-wareec\\     
go-continue-\textsc{cont} [woods-\textbf{inside}  \textsc{rel.n}-thick-\textsc{emph-loc}] sleep-\textsc{cont} be.there-\textsc{temp} arrive-\textsc{ne}\\
\trans `Going along in the woods where it is very thick, he (First Worker) arrived while (Spotted Tail) was still sleeping.' (\citealt{Lowie1939}, II:54)
\z

In examples \REF{boyle2}-\REF{boyle5}, each of the RCs act as a DP for the main clause, serving as an argument in the same manner as a common DP.

\subsection{Specificity in \ili{Hidatsa} RCs}\label{sec:boyle:2.2}

\ili{Hidatsa} RCs are all formed in a similar manner. They are nominalized \isi{clauses} that can take zero, one or two overt DP arguments depending on the transitivity of the nominalized verb. This clause then acts as a DP for a superordinate clause. It is the entire RC that takes a final determiner, thus making it available as an argument for another \isi{predicate}. The overt syntactic structure is shown in \REF{boyle6}.

\ea \textbf{The \ili{Hidatsa} RC structure} \label{boyle6}

[(DP) (DP) \textsc{rel}-verb]-\textsc{det}
\z

As shown in \REF{boyle6}, the verb (or \isi{predicate}) is relativized by a relative marker. \ili{Hidatsa} has two relative markers, \textit{aku}- (shown in \REF{boyle2} and \REF{boyle3}) and \textit{aru}- (shown in \REF{boyle4} and \REF{boyle5}). The first marker, \textit{aku}-, marks for \isi{specificity} and speakers prefer to use it for animate entities. An example of this is shown in \REF{boyle7}.

\ea \textbf{The \ili{Hidatsa} specific relative marker} \label{boyle7}

\glll wa\v{s}\'uka aku\textipa{P}\'awaka\v{s} wak\'itʰaac\\
wa\v{s}\'uka \textbf{aku}-\textipa{P}-\'awaka-\v{s}  wa-k\'ia-tʰaa-c\\
dog \textbf{\textsc{rel.s}}-\textsc{epe}-\textsc{1a}.see-\textsc{det.d} \textsc{1a}-fear-\textsc{neg-decl}\\
\trans `I am not afraid of that dog that I see.' 
\z

The second marker, \textit{aru}-, marks nonspecific arguments and is often used for inanimate objects or entities. An example of this is shown in \REF{boyle8}.

\ea \textbf{The \ili{Hidatsa} non-specific relative marker} \label{boyle8}

\glll wa\v{s}\'uka aru\textipa{P}aw\'aka\v{s} wak\'itʰaac\\
wa\v{s}\'uka \textbf{aru}-\textipa{P}-aw\'aka-\v{s}  wa-k\'ia-tʰaa-c\\
dog \textbf{\textsc{rel.n}}-\textsc{epe}-\textsc{1a}.see-\textsc{det.d} \textsc{1a}-fear-\textsc{neg-decl}\\
\trans `I am not afraid of a dog that I see.' 
\z

The overriding attribute of these markers is one of \isi{specificity} and not animacy. It should be noted that neither of these markers is a relative pronoun. There is no gap in the \isi{relative clause}, as in \ili{English}, nor do they act as an argument projected by the verb. This is to say, they do not satisfy the subcategorization frame of the verb.  The subcategorization frame of the verb can only be satisfied by an overt DP argument or signaled by a pronominal prefix, which agrees with a pro.\footnote{\ili{Hidatsa} is an active-stative language with regards to its pronominal system and only the first and second person pronominal prefixes are overt. The third person marker is null. In this paper, I treat the pronominal prefixes as agreement markers on the verb that agree with a null pro that occupies an argument position.} The difference in \isi{specificity} does nothing to alter any aspect of the \isi{syntax} and as such, they can be treated identically in the syntactic analysis presented below.

\section{The notion of head and internal heads}\label{sec:boyle:3}

The term internally headed \isi{relative clause} was first coined by \citet{Gorbet1976} in his description of Diegue\~no \isi{nominals}.  In some of the subsequent literature, there has been confusion about the notion of `head'. \citet{Coleetal1982} and \citet{Weber1983} (among others) have referred to IHRCs as `headless relative \isi{clauses}'. This was based on the notion that the head noun that was being modified by the \isi{relative clause} was not external to the RC as in languages like \ili{English}. In this paper, the notion of `head' is a semantic one. It refers to the noun being modified by the \isi{relative clause}. It is not to be confused with the `head' in an X-bar construction (i.e. a X\textsuperscript{0} category), which is a syntactic notion. As IHRCs are sentences, and sentences never have DPs as their head, the head of the IHRC cannot be the syntactic head of the sentence.  The term `headless \isi{relative clause}', as used by the authors mentioned above, is misleading for another reason as well.  This reason is that the head noun modified by the clause may be null.  An example of a RC in \ili{Hidatsa} with an overt head is shown in \REF{boyle9}.

\ea \textbf{\ili{Hidatsa} \isi{relative clause} with an overt head} \label{boyle9}

\glll Mary \textbf{uuw\'aki} akuh\'iri\v{s} war\'ucic\\
Mary \textbf{uuw\'aki} aku-h\'iri-\v{s}      wa-r\'uci-c\\
Mary \textbf{quilt} \textsc{rel.s}-make-\textsc{det.d} \textsc{1a}-buy-\textsc{decl} \\
\trans `I bought the quilt that Mary made.' 	
\z
This construction can be juxtaposed with the example in \REF{boyle10}, where the head that is modified by the RC is null.

\ea \textbf{\ili{Hidatsa} \isi{relative clause} with a null head} \label{boyle10}

\glll Mary \textbf{e} akuh\'iri\v{s} war\'ucic\\
Mary \textbf{e} aku-h\'iri-\v{s}   wa-r\'uci-c\\
Mary \textbf{e} \textsc{rel.s}-make-\textsc{det.d} \textsc{1a} -buy -\textsc{decll}\\ 
\trans `I bought (what/something/it) Mary made.'
\z

As stated above, the subcategorization frame of the verb is satisfied by the projection of a \textit{pro}. There is also no overt agreement marker on the verb as the third person agreement prefix is a null affix. This is not possible in \ili{English} as is clearly reflected in the gloss with the insertion of the \ili{English} third person indefinite (\textit{what/something/it}). For the purposes of this paper, the example shown in \REF{boyle10} is an example of a `headless' \isi{relative clause} (i.e. the head that is modified by the RC is null or not phonologically overt). Given this evidence, we can now define an IHRC (following \citealt[27]{Culy1990}) as: ``A (restrictive) internally headed \isi{relative clause} is a nominalized sentence, which modifies a nominal, overt or not, internal to the sentence.'' 

\subsection{Internal heads}\label{sec:boyle:3.1}

In languages with internally headed relative \isi{clauses}, the head of the RC is always internal to the RC and not part of the superordinate clause. This is counter to the claims of \citet{Kayne1994}, which were elaborated on by \citet{Bianchi1999} and \citet{DiSciullo2005}. Kayne claims that the heads of these \isi{clauses} are always the left-most element, and thus are really outside of the relative construction itself.  While this may appear to be correct for IHRCs with intransitive verbs and transitive verbs where the head noun corresponds to the \isi{subject} of the RC, it cannot account for IHRCs where the head noun corresponds to the \isi{object} of the RC. Following the copy theory of movement, it could be claimed that the copy at the relative head position is deleted in IHRCs, in contrast to EHRCs, in which the copy in the tail of the movement chain is deleted. This would suggest that there is a choice in pronunciation between the head and the tail of the chain in the formation of relative \isi{clauses}. However, there is no evidence or motivation for the head of the RC to move out of the clause to the left periphery and then not be pronounced in that `external' position, contravening the typical pattern for copy deletion. This is particularly problematic when the head noun is an internal argument in the clause.\footnote{A more detailed critique of Kayne's analysis of IHRCs can be found in \citet{Boyle2007}.}  A clear example of this is shown in \REF{boyle11}.

\ea \textbf{\ili{Hidatsa} internally headed relative clause} \label{boyle11}

\glll {\ob}wac\'ee\v{s} \textbf{wa\v{s}\'ukawa} akut\'ihee\v{s}{\cb} \v{s}ip\'i\v{s}ac\\
[wac\'ee-\v{s}    \textbf{wa\v{s}\'uka-wa}  aku-t\'i-hee-\v{s}]  \v{s}ip\'i\v{s}a-c\\
[man-\textsc{det.d} \textbf{dog-\textsc{det.i}}  \textsc{rel.s}-die-\textsc{3.caus.d.sg}-\textsc{det.d}] black-\textsc{decl}\\
\trans `The dog [that the man killed] is black.' 
\z

In this sentence, the \isi{relative clause} \textit{wac\'ee\v{s} wa\v{s}\'ukawa akut\'ihee\v{s}} contains the noun that is modified, namely \textit{wa\v{s}\'ukawa} `a dog'. In the \ili{English} translation, the \isi{relative clause} \textit{that the man killed} does not contain the noun that is modified by the \isi{relative clause}. In \ili{English}, the head is outside of the clause. This is not the case in \ili{Hidatsa} and languages like it.  In these languages, the head stays low and is internal to the subordinate clause, hence the name \textit{internally headed relative clause}.
	
This type of \isi{relative clause} is clearly subordinate to the matrix (or superordinate) clause. Number marking on the different verbs in \REF{boyle12} provides evidence of this subordinate relationship. In \ili{Hidatsa}, verbs agree in number (singular or plural) with their \isi{subject}. This is shown in \REF{boyle12}.

\ea \textbf{Plural agreement in a \ili{Hidatsa} relative clause} \label{boyle12}

\glll Alex {\ob}w\'iaaku\textipa{P}o uuw\'aki akuh\'ira\textipa{P}a\v{s}{\cb} \'ikaac\\
Alex [w\'ia-aku-\textipa{P}o    uuw\'aki  aku-h\'iri-\textipa{P}a-\v{s}]  \'ikaa-c\\
Alex [woman-\textsc{det.spec-pl.i} quilt  \textsc{rel.s}-make-\textsc{pl.d}-\textsc{det.d}] see-\textsc{decl}\\
\trans `Alex saw [those women who made the quilt].' 
\z
In this sentence, the matrix verb \textit{\'ikaa-} `see' agrees with the \isi{subject} of the main clause and is marked as singular in number. This is shown with zero marking, and it agrees with the \isi{subject} \textit{Alex}. The subordinate verb in the \isi{relative clause} \textit{h\'iri-} `make' shows plural agreement. This is shown with the definite plural marker \textit{-\textipa{P}a-}.  The verb \textit{h\'iri-} agrees with its \isi{subject}, \textit{w\'iaaku\textipa{P}o} `those women'. This argument, \textit{w\'iaaku\textipa{P}o}, is also the head of the \isi{relative clause}, which is marked with the indefinite plural marker \textit{-\textipa{P}o-}. In IHRCs, the head occupies an argument position that is determined by its role in the subordinate \isi{relative clause}, in this case the \isi{subject}. In this example there is agreement in number between the subjects and the verbs in both \isi{clauses}. The entire RC, not merely the head noun, then acts as an argument for a superordinate verb. In \REF{boyle12} the RC is the \isi{object} of the main verb.

\section{The \isi{syntax} of IHRCs}\label{sec:boyle:4}

Although a number of people\footnote{These include \citet{HalePlatero1974}, \citet{Gorbet1976}, \citet{Fauconnier1979}, \citet{Cole1987}, \citet{Culy1990}, \citet{Kayne1994} \citet{Bianchi1999}, \citet{Citko2001} among others.} have worked on IHRCs, there are two analyses that I will review and build upon. These are \citet{Williamson1987} who developed the notion that IHRCs have an \isi{indefiniteness restriction}, and \citet{Culy1990} who argues for a wh element that triggers movement of the \isi{internal head} to [SPEC, CP] at LF.

\subsection{The indefiniteness restriction}\label{sec:boyle:4.1}

In her 1987 article, Williamson observed that in \ili{Lakota} the head in an IHRC cannot be marked as definite.\footnote{It should be noted that while the head of an IHRC cannot be marked as definite, the head may take either an indefinite determiner or no determiner at all.} This ``\isi{indefiniteness restriction}'' has been generalized to IHRCs in general and to date no evidence has emerged to counter this claim. This \isi{indefiniteness restriction} holds true for \ili{Hidatsa}.  Examples \REF{boyle13} and \REF{boyle14} are both grammatical in \ili{Hidatsa}.

\ea  \label{boyle13}
\glll wac\'eewa aku\textipa{P}aw\'aka\v{s} maap\'aahic\\
\textbf{wac\'ee-wa}     aku-aw\'aka-\v{s}       maa-p\'aahi-c\\
\textbf{man-\textsc{det.i}} \textsc{rel.s-1a}.see-\textsc{det.d} \textsc{indef}-sing-\textsc{decl}\\ 
\trans `The man that I saw sang (something).' 
\z

\ea \label{boyle14}
\glll wac\'ee aku\textipa{P}aw\'aka\v{s} maap\'aahic\\
\textbf{wac\'ee}  aku-aw\'aka-\v{s} maa-p\'aahi-c\\
\textbf{man} \textsc{rel.s-1a}.see-\textsc{det.d} \textsc{indef}-sing-\textsc{decl}\\
\trans `The man that I saw sang (something).' 
\z

It is important to note that the \ili{English} gloss of both \REF{boyle13} and \REF{boyle14} uses the definite article for the head of the \isi{relative clause} [\textit{the man}]. This is because \textit{the man} gets its definiteness from the determiner that nominalizes the \isi{predicate} \textit{aw\'aka} `see', which in both examples is the definite determiner \textit{-\v{s}}. While both \REF{boyle13} and \REF{boyle14} are grammatical in \ili{Hidatsa}, example \REF{boyle15} is not.

\ea \label{boyle15}
\glll *wac\'ee\v{s} aku\textipa{P}aw\'aka\v{s} waap\'aahic\\
\textbf{wac\'ee-\v{s}}    aku-aw\'aka-\v{s}   waa-p\'aahi-c\\
\textbf{man-\textsc{det.d}} \textsc{rel.s-1a}.see-\textsc{det.d} \textsc{indef}-sing-\textsc{decl}\\
\trans intended: `The man that I saw sang.' 
\z

In \REF{boyle15} we see that Williamson's claim that the head of an IHRC cannot be marked definite holds true for \ili{Hidatsa}. The definite determiner on the head of the IHRC makes the sentence ungrammatical.

\subsection{Williamson's analysis of Lakota}\label{sec:boyle:4.2}

Williamson's \citeyearpar{Williamson1987} analysis of the relative structure is that shown in \REF{boyle16}. In this structure the S\footnote{Although I will use a Minimalist framework and the associated terminology for my analysis, I will employ the older terminology used by the various authors cited so as to not confuse their arguments.  For example many authors use S for TP and S$'$ for CP.  This older terminology will be retained when showing older examples written in early P \& P or pre P \& P frameworks.} has a determiner for a sister. 

\ea \textbf{\citegen{Williamson1987} model at S-structure}\label{boyle16}

\Tree [ .NP\textsubscript{i} [ .S [ .- ] [ .NP\textsubscript{i} ] [ .- ] ] [ .DET ] ]	
\z                 

This analysis is what Williamson posits for a representation of the overt \isi{syntax}, or S-Structure. We can see that the IHRC acts as a sentence or subordinate clause in its own right. This entire structure then acts as an argument or DP for the matrix verb. Williamson posits that the \isi{internal head} obligatorily moves outside of the IHRC at LF, giving the structure shown in \REF{boyle17}.

\ea \textbf{\citegen{Williamson1987} model after movement at LF} \label{boyle17}

\Tree [ .NP\textsubscript{i} [ .S$'$ [ .S [ .- ] [ .t\textsubscript{i} ] [ .- ] ] [ .NP\textsubscript{i} ] ][ .DET ] ]		   
\z              
	
\citeauthor{Williamson1987} further posits a cyclic rule that co-indexes the \isi{internal head} with the DP dominating the IHRC. Although Williamson's syntactic analysis is adequate, both \citet{Hoeksema1989} and \citet{Culy1990} point out that the movement at LF is unmotivated in general. Nonetheless, many researchers have adopted Williamson's analysis. Williamson's major insight into the structure of IHRCs is her observation and explanation of the fact that the head of an IHRC cannot be marked with a definite determiner.  

\subsection{Culy's analysis of IHRCs}\label{sec:boyle:4.3}

\citet{Culy1990} describes the syntactic structure of an IHRC as that shown in \REF{boyle18}.
 
\ea \label{boyle18} \textbf{\citegen{Culy1990} structure of an IHRC.}

\Tree [ .NP\textsubscript{i} [ .- ] [ .N$'$  [ .S [ .-  ] [ .NP\textsubscript{i} ] [ .- ] ]  ] [ .- ]  ] 
\z                        
                        
The main problem with this structure is that it violates the endocentric constraint of X-bar \isi{syntax}, which states that all phrases must project a head and all heads must project a phrase. In the schema proposed above in \REF{boyle18}, the N$'$ exhaustively dominates an S, not an N. Culy claims, ``while this structure is unusual, it is similar to a rule proposed by \citet{Jackendoff1977a} that allows `category-switching'\,''. That rule is:

\begin{center}
	x\textsuperscript{i}  $\rightarrow$  af  $\rightarrow$ y\textsuperscript{i}
\end{center}

The difference, according to \citeauthor{Culy1990}, is that in an IHRC construction the two categories N$'$ and S are not at the same level. Culy's argument for this structure is as follows; N$'$ dominates S because IHRCs are nominalized sentences and as such occur with the elements of a NP that also occur with N$'$. That is to say, IHRCs have an internal structure of S but an external distribution of N$'$.

	Culy also states that there does not seem to be any framework-independent evidence about whether the IHRC is an S (TP) or an S$'$ (CP) and that there is no strong evidence that IHRCs have overt complementizers.  
	
	\citeauthor{Culy1990} makes one other important prediction about IHRCs that is relevant to this discussion. He proposes that IHRCs have a \textit{wh} element which is not overt. He uses this notion to provide motivation for the movement at LF that Williamson proposes. As shown in \REF{boyle18}, Culy (like \citeauthor{Williamson1987}) argues that there is coindexation between the DP dominating the IHRC and the internal DP being modified (the head). This coindexation is similar to that which exists between a relative pronoun and its antecedent. Culy bases his argument on \citet{Safir1986} who proposes that in \ili{English} relative \isi{clauses} without relative pronouns, there is a null \textit{wh} operator\footnote{Safir terms this operator an ``abstract A$'$ binder'', which he then represents as a \textit{wh}.} which functions like a relative pronoun. Given this assumption, Culy shows that the \isi{relative clause} in \REF{boyle19a} has the S-structure shown in \REF{boyle19b}.

\ea \textbf{\ili{English} \isi{relative clause} without a relative pronoun}
\begin{xlist}
\ex the dog I ran away from	 \label{boyle19a}
\ex {[\textsubscript{NP} the dog\textsubscript{i} [\textsubscript{S$'$} \textit{wh}\textsubscript{i} [\textsubscript{S} I ran away from e\textsubscript{i} ] ] ]} \label{boyle19b}
\end{xlist}
\z

Culy proposes that any common noun can optionally act as a \textit{wh} operator in languages with IHRCs. This operator then moves from the head to COMP (or C) at LF via the rule of wh construal, just as in-situ \textit{wh} elements must move at LF.  Culy then proposes a tree structure for both D-structure and S-structure that is the same in all relevant respects. This is shown in \REF{boyle20}.

\ea \textbf{D- and S-structure of an IHRC (\citealt{Culy1990})} \label{boyle20}

\Tree [ .NP\textsubscript{i} [ .- ] [ .N$'$ [ .S$'$ [ .COMP ] [ .S [ .- ] [ .NP\textsubscript{i} [ .- ] [ .N\textsubscript{i} [ .\textit{wh}\textsubscript{i} ] ] [ .- ] ] [ .- ] ] ] ] [ .- ] ]          		   
\z 
   
Additionally, to account for the construction in \REF{boyle20}, Culy proposes a general rule that coindexes the NP (that is the IHRC) and the \textit{wh} operator. To do this he proposes the LF structures for EHRCs shown in \REF{boyle21} and the LF structure for IHRCs shown in \REF{boyle22}.

\ea	\textbf{LF structure for an EHRC (\citealt{Culy1990})} \label{boyle21}

\Tree [ .NP\textsubscript{i} [ .NP\textsubscript{i} ] [ .S$'$ [ .COMP [ .wh\textsubscript{i} ] [ .X ] ] [ .S [ .- ] [ .NP\textsubscript{i} [ .e\textsubscript{i} ] ] [ .- ] ] ] ]   
\z

\ea \textbf{LF structure of an IHRC (\citealt{Culy1990})} \label{boyle22}

\Tree [ .NP\textsubscript{i} [ .- ] [ .N$'$   [ .S$'$ [ .COMP [ .wh\textsubscript{i} ] [ .X ] ] [ .S [ .- ] [ .NP\textsubscript{i} [ .- ] [ .N\textsubscript{i} ] [ .- ] ] [ .- ] ] ] ]  [ .- ] ]
\z

\citeauthor{Culy1990} points out that these structures both have an NP dominating an S$'$ with a \textit{wh} element in its COMP that is coindexed with the NP internal to the clause. He further states that while it has generally been assumed that coindexation in EHRCs is between a \textit{wh} element and an NP that is the head, this is not a necessary assumption.  The same effect can be accomplished by coindexing the \textit{wh} element with the NP dominating the \isi{relative clause}, since this NP will have the same index as its daughter NP by general feature passing conventions (i.e. that a head and its mother share the same features). By taking this approach, Culy subsumes coindexation in EHRCs and IHRCs under the same rule, which he formalizes as the Relative Coindexing Constraint, shown in \REF{boyle23}.

\ea \textbf{Relative coindexing constraint (RCC) (\citealt{Culy1990})}  \label{boyle23}

It must be the case that m = p.

\Tree [ .NP\textsubscript{m} [ .X ] [ .S$'$ [ .COMP [ .wh\textsubscript{p} ] [ .Y ] ] [ .S ] ] [ .Z ] ]
\z
 
This allows a generalization about the coindexing that occurs in both EHRCs and IHRCs, thus providing a unifying account of both internally headed and externally headed relative \isi{clauses}.

\section{The \isi{syntax} of the \ili{Hidatsa} IHRC}\label{sec:boyle:5}

After having reviewed the above approaches to IHRCs, a theoretical analysis of the \ili{Hidatsa} \isi{relative clause} is now possible. As shown above in \sectref{sec:boyle:4.1}, \ili{Hidatsa} obeys the \isi{indefiniteness restriction} first proposed by Williamson. In this respect, it is typical of IHRCs in other languages. However, \ili{Hidatsa} is unusual in that its RCs are prefixed with a relative marker that distinguishes \isi{specificity} of the nominalized clause. In \sectref{sec:boyle:5.1}, I will show that these prefixes are not relative pronouns. In \sectref{sec:boyle:5.2}, I will argue that they are complementizers and provide a syntactic analysis of IHRCs in a Minimalist framework.

\subsection{The syntactic status of the \ili{Hidatsa} relative markers}\label{sec:boyle:5.1}

The \ili{Hidatsa} RC is a nominalized sentence with the structure shown above in \REF{boyle6} and repeated here as \REF{boyle24}.

\ea  \label{boyle24}
{[(DP) (DP) \textsc{rel}-verb]-\textsc{det}}
\z

In addition to overt DPs, the argument positions can also be filled with a \textit{pro}. Like full DPs, \textit{pro} triggers number and person agreement on the verb. Given this fact, the \textit{aku-} and \textit{aru-} relative markers cannot be relative pronouns in the common sense. That is to say, they do not serve the same function as relative pronouns in languages like \ili{English}. They do not serve as arguments. Consider again example \REF{boyle9} repeated here as \REF{boyle25}.

\ea \label{boyle25}
 \glll Mary uuw\'aki akuh\'iri\v{s} war\'ucic\\
Mary uuw\'aki aku-h\'iri-\v{s}         wa-r\'uci-c\\
Mary quilt   \textsc{rel.s}-make-\textsc{det.d} \textsc{1a}-buy-\textsc{decl}\\
\trans `I bought the quilt that Mary made.' 
\z

In this example, we see that the matrix verb \textit{r\'uci-} `buy' projects two arguments and two \textipa{T} roles. The argument positions are filled by a \textit{pro} \isi{subject}, which is first person (shown by the agreement marker \textit{ma-}), and the \isi{relative clause} itself. Likewise, the verb in the \isi{relative clause}, \textit{h\'iri-} `buy', also projects two arguments and \textipa{T} roles. These are filled by the \isi{subject} of the clause, \textit{Mary}, and by the \isi{object}, \textit{uuw\'aki} `quilt'. If \textit{aku-}, and by extension \textit{aru-}, were relative pronouns, they would fill an argument slot, but this is not possible since both verbs project only two arguments each, and these are filled with either full DPs or \textit{pro}. It is thus not possible for either of the \ili{Hidatsa} relative markers to be a relative pronoun. If this was their role, it would be a violation of the theta criterion \citep[36]{Chomsky1981}. As they cannot be relative pronouns, I propose that they are complementizers. They signal that the RC is a \isi{complement} clause of a superordinate verb and in that role they function like the \ili{English} complementizer \textit{that}. 

\subsection{The \ili{Hidatsa} relative markers as complementizers}\label{sec:boyle:5.2}

\citet{Culy1990} claims that 1) no languages have overt complementizers in IHRC constructions; and 2) as a result of this lack of overt complementizers, there is no framework independent evidence as to whether the IHRC is an S (TP) or an S$'$ (CP).  

	All Siouan languages have an extensive set of morphemes that serve as clause final markers. In \ili{Hidatsa}, a \isi{predicate} must have one of these markers to be grammatical. In main \isi{clauses}, these morphemes show illocutionary force: declarative \textit{-c}, emphatic \textit{-ski}, speculative \textit{-t\'ook}, past definite \textit{-\v{s}t}, and permission \textit{-ahka} among others (\citealt{Matthews1965}; \citealt{Boyle2007}). Following \citet{Rizzi1997}, I assume that complementizers are the syntactic elements that express this type of illocutionary information. As illocutionary force markers, these morphemes indicate the clause type (\citealt{Cheng1997}) or Force (\citealt{Chomsky1995}) of the sentence and must be projected in C. In subordinate \isi{clauses}, these morphemes can include complementizers that include conditional and temporal subordinators. As in matrix \isi{clauses}, these morphemes are also projected in C.\footnote{\ili{Hidatsa} also has co-subordinate \isi{clauses} (\citealt{VanValinLaPolla1997}). These \isi{clauses} are connected with switch-reference (SR) markers. \citet{Boyle2007} argues that these are coordinate structures with the SR markers coordinating vPs (same-\isi{subject} markers) or TPs (different \isi{subject} markers). These co-subordinate \isi{clauses} receive their illocutionary force from the matrix verb. See \citet{GordonTorres2012} for a similar analysis of SR marking in \ili{Mandan}.}
	
Only relative \isi{clauses} do not have a clause final illocutionary force marker.  While there is a determiner that nominalizes the entire clause, which appears to be clause final, I propose that this is not the case. That determiner is the head of the DP that takes the RC as its \isi{complement}. The determiner is thus outside of the RC. It is not a complementizer. To unify the syntactic structure of \ili{Hidatsa}, I believe we can analyze the RC markers \textit{aku-} and \textit{aru-} as complementizers that mark the clause type as relative (REL). This feature satisfies the underspecified clause type feature in T. If this analysis is correct, we now have evidence that IHRCs are CPs, not TPs. The mechanics of how these complementizers function will be detailed below.   
	
While \citeauthor{Culy1990} needs the structure presented in \REF{boyle18} above for his analysis, it is highly unusual as it is exocentric --- that is to say, the head does not project a phrasal level. In \REF{boyle18}, the S projects an N$'$ and then NP. Culy's analysis could have been simplified but he employs older syntactic notations that do not allow him to capture greater generalities. The analysis presented here follows from the structure proposed by \citeauthor{Williamson1987} with only slight alterations. An IHRC in \ili{Hidatsa} has the base-generated structure shown in \REF{boyle26}. 

\ea \textbf{Proposed structure for IHRCs in Hidatsa} \label{boyle26}

\Tree [ .DP [ .CP [ .TP [ .vP [ .{SubDP} ] [ .v$'$ [ . \isi{VP} [ .{ObjDP}{\hspace{1em}} ] [ .V ] ] [ .v ] ] ] [ .T ] ] [ .{C [aku-/aru-]} ] ] [ .DET ] ]
\z           

This structure allows for a straightforward syntactic analysis without postulating an exocentric construction.\footnote{In \ili{Hidatsa} DPs select NPs or nominalized verbs, which include relative \isi{clauses}, as complements.} It captures the insights of both Williamson and Culy. In addition, it shows the placement of the \ili{Hidatsa} complementizers, \textit{aku-} and \textit{aru-}, as projections of C. These relative complementizers surface as prefixes to the verb. The motivation for this ordering is discussed below. The overt syntactic data provides framework-independent evidence for the CP status of IHRC. If this analysis is correct, it is not consistent with Culy's claims presented at the beginning of this section.

\subsection{\ili{Hidatsa} \textit{aku-} and \textit{aru-} as strong features in C}\label{sec:boyle:5.3}

Above I have argued that \ili{Hidatsa} has overt complementizers in relative \isi{clauses}, namely the \textit{aku-} and \textit{aru-} morphemes. In \ili{Hidatsa} IHRCs, these morphemes carry strong features, and as such they trigger movement of the verbal complex to T and then C. It is important to note that this type of cyclic roll-up is head-to-head movement, rather than phrasal movement. Consider the structure (shown in example \REF{boyle27} of the \ili{Hidatsa} IHRC, here simplified for only one DP in the \isi{relative clause}.

\ea \textbf{Proposed structure for IHRCs in Hidatsa} \label{boyle27}

\Tree [ .DP\textsubscript{i} [ .CP [ .TP [ .vP [ .DP\textsubscript{i} ] [ .v$'$ [ .\isi{VP} ] [ .v ] ] ] [ .T ] ] [ .{C [*\textsc{rel}]} ] ] [ .DET ] ]			       		        
\z

\citeauthor{Culy1990} (like \citeauthor{Williamson1987}) argues that there is coindexation between the DP dominating the IHRC and the internal DP being modified (the head). In \ili{Hidatsa}, the head of the RC is coindexed with the DP that dominates the entire IHRC. 
	
Evidence of coindexation between the head of the IHRC and the DP dominating the IHRC can be seen in number agreement.\footnote{In some languages, person agreement also provides evidence for this coindexation. While Williamson\ia{WIlliamson, Janis Shirley} shows this evidence in \ili{Lakota}, it is not seen in \ili{Hidatsa} since the type of person agreement morphology that exists in \ili{Lakota} does not exist in \ili{Hidatsa}.}  An example of this agreement is shown in \REF{boyle28} and \REF{boyle29}.

\ea \textbf{Plural agreement with overt subject} \label{boyle28}

\glll w\'iaku\textipa{P}o uuw\'aki akuh\'ira\textipa{P}a\v{s} \'iiwia\textipa{P}ac\\
w\'ia-aku-\textipa{P}o  uuw\'aki aku-h\'iri-\textipa{P}a-\v{s}    \'iiwia-\textipa{P}a-c\\
woman-\textsc{det.s-pl.i}  quilt  \textsc{rel.s}-make-\textsc{pl.d-det.d} cry-\textsc{pl.d-decl}\\
\trans `[Those women who made the quilt] cried.'
\z

\ea \textbf{Plural agreement with null subject} \label{boyle29}

\glll uuw\'aki akuh\'ira\textipa{P}a\v{s} \'iiwia\textipa{P}ac\\
uuw\'aki aku-h\'iri-\textipa{P}a-\v{s}   \'iiwia-\textipa{P}a-c\\
quilt     \textsc{rel.s}-make-\textsc{pl.d}-\textsc{det.d} cry-\textsc{pl.d-decl}\\
\trans `(The ones who) made the quilt cried.'	
\z

In these examples, we see that the matrix verbs agree with their subjects, which are IHRC. The head in each of these \isi{clauses}, whether overt (as in \ref{boyle28}) or null (as in \ref{boyle29}), is plural and this plural number is marked on both the matrix and subordinate verbs. Thus, both the head and the IHRC must have the same number feature. Coindexation is the usual way for two DPs to have the same number marking. The overt interaction of these elements offers convincing evidence that there is indeed a coindexing relationship between the head and the IHRC.  
	
\subsection{Move and Merge in the \ili{Hidatsa} IHRC}\label{sec:boyle:5.4}

Consider again example \REF{boyle11} repeated here as \REF{boyle30}.

\ea \textbf{\ili{Hidatsa} internally headed relative clause} \label{boyle30}

\glll {\ob}wac\'ee\v{s} wa\v{s}\'ukawa akut\'ihee\v{s}{\cb} \v{s}ip\'i\v{s}ac\\
[wac\'ee-\v{s} wa\v{s}\'uka-wa aku-t\'i-hee-\v{s}] \v{s}ip\'i\v{s}a-c\\
[man-\textsc{det.d} dog-\textsc{det.i}  \textsc{rel.s}-die-\textsc{3.caus.d.sg-det.d}] black-\textsc{decl}\\
\trans `The dog [that the man killed] is black.' 
\z

The direct \isi{object} is built by the indefinite determiner selecting for a NP. These two elements merge to form a DP as shown in \REF{boyle31}.

\ea	 \label{boyle31}  
\hspace{1em}\newline
\begin{tikzpicture}
\Tree [ .DP [ .{wa\v{s}\'uka [\textsc{n, 3, sg}]}\\{\hspace{3.2em}[\textit{u} \textsc{case:}]} ] [ .\hspace{2em}{-wa [\textsc{d, indef}, \sout{\textit{u} \textsc{n}]}} ] ]
\end{tikzpicture}
\z

The verb \textit{ti-} `die' then selects for an argument, satisfying the [\textit{u} \textsc{d}] feature of the verb as shown in \REF{boyle32}.

\ea	\label{boyle32}
\Tree [ .\isi{VP} [ .DP \edge[roof]; {wa\v{s}\'ukawa [\textit{u} \textsc{case:}]} ]  [ .\hspace{2em}{ti [\textsc{v}, \sout{\textit{u} \textsc{d}}, \textit{u} \textsc{infl:}]}... ] ]		               	
 \z

In \ili{Hidatsa}, the direct causative is projected as the head of v. Direct causatives take stative verbs as their complements and add an agentive argument to the subcategorization of the verb. The lexical verb then moves up to adjoin to the v forming the derived verb \textit{ti+\textsc{caus.d-}} and projecting a v$'$. The verb \textit{ti+\textsc{caus.d-}} in \ili{Hidatsa} is literally `cause to die'. Accusative [\textsc{acc}] case is then licensed and checked on the DP \textit{wa\v{s}ukawa}, making it a direct \isi{object} as shown in \REF{boyle33}.

\ea	    \label{boyle33} 
\Tree [ .{v$'$ [\textit{u} \textsc{d}]} [ .\isi{VP} [ .DP \edge[roof]; {wa\v{s}\'ukawa [\sout{\textit{u}\textsc{case: acc}}]} ]  [ .<ti> ] ] [ .{v [\textit{u} \textsc{d}]} [ .t\'i ] [ .\hspace{3em}{-\textsc{cause.d} [\textsc{v}, \textit{u} \textsc{infl:}]} ] ] ]     
\z

The [\textit{u} \textsc{d}] feature in v is passed up to v$'$ and projects a SPEC position where the agent is merged, thus checking the [\textit{u} \textsc{d}]. This is shown in \REF{boyle34}.

\ea	 \label{boyle34}
\Tree [ .vP [ .DP \edge[roof]; {wac\'ee\v{s} \\ {[\textsc{n, 3, sg} \textit{u} \textsc{case}]}} ] [ .{v$'$ [\sout{\textit{u} \textsc{d}}]} [ .\isi{VP} [ .DP \edge[roof]; {wa\v{s}\'ukawa [\textsc{acc}]} ] [ .<ti> ] ] [ .V [ .t\'i ] [ .{\hspace{2em}-\textsc{caus.d} \\ {\hspace{2em}[\textsc{v}, \textit{u} \textsc{infl}:]}} ] ] ] ] ]		
\z

The T head then merges with vP. T has the features [\textsc{*pres, nom, *epp} ([\textit{u} \textsc{d}]), \textit{u} \textipa{F}: , \textit{u} \textsc{clause}: ]. This gives us the structure shown in \REF{boyle35}.

\ea \label{boyle35}
{\hspace{1em}}\newline

\begin{tikzpicture}[scale=0.8]
\Tree [ .TP [ .DP \edge[roof]; {wac\'ee\v{s} \\{[\sout{\textit{u} CASE}:NOM]}} ] [ .{T$'$ \sout{[\textit{u} D}}] [ .vP [ .DP \edge[roof]; {{wac\'ee\v{s}}\\{[N, 3, SG,}\\ {\textit{u} CASE:NOM]}} ]  [ .{v$'$ [\sout{\textit{u} D}]} [ .\isi{VP} [ .DP \edge[roof]; {{wa\v{s}\'ukawa} \\ {[ACC]}} ] [ .<ti> ] ] [ .v \edge[roof]; {t\'i-hee \\ {[v, \sout{\textit{u} INFL}:} \\ {*PRES, 3, SG]} } ] ] ] [ .{\hspace{3em}T \\ {[*PRES, NOM, \sout{\textit{u}\textipa{F}}:3,} \\ {SG, \textit{u} CLAUSE:]}} ] ] ]	
\end{tikzpicture}		 		 
\z

In this structure, nominative case [\textsc{nom}] is passed down from T to value the [\textit{u} \textsc{case}:] feature of the DP in [SPEC, \isi{VP}]. This agentive DP, \textit{wacee\v{s}} `the man', then passes its phi-features up to value the [\textit{u} \textipa{F}:] in T as [3, SG].  The features [*\textsc{pres, 3, sg}] are then passed down to value the [\textit{u} \textsc{infl}:] feature on the verb. Once the \textsc{caus.d} is valued for person and number, it is realized as \textit{hee-} at Spellout. The strong [*\textsc{pres}] tense feature then attracts the verb to T where the tense features are checked (shown below in \ref{boyle36}). The EPP feature attracts the agentive DP to [SPEC, TP] where the [\textsc{nom}] case feature is checked and the [EPP (\textit{u} \textsc{d}]) feature is satisfied. These are all straightforward Move and Merge operations.
	
The TP then merges with a C. Normally, if there is no [\textit{wh} feature] in C, there is no [SPEC, CP] position projected. Complementizers for IHRCs are different.  I propose that the \ili{Hidatsa} relative complementizers have the features [\textsc{c, *rel,} \textit{u} \textsc{d:indef}].  Additionally, the \textit{aku-} complementizer has a value of [\textsc{spec}] (specific) and the \textit{aru-} complementizer has a value of [\textsc{non.spec}] (nonspecific). This gives us the structure shown in \REF{boyle36}.

\ea \label{boyle36}
{\hspace{1em}}\newline

\begin{tikzpicture}[scale=0.9]	
\Tree [ .{C$'$ [\textit{u} \textsc{d:indef}]}  [ .TP [ .DP \edge[roof]; {wac\'ee\v{s}\\ {[*\textsc{nom}]}} ] [ .T$'$ [ .vP \edge[roof]; {<wac\'ee\v{s}>  wa\v{s}\'ukawa <t\'ihee>} ] [ .{{\hspace{3em}tihee-} \\ {\hspace{1em}[\textit{u} \textsc{clause:*rel}]}} ] ] ] [ .{{\hspace{2em}aku-} \\ {[\textsc{c, spec, *rel,}} \\ { \textit{u} \textsc{d:indef}]}} ] ]       
\end{tikzpicture}         
\z

As the [*\textsc{rel}] illocutionary feature in C is strong it triggers Move of the verbal head in T, so that this feature can be checked. The verb moves to C where the \textit{aku-} is prefixed to it. This is shown in \REF{boyle37}.

\ea	\label{boyle37}
{\hspace{1em}}\newline

\begin{tikzpicture}[scale=0.9] 	 		
\Tree [ .{C$'$ [\textit{u} \textsc{d:indef}]} [ .TP [ .DP \edge[roof]; {wac\'ee\v{s} \\{[\textsc{nom}]}} ] [ .{T$'$ [\sout{\textit{u} \textsc{d}}]} [ .vP \edge[roof]; {<wac\'ee\v{s}>  wa\v{s}\'ukawa <t\'ihee->} ]  [ .{\hspace{2em}<tihee-> \\ {[\textit{u} \textsc{clause:*rel}]}} ] ] ] [ .{{\hspace{2em}akutihee} \\ { [\textsc{c, spec, *rel,}} \\ {\textit{u} \textsc{d:indef}]}} ] ]	
\end{tikzpicture}                        
\z

Normally, illocutionary features in C are weak and don't trigger movement. Because of this the illocutionary force markers concatenate onto the verbal complex as a suffix due to roll-up movement.  

The [\textit{u} \textsc{d:indef}] feature in C moves up to C$'$ and creates a [SPEC, CP] position. This is a weak feature, so movement of the indefinite DP is triggered after Spellout at LF. The semantic motivations for this will be discussed below in \sectref{sec:boyle:6}. The determiner then nominalizes the clause, selecting for a C rather than N, giving us the complete \isi{relative clause} structure. This is shown in \REF{boyle38}.  

\ea\label{boyle38}
{\hspace{1em}}\newline

\begin{tikzpicture}[scale=0.7] 	    
\Tree  [ .DP [ .CP [ . wa\v{s}uka-wa ] [ .{C$'$ [\sout{\textit{u} \textsc{d:indef}}]} [ .TP [ .DP {wac\'ee\v{s} \\ {[*\textsc{nom}]}} ] [ .T$'$ [ .vP \edge[roof]; {<wac\'ee\v{s}> <wa\v{s}\'ukawa> <t\'ihee>} ]  [ .{<t\'ihee-> \\{[\textit{u} \textsc{clause:*rel}]}} ] ] ] [ .{akutihee \\ {[\textsc{c, spec, *rel,}} \\ {\textit{u} \textsc{d:indef}]}} ] ] ] [ .{{-\v{s} [\textsc{d},} \\ {\textsc{def,} \sout{\textit{u} \textsc{c}}]}} ] ]                                                             
\end{tikzpicture}
\z          
      
\section{Semantic constraints and motivations}\label{sec:boyle:6}

Prior to \citet{Williamson1987} there was very little in the way of semantic explanation for IHRCs. Williamson and most linguists after her built on Heim's \citeyearpar{Heim1982} ideas about definites and indefinites and applied them to IHRCs. According to Heim, indefinites act not as quantifiers, but as variables.  In this section, I will review Heim's basic assumptions. I will then review Williamson's account of IHRC \citeyearpar{Williamson1987} as well as Basilico's \citeyear{Basilico1996} account, which employs not only Heim's ideas but also Diesing's mapping theory \citeyearpar{Diesing1990, Diesing1992a, Diesing1992b} to account for how IHRCs function at LF. Adopting Heim's framework, and building on Williamson and Basilico, I develop an account of IHRCs that simplifies previous work and provides an explanation as to the nature of IHRCs and their heads with regard to definiteness and movement at LF. I then show how this account unifies our understanding of IHRCs and EHRCs.

\subsection{Heim's account of indefinite determiners}\label{sec:boyle:6.1}

Heim's \citeyear{Heim1982} dissertation has proven to be very important in the theoretical explanation for the semantics of IHRCs. This work explores how the logical form of a sentence is constructed. Although Heim accepts the commonly held view that noun phrases headed by a common noun are generalized quantifiers, her major contribution to semantic analysis (particularly for IHRCs) is that indefinites are variables, not quantifiers.  So a sentence like \REF{boyle39a} will have the semantic representation seen in \REF{boyle39b}:

\ea \textbf{DPs as Generalized Quantifiers}
\begin{xlist}
\ex Every dog is barking \label{boyle39a}
\ex every [(dog(x)) (is-barking(x))] \label{boyle39b}
\end{xlist}
\z

In \REF{boyle39b}, `every' is the quantifier, `(dog(x))' is the restriction of the quantifier, and `(is barking(x))' is the nuclear scope of the quantifier.  

According to \citet{Heim1982}, definites and indefinites can be distinguished by three properties. First, only indefinites can undergo Operator Indexing. This means that an indefinite can be indexed with another element in the sentence. Definites, along with proper names and pronouns, are not \isi{subject} to Operator Indexing. Second, only indefinites are constrained by the Novelty Condition. The Novelty Condition states that an indefinite NP must not have the same index as any NP to its left. Third, definites, but not indefinites, presuppose their descriptive content, if they have any.  That is to say, a definite presupposes the existence of an entity with the properties of its descriptive content, while an indefinite does not. The crucial property in understanding the distribution of determiners in IHRCs is that only indefinites undergo Operator Indexing. Lastly, as indefinites are variables and not quantifiers, they must be bound by an existential operator inserted in the sentence by an operation she calls Existential Closure.

\subsection{Williamson's treatment of IHRCs}\label{sec:boyle:6.2}

\citet{Williamson1987} claims that all languages that have IHRCs will have an \isi{indefiniteness restriction}. According to this restriction, only indefinite NPs may be heads in an IHRC. This is to say that the head of an IHRC cannot be marked with a definite determiner. Williamson claims that the \isi{indefiniteness restriction} cannot be attributed to some inherent (i.e. lexical) property requiring wide scope of the indefinite NP. In addition, one cannot attribute this restriction to the traditional distinction between quantifiers, on the one hand, and proper nouns and definite NPs, on the other. To understand the \isi{indefiniteness restriction}, Williamson claims that we must understand that both simple declaratives containing an indefinite and RCs indicate the intersection of two sets. The traditional view of this can be seen in \REF{boyle40} and \REF{boyle41}.

\ea I bought a dog	\label{boyle40}

$\exists$ x (Dog (x) \& Buy (I, x))
\ex dog that I bought \label{boyle41}

(Dog (x) \& Buy (I, x))
\z	
	
In \REF{boyle40} we see a proposition with a bound variable and in \REF{boyle41} we see a propositional function with a free variable. Williamson suggests that we reconsider the traditional view of indefinites as existential quantifiers. Following \citet{Heim1982}, Williamson proposes that indefinites are `quantifier-free'. That is to say that they are essentially free variables. This then gives the example in \REF{boyle40} the semantic interpretation of the example in \REF{boyle41}. The quantifier force of indefinites in simple declaratives is determined by the rule of Existential Closure. Thus, IHRCs have the interpretation of a propositional function. Williamson\ia{Williamson, Janice Shirley} claims that universal quantifiers are excluded as heads because semantically such a quantifier is interpreted as a restrictive term. A definite is familiar (known) and presupposes the content of its \isi{predicate}.  This property is at variance with the meaning of restrictive RCs, for if the head is already familiar to the hearer, further specification by the RC is, at best, unnecessary.  While I agree with Williamson's analysis of IHRCs, she does not provide a motivation for why it is true.

\subsection{Basilico's account of IHRCs and Diesing's mapping theory}\label{sec:boyle:6.3}

\citet{Basilico1996} notes that most theorists working in a transformational framework posit that the \isi{internal head} moves to an external position at some point in the derivation. Examples of this have been shown above with the work of \citet{Cole1982, Cole1987}, \citet{Williamson1987}, \citet{Culy1990}, \citet{Kayne1994} and \citet{Bianchi1999}, among others. With the exception of Kayne and Bianchi, most researchers working on IHRCs have posited that this movement takes place at LF (or its predecessor D-Structure). 

Prior to \citet{Basilico1996} there were two general approaches to head movement in IHRCs. In the first approach, advocated by \citet{Broadwell1985, Broadwell1987}, \citet{Cole1987}, \citet{LefebvreMuysken1988}, and \citet{ColeHermon1994}, the head moves to a position external to the CP of the \isi{relative clause}. The second approach, advocated by \citet{Williamson1987}, \citet{Brassetal1989}, and \citet{Bonneau1992}, postulates that the head moves to the [SPEC, CP] of the RC but not out of the clause itself. While all of these works have arguments supporting the nature of the movement, none of them provide a detailed explanation as to why the head needs to move.  

Basilico presents evidence that in some languages with IHRCs, movement of the head occurs in the overt \isi{syntax}. He argues that the head need not necessarily move to a position external to the clause and that while the head is not in its usual place it nevertheless remains within the RC in the overt \isi{syntax}. Drawing from the previous work of \citet{Williamson1987}, \citet{Jelinek1987}, and \citet{Culy1990}, Basilico adopts the notion that IHRCs are not cases of relativization semantically, but cases of quantification. Following \citet{Heim1982}, Basilico argues that IHRCs are associated with quantificational elements that bind variables within the subordinate clause itself. The sentential part of the IHRC is interpreted semantically as an open sentence. According to Basilico the \ili{Hidatsa} \isi{relative clause} in \REF{boyle42} would have the semantic interpretation shown in \REF{boyle43}.

\ea \label{boyle42}
\glll w\'aceewa     aku\textipa{P}aw\'akaa\v{s}\\
w\'acee-wa       aku-aw\'akaa-\v{s}\\
man-\textsc{det.i}  \textsc{rel.s-1a}.see-\textsc{det.d}\\
\trans 'The man that I saw'
\z

\ea \label{boyle43}
 $\iota$ x [man (x) \& I saw (x)]  
\z

In this example, the sentential part of the IHRC `man I saw' should be interpreted semantically as \textit{man (x) \& I saw (x)}, an open sentence with two unbound variables.  According to \citeauthor{Basilico1996}, the definite determiner -\v{s} functions as an (iota) operator that binds the variables within the \isi{relative clause}.\footnote{See \citet{Jelinek1987} for the proposal concerning the use of the iota operator with IHRCs.} Following \citet{Culy1990}, the sentential part of the IHRC functions as the restriction on the operator associated with the \isi{relative clause}.  

In this analysis, one of the variables associated with the sentence is provided by the head noun. The importance of this, namely the \isi{indefiniteness restriction} on the head NP, was first noted by \citet{Williamson1987}. She showed that the head NP in IHRCs is not allowed to be marked as definite. According to Basilico, this follows from Heim's \citeyearpar{Heim1982} analysis that indefinite NPs are not associated with quantificational force (as presented above) and Kratzer's \citeyearpar{Kratzer1989} Prohibition Against Vacuous Quantification. In a similar manner to \citet{Culy1990}, Basilico follows \citet{Heim1982} in treating indefinites as having no quantificational force. He argues that they provide only a variable, which must be bound by another operator in the representation. In IHRCs this operator is the determiner associated with the entire IHRC itself; it comes to bind the variable associated with the indefinite head. Basilico argues that if there was a definite marker on the head then the variable provided by the head would be unavailable for \isi{binding}. Since the operator associated with the IHRC would not bind a variable, this would be a violation of the prohibition against vacuous quantification (as shown in \ref{boyle44}).

\ea For every quantifier Q, there must be a variable x such that Q binds an occurrence of x in both its restrictive clause and its nuclear scope (\citealt{Kratzer1989}). \label{boyle44} 
\z

Since the sentential part of the IHRC forms the restriction on the operator, there would be no variable for the operator to bind if there were no indefinite within the subordinate sentence to provide this variable.

\citeauthor{Basilico1996} then goes on to apply Diesing's \citeyearpar{Diesing1990, Diesing1992a, Diesing1992b} Mapping Hypothesis to the head movement in IHRCs. Her mapping hypothesis \citeyearpar{Diesing1992a, Diesing1992b}, which holds at LF, proposes two notions:

\ea \label{boyle45}
\begin{xlist}
\ex Material from \isi{VP} (vP) maps into nuclear scope, which is the domain of \isi{existential closure}.
\ex Material from TP maps into a restrictive clause.
\end{xlist}
\z
 
A restrictive clause is that part of the representation which forms the restriction on some operator. That is, an indefinite that restricts some operator will be in a different syntactic position at LF than an indefinite that receives an existential interpretation by VP-level \isi{existential closure}. The former indefinite NPs (the heads of the IHRCs) must not be within the \isi{VP} at LF, while the latter must be in the \isi{VP} at LF (\citealt{Basilico1996}). Therefore, the indefinite head of an IHRC must move out of \isi{VP} simply because it is indefinite. RCs are quantificational and are selected by the determiner. The head must move out of its argument position in order for the quantificational operator that is associated with it (as shown by the final determiner of the RC) to bind the variable introduced by the head (which is indefinite or null).   
	
Basilico\ia{Basilico, David} argues that there is an operator associated with the IHRC, which must come to bind the variable associated with the indefinite head. In order for an indefinite to become bound by an operator and not undergo \isi{existential closure}, it must move out of the \isi{VP} by LF. Thus, the quantificational approach to IHRCs and the mapping hypothesis provide a motivation for head movement. The head must move in order to be bound by the operator associated with the IHRC. If there is no head movement, and the head remains in the \isi{VP}, then there will be no variable to bind, and as a result, this will violate the prohibition against vacuous quantification.  
	
Basilico, like others before him, claims that IHRCs are DPs. Like other DPs, IHRCs can appear as arguments. For Basilico the difference between IHRCs and other DPs lies in what the head D of the DP takes as its \isi{complement}. Noun phrase DPs must take NPs as the \isi{complement} to the head D; this NP functions as the restriction on the head of D. IHRC DPs on the other hand take sentences (TPs) as their complements and these sentences function as the restriction on the head D (\citealt{Basilico1996}). 
	
Unfortunately, Basilico is only examining languages (Diegue\~no, \ili{Mojave}, and \ili{Cocopa}) that have evidence of movement in the overt \isi{syntax} of their IHRCs. As a result of this, the structures he posits show movement taking place prior to Spellout. In these structures, either the head or the entire IHRC move, thus allowing the head to escape \isi{existential closure}.   
	
By only examining IHRCs that show some evidence of movement overtly in the \isi{syntax}, \citeauthor{Basilico1996} avoids the more general consideration of what happens in languages with IHRCs that show no evidence of movement. As a result, he need not posit any structure for the majority of languages with IHRCs where head movement is done covertly at LF.  

\subsection{The semantic motivation for movement at LF}\label{sec:boyle:6.4}

Consider again example \REF{boyle11} repeated here as \REF{boyle46}.

\ea \textbf{\ili{Hidatsa} internally headed relative clause} \label{boyle46}

\glll {\ob}wac\'ee\v{s} wa\v{s}\'ukawa akut\'ihee\v{s}{\cb} \v{s}ip\'i\v{s}ac\\
[wac\'ee-\v{s}   wa\v{s}\'uka-wa  aku-t\'i-hee-\v{s}]  \v{s}ip\'i\v{s}a-c\\
[man-\textsc{det.d} dog-\textsc{det.i}  \textsc{rel.s}-die-\textsc{3.caus.d.sg}-\textsc{det.d}] black -\textsc{decl}\\
\trans `[The dog that the man killed] is black.' 
\z

This is an unambiguous IHRC. In this sentence the head of the RC is wa\v{s}\'ukawa `a dog'. The structure of this sentence given above in example \REF{boyle38} is repeated here as \REF{boyle47}. 

\ea 	\label{boyle47}
{\hspace{1em}}\newline

\begin{tikzpicture}[scale=0.7] 	    
\Tree  [ .DP [ .CP [ . wa\v{s}uka-wa ] [ .{C$'$ [\sout{\textit{u} \textsc{d:indef}}]} [ .TP [ .DP {wac\'ee\v{s} \\ {[*\textsc{nom}]}} ] [ .T$'$ [ .vP \edge[roof]; {<wac\'ee\v{s}> <wa\v{s}\'ukawa> <t\'ihee>} ]  [ .{<t\'ihee-> \\{[\textit{u} \textsc{clause:*rel}]}} ] ] ] [ .{akutihee \\ {\textsc{c, spec, *rel},} \\ {\textit{u} \textsc{d:indef}]}} ] ] ] [ .{{-\v{s} [\textsc{d},} \\ {\textsc{def}, \sout{\textit{u} \textsc{c}}]}} ] ]                                                             
\end{tikzpicture}
\z                                                   

As \citet{Basilico1996} has shown, relative \isi{clauses} are quantificational. A Determiner selects a \isi{relative clause} as its \isi{complement}. In addition, all IHRCs are restrictive, which means it is part of the representation which forms the restriction on some operator. An indefinite that restricts some operator must be in a different syntactic position at LF than an indefinite that receives existential interpretation by VP-level \isi{existential closure}. Given this, the head of the IHRC, which must be indefinite, must not be within \isi{VP} at LF. The head must move out of its argument position in order for the quantificational operator that is associated with it (as shown by the final determiner on the RC) to bind the variable introduced by the head (which is indefinite or null). The indefinite determiner is not associated with any quantificational force: it is an identity function. The semantics of the \ili{Hidatsa} indefinite can be seen in \REF{boyle48}.

\ea \label{boyle48}
 [-wa] = $\lambda$ P<et>.P
\z
	
Given this, the head must move; if it does not, it will not escape the \isi{existential closure} of the \isi{VP}. The head must be indefinite (or generic with null morphology) if it is to be bound by this outside operator. Given that the head of the IHRC is in-situ at Spellout, any movement must take place at LF. If this movement does not take place, the derivation will crash.
	
Although \citet{Basilico1996} postulates structures with either IP or \isi{VP} adjunction, this cannot be correct for \ili{Hidatsa}, as IHRCs show no evidence of movement. However, I have postulated that \ili{Hidatsa} does show clear evidence for a CP structure as the overt relative markers act as complementizers. This overt evidence for a complementizer shows that \citet{Culy1990} was correct in postulating the complementizer position for IHRCs and while it is rarely filled overtly in many of the world's languages that have these structures, it is in \ili{Hidatsa}. In \ili{Hidatsa} the head of the IHRC moves to the [SPEC, CP] position at LF.

\subsection{A unified account of IHRCs and EHRCs}\label{sec:boyle:6.5}

Given the above analysis, we can see that IHRCs and EHRCs are remarkably similar. EHRCs serve as complements (restrictive RCs) or adjuncts (non-restrictive RCs) to NPs, which are arguments of a \isi{predicate}. IHRCs are DPs that serve as arguments of a \isi{predicate}. In EHRCs, the head that is modified by the clause is coindexed with an element inside the RC. This is either a relative pronoun or an operator. This relative pronoun or operator is coindexed with the head outside of the RC. In IHRCs, the head stays inside of the RC, but it is coindexed with the determiner that takes the IHRC as its \isi{complement}. This head must move to [SPEC, CP] at LF to escape \isi{existential closure}, just as the relative pronoun or operator moves to [SPEC, CP] in an EHRC.

\section{Conclusion}\label{sec:boyle:7}

In this paper I have shown that \ili{Hidatsa} has IHRCs. I have examined previous attempts at describing their syntactic structure and provided a new one based on data from \ili{Hidatsa}. In addition, I have provided theory external evidence for the possibility of complementizer in these \isi{clauses} (namely the \textit{aku-} and \textit{aru-} markers). Following \citet{Culy1990}, I have argued that this complementizer has a strong feature, and that in \ili{Hidatsa} this is different from all other complementizers triggering movement. This accounts for the morpheme order in \ili{Hidatsa} relative \isi{clauses}.  Following previous work on IHRCs (most notably \citealt{Williamson1987} and \citealt{Culy1990}) I have expanded and simplified how the semantics of IHRCs functions. I have provided motivation for the movement of the head of the IHRC at LF in addition to explaining why Williamson's \isi{indefiniteness restriction} holds true.

\section*{Acknowledgment}

I would like to thank the late Alex Gwin\ia{Gwin, Alex} for the many hours spent discussing \ili{Hidatsa}. He was a wise and generous man. I would also like to thank Jason Merchant\ia{Jason Merchant} and Catherine Rudin\ia{Rudin, Catherine} for their comments on an earlier draft of this paper as well as Brian Agbayani's\ia{Agbayani, Brian} comments and a later draft. Thank you also to two anonymous reviewers for their helpful comments. Lastly, I would like to thank Robert L. Rankin\ia{Rankin, Robert L.}, who was always so encouraging and helpful to students and junior scholars.

\section*{Abbreviations}

\textsc{1.a} = first person active, \textsc{3.caus.d.sg} = third person causative, \textsc{3.poss.a} = third person alienable possessive, \textsc{cont} = continuative, \textsc{decl} = declarative, \textsc{det.d} = definite determiner, \textsc{det.i} = indefinite determiner, \textsc{det.s} = specific determiner, EHRC = externally headed \isi{relative clause}; \textsc{emph} = emphatic, \textsc{epe} = epenthetic consonant, IHRC = internally headed \isi{relative clause}; \textsc{indef} = indefinite, \textsc{loc} = locative, \textsc{ne} = narrative ending, \textsc{neg} = negative, \textsc{pl.d} = plural definite, \textsc{pl.i} = plural indefinite, RC = \isi{relative clause}; \textsc{rel.n} = nonspecific relative, \textsc{rel.s} = specific relative, \textsc{temp} = temporal, \textsc{top} = \isi{topic}

\printbibliography[heading=subbibliography,notkeyword=this]

\end{document}