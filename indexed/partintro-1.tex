\addchap{Introduction to Part I}
\begin{refsection}

%content goes here
The relative degree of ``genetic'' relatedness of the major branches of the Siouan language family is quite well-established: the \ili{Catawban} languages split off first, then the \ili{Missouri Valley Siouan} languages, followed by the \ili{Southeastern Siouan} and \ili{Mississippi Valley Siouan} languages. Among the latter branch, the \ili{Dakotan} languages split off first, followed by the \ili{Dhegiha} and \ili{Jiwere}-\ili{Ho-Chunk} sub-branches. Rankin (\citeyear{Rankin1988}; \citeyear{Rankin1998macrosiouan}; \citealt{RankinEtAl1998}; etc.) contributed much to developing and supporting this understanding, alongside advances in rigorous application of the \isi{comparative method}. Open questions include the possibility of relationships with \ili{Yuchi}, \ili{Iroquoian} languages and \ili{Caddoan} languages, and areal connections. Chapters in Part 1 of this volume address some of these issues, as well as considering what we can learn from early attempts to write Siouan languages.

Ryan Kasak\ia{Kasak, Ryan M.} (``A distant genetic relationship between Siouan-\ili{Catawban} and \ili{Yuchi}'') argues that evidence exists to link \ili{Yuchi} to the Siouan family. Though scarce and not fully conclusive, the evidence includes phonological and morphological correspondences strong enough to make a reasonable case for a genetic relationship between \ili{Yuchi} and Siouan-\ili{Catawban}. 

David Kaufman\ia{Kaufman, David V.} (``Two Siouan languages walk into a Sprachbund'') details the effects on \ili{Ofo} and \ili{Biloxi} of their participation in the \isi{Lower Mississippi Valley} (LMV) \isi{language area}; these languages share many lexical, phonetic and grammatical traits with genetically unrelated languages across the southeastern present-day United States. 

Rory Larson\ia{Larson, Rory} (``Regular sound shifts in the history of Siouan'') summarizes the current state of knowledge of the sound changes and correspondences distinguishing each branch and sub-branch of the Siouan family. These phonetic correspondences were worked out as part of the \isi{Comparative Siouan Dictionary} project (recently made available on line as \citet{Rankinetal2015AccessSeptember}), of which Bob Rankin was a central member. This concise catalog of all the known sound cwill be invaluable to anyone working with Siouan etymologies or \isi{cognates} in the future.

Kathleen Danker\ia{Danker, Kathleen} (``Ba-be-bi-bo-ra: Refinement of the \ili{Ho-Chunk} \isi{syllabary} in the 19\textsuperscript{th} and 20\textsuperscript{th} centuries'') presents a glimpse into the process of formation of a Native writing system. This \isi{syllabary} was inspired by one used by neighboring \ili{Algonquian} peoples, then progressively changed to better represent the phonologically quite different \ili{Ho-Chunk} language before being supplanted by writing in \ili{English}, and by an alphabetic \ili{Ho-Chunk} \isi{orthography}.

Anthony Grant\ia{Grant, Anthony} (``A forgotten figure in Siouan and \ili{Caddoan} linguistics: Samuel Stehman Haldeman (1812--1880)'') is another study of writing systems, in this case an early attempt to write \ili{Kanza} and \ili{Osage}. Haldeman\ia{Haldeman, Samuel Stehman} developed a universal phonetic \isi{orthography}, one of several precursors to the modern IPA, which he tried out on a variety of languages including these two Siouan ones. While not entirely successful at representing all the sounds of \ili{Kanza} and \ili{Osage}, Haldeman's word lists do provide some insights into the pronunciation of these languages at a time earlier than other available information on them. 

 

\printbibliography[heading=subbibliography]
 

\end{refsection}
