\addchap{Introduction to Part III}
\begin{refsection}

Siouan languages have received relatively little attention in general linguistic theory, and so pose a challenge even to theories based on typological\is{typology} generalizations from broad selections of the world's languages. The head-final, partially polysynthetic nature of Siouan languages raises issues for some claims in syntactic\is{syntax} theory, and interesting issues arise in \isi{phonology} as well. This section of the volume comprises five chapters applying formal linguistic theory to problems in the phonological\is{phonology} or syntactic structure of a single Siouan language. (Cross-linguistic studies are in the next section, Part IV.)

David Rood\ia{Rood, David~S.} (``The \isi{phonology} of \ili{Lakota} voiced stops'') reexamines a longstanding problem, the phonological status of voiced stops in \ili{Lakota}. He proposes a new analysis drawing on autosegments\is{autosegmental phonology} and \isi{feature geometry} to account for the sonorant-like behavior of /b/ and /g/ in \isi{lenition}, \isi{nasalization}, and \isi{cluster} contexts, and concludes that \ili{Lakota} is not a voicing language but an \isi{aspiration language}. 

John Boyle\ia{Boyle, John~P.} (``The \isi{syntax} and semantics of internally headed relative clauses\is{clauses, internally headed relative} in \ili{Hidatsa}'') analyzes the internally-headed-relative-clause construction in \ili{Hidatsa}, from both formal syntactic and formal semantic perspectives. Working within the Minimalist framework, he demonstrates that \ili{Hidatsa} IHRC are nominalized \isi{clauses}; using Heim's\ia{Heim, Irene} framework, he then presents a formal semantic explanation for the well-known \isi{indefiniteness restriction} on the heads of IHRC. 

This section of the volume concludes with three interrelated chapters with overlapping authors, all dealing with \ili{Ho-Chunk} (Hoc\k{a}k)\footnote{As discussed in the volume preface, numerous spellings exist for the name of this language. \textit{Ho-Chunk} is the usage on the Tribe's web site, but the three authors in this section all opted to use the spelling \textit{Hoc\k{a}k}.} \isi{syntax}, especially the existence and structure of verb and \isi{adjective} phrases (\is{verb phrase}VP and AP\is{adjective}) in the language.

Meredith Johnson\ia{Johnson, Meredith} (``A description of verb-phrase ellipsis in Hoc\k{a}k\il{Ho-Chunk}'') demonstrates that \ili{Ho-Chunk} does have true verb-phrase ellipsis\isi{ellipsis, verb-phrase}, with cross-linguistically typical characteristics. This argues strongly for the existence of \is{verb phrase}VP in \ili{Ho-Chunk}.

Bryan Rosen\ia{Rosen, Bryan} (``On the structure and constituency of Hoc\k{a}k\il{Ho-Chunk} \isi{resultative}s'') continues the theme of arguing that \ili{Ho-Chunk} has a full range of syntactic categories, this time including both \is{verb phrase}VP and AP\is{adjective}. The claim that \ili{Ho-Chunk} has adjectives is controversial: nearly all work on \ili{Ho-Chunk} and other Siouan languages argues or assumes that these words are stative verbs.

Meredith Johnson\ia{Johnson, Meredith}, Bryan Rosen\ia{Rosen, Bryan} and Mateja Schuck\ia{Schuck, Mateja} (``Evidence for a \is{verb phrase}VP constituent in Hoc\k{a}k\il{Ho-Chunk}'') rounds out this part of the volume by cataloguing the arguments for a configurational\is{configurationality} analysis of \ili{Ho-Chunk} (and, by extension, other Siouan languages as well). Subjects\is{subject} and \isi{object}s are shown to behave differently with respect to a number of tests, including scope as well as the elliptical\is{ellipsis} and \isi{resultative} constructions discussed in the previous two chapters.

 
\end{refsection}

