% 13
\documentclass[output=paper]{LSP/langsci}
\author{Bryan Rosen}
\title{On the structure and constituency of {Hocąk} resultatives}

\abstract{Abstract: This paper explores the structure and constituency of \il{Ho-Chunk}Hocąk (Siouan) ``adjectival'' \isi{resultative}s. I argue that \il{Ho-Chunk}Hocąk \isi{resultative}s project a phrasal \isi{XP} as the complement\is{complement} of the verb in a Larsonian ``\isi{VP-shell}'' (\citealt{Larson1988}), while the object of the \isi{resultative} is in Spec,\is{specifier}VP. First, I show that the result is an \isi{XP} and is not a full clause\is{clauses} (i.e., a CP). Second, I provide evidence that the result is in a VP\is{verb phrase}-internal position. While the focus of this paper is the structure of \isi{resultative}s in \il{Ho-Chunk}Hocąk, \isi{resultative}s as a construction tend to highlight other important characteristics of a language's grammar. I argue that the result \isi{predicate} is an AP\is{adjective}. This puts \il{Ho-Chunk}Hocąk \isi{resultative}s in line with English adjectival \isi{resultative}s. The data from \isi{resultative}s thus suggest that \il{Ho-Chunk}Hocąk has the lexical category \emph{\isi{adjective}}, contra the previous descriptions of lexical categories in \il{Ho-Chunk}Hocąk (see \citealt{Lipkind1945}; \citealt{1943}; and \citealt{Helmbrecht2006b}). The goal of this paper is therefore to present new \il{Ho-Chunk}Hocąk data, provide a structural analysis of \isi{resultative}s, and then explore the adjectival nature of \isi{resultative} predicates in the language. 
% KEYWORDS: [Ho-Chunk, resultative construction, adjective, lexical categories]
}
\ChapterDOI{ 10.17169/langsci.b94.176}

\maketitle

\begin{document}

\section{Introduction}\label{sec:rosen:1}
 
This paper explores the structure and constituency of \il{Ho-Chunk}Hocąk ``adjectival'' \isi{resultative}s. In \il{Ho-Chunk}Hocąk \isi{resultative}s, the result \isi{predicate} appears to the left of the verb, as exemplified in \REF{ex:rosen:1} with \textit{paras} `flat' and \textit{šuuc} `red'.\footnote{Unless noted otherwise, the data comes from  elicitation with Cecil Garvin\ia{Garvin, Cecil}. My methodology follows the standard techniques of translation and acceptability judgment tasks (see \citealt{Matthewson2004} for more details).}

\let\eachwordtwo=\itshape
\ea\label{ex:rosen:1}
\ea
\glll Meredithga mąąsra paras gistakšąną. \\
 Meredith-ga mąąs-ra paras {$\varnothing$}-gistak-šąną\\
Meredith-\textsc{prop} metal-\textsc{def} flat  \textsc{3s/o}-hit-\textsc{decl}\\
% \glt `Meredith hit the metal flat.'

\ex 
\glll Cecilga wažątirera šuuc hogiha. \\
Cecil-ga  wažątire-ra šuuc {$\varnothing$}-hogiha \\
Cecil-\textsc{prop} car-\textsc{def} red \textsc{3s/o}-paint\\
% \glt `Cecil painted the car red.'
\z
\z

 
The analysis of examples like those in \REF{ex:rosen:1} is as follows: I propose that \il{Ho-Chunk}Hocąk \isi{resultative}s project a phrasal AP\is{adjective} as the \isi{complement} of the verb in a Larsonian ``\isi{VP-shell}'' (i.e., a recursive VP\is{verb phrase} structure; \citealt{Larson1988}). The \isi{object} of the \isi{resultative} is in the \isi{specifier} of VP\is{verb phrase}. Thus, the sentence in (1b) has the basic structure in \REF{ex:rosen:2}.\footnote{I assume the Principles and Parameters framework (see also the Minimalist Program and X-bar theory; \citealt{Chomsky1995}). A phrase in this framework consists of three basic layers. The head (X\textsuperscript{0}) specifies the syntactic type or lexical category of the phrase (e.g., V for verb, N for noun, and A for \isi{adjective}). Complements\is{complement} are arguments (e.g., objects) of X\textsuperscript{0} and are sisters to the X head. Specifiers (Spec for short) are often reserved for subjects of the the phrase. They are sisters to X$'$.}
 

\begin{exe}
\ex\label{ex:rosen:2}
{\hspace{1em}}\newline
\begin{tikzpicture}
\Tree [ .vP [ .NP \edge[roof]; {Cecilga} ] [ . v$'$ [ .VP\is{verb phrase} [ .NP \edge[roof]; {wažątirera\\`the car'} ] [ .V$'$ [ .AP\is{adjective} \edge[roof]; {šuuc\\`red'} ] [ .V  hogiha\\`paint' ] ] ] [ .v ] ] ]
\end{tikzpicture}
\end{exe}
 
While the focus of this paper is to propose a structure of \isi{resultative}s in \il{Ho-Chunk}Hocąk, \isi{resultative}s as a construction tend to highlight other important characteristics of a language's grammar. \il{Ho-Chunk}Hocąk \isi{resultative}s are no exception. I argue that the result \isi{predicate} is an AP\is{adjective}. This puts \il{Ho-Chunk}Hocąk \isi{resultative}s in line with \ili{English} adjectival \isi{resultative}s. The data from \isi{resultative}s
thus suggest that \il{Ho-Chunk}Hocąk has the lexical category \isi{adjective}, contra the previous descriptions of lexical categories in \il{Ho-Chunk}Hocąk (see \citealt{Lipkind1945}, \citealt{Susman1943}, and \citealt{Helmbrecht2006b}). The goal of this paper is therefore to present new \il{Ho-Chunk}Hocąk data, provide a structural analysis of \isi{resultative}s, and then explore the adjectival nature of \isi{resultative} \isi{predicate}s in the language. The rest of this paper is organized as follows: \sectref{sec:rosen:2} provides background on \il{Ho-Chunk}Hocąk \isi{syntax} and \isi{resultative}s in \il{Ho-Chunk}Hocąk. \sectref{sec:rosen:3} examines the constituency of \il{Ho-Chunk}Hocąk \isi{resultative}s. In \sectref{sec:rosen:4}, I give a syntactic representation of \isi{resultative}s in \il{Ho-Chunk}Hocąk. In \sectref{sec:rosen:5}, I argue that the result \isi{predicate} projects as an AP\is{adjective}. \sectref{sec:rosen:6} concludes the paper.

\section{Overview of \il{Ho-Chunk}Hocąk \isi{syntax}}\label{sec:rosen:2}

In this section, I first present background information on \isi{word order} in \il{Ho-Chunk}Hocąk, and then I discuss some preliminary characteristics of \il{Ho-Chunk}Hocąk \isi{resultative}s.
 

\subsection{Word order in \il{Ho-Chunk}Hocąk}

Unmarked \isi{word order} in \il{Ho-Chunk}Hocąk is SOV, as in \REF{ex:rosen:3}. Variation in \isi{word order} has \isi{discourse} effects: a rightward displaced noun phrase is interpreted as discourse-old in (4a), while a leftward moved noun phrase serves a different \isi{discourse} function (e.g., \isi{topic} or focus) in (4b). Note that the interpretation in (4b) with OSV \isi{word order} is possible because there is a pause (represented by the comma) that offsets the fronted \isi{object}.

\begin{exe}

\ex\label{ex:rosen:3} \glll Wijukra šųųkra haja \\
 wijuk-ra šųųk-ra {$\varnothing$}-haja\\
cat-\textsc{def} dog-\textsc{def} \textsc{3s/o}-see\\
\glt `The cat saw the dog.'

\end{exe}

\begin{exe}
\ex\label{ex:rosen:4}
\begin{xlist}

\ex \glll Wijukra  haja, šųųkra \\
 wijuk-ra {$\varnothing$}-haja  šųųk-ra\\
cat-\textsc{def}  \textsc{3s/o}-see dog-\textsc{def}\\
\glt `The cat saw something, the dog.' 

\ex \glll \v{S}ųųkra, wijukra haja  \\
 šųųk-ra wijuk-ra {$\varnothing$}-haja  \\
dog-\textsc{def} cat-\textsc{def}  \textsc{3s/o}-see \\
\glt `The dog, the cat saw (it).' 

\end{xlist}
\end{exe}

In double \isi{object} constructions, the canonical \isi{word order} is \isi{subject}--indirect object--direct object--verb. This is shown below in \REF{ex:rosen:5}.

\begin{exe}

\ex \label{ex:rosen:5}\glll Hinųknįkhižą hocįcįhižą wiiwagaxhižą hok'ų.\\
hinųknįk-hižą hocįcį-hižą wiiwagax-hižą {$\varnothing$}-hok'ų\\
girl-\textsc{indef} boy-\textsc{indef} pencil-\textsc{indef} \textsc{3s/o}-give\\
\glt `A girl gave a boy a pencil.'

\end{exe}

In \il{Ho-Chunk}Hocąk, \isi{word order} is crucial to disambiguate the \isi{subject} from the \isi{object}: the first argument is interpreted as the \isi{subject}. In \REF{ex:rosen:6}, the first interpretation of the sentence (although pragmatically unlikely) is the only one with neutral \isi{intonation}; however, the second interpretation is  possible if there is a pause after `car'.

\begin{exe}
\ex\label{ex:rosen:6}
 \glll Wažątirera hinųkra ruwį.\\
wažątire-ra hinųk-ra {$\varnothing$}-ruwį\\
car-\textsc{def} woman-\textsc{def} \textsc{3s/o}-buy\\
\glt `The car bought the lady.' \textsc{or} `The lady bought the car.'

\end{exe}
 
\citet{JohnsonRosen2014} argue that \il{Ho-Chunk}Hocąk is underlying head-final, by providing evidence from \is{scope, quantifier}quantifier scope and postverbal \isi{predicate}s. Thus, I represent \il{Ho-Chunk}Hocąk as head-final here.

\subsection{Resultatives in \il{Ho-Chunk}Hocąk: Some preliminaries} 

\is{resultative}Resultatives are complex predicates that put together a means \isi{predicate} (always a verb) and a result \isi{predicate}, where neither is licensed by a \isi{conjunction} or an adposition (\citealt{Williams2008}). In \REF{ex:rosen:7}, the result \textit{šuuc} `red' immediately precedes the means \textit{hogiha} `paint', and the direct \isi{object} \textit{wažątirera} `the car' surfaces to the left of the result. Since the result is typically analyzed as the \isi{complement} of the means (\citealt{Li1999}, \citealt{Williams2008}), the result-means order would be expected in a head-final language.

\begin{exe}

\ex \label{ex:rosen:7}\glll Cecilga wažątirera šuuc hogiha. \\
Cecil-ga  wažątire-ra šuuc {$\varnothing$}-hogiha \\
Cecil-\textsc{prop} car-\textsc{def} red \textsc{3s/o}-paint\\
\glt `Cecil painted the car red.'

\end{exe}

The \isi{word order} of \isi{resultative}s and sentences with object-internal attributive modifiers is similar. Compare the position of the result phrase in \REF{ex:rosen:7} with the position of the attributive modifier in \REF{ex:rosen:8}.

\begin{exe}
\ex\label{ex:rosen:8}
 \glll Cecilga wažątire šuucra hogiha. \\
Cecil-ga  wažątire šuuc-ra  {$\varnothing$}-hogiha\\
Cecil-\textsc{prop} car red-\textsc{def}  \textsc{3s/o}-paint\\
\glt `Cecil painted the red car.'

\end{exe}

In \REF{ex:rosen:8}, the modifier \textit{šuuc} `red' is located to the right of the noun it modifies, \textit{wažątire} `car'. This attributive modifier cannot be to the right of the definite article \textit{-ra}. This entails that \textit{šuuc} `red' in \REF{ex:rosen:8} in an NP-internal position. By comparison, the result in \REF{ex:rosen:7} (\textit{šuuc} `red') is to the right of the definite article \textit{-ra}, which indicates that the result is in an NP-external position.

Moreover, the result AP\is{adjective} can ``scramble,'' or move leftward, to a position before the \isi{object} or \isi{subject}, as illustrated in \REF{ex:rosen:9}. In contrast, attributive modifiers do not have this option, as in \REF{ex:rosen:10}. This contrast demonstrates that \isi{resultative} \isi{predicate}s are not treated as part of the NP-\isi{object}, and provides further evidence that they are not in an NP-internal position.

\begin{exe}
\ex\label{ex:rosen:9}
\begin{xlist}

\ex \glll Cecilga \textbf{šuuc}, wažątirera woogiha. \\
 Cecil-ga šuuc  wažątire-ra wa-{$\varnothing$}-hogiha\\
 Cecil-\textsc{prop} red car-\textsc{def} \textsc{3o.pl}-\textsc{3s}-paint \\
\glt `Cecil painted the cars red.'

\ex \glll \textbf{\v{S}uuc}, Cecilga wažątirera woogiha. \\
  šuuc Cecil-ga wažątire-ra wa-{$\varnothing$}-hogiha\\
red Cecil-\textsc{prop} car-\textsc{def} \textsc{3o.pl}-\textsc{3s}-paint \\
\glt `Cecil painted the cars red.'

\end{xlist}


\ex[*]  {\label{ex:rosen:10}\glll Meredithga \textbf{šuuc}, wiišgacra ruwį. \\
Meredith-ga šuuc wiišgac-ra {$\varnothing$}-ruwį\\
Meredith-\textsc{prop} red toy-\textsc{def} \textsc{3s/o}-buy\\
\glt (Intended: `Meredith bought the red toy.')}


\end{exe}

It should be noted that \isi{resultative} constructions have been categorized cross-linguistically based on whether the lexical semantics of the verb and the result are independent of each other. In his \isi{typology} of \ili{Japanese} and \ili{English} \isi{resultative} \isi{predicate}s, \citet{Washio1997} presents two types of \isi{resultative}s: \textit{weak} and \textit{strong}. When the lexical semantics of the verb entails a change, it is called a weak \isi{resultative}. When the verb in \isi{resultative} constructions does not entail a change, Washio refers to this class as a strong \isi{resultative}. In other words, the classification between weak and strong \isi{resultative}s depends on whether the matrix verb denotes a result. Consider the two \ili{English} examples in \REF{ex:rosen:11}.

\begin{exe}
\ex\label{ex:rosen:11}
\begin{xlist}

\ex {Sam painted the wall red.}

\ex {Alex pounded the metal thin.}

\end{xlist}
\end{exe}
 
In (11a), the verb \textit{paint} entails that there is some change, since \textit{to paint} means to apply color. \textit{Paint} represents an example of a weak \isi{resultative}. In (11b), however, the verb \textit{pound} does not entail that the \isi{object} being pounded will become flat. That is, pounding metal could result in the metal being bumpy. Thus, there is no entailed change with \textit{pound}. The verb \textit{pound} is an example of a strong \isi{resultative}. In \il{Ho-Chunk}Hocąk, \isi{resultative}s are possible when the verb lexically specifies a change, as with \textit{hogiha} `paint' in \REF{ex:rosen:7} above and with \textit{gižap} `polish' in \REF{ex:rosen:12} below.
 

\begin{exe}
\ex\label{ex:rosen:12}
 \glll Meredithga mąąsra gišįnįšįnį gižapšąną.  \\
 Meredith-ga mąąs-ra gišįnįšįnį {$\varnothing$}-gižap-šąną\\
 Meredith-\textsc{prop} metal-\textsc{def} shiny \textsc{3s/o}-polish-\textsc{decl}\\
\glt `Meredith polished the metal shiny.'

\end{exe}

A verb like \textit{gižap} `polish' strongly denotes an activity whereby its \isi{object} (the\-me) changes its state to become `shiny.' Because \textit{gižap} implies this change of state, it is considered a weak \isi{resultative}. 

They can also be formed with verbs that do not specify a change, as with \textit{gistak} `hit' and \textit{rucgis} `cut' in \REF{ex:rosen:13}.

\begin{exe}
\ex\label{ex:rosen:13}
\begin{xlist}

\ex \glll Meredithga mąąsra paras gistakšąną. \\
 Meredith-ga mąąs-ra paras {$\varnothing$}-gistak-šąną\\
Meredith-\textsc{prop} metal-\textsc{def} flat \textsc{3s/o}-hit-\textsc{decl}\\
\glt `Meredith hit the metal flat.'

\ex \glll Matejaga peešjįra žiipįk rucgisšąną.\\
Mateja-ga peešjį-ra žiipįk {$\varnothing$}-rucgis-šąną\\
Mateja-\textsc{prop} hair-\textsc{def} short \textsc{3s/o}-cut-\textsc{decl}\\
\glt `Mateja cut the hair short.'

\end{xlist}
\end{exe}
 
Similar to \textit{pound} in \ili{English}, \textit{gistak} `hit' in \il{Ho-Chunk}Hocąk does not denote an event whereby its \isi{object} results in a particular state (e.g., flat). Thus we can consider this verb a strong \isi{resultative}. The verb \textit{rucgis} `cut' belongs to the class of strong \isi{resultative}s for the same reasons: the event denoted by \textit{rucgis} `cut' does not contain the notion of being short. Thus, \il{Ho-Chunk}Hocąk exhibits both strong and weak \isi{resultative}s. With this background in mind, I turn to the next section, where I discuss more about the constituency of \il{Ho-Chunk}Hocąk \isi{resultative}s.

\section{The constituency of \il{Ho-Chunk}Hocąk Resultatives}\label{sec:rosen:3}

This section outlines some diagnostics that support the structure presented in \REF{ex:rosen:2} for \il{Ho-Chunk}Hocąk \isi{resultative}s. In \sectref{sec:rosen:3.1}, I provide evidence that the result is a phrase and not a \is{clauses}clause, while in \sectref{sec:rosen:3.2} I show that the result is in a VP-internal position.
 

\subsection{The result \isi{predicate} as a phrase}\label{sec:rosen:3.1}

In this subsection, I show that the result is an \isi{XP} and is not a full clause\is{clauses} (i.e., a CP). First, it should be noted that the result is not a head that forms a \isi{compound} with the matrix verb; that is, the verb and the result in the construction should not be considered a single lexical unit, such as V\textsuperscript{0} or A\textsuperscript{0}. The result can include adverbial modifiers, such as \textit{hikųhe} `quickly' in (14a), and the intensifier suffix \textit{-xjį} in (14b).

\begin{exe}
\ex\label{ex:rosen:14}
\begin{xlist}

\ex \glll Meredithga mąąsra paras \textbf{hikųhe} gistakšąną. \\
 Meredith-ga mąąs-ra paras hikųhe {$\varnothing$}-gistak-šąną\\
Meredith-\textsc{prop} metal-\textsc{def} flat quickly \textsc{3s/o}-hit-\textsc{decl}\\ 
\glt `Meredith hit the metal flat quickly.'


\ex \glll Meredithga mąąsra parasxjį gistakšąną.\\
 Meredith-ga mąąs-ra paras-\textbf{xjį} {$\varnothing$}-gistak-šąną\\
Meredith-\textsc{prop} metal-\textsc{def} flat-very  \textsc{3s/o}-hit-\textsc{decl}\\
\glt `Meredith hit the metal very flat.'

\end{xlist}
\end{exe}

A piece of evidence that the result\is{resultative} \isi{predicate} is not a clause\is{clauses} comes from the fact the result phrase cannot take declarative (15a), or complementizer (15b) suffixes.

\begin{exe}
\ex\label{ex:rosen:15}
\begin{xlist}

\ex[*] {\glll Matejaga peešjįra žiipįkšąną rucgisšąną.\\
Mateja-ga peešjį-ra  žiipįk-šąną {$\varnothing$}-rucgis-šąną\\
Mateja-\textsc{prop} hair-\textsc{def}  short-\textsc{decl}  \textsc{3s/o}-cut-\textsc{decl}\\
\glt (Intended: `Mateja cut the hair short.')}

\ex[*] {\glll Matejaga peešjįra  žiipįkra rucgisšąną.\\ 
Mateja-ga peešjį-ra  žiipįk-ra  {$\varnothing$}-rucgis-šąną\\
Mateja-\textsc{prop} hair-\textsc{def}  short-\textsc{comp}   \textsc{3s/o}-cut-\textsc{decl}\\
\glt (Intended: `Mateja cut the hair short.')}

\end{xlist}
\end{exe}

The result also cannot take the future tense marker \textit{kjane}, as in \REF{ex:rosen:16}, even though the hair becoming short would necessarily take place after cutting it.

\begin{exe}

\ex[*] {\label{ex:rosen:16}\glll Matejaga peešjįra  žiipįk ikjane rucgisšąną.\\
Mateja-ga peešjį-ra  žiipįk kjane {$\varnothing$}-rucgis-šąną\\
Mateja-\textsc{prop} hair-\textsc{def}  short \textsc{fut} \textsc{3s/o}-cut-\textsc{decl}\\
\glt (Intended: `Mateja cut the hair short.')}

\end{exe}

In addition, the result cannot bear the \isi{negation} suffix \textit{-nį}. Negation in \il{Ho-Chunk}Hocąk is bipartite: the free particle \textit{hąąke} and the suffix \textit{-nį} are both required to form the negative. The example in (17a) shows that \textit{-nį} attaches to the matrix verb, while (17b) illustrates that the result cannot appear with \textit{-nį}.

\begin{exe}
\ex\label{ex:rosen:17}
\begin{xlist}

\ex[] {\glll Meredithga hąąke mąąsra paras gistaknį.\\
Meredith-ga hąąke mąąs-ra paras {$\varnothing$}-gistak-\textbf{nį}\\
Meredith-\textsc{prop} \textsc{neg} metal-\textsc{def} flat \textsc{3s/o}-hit-\textsc{neg}\\
\glt `Meredith did not hit the metal flat.'}

\ex[*] {\glll Meredithga hąąke mąąsra parasnį gistak.\\
Meredith-ga hąąke mąąs-ra paras-\textbf{nį} {$\varnothing$}-gistak\\
Meredith-\textsc{prop} \textsc{neg} metal-\textsc{def} flat-\textsc{neg} \textsc{3s/o}-hit\\
\glt (Intended: `Meredith did not hit the metal flat.' \textsc{or}\\
`Meredith hit the metal such that its surface didn't get fully flat.')}

\end{xlist}
\end{exe}

If the result could take one of these suffixes, this would mean that it would have the syntactic status of a clause\is{clauses}. Since the examples in \REF{ex:rosen:15}--\REF{ex:rosen:17} are ungrammatical, the result must not be a clause\is{clauses}.
 
Third, \il{Ho-Chunk}Hocąk \isi{resultative}s respect the \textit{Direct Object Restriction} (DOR): the result \isi{predicate} must be predicated on the NP in \isi{object} position (\citealt{LevinRappaportHovav1995}). That is, the result must be predicated of a transitive \isi{object} or the \isi{subject} of an unaccusative, but not the \isi{subject} of a transitive or an unergative verb.\footnote{Note that the DOR can also apply to so-called ``fake'' objects (e.g., reflexives) of unergative verbs. For example, the result phrase \textit{hoarse} can be predicated on \textit{herself} in (i). See \citet{Carrier1992}, \citet{Li1999}, and \citet{Wechsler2005} for more details on the DOR. (i) The woman sang herself hoarse.} This restriction is shown in \REF{ex:rosen:18} with the transitive verb \textit{gistak} `hit'.
 

\begin{exe}
\ex \label{ex:rosen:18}
\glll Rockyga wanįra šuuc gistakšąną.\\ 
Rocky-ga wanį-ra šuuc {$\varnothing$}-gistak-šąną\\
Rocky-\textsc{prop} meat-\textsc{def} red \textsc{3s/o}-hit-\textsc{decl}\\
\glt $=$ `Rocky hit the meat red.' \vspace{-3pt} \\ 
$\not=$ `Rocky hit the meat  and he was red as a result.'

\end{exe}

\largerpage[-1]
As seen in \REF{ex:rosen:18}, since \textit{wanįra} `the meat' is in \isi{object} position, it can be the \isi{subject} of the result, while the \isi{subject} of the matrix verb, \textit{Rocky}, cannot. The contrast in \REF{ex:rosen:18} points to the fact that the result is not a clause\is{clauses} (i.e., a CP). I follow \citet{Li1999} and assume that when the result can be linked to either the \isi{subject} or the \isi{object} and the result plus the means \isi{predicate} is not formed in the lexicon (i.e., they do not form a \isi{compound}), the \isi{resultative} phrase is a clause with a \textit{pro}-controlled \isi{subject} (see also \citealt{Song2005}). According to \citet{Chomsky1982}, \textit{pro} is an empty category of the type [+pronominal, --anaphoric], and Binding\is{binding} Theory states that it cannot be bound within its governing category. Thus, \textit{pro} could be bound by either the matrix \isi{subject} or \isi{object}. Since the result in \REF{ex:rosen:18} cannot be linked to the \isi{subject}, the result cannot be a clause\is{clauses}.
 
Moreover, \il{Ho-Chunk}Hocąk \isi{resultative}s show a contrast in availability between prototypical unaccusative and unergative verbs. \citet{Perlmutter1978} defines unaccusative verbs as ones where the single argument is an underlying \isi{object}, whereas the argument of an unergative verb is an underlying \isi{subject}. Typically, unaccusative verbs denote change (e.g., \textit{break, melt}) while unergative verbs indicate manner of motion (e.g., \textit{run}) or other bodily functions (e.g., \textit{cry}). In \il{Ho-Chunk}Hocąk, intransitive verbs that take stative agreement morphemes correspond to unaccusatives, and the set of intransitive verbs that bear active agreement morphemes are parallel to unergative verbs (see e.g., \citealt{Williamson1984}, \citealt{Woolford2010}). Prototypical unaccusatives in \REF{ex:rosen:19}, such as \textit{ziibre} `melt' and \textit{taaxu} `burn', can serve as the matrix verb of \isi{resultative}s. On the other hand, prototypical unergative verbs in \REF{ex:rosen:20}, such as \textit{nąąwą} `sing' and \textit{nąąk} `run', cannot.
 

\begin{exe}
\ex\label{ex:rosen:19}
\begin{xlist}

\ex \glll Xaigirara sgaasgap {ziibre}. \\
 xaigira-ra sgaasgap {$\varnothing$}-ziibre\\
chocolate-\textsc{def} sticky \textsc{3s}-melt\\
\glt `The chocolate melted sticky.'

\ex \glll Waisgapra seep {taaxu}.\\
 waisgap-ra seep {$\varnothing$}-taaxu\\
bread-\textsc{def} black \textsc{3s}-burn\\
\glt `The bread burned black.'

\end{xlist}
\end{exe}

\begin{exe}
\ex\label{ex:rosen:20}
\begin{xlist}

\ex[*] {\glll Hinųkra nįįra teek {nąąwą}. \\
hinųk-ra nįį-ra teek {$\varnothing$}-nąąwą\\
woman-\textsc{def} throat-\textsc{def} sore 3\textsc{s}-sing\\
\glt (Intended: `The woman sang her throat sore.')}

\ex[*] {\glll Henryga wagujirera paras {nąąkšąną}. \\
Henry-ga wagujire-ra paras {$\varnothing$}-nąąk-šąną\\
Henry-\textsc{prop} shoe-\textsc{def} flat \textsc{3s}-run-\textsc{decl}\\
\glt (Intended: `Henry ran the shoe(s) flat.')}

\end{xlist}
\end{exe}
 
\newpage  
Note that the restriction with unergative verbs also holds when the reflexive morpheme \textit{kii-} denotes the so-called `fake' reflexive/\isi{object} of the \isi{predicate}; see \REF{ex:rosen:21}.\footnote{Under \citegen{Washio1997} \isi{typology}, intransitive \isi{resultative}s are a type of weak \isi{resultative}. For example, \isi{resultative}s with an unergative verb like `run' can form a weak \isi{resultative}. Recall that \il{Ho-Chunk}Hocąk has transitive strong \isi{resultative}s (see \REF{ex:rosen:13} above). \il{Ho-Chunk}Hocąk \isi{resultative}s thus present a counterexample to Washio's \isi{typology}: \il{Ho-Chunk}Hocąk has transitive strong \isi{resultative}s but lacks intransitive strong \isi{resultative}s. I leave further discussion of these examples with respect to Washio's \isi{typology} for future work.}
 

\begin{exe}

\ex[*] {\label{ex:rosen:21}\glll Hunterga hoix'įk kiinąąkšąną.\\
Hunter-ga hoix'įk <\textbf{kii}>{$\varnothing$}-nąąk-šąną\\
Hunter-\textsc{prop} tired <\textsc{refl}>\textsc{3s}-run-\textsc{decl}\\
\glt (Intended: `Hunter ran himself tired.')}

\end{exe}

The DOR states the result must be \isi{predicate}d of the \isi{object}. If we assume that the subjects of the verbs in \REF{ex:rosen:20} are underlying objects, we can maintain the DOR. On the other hand, since unergative verbs do not have an underlying \isi{object}, no \isi{resultative} interpretation is possible in \REF{ex:rosen:20} and \REF{ex:rosen:21}.\footnote{The DOR holds consistently in \ili{English} for transitive objects. In the case of unergative verb phrases, a fake reflexive/\isi{object} ensures that there is an \isi{object} that the result can be linked to. (See the translations in \REF{ex:rosen:20} and \REF{ex:rosen:21}).}

\subsection{VP-internal status of the result \isi{predicate}}\label{sec:rosen:3.2}

In this subsection, I argue that the result \isi{predicate} is the \isi{complement} of the verb. I first show that the result \isi{predicate} must be VP-internal, and then I provide evidence that \isi{resultative}s in \il{Ho-Chunk}Hocąk project as a binary structure. \citet{LevinRappaportHovav1995} use VP-ellipsis\is{ellipsis, verb-phrase} in order to show that \isi{resultative}s are VP-internal, and that they are part of the eventuality of the VP\is{verb phrase}. \il{Ho-Chunk}Hocąk has a type of VP-ellipsis\is{ellipsis, verb-phrase} shown in \REF{ex:rosen:22} and \REF{ex:rosen:23}: the \isi{light verb} \textit{ųų} can replace either a minimal VP\is{verb phrase} or a multi-segmental VP\is{verb phrase}, resulting from adjunction to VP\is{verb phrase}. Example \REF{ex:rosen:22} shows an example of VP-ellipsis\is{ellipsis, verb-phrase} that targets on the \isi{object} and the verb, while in \REF{ex:rosen:23}, VP-ellipsis\is{ellipsis, verb-phrase} targets a VP-level \isi{adjunct}, such as \textit{xjanąre} `yesterday'.

\begin{exe}
\ex \label{ex:rosen:22}
\glll Cecilga {\ob}{\sVP} wažątirehižą ruwį{\cb} kjane anąga nee šge {\ob}haųų{\cb} kjane.\\
Cecil-ga {} wažątire-hižą {$\varnothing$}-ruwį kjane anąga nee šge {\db}ha-ųų kjane\\
Cecil-\textsc{prop} {} car-\textsc{indef} \textsc{3s/o}-buy \textsc{fut} and I also {\db}\textsc{1s}-do \textsc{fut}\\
\glt `Cecil will buy a car, and I will too.' \citep[5]{Johnson2013}

\ex \label{ex:rosen:23}
\glll Cecilga {\ob}{\sVP} xjanąre waši{\cb} anąga Bryanga šge {\ob}ųų{\cb}.\\
Cecil-ga {} xjanąre {$\varnothing$}-waši anąga Bryan-ga šge {\db}{$\varnothing$}-ųų\\
Cecil-\textsc{prop} {} yesterday \textsc{3s}-dance and Bryan-\textsc{prop} also {\db}\textsc{3s}-do\\
\glt `Cecil danced yesterday, and Bryan did too.' \citep[6a]{Johnson2013}

\end{exe}

As shown in (24b), it is not possible to ``strand'' the result \isi{predicate} \textit{šuuc} `red' under VP-ellipsis\is{ellipsis, verb-phrase}. It thus follows that the result is inside the VP\is{verb phrase}, rather than adjoined to VP\is{verb phrase}.


\ea\label{ex:rosen:24}
\ea[] {\glll Hunterga {\ob}{\sVP} nąąju seep hogiha{\cb} anąga Bryanga šge {\ob}ųų{\cb}.\\
Hunter-ga {} nąąju seep {$\varnothing$}-hogiha anąga Bryan-ga šge {\db}{$\varnothing$}-ųų\\
Hunter-\textsc{prop} {} hair black \textsc{3s/o}-dye and Bryan-\textsc{prop} too {\db}\textsc{3s}-do\\ 
\glt `Hunter dyed the hair black and Bryan did, too.'}

\ex[*] {\glll Hunterga nąąju seep hogiha anąga Bryanga šge \textbf{šuuc} ųų.\\
Hunter-ga nąąju seep {$\varnothing$}-hogiha anąga Bryan-ga šge šuuc {$\varnothing$}-ųų\\
Hunter-\textsc{prop} hair black \textsc{3s/o}-dye and Bryan-\textsc{prop} too red \textsc{3s}-do\\ 
\glt (Intended: `Hunter dyed the hair black and Bryan did red, too.')}

\z
\z

Example \REF{ex:rosen:24} contrasts with \REF{ex:rosen:25}.  \REF{ex:rosen:25} contains the adverb \emph{wasisik} `energetically' as a depictive. Since depictives are typically analyzed as adjuncts that occupy a VP-external position (\citealt{LevinRappaportHovav1995}), they can be stranded.

\ea\label{ex:rosen:25} 
\glll Bryanga {\ob}{\sVP} waarucra hoix'įk waža{\cb} anąga  Meredithga \textbf{wasisik} {\ob}ųų{\cb}. \\
 Bryan-ga {} waaruc-ra hoix'įk {$\varnothing$}-waža anąga Meredith-ga  wasisik {$\varnothing$}-ųų\\
Bryan-\textsc{prop} {} table-\textsc{def} tired \textsc{3s/o}-wipe and Meredith-\textsc{prop}  energetic {\db}\textsc{3s}-do\\
\glt `Bryan wiped the table tired(ly) and Meredith did energetically.'
\z

As we saw in \REF{ex:rosen:22}, \textit{ųų} affects the verb and its \isi{complement}. Since a result \isi{predicate} is not strandable with \textit{ųų}, it must be the case that the result is inside the minimal VP\is{verb phrase}, and thus is part of the core eventuality of the VP\is{verb phrase}. In other words, it follows that the result is inside the verb phrase.

 
Another option for the structure of \isi{resultative}s could be that the verb, the result, and direct \isi{object} are all sisters in a \isi{flat structure}.  \citet{Carrier1992} provide such a ternary analysis for \ili{English} \isi{resultative}s. However, \citet{Bowers1997} argues that a ternary structure cannot account for structures involving Across the Board movement. This type of movement describes a situation when a syntactic element moves from multiple base positions to a single terminal position. In this conjunctive\is{conjunction} test, the \isi{object} and result of both conjuncts form a single constituent (see also \citealt{Li1999}). An Across the Board structure is possible with \il{Ho-Chunk}Hocąk \isi{resultative}s, as seen in \REF{ex:rosen:26}, where the verb is moving across conjuncts\is{conjunction}.
 

\ea
\label{ex:rosen:26}
\ea \glll Meredithga mąąsra paras gistak anąga waisgap pereįk. \\
Meredith-ga mąąs-ra paras {$\varnothing$}-gistak anąga waisgap pereįk\\
Meredith-\textsc{prop} metal-\textsc{def} flat \textsc{3s/o}-hit and bread thin\\
\glt `Meredith hit the metal flat and the bread thin.'

\ex 
\glll Meredithga  mąąsra   gišįnį{s}įnį gižap                 anąga wažątirera sgee.\\
      Meredith-ga mąąs-ra  gišįnį{s}įnį {$\varnothing$}-gižap anąga wažątire-ra sgee\\
      Meredith-\textsc{prop} metal-\textsc{def} shiny  \textsc{3s/o}-polish and  car-\textsc{def} clean \\
\glt `Meredith polished the metal shiny and the car clean.'
\z
\z

The ability of \il{Ho-Chunk}Hocąk \isi{resultative}s to participate in Across the Board movement is consistent with an analysis that argues for a binary structure (\citealt{Bowers1997}). I conclude that \il{Ho-Chunk}Hocąk \isi{resultative}s are straightforwardly analyzable under a binary branching approach. This provides another argument that the result is in a VP-internal position.
 
\section{Syntactic representation of  Hocąk  resultatives}\label{sec:rosen:4}
 
In this section, I propose that \isi{resultative}s are in a Larsonian \isi{VP-shell} structure (\citealt{Larson1988}): a VP\is{verb phrase} structure takes another VP\is{verb phrase} as its \isi{complement}. This approach follows \citegen{Li1999} structure for \ili{English} \isi{resultative}s (cf. \citealt{Hoekstra1988,Carrier1992,LevinRappaportHovav1995}). \citegen{Larson1988} \isi{VP-shell}s are intended to accommodate the double-\isi{object} construction, where the left-most \isi{object} is in a higher position than the right-most. If we maintain a binary branching structure, then a \isi{resultative} has the same structure as the double-\isi{object} construction. I claim that the structure for \il{Ho-Chunk}Hocąk \isi{resultative}s is depicted in \REF{ex:rosen:27}. The result \isi{predicate} is the \isi{complement} of the verb, and I assume that the \isi{object} is base-generated in Spec,\is{specifier}VP\is{verb phrase}. The \isi{subject} is generated in Spec,\is{specifier}vP, where ``little v'' is a semi-\isi{functional head} that licenses external arguments (\citealt{Chomsky1995}).
 

\begin{exe}
\ex\label{ex:rosen:27}
\begin{xlist}

\ex \glll Cecilga wažątirera šuuc hogiha. \\
Cecil-ga  wažątire-ra šuuc {$\varnothing$}-hogiha \\
Cecil-\textsc{prop} car-\textsc{def} red \textsc{3s/o}-paint\\
\glt `Cecil painted the car red.'

\ex 
{\hspace{1em}}\newline
\begin{tikzpicture}
\Tree [ .vP [ .NP \edge[roof]; {Cecilga} ] [ .v$'$ [ .VP\is{verb phrase} [ .NP \edge[roof]; {wažątirera\\`the car'} ] [ .V$'$ [ .AP\is{adjective} \edge[roof]; {šuuc}\\`red' ] [ .V hogiha\\'`paint' ]] ][ .v ] ] ]
\end{tikzpicture}
\end{xlist}
\end{exe}
 
The structure in (\ref{ex:rosen:27}b) straightforwardly explains the facts with respect to \il{Ho-Chunk}Ho\-cąk \isi{resultative}s. First, the result is not a head that forms a \isi{compound} with the matrix verb since adverbs and intensifiers can intervene. The structure in (27b) shows that the result is an AP\is{adjective} and not a CP. This accounts for why the result cannot take complementizer, tense, \isi{negation}, or declarative suffixes: the result is an AP, which does not contain clause\is{clauses}-level heads or morphology. This property of result APs is also reflected in the fact that \il{Ho-Chunk}Hocąk \isi{resultative}s obey the DOR. In \REF{ex:rosen:18}, only the \isi{object} `meat' can be modified by the result `red'. This restriction predicts that the result is not a clause. If the result were a clause\is{clauses}, the \isi{subject} in \REF{ex:rosen:18} could also be modified by `red' because \isi{resultative} phrases that project as CPs have a \textit{pro} \isi{subject}, which could be linked to the matrix \isi{subject}. However, this is not the case. To formalize the relationship between the NP \isi{object} and the \isi{adjective}, I follow \citegen{Li1999} analysis. The AP can assign its theta-role to the \isi{object} through mutual m-command.\footnote{I assume that m-command refers to a syntactic\is{syntax} relation where X m-commands Y if and only if the first maximal projection that dominates X also dominates Y and X does not dominate Y. In (31b), X is the NP \textit{wažątirera} `the car', and Y is the AP\is{adjective} \textit{šuuc `red'.}} In the case of \REF{ex:rosen:31}, AP and the NP in Spec,\is{specifier}VP\is{verb phrase} are both dominated by the same VP\is{verb phrase} node, and they do not dominate each other. Thus, the AP and the \isi{object} NP mutually m-command each other. On the other hand, the AP does not hold a mutual m-command relationship with the \isi{subject} in Spec,\is{specifier}vP; thus, the AP\is{adjective} cannot assign its theta-role to the \isi{subject}. This results in the DOR effect.
 

This situation applies to \isi{resultative}s with unaccusative matrix verbs, as depicted in (\ref{ex:rosen:28}b).

\ea\label{ex:rosen:28}
\ea
\glll Waisgapra seep {taaxu}.\\
 waisgap-ra seep {$\varnothing$}-taaxu\\
bread-\textsc{def} black \textsc{3s}-burn\\
\glt `The bread burned black.'


\ex
\Tree [ .VP\is{verb phrase} [ .NP \edge[roof]; {waisgapra} ] [ .V$'$ [ .AP\is{adjective} \edge[roof]; {seep} ] [ .V taaxu ] ] ]

\z
\z

The AP\is{adjective} in (28b) has the same position that it has in (27b); that is, it is the \isi{complement} of the verb. Thus, the AP maintains the same relationship with the \isi{object} in Spec,\is{specifier}VP\is{verb phrase} whether the verb is transitive or intransitive. Consequently, the AP\is{adjective} \textit{seep} `black' and the \isi{object} \textit{waisgapra} `the bread' are within the same VP\is{verb phrase}, and the DOR effect is preserved. Data from VP-ellipsis\is{ellipsis, verb-phrase} has also demonstrated that the result phrase is inside the VP\is{verb phrase}. This is in contrast to depictive phrases, where the depictive can be stranded by VP-ellipsis\is{ellipsis, verb-phrase}. Assuming the structure presented above, this contrast falls out naturally. Depictives have been analyzed as VP-\isi{adjunct}s in \ili{English} (\citealt{LevinRappaportHovav1995}); thus, I suggest that a depictive phrase, such as \textit{wasisik} `energetically' in \REF{ex:rosen:25}, is adjoined to the upper \isi{VP-shell} (i.e., vP) in (27b).\footnote{In this paper, I leave it open whether depictives can adjoin to the lower \isi{VP-shell}.}

 
To summarize, I have argued that the \isi{resultative} secondary \isi{predicate} is the \isi{complement} to the main verb, and is a phrase. This accounts for a constellation of facts that concern the properties of \il{Ho-Chunk}Hocąk \isi{resultative}s, including the DOR.
 

\section{The result  {predicate} and  {adjective}s in  Hocąk} \label{sec:rosen:5}

 
Thus far I have assumed without comment that the result \isi{predicate} is an \isi{adjective} phrase. This section provides evidence that the result is in fact an AP\is{adjective}, and thus that \il{Ho-Chunk}Hocąk has \isi{adjective}s. Traditional grammars (e.g., \citealt{Lipkind1945} and \citealt{Susman1943}) and more recently \citet{Helmbrecht2006b} have claimed that \il{Ho-Chunk}Hocąk lacks the lexical class adjectives since there is no distinct inflectional\is{conjugation} morphology for adjectives and verbs. Instead these works claim that \isi{adjective}s are a class of stative verbs. For reasons of space, I consider only two of these arguments in detail.

First, \citet{Helmbrecht2006b} shows that there is no category establishing morphology with respect to \isi{adjective}s. Recall that \il{Ho-Chunk}Hocąk has an active-stative split between intransitive verbs. Helmbrecht notes that purported \isi{adjective}s and stative verbs exhibit parallel agreement morphology, as shown in \REF{ex:rosen:29} and \REF{ex:rosen:30}, respectively.
 
\let\eachwordtwo=\upshape
 
\ea\label{ex:rosen:29}
\ea \gll
hį-xete \hspace{36pt} b. {}  nį-xete \hspace{48pt} c. {} xete-ire\\
1-big {} {} {} 2-big {} {}  {} big-\textsc{3s.pl}\\
\glt `I am big.' \hspace{1.2cm} `You are big.' \hspace{1.1cm} `They are big.'

\z
\z

\ea\label{ex:rosen:30}
\ea
\gll
hį-šiibre \hspace{30pt}  b. {} nį-šiibre \hspace{43pt} c. {} šiibre-ire\\
1-fall {} {} {} 2-fall {} {} {} fall-\textsc{3s.pl}\\
\glt `I fell.' \hspace{1.8cm} `You fell.'  \hspace{1.8cm} `They fell.'
\z
\z

 
Example \REF{ex:rosen:29} illustrates that the stative set of agreement markers may be used with \isi{adjective}s: in (29a,b), the prefixes \textit{hiį-} and \textit{niį-} mark 1\textsuperscript{st} and 2\textsuperscript{nd} person respectively, and in (29c) \textit{-ire} encodes third person plural. The example in \REF{ex:rosen:30} with the stative verb \textit{šiibre} `fall' shows that this verb bears the same agreement markers. Since \il{Ho-Chunk}Hocąk is an active-stative language, the similarities between \REF{ex:rosen:29} and \REF{ex:rosen:30} follow if apparent adjectives are stative verbs. Second, apparent adjectives can be used predicatively without any morphological modification or without the help of auxiliaries, as seen in (31a). \citet{Helmbrecht2006b} asserts that the lack of auxiliaries is possible for all \isi{adjective}s in \il{Ho-Chunk}Hocąk. This possibility extends to verbs as well. (31a) shows an example of the verb \textit{nįį} `swim'.
 

\ea
\label{ex:rosen:31}
\ea \glll Wijukra seepšąną. \\
wijuk-ra {$\varnothing$}-seep-šąną\\
cat-\textsc{def} \textsc{3s}-black-\textsc{decl}\\
\glt `The cat is black.'


\ex \glll Hocįcįkra nįį eeja nįįpšąną.\\
hocįcįk-ra nįį eeja {$\varnothing$}-nįįp-šąną\\
boy-\textsc{def} water there \textsc{3s}-swim.\textsc{act}-\textsc{decl}\\
\glt `The boy swam in the lake.'
\z
\z

Thus, since verbs and purported \isi{adjective}s may also be the main \isi{predicate} of the clause\is{clauses}, there is no structural difference between adjectives and verbs.

 
In the following subsections, I 
will \todo{added `will' for hyphenation}
present two arguments that the \isi{resultative} phrase projects as an AP\is{adjective} in \il{Ho-Chunk}Hocąk \isi{resultative}s.\footnote{\citet{Baker2003} has previously argued that a main characteristic of \isi{adjective}s is that they can occur as secondary \isi{resultative} \isi{predicate}s.} In the first subsection, I argue that the linear ordering of the result and the matrix verb indicates that the result is an AP\is{adjective}. In the second subsection, I turn to the fact that (stative) verbs are ungrammatical as a result \isi{predicate}. I argue that only gradable \isi{predicate}s (i.e., adjectives) can participate in \isi{resultative}s. 
 

\subsection{The Temporal Iconicity Constraint and \isi{resultative}s}
 
Following \citet{Li1993}, I suggest that the fact that the result precedes the verb in \isi{resultative} predication provides evidence that the result is an \isi{adjective} in \il{Ho-Chunk}Hocąk.\footnote{Thanks to Yafei Li (personal communication) for bringing this diagnostic to my attention.} Specifically, I argue that since the result precedes the matrix verb in \isi{resultative}s, \citegen{Li1993} Temporal Iconicity Constraint would be violated if the result were a verb. Rather, since the result must precede the verb in \il{Ho-Chunk}Hocąk \isi{resultative}s, the result must not be a verb. Instead, I claim that the result is an \isi{adjective}.
 

\citet[499]{Li1993} proposes his constraint in order to account for the restrictions on the order of verbs in V-V \isi{resultative} \isi{compound}s in \ili{Chinese} and \ili{Japanese}. The first V (V-cause) always encodes the event, while the second V (V-result) indicates the result of the event. 

Li shows that V-cause must temporally and morphologically precede V-result. Li formalizes this constraint as in \REF{ex:rosen:32}.

\begin{exe}

\ex\label{ex:rosen:32}
 \emph{Temporal Iconicity Constraint (TIC)}:\\
 Let A and B be two subevents (activities, states, changes of states, etc.) and let A$'$ and B$'$ be two verbal constituents denoting A and B, respectively; then the temporal relation between A and B must be directly reflected in the surface linear order of A$'$ and B$'$ unless A$'$ is an argument of B$'$ or vice versa.
 
 \end{exe}

For example, Li notes that in both \ili{Chinese} and \ili{Japanese}, V-cause is the first verb of the \isi{compound}. Consider the \ili{Chinese} example in \REF{ex:rosen:33} and the \ili{Japanese} example in \REF{ex:rosen:34}.

 \begin{exe}
 \ex \label{ex:rosen:33}
 \gll Táotao tiào-fán-le {\op}Y$\overline{\rm{o}}$uyou le{\cp}.\\
 Taotao jump-bored-\textsc{asp} {\db}Youyou \textsc{le}\\
 \glt `Taotao jumped and as a result he/(Youyou) got bored.' (\citealt[480 (1b)]{Li1993})
 
 \ex \label{ex:rosen:34}
\gll John-ga Mary-o karakai-akiru-ta.\\
 John-\textsc{nom} Mary-\textsc{acc} tease-bored-\textsc{past}\\
 \glt `John teased Mary and as a result John got bored.'  (\citealt[481 (2b)]{Li1993})
 
 \end{exe}

What is important to note here is that V-cause always precedes V-result. In \REF{ex:rosen:33}, the V-cause \textit{tiào} `jump' necessarily precedes V-result \textit{f\'an} `bored'. Without the parentheses in \REF{ex:rosen:33}, Taotao's jumping causes Youyou to become bored. With the parentheses in \REF{ex:rosen:33}, Taotao's jumping makes himself become bored. In \REF{ex:rosen:34}, the V-cause \textit{karakai} `tease' must appear to the left of the V-result \textit{akiru} `bored'. A further piece of evidence for the TIC comes from serial-verbs in \ili{Sranan} and \textsubdot{I}j\textsubdot{o}. \ili{Sranan} is syntactically a head-initial language, whereas \textsubdot{I}j\textsubdot{o} is head-final. Both examples in \REF{ex:rosen:35} illustrate that the verb phrase that denotes getting ahold of the instrument linearly precedes the central action. That is, `take the knife' in \ili{Sranan} comes before `cut the bread', and the same pattern is seen in \textsubdot{I}j\textsubdot{o} with `basket take' preceding `yam cover'.

\ea\label{ex:rosen:35}
\ea
\ili{Sranan}; SVO\\
\gll Mi e teki a nefi koti a brede. \\
I \textsc{asp} take the knife cut the bread \\
\glt `I cut the bread with the knife.'

\ex 
\textsubdot{I}j\textsubdot{o}; SOV\\
 \gll \'Aràú su-ye ákì buru teri-mí. \\
she basket take yam cover-\textsc{past}\\
\glt `She covered a yam with a basket.' (\citealt[500]{Li1993}, \REF{ex:rosen:38})
\z
\z

We find similar evidence from manner-of-directed motion serial verbs in \il{Ho-Chunk}Ho\-cąk. These serial verbs consist of a manner-of-motion verb (e.g., \textit{nųųwąk} `run') and a directional motion verb (e.g., \textit{hii} `arrive'). In \il{Ho-Chunk}Hocąk, the order of these two verbs cannot be reversed. Example \REF{ex:rosen:36} shows that the linear order of \textit{nųųwąk} `run' and \textit{hii} `arrive' must be \textit{nųųwąk-hii}. The verb \textit{hii} `arrive' must always be the second verb. This directly follows from the TIC: a running event must logically precede the arriving event.

\let\eachwordtwo=\itshape
\begin{exe}
\ex\label{ex:rosen:36}
\begin{xlist}

\ex[] {\glll Matejaga Teejop eeja nųųwąk \textbf{hii}.\\
Mateja-ga Teejop eeja nųųwąk {$\varnothing$}-hii\\
Mateja-\textsc{prop} Madison there run \textsc{3s}-arrive\\
\glt `Mateja ran to Madison.'}

\ex[*] {\glll Matejaga Teejop eeja  \textbf{hii} nųųwąk.\\
Mateja-ga Teejop eeja  hii {$\varnothing$}-nųųwąk\\
Mateja-\textsc{prop} Madison there  arrive \textsc{3s}-run\\
\glt (Intended: `Mateja ran to Madison.')}

\end{xlist}
\end{exe}

 
Despite the strong predictions that the TIC makes, it is not intended to account for all \isi{resultative} constructions. According to Li's proposal, the TIC applies only if two conditions are met: one, the constituents involved are both verbal, and two, the verbal constituents must not be in a \isi{predicate}-argument relation (e.g., causatives). Here I am only concerned with the first condition, as this second condition does not apply to \il{Ho-Chunk}Hocąk \isi{resultative}s. Li presents an example from \ili{German} to illustrate the first constraint, as in \REF{ex:rosen:37}.
 

\let\eachwordtwo=\upshape 

\begin{exe}
\ex\label{ex:rosen:37}
 \gll Er will das Eisen flachschlagen.\\
he wants the iron flat.pound\\
\glt `He wants to pound the iron flat.' (\citealt[501 (41)]{Li1993})

\end{exe}

The result encoded by \textit{flach} `flat' linearly precedes the activity \textit{schlagen} `pound'. Since \textit{flach} `flat' is an \isi{adjective}, Li claims that the TIC does not apply. Rather the head-final structure of \ili{German} determines the order of \textit{flach} `flat' and \textit{schlagen} `pound'. 

In summary, while the TIC applies to verbal constituents, the TIC has nothing to say about when \isi{adjective}s form similar events with verbs.

Let us return to the \il{Ho-Chunk}Hocąk data. We see that the result precedes the matrix verb, as in (38a). That is, \textit{paras} `flat' linearly precedes \textit{gistak} `hit'. In fact it is ungrammatical for the result to be postverbal, as shown in (38b).


\let\eachwordtwo=\itshape
\begin{exe}
\ex\label{ex:rosen:38}
\begin{xlist}

\ex[] {\glll Meredithga mąąsra \textbf{paras} gistakšąną.\\
Meredith-ga mąąs-ra paras {$\varnothing$}-gistak-šąną\\
Meredith-\textsc{prop} metal-\textsc{def} flat \textsc{3s/o}-hit-\textsc{decl}\\
\glt `Meredith hit the metal flat.'}

\ex[*] {\glll Meredithga mąąsra  gistakšąną, \textbf{paras}.\\
Meredith-ga mąąs-ra  {$\varnothing$}-gistak-šąną paras\\
Meredith-\textsc{prop} metal-\textsc{def}  \textsc{3s/o}-hit-\textsc{decl} flat\\
\glt (Intended: `Meredith hit the metal flat.')}

\end{xlist}
\end{exe}

 
Accordingly, if apparent \isi{adjective}s in \il{Ho-Chunk}Hocąk are stative verbs, then the grammaticality of examples like (38a) is surprising. We expect (38a) to be ungrammatical, given the TIC. Since the TIC does not rule out examples like (38a), we can conclude that the result is not a verb. This is similar to the \ili{German} example in \REF{ex:rosen:37}. Moreover, the fact that the order that the TIC predicts, as in (38b), is ungrammatical also leads to the conclusion that the result is not a verb.\footnote{More needs to be said as to why the result cannot be postverbal. \citet{JohnsonRosen2014} propose that constituents are moved to a postverbal position via an EPP feature that can only attract DPs. I leave a full explanation of this issue open for now.} I take this as evidence that the result is an AP\is{adjective}.
 

\subsection{Barring verbs as the result}

 
In this section, I show that \isi{adjective}s can appear in \isi{resultative} secondary predication, while verbs cannot. In order to account for the contrast, I argue that we need to slightly refine the structure of the result phrase: the result phrase in \il{Ho-Chunk}Hocąk is an AP\is{adjective} that contains a degree phrase. Following \citet{Corver1997},  I assume that only gradable adjectives have a degree argument, and that degree heads need to bind such a degree argument. I show that non-gradable \isi{adjective}s are incompatible with \isi{resultative}s in \il{Ho-Chunk}Hocąk. Thus, if verbs do not have a degree argument to be discharged, the structure will be ruled out as an instance of vacuous quantification. Compare \REF{ex:rosen:39} that has \textit{žiipįk} `short' as the result with \REF{ex:rosen:40} that uses the verb \textit{šiibre} `fall'.
 

\begin{exe}

\ex[] {\label{ex:rosen:39}\glll Matejaga peešjįra \textbf{žiipįk} rucgisšąną. \\
 Mateja-ga peešjį-ra žiipįk {$\varnothing$}-rucgis-šąną\\
Mateja-\textsc{prop} hair-\textsc{def} short \textsc{3s/o}-cut-\textsc{decl}\\
\glt `Mateja cut the hair short.'}

\ex[*]
{\label{ex:rosen:40}\glll Matejaga peešjįra \textbf{šiibre} rucgisšąną. \\
 Mateja-ga peešjį-ra šiibre {$\varnothing$}-rucgis-šąną\\
Mateja-\textsc{prop} hair-\textsc{def} fall \textsc{3s/o}-cut-\textsc{decl}\\
\glt (Intended: `Mateja cut the hair (so that) it falls.')}

\end{exe}

 
The ungrammaticality of a verb like \textit{šiibre} `fall' in a \isi{resultative} construction \REF{ex:rosen:40} indicates that this \isi{predicate} is somehow fundamentally different than the one in \REF{ex:rosen:39}. If we take a closer look at \il{Ho-Chunk}Hocąk, we notice that verbs are not the only elements that cannot be a secondary \isi{resultative} \isi{predicate}. While I argue that only \isi{adjective}s can be \isi{resultative} predicates in \il{Ho-Chunk}Hocąk, not all adjectives are available in this position. Crucially, non-gradable adjectives cannot appear as a result \isi{predicate}. The example in \REF{ex:rosen:41} illustrates this for the non-gradable \isi{adjective} \textit{t'ee} `dead', which is ungrammatical as a result \isi{predicate}. Note that the \ili{English} equivalent is grammatical, as indicated by the translation in \REF{ex:rosen:41}.
 

\begin{exe}
\ex[*] {\label{ex:rosen:41}\glll Bryanga caara t'ee guucšąną.\\
Bryan-ga caa-ra t'ee {$\varnothing$}-guuc-šąną\\
Bryan-\textsc{prop} deer-\textsc{def} dead \textsc{3s/o}-shoot-\textsc{decl}\\
\glt (Intended: `Bryan shot the deer dead.')}
\end{exe}
 
To account for the restriction seen in \REF{ex:rosen:41}, I claim that the \isi{resultative} \isi{predicate} in \il{Ho-Chunk}Hocąk takes a DegP in its \isi{specifier}, as shown in \REF{ex:rosen:42}. I label this degree phrase ``Deg\textsubscript{RES}P.''\footnote{\citet{Corver1997}
 argues that DegP dominates the AP\is{adjective} (as in (i)). Differently than Corver, the structure in \REF{ex:rosen:42} follows \citet{Jackendoff1977b} and  \citet{BhattPancheva2004}, among others, and places DegP in Spec,\is{specifier}AP\is{adjective}. Nothing crucially hinges on the placement of the degree phrase, however. 
\ea \upshape \Tree [ .DegP [ .AP\is{adjective} ]  [ .Deg ] ] \z
}

\begin{exe}
\ex \label{ex:rosen:42}
{\hspace{1em}}\newline
\begin{tikzpicture}
\Tree [ .VP\is{verb phrase} [ .NP \edge[roof]; {wažątirera\\`the car'} ] [ .V$'$ [ .AP\is{adjective} [ .Deg\textsubscript{RES}P \edge[roof]; {{$\varnothing$}} ] [ .A šuuc\\`red' ] ] [  .V hogiha\\`paint' ]] ]
\end{tikzpicture}
\end{exe}
 
\il{Ho-Chunk}Hocąk \isi{resultative}s are thus obtained by specifying the eventuality of the result to the highest degree. This is consistent with \citegen{Wechsler2005} proposal on the constraints on the result \isi{predicate}. Wechsler asserts that the result must express a gradable property with a maximum degree, when the \isi{object} NP is the argument of the matrix verb. I assume that only gradable \isi{adjective}s can take DegPs in their \isi{specifier}s, while non-gradable adjectives lack this ability. The degree head is an operator and thus has to bind a variable. If gradable adjectives have a degree argument (or grade-role) in its argument structure, then the degree head will be able to bind it. On the other hand, if non-gradable adjectives lack this degree argument, then the structure will be ruled out since all operators have to bind a variable. Consider the contrast between the gradable \isi{adjective} \textit{sgįgre} `heavy' in (43a) and the non-gradable \isi{adjective} \textit{t'ee} `dead' in (43b) with the degree element \textit{eegišge} `too'.
 

\begin{exe}
\ex\label{ex:rosen:43}
\begin{xlist}

\ex[] {\glll Henryga eegišge sgįgre.\\ 
Henry-ga eegišge {$\varnothing$}-sgįgre\\
Henry-\textsc{prop} too \textsc{3s}-heavy\\
\glt `Henry is too heavy.'}


\ex[*] {\glll Caara eegišge t'ee\\
Caa-ra eegišge {$\varnothing$}-t'ee\\
deer-\textsc{def} too \textsc{3s}-dead\\
\glt (Intended: `The deer is too dead.')}

\end{xlist}
\end{exe}

I propose that \textit{eegišge} realizes a Deg head; thus, an example like in (43a) has the structure in \REF{ex:rosen:44}.\footnote{As noted in footnote 11, it could also be the case that DegP dominates the AP\is{adjective}. I suggest that \textit{eegišge} `too' would be in the \isi{specifier} of a head-final and phonologically\is{phonology} null Deg. See \citet{Rosen2015} for more information.}

\begin{exe}
\ex \label{ex:rosen:44}
{\hspace{1em}}\newline
\begin{tikzpicture}
\Tree [ .AP\is{adjective} [ .DegP \edge[roof]; {eegišge}\\`too' ] [ .A  sgįgre\\`heavy' ] ]   
\end{tikzpicture}
\end{exe}

 
I attribute the ungrammaticality of non-gradable \isi{adjective}s with \textit{eegišge} `too' to the hypothesis that the degree head associated with eegišge `too' must bind the degree argument of a lexical item. Since non-gradable adjectives in \il{Ho-Chunk}Hocąk do not have degree arguments as part of their lexical entry (cf. \citealt{Higginbotham1985}; \citealt{Corver1997}), the degree head does not have a degree argument to bind. This results in ungrammaticality. Under the present analysis, since the result takes a degree phrase in its \isi{specifier}, it is expected that a non-gradable \isi{adjective} is not allowed as a result \isi{predicate}. In the case of the \isi{resultative} in \REF{ex:rosen:41}, \textit{t'ee} `dead' is ill-formed because the degree operator in Deg does not have a variable in its scope that it can bind.\footnote{I assume that color adjectives, such as \textit{šuuc} `red', are gradable (\citealt{KennedyMcNally2010}).} Let us return to the fact that verbs are ungrammatical as \isi{resultative} \isi{predicate}s. Following \citet{Higginbotham1985} and \citet{Corver1997}, I assume that verbs do not have a grade-role; rather they have an event-role. This is evidenced by the ungrammaticality of the verb \textit{šiibre} `fall' with \textit{eegišge} `too' in \REF{ex:rosen:45}.\footnote{The contrast between \REF{ex:rosen:43} and \REF{ex:rosen:45} illustrates another way in which stative verbs and adjectives differ. In this paper, I am only concerned with how they differ with respect to \isi{resultative}s. \citet{Rosen2014,Rosen2015} presents more diagnostics for the existence of \isi{adjective}s in \il{Ho-Chunk}Hocąk.}
 

\begin{exe}

\ex[*] {\label{ex:rosen:45}\glll Hunterga eegišge šiibre.\\
Hunter-ga eegišge {$\varnothing$}-šiibre\\
Hunter-\textsc{prop} too \textsc{3s}-fall.\textsc{stat}\\
\glt (Intended: `Hunter fell too much/a lot.')}

\end{exe}
  
I argue that \isi{resultative} examples like \REF{ex:rosen:40} with verbs are ungrammatical since a degree head can measure the state of the \isi{adjective}, but it cannot link the event of a verb. In other words, example \REF{ex:rosen:40} is ruled out because there is a mismatch between the selectional restrictions of DegP and a verb phrase. This explains why verbs are barred from \isi{resultative}s in \il{Ho-Chunk}Hocąk. In this subsection, we see that verbs cannot appear as the result \isi{predicate} in \il{Ho-Chunk}Hocąk. The reason that verbs cannot appear as the result, I claim, is that the result \isi{predicate} takes a special degree phrase that I labeled ``Deg\textsubscript{RES}P'' in its \isi{specifier}. A straightforward explanation arises if we assume that degree phrases in \il{Ho-Chunk}Hocąk must bind a degree argument. Since I am assuming that verbs lack a degree argument, verbs are not allowed as a result \isi{predicate}. Thus, I contend that the result \isi{predicate} in \il{Ho-Chunk}Hocąk is an AP\is{adjective}.
 

\subsection{Implications: Status of \isi{adjective}s}

 
I have presented evidence that the result \isi{predicate} in \il{Ho-Chunk}Hocąk \isi{resultative}s projects as an AP\is{adjective} (an \isi{adjective}). This puts \isi{resultative}s in \il{Ho-Chunk}Hocąk in line with \isi{resultative} constructions cross-linguistically that use AP\is{adjective}s as result \isi{predicate}s (cf. \ili{English} \isi{resultative}s). Moreover, these data indicate that \il{Ho-Chunk}Hocąk has the lexical category \isi{adjective}. This is a significant result since \il{Ho-Chunk}Hocąk has been previously described as only having nouns and verbs (see \S 2.3). The previous traditional literature (e.g., \citealt{Helmbrecht2006b}) has focused primarily on the morphological similarities between stative verbs and adjectives. The data from \isi{resultative}s have shown that these similarities can be misleading. Rather \isi{adjective}s surface in at least one environment in \il{Ho-Chunk}Hocąk; namely, \isi{resultative}s (see \citealt{Rosen2014,Rosen2015} for further discussion of these issues in \il{Ho-Chunk}Hocąk, and \citealt{Baker2003} and \citealt{Dixon2004} cross-linguistically).

\section{Conclusion}\label{sec:rosen:6}
This paper has offered a description and an analysis of \il{Ho-Chunk}Hocąk \isi{resultative}s. I have shown that the result \isi{predicate} must not be a clause\is{clauses} and must be in a VP-internal position. I have argued that \il{Ho-Chunk}Hocąk \isi{resultative}s project a phrasal AP\is{adjective} as the \isi{complement} of the verb in a Larsonian ``\isi{VP-shell}'' (\citealt{Larson1988}). This proposal is supported by the fact that \isi{resultative}s in \il{Ho-Chunk}Hocąk have many of the properties that have been attributed to \isi{resultative}s cross-linguistically, such as in \ili{English}. In particular, \isi{resultative}s in \il{Ho-Chunk}Hocąk obey the DOR, and the \isi{resultative} phrase is adjectival. I conclude with the hope that this paper will continue to improve our understanding of \isi{resultative}s and the structure of predication in \il{Ho-Chunk}Hocąk and Siouan languages.
 

\section* {Acknowledgments}
I am extremely grateful to Cecil Garvin\ia{Garvin, Cecil} for teaching me about his language. Thanks also to Meredith Johnson\ia{Johnson, Meredith}, Yafei Li\ia{Li, Yafei}, Becky Shields\ia{Schields, Becky}, Catherine Rudin\ia{Rudin, Catherine}, Rand Valentine\ia{Valentine, Rand}, and an anonymous reviewer for valuable discussion and comments, and to Iren Hartmann\ia{Hartmann, Iren} for allowing me to use her Lexique Pro \isi{dictionary}. Lastly, material in this paper was presented at SSILA 2014. I would like to thank the audience there.

\section*{Abbreviations}
\begin{tabularx}{.45\textwidth}{lX}
1, 2, 3 & first, second, third person\\
\textsc{comp} & complementizer \\
 \textsc{decl} & declarative \\
 \textsc{def} & definite \\
 \textsc{indef} & indefinite \\
\end{tabularx}
\begin{tabularx}{.45\textwidth}{lX}
 \textsc{o} & {object} agreement \\
 \textsc{pl} & plural \\
 \textsc{prop} & proper noun \\
 \textsc{refl} & reflexive \\
 \textsc{s} & {subject} agreement\\ 
 \\
\end{tabularx}


{\sloppy 
\printbibliography[heading=subbibliography,notkeyword=this]
}

\let\eachwordtwo=\upshape
\end{document}