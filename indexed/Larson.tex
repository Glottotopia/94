%3
\documentclass[output=paper]{LSP/langsci}
\author{Rory Larson}
\title{Regular sound shifts in the history of {Siouan}}

\abstract{The team of contributors to the Comparative Siouan Dictionary (CSD) reconstructed a phonemic set for Proto-Siouan, together with the necessary reflexes to produce the actual speech sounds found in the various daughter languages.  Until recently, this system was common knowledge within the Siouanist community, since the participants in the CSD project were active as the leaders of that community, and were available to explain the predicted sound shifts.  With the passing, retirement, or disappearance of most of the CSD team, however, it seems that it might be useful to document the reconstructed system and its most important regular reflexes, as an aid to comparative studies.  This paper will rely primarily on the CSD, edited until his passing by Dr. Robert Rankin\ia{Rankin, Robert L.}, to summarize the regular sound shifts known to have occurred in Siouan.  It will prioritize sound shifts in which separate phonemes or clusters have collapsed together to become indistinguishable in the daughter languages, since this is where interesting confusion is most likely to occur. 
% KEYWORDS: [Siouan, proto-Siouan, sound shifts, regular reflexes, historical phonology]
} 
\ChapterDOI{10.17169/langsci.b94.122}

\maketitle

\begin{document}

\section{Introduction}  
In 1984, a group of linguists studying Siouan languages began a  project under NEH and NSF sponsorship to assemble a comparative \isi{dictionary} of the Siouan language family.  The principal investigator was David S. Rood\ia{Rood, David S.}. The team included Richard T. Carter\ia{Carter Jr., Richard T.}, A. Wesley Jones\ia{Jones, A. Wesley} and Robert L. Rankin as senior editors, along with Rood and John E. Koontz\ia{Koontz, John E.}.  Together with Willem de Reuse\ia{Reuse, Willem de}, Randolph Graczyk\ia{Graczyk, Randolph}, Patricia A. Shaw\ia{Shaw, Patricia A.} and Paul Voorhis\ia{Voorhis, Paul}, the \isi{dictionary} team began their project at the Comparative Siouan Workshop held at the University of Colorado in 1984.  A number of other scholars, including Louanna Furbee\ia{Furbee, Nonnie Louanna}, Jimm Goodtracks\ia{Goodtracks, Jimm G.}, Jill Hopkins Greer\ia{Greer, Jill D.}, Kenneth Miner\ia{Miner, Kenneth L.}, Carolyn Quintero\ia{Quintero, Carolyn}, Kathleen Shea\ia{Shea, Kathleen} and Mark Swetland\ia{Awakuni-Swetland, Mark} also contributed information.

This undertaking was huge.\footnote{For an enlightening and wryly humorous history of the project, see ``The \isi{Comparative Siouan Dictionary} Project'' \citep{RoodKoontz2002} written by two of its principal participants.} The present writer regrettably turned down an offer by Richard Carter\ia{Carter Jr., Richard T.} in about 1997 of an advance copy of the \isi{dictionary} manuscript, on the assurance that it would be published within a year.  In fact, it was never completed to the editors' satisfaction or published in book form. Carter himself retired from active work in the Siouan field after around 2002, and Robert Rankin\ia{Rankin, Robert L.} became the principal steward of the project.  In \citeyear{Rankinetal2006PDF}, Rankin distributed a .pdf file of the manuscript as it stood so far to interested members of the Siouanist community, on condition that any further requests be submitted to himself or David Rood.  The file runs to nearly a thousand pages, and is full of working notes and comments about the various words and their relationships, mostly by Rankin.

Rankin, Carter, and their colleagues developed a sophisticated understanding of the phonological and phylogenetic relationships among the various groups of Siouan.  Until recently, they formed a body of respected linguistic ``elders'' who freely shared this lore on request with more junior scholars.  With the untimely passing of Robert Rankin in February of 2014, however, and the retirement, disappearance, or focus shift of most of the other leading members of the team, the framework they developed seems in danger of being forgotten by the Siouanist community.  This paper is intended to address that concern.  Drawing on working notes found throughout the CSD, as well as years of discussions on the \isi{Siouan List}, it will attempt to summarize the model of Siouan \isi{phonology} and its standard sound shifts built by Rankin and the CSD team, with occasional comments and additions from the writer.  All references to the CSD are to the \citeyear{Rankinetal2006PDF} version. (In May 2015, after this paper was completed, the most recent version was made available online at csd.clld.org (\citealt{Rankinetal2015AccessMay}) where most of the original notes and comments can now be found.)


\section{The Siouan family tree}

The CSD recognizes four major branches of Siouan. In the far northwest is Missouri Valley, or \ili{Crow}-\ili{Hidatsa}, consisting of the \ili{Crow} and \ili{Hidatsa} languages. Next is \ili{Mandan}, an isolate within Siouan. Third is \ili{Mississippi Valley Siouan}, or ``MVS'', which itself has three branches. Fourth is Southeastern, or \ili{Ohio Valley Siouan}, at the southeastern end of the Siouan span.

MVS branches into \ili{Dakotan}, which includes the five ``\ili{Sioux}'' \isi{dialects} of \ili{Santee}-Sisseton (\ili{Dakota}), \ili{Yankton-Yanktonai}, \ili{Teton} (\ili{Lakota}), \ili{Assiniboine} and \ili{Stoney};\footnote{\citet{ParksDeMallie1992} present the results of a major dialect survey undertaken to clarify the relations among the various \ili{Dakotan} \isi{dialects}, a field in which considerable confusion had prevailed in the literature prior to their work.  The CSD itself is rather deficient in \ili{Dakotan} material other than \ili{Lakota} and \ili{Dakota}, although it contains a few words from four other categories: \ili{Stoney}, \ili{Assiniboine}, Yankton, and ``\ili{Sioux} Valley''.  The latter presumably refers to the \ili{Sioux} Valley reservation in southwestern Manitoba, which Parks and DeMallie classify along with the Minnesota ``\ili{Dakota}'' within the broad \ili{Santee}-Sisseton dialect group.  The present paper will follow Parks and DeMallie in sub-classification of \ili{Dakotan}.}  \ili{Winnebago}-\ili{Chiwere}, composed of Hooc\k{a}k and the \ili{Iowa}, \ili{Oto} and Missouria languages; and \ili{Dhegiha}, comprising \ili{Omaha}-\ili{Ponca}, \ili{Kaw}, \ili{Osage} and \ili{Quapaw}.  \ili{Southeastern Siouan} contains \ili{Biloxi} and \ili{Ofo} as one branch, and \ili{Tutelo} and \ili{Saponi} as another.

\ili{Catawba} is the language most closely related to Siouan.  Though sound-historical relationships are not very clear, \ili{Catawba} examples are often included in an entry's word list.  The language probably next most closely related is \ili{Yuchi} (cf. \citealt{Kasak2016} (in this volume)), but few examples of \ili{Yuchi} are given.


\section{The Reconstructed \ili{Proto-Siouan} phoneme set}

The CSD team recognizes eight vowels for \ili{Proto-Siouan}, five oral and three nasal, which were distinguished also by length \citep{RankinEtAl1998}.%} ``Proto Siouan Phonology and Grammar''.}
 

\begin{center}
\begin{tabular}[t]{c c c c c c c c} 
i &  & u &&& \k{i} &    & \k{u}\\

e &  & o \\

 & a &   &&&      & \k{a} \\ 
\end{tabular}
\end{center}		

Basic stops are *p, *t, *k, and the \isi{glottal} stop.  \ili{Proto-Siouan} had a series of alveolar, palatal and velar fricatives: *s, *š, and *x, as well as *h.  It also had three resonants, *w, *r and *y.  Minimally, its consonant structure was as follows: 
\begin{table}
\begin{tabular}[t]{r c c c c c c}
\lsptoprule
& Labial & Alveolar & Palatal & Velar & Glottal \\
\midrule
Stops  & p & t & & k & \textipa{P} \\

Fricatives   & & s & š & x & h \\

Resonants  & w & r & y \\
\lspbottomrule
\end{tabular}
\caption{Consonants}
\end{table} 

Many of the consonants of Siouan have occurred in clusters, however, so the actual historical picture is more complex than this.  Stops can be adjoined to other stops in almost any order, all non-\isi{glottal} stops and fricatives can be glottalized, aspiration (h) can occur either before or after stops, and combinations can occur involving fricatives, stops and resonants.  In particular, there exist two historical phonemes that manifest as either stops or resonants in the daughter languages, called ``funny w'' and ``funny r''.  We symbolize these sounds as *W and *R.  *R was found in \ili{Proto-Siouan}, and *W in MVS only.  Rankin believed that *R was originally a combination of *r with a laryngeal, either *h or the \isi{glottal} stop.

Notably, Siouan had no distinct nasal consonant series.  When *w or *r occurred in the environment of a nasal vowel, they usually manifested as |m| or |n|, respectively.

Accent in \ili{Proto-Siouan} was normally on the second syllable of a word.

\section{Historical Siouan sound shifts}

One of the first sound shifts affecting Siouan was a process called ``\isi{Carter's Law}'.  Wherever a simple stop, *p, *t or *k, occurred before the vowel of an accented syllable, the stop itself was more prominently ``marked'', either by lengthening it or by preaspirating it.  In the CSD, these are considered to be preaspirated.  Thus, *p, *t and *k become *hp, *ht and *hk before an accented syllable.  Since accent normally was on the second syllable of a word, these preaspirated stops and their derivatives are usually found inside the word rather than at the beginning.  When they are found at the beginning of a word, it may be an indication of a lost initial syllable.
%begin{center}


\begin{tabular}[t]{c c c c c }

\isi{Carter's Law}: & *p\'V & > & *hp\'V  \\

& *t\'V & >  & *ht\'V \\

&  *k\'V & >  & *hk\'V\\
\end{tabular}
%\end{center}

\subsection{Missouri Valley (\ili{Crow}-\ili{Hidatsa}) reflexes}

In \ili{Missouri Valley Siouan}, loss of historical aspiration, loss of nasal vowels and the merger of *y with *r are the most sweeping transformations of the \ili{Proto-Siouan} phonemic inventory.  Several other changes also occur.

\begin{itemize}
\item As in \ili{Mandan}, \ili{Proto-Siouan} aspiration is lost.\footnote{\citealt[50, 85]{Rankinetal2006PDF}.} This notably includes the preaspirate series produced by the operation of \isi{Carter's Law}.
%begin{center}

				
\begin{tabular}[t]{c c c c c c }
Loss of aspiration: & *hp & > & *|p|	& < & 	*p \\
& *ht	 & > & *|t|	& <	 & *t \\
& *hk & > & *|k|	 & < & *k \\
\end{tabular}
%\end{center}

\item Phonemic \isi{nasalization} is completely lost.  The three \ili{Proto-Siouan} nasal vowels merge with their oral counterparts, and neither vowels nor consonants are distinguished by nasality.\footnote{\citealt[109]{Rankinetal2006PDF}.}
%begin{center}


\begin{tabular}[t]{c c c c c c }
Loss of nasal vowels: &	*\k{a} & >	& *|a| & < & *a \\
& *\k{i} & > & *|i| & < & *i \\
& *\k{u} & > & *|u| & < & *u \\
\end{tabular}
%\end{center}

\item As in \ili{Mandan} and Hooc\k{a}k, \ili{Proto-Siouan} *y merges with *r.

%begin{center}


\begin{tabular}[t]{c c c c c c c}
*y/*r merger: & & *y	 & >  & *|r| &  <	 & *r 
\end{tabular}
%\end{center}

\item Between vowels at the end of a word, *h is lost .

%begin{center}


\begin{tabular}[t]{c c c c c c c}
Loss of intervocalic *h: & & *V\textsubscript{1}hV\textsubscript{2} & > & *|V\textsubscript{1}V\textsubscript{2}|	
\end{tabular}
%\end{center}

\item Rightward vowel exchange, in which the first two vowels of a word are swapped.\footnote{\citealt[193, 788]{Rankinetal2006PDF}.}  Both \ili{Crow} and \ili{Hidatsa} show this feature, but not necessarily in the same words, which suggests that this change was spreading at the time \ili{Crow} and \ili{Hidatsa} separated.

%begin{center}


\begin{tabular}[t]{c c c c c c c}
Rightward vowel exchange:	 & & *CV\textsubscript{1}CV\textsubscript{2} & > & |CV\textsubscript{2}CV\textsubscript{1}|	
\end{tabular}
%\end{center}
\end{itemize}

\subsubsection{\ili{Hidatsa} reflexes}

Few changes are specific to \ili{Hidatsa}.  There may be a few vowel shifts and \isi{cluster} changes.  \ili{Proto-Siouan} *w generally manifests as |m|.

\begin{itemize}
\item Short *o is raised to |u|:\footnote{\citealt[137, 922]{Rankinetal2006PDF}.}	 \hspace{1em} *o	>	|u|
\item *xk becomes |hk|:\footnote{\citealt[193]{Rankinetal2006PDF}.} \hspace{3.2em}  *xk	>	|hk|
\item *w becomes |m|: \hspace{4.1em} *w	>	|m|
\end{itemize}

\subsubsection{\ili{Crow} reflexes}

\ili{Crow} is more innovative.  The biggest change is complete loss of glottals, usually with lengthening of the following vowel.  \ili{Proto-Siouan} *x manifests as |xš|.  \ili{Proto-Siouan} *t becomes |s|.

\begin{itemize}
\item Glottalization is lost, but is reflected in the lengthening the following vowel, usually with rising pitch.\footnote{\citealt[232]{Rankinetal2006PDF}.}

%begin{center}


\begin{tabular}[t]{c c c c c c c}
Loss of glottals:	 & & *(C)\textsuperscript{\textipa{P}}V & > & |(C)V\'V|
\end{tabular}
%\end{center}

\item *x becomes |xš|:\footnote{	\citealt[124]{Rankinetal2006PDF}.}	\hspace{1em} *x	>	|xš|
\item *t becomes |s|: \hspace{2em}  *t	>	|s|
\end{itemize}

\subsection{\ili{Mandan} reflexes}

In \ili{Mandan}, loss of historical aspiration\footnote{\citealt[50]{Rankinetal2006PDF}.} and the merger of *y with *r are the most notable sound shifts, as well as a peculiar reversal of sibilants.\footnote{\citealt[126]{Rankinetal2006PDF}.}

\begin{itemize}
\item As in \ili{Crow} and \ili{Hidatsa}, historical aspiration is lost, including the preaspirate series.

%begin{center}


\begin{tabular}[t]{c c c c c c }
 Loss of aspiration: & *hp & > & |p| & < & *p \\
& *ht & >	 & |t| & < & *t \\
& *hk & > & |k| & < & *k \\
\end{tabular}
%\end{center}

\item As in \ili{Crow}, \ili{Hidatsa}, and Hooc\k{a}k, \ili{Proto-Siouan} *y merges with *r.
%begin{center}


\begin{tabular}[t]{c c c c c c c}
*y/*r merger: & & *y	 & > & |r| & < & *r
\end{tabular}
%\end{center}
\item \ili{Proto-Siouan} *s and *š swap phonetic value.  *s becomes |š| and *š becomes |s|.

%begin{center}


\begin{tabular}[t]{c c c c c c }
*s/*š reversal: & & *s & > & |š| \\
& & *š & > & |s| \\
\end{tabular}
%\end{center}

\item The \isi{cluster} *sp metathesizes to become |ps|. More generally, there seems to be a usual, but not quite complete, constraint against having |p| as the second element of a \isi{cluster}.\footnote{\citealt[275]{Rankinetal2006PDF}. }
%begin{center}


\begin{tabular}[t]{c c c c c}
*sp metathesis:	 & & *sp & > & |ps|
\end{tabular} 
%\end{center}

\item Before a consonant, the absolutizing or generalizing *wa- prefix loses its vowel through \isi{syncopation}, and the *w becomes |p|.\footnote{\citealt[793]{Rankinetal2006PDF}.}
%begin{center}


\begin{tabular}[t]{c c c c c }
*wa- \isi{syncopation}: & & *waC & > & |pC|
\end{tabular}
%\end{center}
\end{itemize} 

\subsection{MVS reflexes}

In \ili{Mississippi Valley Siouan} (MVS), the fricatives are divided between a voiceless series and a voiced series.  This is also the only branch of Siouan in which the preaspirates are clearly distinguishable.  Another major transformation is the loss of short, unaccented vowels in the initial syllable,\footnote{\citealt[10]{Rankinetal2006PDF}.} and the production of clusters that result from this \isi{syncopation}.  This frequently involves the absolutizing or generalizing *wa- prefix, as well as the first person \isi{subject} pronoun *wa\textsuperscript{1}- prefix.  Also, the *hr \isi{cluster} becomes *ht,\footnote{\citealt[199]{Rankinetal2006PDF}.} merging with the original preaspirate *ht.  For this group, we may restate the basic consonant set as follows:
\vspace{1em}
%begin{center}

\begin{table}
\begin{tabular}{r r c c c c c}
\lsptoprule
& & Labial & Alveolar & Palatal & Velar & Glottal \\
\midrule
Stops: & \\
& Simple:	& p	 & t & & k & \textipa{P} \\
& Preaspirate: & hp & ht	 & & hk \\
& Postaspirate: & ph & th & & kh \\
& Glottalized:	& p\textsuperscript{\textipa{P}}	& t\textsuperscript{\textipa{P}} & & k\textsuperscript{\textipa{P}} \\

Fricatives: & \\
& Voiceless: & & s	& š	& x	 & h \\
& Voiced:	& & z & \v{z} & \textipa{G} \\
& Glottalized:	& & s\textsuperscript{\textipa{P}} & š\textsuperscript{\textipa{P}}	 & x\textsuperscript{\textipa{P}} \\
Resonants: & \\
& Normal:	& w & r & y \\
& ``Funny'': & W & R \\
\lspbottomrule
\end{tabular}
\caption{Basic consonant set}
%\end{center}
\end{table}

\begin{itemize}
\item The \ili{Proto-Siouan} fricatives are divided between a voiced and a voiceless set, possibly according to phonological conditions.
%begin{center}


\begin{tabular}[t]{c c c c c c }
Voiced/voiceless \isi{fricative} split: & *s	 &  >  & *|s| and *|z| \\
& *š & > & *|š| and *|\v{z}| \\
& *x & > & *|x| and *|\textipa{G}| \\
\end{tabular}
%\end{center}

\item \ili{Proto-Siouan} *pr merges with syncopated *w-r to become MVS *br.
%begin{center}


\begin{tabular}[t]{c c c c c c c}
*pr/*w-r \isi{syncopation}: & & *w-r & > & *|br| & < & *pr
\end{tabular}
%\end{center}

\item Syncopated \ili{Proto-Siouan} *w-w usually becomes MVS *W.\footnote{\citealt[164, 193, 213]{Rankinetal2006PDF}}
%begin{center}


\begin{tabular}[t]{c c c c c }
*w-w \isi{syncopation}: & & *w-w & > & *W
\end{tabular}
%\end{center}

\item Syncopated \ili{Proto-Siouan} *\textit{wa}\textsuperscript{1}- used as the first person affixed pronoun `I', however, becomes MVS *m when it precedes *w or the \isi{glottal} stop.\footnote{\citealt[10]{Rankinetal2006PDF}.}
%begin{center}


\begin{tabular}[t]{c c c c c c c}
I-*\textit{wa}\textsuperscript{1}-w \isi{syncopation}: & & *\textit{wa}\textsuperscript{1}-w & > & *m & < & *\textit{wa}\textsuperscript{1}-\textipa{P} 
\end{tabular}
%\end{center}
\item Syncopated \ili{Proto-Siouan} *w-C, where C is a voiceless contoid, becomes MVS *pC.\footnote{\citealt[793]{Rankinetal2006PDF}.} 

%begin{center}


\begin{tabular}[t]{c c c c c c }
*w-C \isi{syncopation}: & & *w-h & > & *ph \\
& & *w-t & > & *pt \\
\end{tabular}
%\end{center}

\item \ili{Proto-Siouan} *hr merges with preaspirate *ht to become *|ht|.\footnote{\citealt[199]{Rankinetal2006PDF}.}
%begin{center}


\begin{tabular}[t]{c c c c c c c }
*hr/ht merger: & &*hr & > & *|ht| & < & *ht
\end{tabular}
%\end{center}
\end{itemize}

\subsubsection{\ili{Dakotan} reflexes}

In \ili{Dakotan}, \isi{vowel length} is lost.  \ili{Proto-Siouan} *y manifests as aspirated |\v{c}h|.  So too do cases in which *r is preceded by *i.  Many inalienably owned nouns beginning with |\v{c}h| in \ili{Dakotan} are explained as *r-initial stems preceded by the *i- of \isi{inalienable possession}.  When *r stands alone without an adjacent consonant, it manifests as |y|.  When *k is preceded by a front vowel, it palatalizes to |\v{c}| in \ili{Dakotan}.  Otherwise, the main sound shifts involve clusters.  In particular, \ili{Proto-Siouan} or MVS preaspirates become postaspirates, merging with that series.\footnote{\citealt[199, 269, 818]{Rankinetal2006PDF}.} The \isi{cluster} *rh, which is important in a few words, becomes plain |h|.\footnote{\citealt[165]{Rankinetal2006PDF}.} In \ili{Dakotan}, clusters of two stops are frequent, and the \isi{cluster} *wR becomes *|br|, merging this with the MVS *br series. In all \ili{Dakotan} languages, *br manifests as |mn| before a nasal vowel.  \ili{Stoney} and \ili{Assiniboine} manifest *br and *kr the same regardless of environment, but these sounds alternate in the other three \isi{dialects} according to whether the following vowel is oral or nasal.

\begin{itemize}
\item \isi{vowel length} is lost: \hspace{4.2em} *VV	>	|V|	<	*V

\item *y- and *ir- merge as |\v{c}h-|: \hspace{2em} *y-	>	|\v{c}h-|	<	*ir-

\item *rV becomes |yV|:	 \hspace{5.3em} *rV	>	|yV|

\item *k after front vowel becomes |\v{c}|: \hspace{1em} *ik	>	|i\v{c}|

\hspace{14.2em} *ek	>	|e\v{c}|

\item preaspirates merge with postaspirates:

%begin{center}


\begin{tabular}[t]{c c c c c c }
*hp & > & |ph| & < & *ph \\
*ht	& > & |th| & < & *th \\
*hk & > & |kh| & < & *kh \\
\end{tabular}
%\end{center}

\item *rh becomes |h|: \hspace{4em} *rh	>	|h|	<	*h

\item *wR merges with MVS *br: \hspace{1em}*wR	>	*|br|	<	*br
\item  \ili{Dakotan} *br then manifests as |mn| before a nasal vowel:              *br\k{V}    >          *|mn\k{V}|
\end{itemize}

\subsubsubsection{\ili{Santee}-Sisseton reflexes}

\begin{itemize}
\item  *br alternates by nasality: \hspace{1em} *brV		>	|md| or |bd|\footnote{In the CSD, ``\ili{Sioux} Valley'' seems to agree with ``\ili{Dakota}'' in practically everything except that the one case recorded of a \ili{Sioux} Valley word with a *br reflex before an oral vowel shows this as |bd|, rather than the usual |md| for \ili{Dakota}.}

\hspace{12em} *br\k{V}	>	|mn|

\item  *kr alternates by nasality:  \hspace{1.2em} *krV       >          |hd|

\hspace{12em} *kr\k{V}	>	|hn|

\item *R manifests as |d|: \hspace{4.5em} *R	>	|d|
\end{itemize}

\subsubsubsection{\ili{Yankton-Yanktonai} reflexes}

\begin{itemize}
\item *br alternates by nasality: \hspace{1em} *brV	>	|bd| or |md|

\hspace{12em} *br\k{V} 	>	|mn|
\item  *kr alternates by nasality:  \hspace{1.2em} *krV       >          |kd| or |gd|

\hspace{12em} *kr\k{V} 	>	|kn| or |gn|

\item *R manifests as |d|: \hspace{4.5em} *R	>	|d|\footnote{The CSD records very few words of Yankton, none of which are useful here.  \citet{ParksDeMallie1992} \isi{stress} that the long-repeated claim that \ili{Yankton-Yanktonai} is ``\ili{Nakota}'', is false; their self-designation, when not misled by confused linguists, is ``\ili{Dakota}'', which means that *R manifests as |d|, not |n|, in their dialect.  The true Nakotas are the Assiniboines and the Stoneys.  The authors clearly illustrate the *kr clusters for this group on pages 245-6, but do not include any *br clusters.  Words listed on the Yankton Reservation pedagogical website, http://www.nativeshop.org/learn-dakota.html, show that *br before an oral vowel normaly manifests as |bd|, or perhaps sometimes as |md| or |mbd|.}
\end{itemize}

\subsubsubsection{\ili{Teton} (\ili{Lakota}) reflexes}

\begin{itemize}
\item  *br alternates by nasality:  \hspace{1em} *brV	>	|bl|

\hspace{12em} *br\k{V}	>	|mn|

\item  *kr alternates by nasality: \hspace{1em}   *krV       >          |gl|

\hspace{12em} *kr\k{V}	>	|gn|
\item *R manifests as |l|: \hspace{4.5em} *R	>	|l|
\item *tp becomes |kp|:\footnote{\citealt[253, 265, 865]{Rankinetal2006PDF}.} \hspace{4.5em} *tp	>	|kp|	 <	*kp
 \end{itemize}
 
 \subsubsubsection{\ili{Assiniboine} reflexes}
 
 \begin{itemize}
 \item  *br always manifests as |mn|:     \hspace{1.5em}  *br       >          |mn|\footnote{The CSD contains only a few words of \ili{Assiniboine}, or \ili{Nakota}.  \citet{ParksDeMallie1992} demonstrate that *R becomes |n|, and that *kr manifests as |kn| before both oral and nasal vowels.  The preliminary \ili{Assiniboine} text developed by \citet{Shields2012} contains words with *br clusters showing that these manifest as |mn| regardless of the nasality of the following vowel.}
 \item *kr always manifests as |kn|:       \hspace{1.6em}        *kr       >          |kn|
 \item *R manifests as |n|: \hspace{5.3em} *R	>	|n|
 \end{itemize}
  
 \subsubsubsection{\ili{Stoney} reflexes}
 
 \begin{itemize}
 \item Fricatives tend to shift forward:  \hspace{.7em}   *s         >          |\textipa{T}|

\hspace{13.6em}      *š         >          |s|

\item Free \isi{simple stops} are voiced:  \hspace{1.6em}   *p        >          |b|

\hspace{13.6em}      *t         >          |d|

\hspace{13.6em}       *k        >          |g|

\item *br always manifests as |mn|: \hspace{1.6em}    *br       >          |mn|

\item *kr always manifests as |hn|:  \hspace{1.7em}   *kr       >          |hn|

\item *R manifests as |n|:  \hspace{6em}     *R       >          |n|

 \item *tk becomes |kt|:  \hspace{7em}      *tk       >          |kt|       <          *kt
 \end{itemize}
 
\subsubsection{\ili{Winnebago}-\ili{Chiwere} reflexes}

Hooc\k{a}k and IOM share a number of innovations.  The \isi{cluster} *pt merges with preaspirate *ht.  \ili{Proto-Siouan} simple stop *p before vowels becomes |w|.  Generally, it appears that the postaspirate stop series merges with the simple stop series.  The *rh \isi{cluster} also merges with the simple stop *t.  As in \ili{Dhegiha}, the presumed \isi{cluster} *wR always seems to reduce to simple *|R|.  

Both languages show a sporadic tendency to nasalize vowels that are not nasal in other MVS languages.\footnote{\citealt[50]{Rankinetal2006PDF}.} Both of them also sometimes replace a \isi{glottal} stop with a glottalized |t\textsuperscript{\textipa{P}}| following *i.  This could be interpreted as an epenthetic |y| being naturalized as *|r|, and then converted to |t| before the \isi{glottal} stop.  The problem is that the \isi{glottal} stop itself would seem to be in the way of obtaining the epenthetic |y| in the first place.  Rankin suggests that in verb paradigms, the \isi{glottal} stop is lost in conjugated forms, and that the conjugated form was recast back into the main verb.

\begin{itemize}
\item *pt becomes |ht|: \hspace{7em} *pt	>	*|ht|	<	*ht
\item *rh becomes *|d|: \hspace{7em} *rh	>	*|d|	<	*t
\item *wR merges with MVS *R: \hspace{3em} *wR	>	*|R|	<	*R
\item *p becomes |w| before a vowel: \hspace{1em} *pV	>	*|wV|
\item *i\textsuperscript{\textipa{P}}V verbs become |it\textsuperscript{\textipa{P}}V|:	\hspace{4em} *i\textsuperscript{\textipa{P}}V	>	*|it\textsuperscript{\textipa{P}}V|
\end{itemize}
 
\subsubsubsection{Hooc\k{a}k reflexes}

Hooc\k{a}k shows quite a number of sound shifts of its own.  One of its biggest is that it levels \isi{vowel length} on monosyllables: the vowel of all monosyllabic words is long.\footnote{\citealt[303, 797]{Rankinetal2006PDF}.} Further, it creates many new monosyllabic words by dropping the trailing final vowel, especially *-e.  On top of this, it creates an extra syllable within an obstruent-\isi{sonorant} \isi{cluster}, by inserting the vowel that follows the \isi{cluster} into the spot between the two consonants as well.\footnote{\citet[123--124]{Helmbrecht2011}. This Hooc\k{a}k pattern of back-filling an obstruent-\isi{sonorant} \isi{cluster} with the following vowel is known as `Dorsey's Law'.}  As in \ili{Mandan}, \ili{Crow} and \ili{Hidatsa}, \ili{Proto-Siouan} *y merges with *r.  The *t series, except for glottalized t *t\textsuperscript{\textipa{P}}, is affricated into a |\v{c}| series.  An *r\textsuperscript{\textipa{P}} \isi{cluster} may become either |t\textsuperscript{\textipa{P}}| or |k\textsuperscript{\textipa{P}}|.\footnote{\citealt[816-817]{Rankinetal2006PDF}.}

\begin{itemize}
\item The vowels in monosyllables are always long.
%begin{center}


Long monosyllables: \hspace{1em} *CV(C)    >	*|CVV(C)|     <	*CVV(C)
%\end{center}
\item Trailing final vowels are often dropped, making even more monosyllables.
%begin{center}


Trailing vowels dropped: \hspace{1em} 	*CVCe    >	*|CVVC|        <	*CVVCe
%\end{center}
\item Obstruent plus \isi{sonorant} clusters are broken up by insertion of the following vowel between the obstruent and the \isi{sonorant}.
%begin{center}


Back insertion of vowel:	 \hspace{1em}  *C\textsubscript{obst}C\textsubscript{son}V\textsubscript{1}    >	*|C\textsubscript{obst}V1C\textsubscript{son}V\textsubscript{1}|
%\end{center}
\item As in \ili{Crow}, \ili{Hidatsa}, and \ili{Mandan}, \ili{Proto-Siouan} *y merges with *r.
%begin{center}


*y/*r merger: \hspace{1em} *y	>	|r|	<	*r
%\end{center}
\item *t series affricatizes: \hspace{1em} *t  >  |\v{j}|  

\hspace{9.2em} *ht	 >  |\v{c}|  

\item *r\textsuperscript{\textipa{P}} becomes |t\textsuperscript{\textipa{P}}| or |k\textsuperscript{\textipa{P}}|: \hspace{1em} *r\textsuperscript{\textipa{P}}	>	|t\textsuperscript{\textipa{P}}| or |k\textsuperscript{\textipa{P}}|
\item *R manifests as |d|:\footnote{\citet{HelmbrechtND} and personal communication. This sound is written `t' in the CSD and in modern Wisconsin Hoo\k{a}k \isi{orthography}. But the `t' is voiced in prevocalic and intervocalic position, where it is the reflex of *R.} \hspace{2em} *R	>	|d|
\end{itemize}

\subsubsubsection{IOM reflexes}

A distinctive features of IOM is its forward shifting of the fricatives.  Siouan *s becomes |\textipa{T}|, and *š becomes |s|.\footnote{\citealt[245]{Rankinetal2006PDF}. } In clusters of *k before a \isi{fricative}, the |k| is replaced by a \isi{glottal} stop.\footnote{\citealt[857]{Rankinetal2006PDF}.}  As in \ili{Kaw} and \ili{Osage}, the *t-series, including *t\textsuperscript{\textipa{P}}, is affricatized before a front vowel *i or *e.  Initial *o- regularly becomes |u-|.\footnote{\citealt[893]{Rankinetal2006PDF}.}

One of the most interesting features of IOM is its treatment of the \ili{Proto-Siouan} *y phoneme.  As in several other Siouan languages, \ili{Proto-Siouan} *y merges with another phoneme.  Uniquely to IOM, however, the *y words are split about evenly between which other phoneme they merge with.  Some of them merge with Siouan *r, as in Hooc\k{a}k, \ili{Mandan}, \ili{Crow} and \ili{Hidatsa}.  Others remain |y|, but these are joined by MVS *\v{z}, which itself becomes |y| in IOM.  The fact that many IOM *y fail to merge with *r is mentioned in the CSD, but the significance of the counter-merger of these *y with MVS *\v{z} seems not to have been noticed.  For IOM only, we must consider the *y phoneme to be two distinct phonemes, *y\textsubscript{1} and *y\textsubscript{2}.

\begin{itemize}
\item Fricatives shift forward:
%begin{center}


\begin{tabular}[t]{c c c c c c }
*s & > & |\textipa{T}| \\
*š & > & |s| \\
\end{tabular}
%\end{center}

\item *k before \isi{fricative} becomes |\textsuperscript{\textipa{P}}|: \hspace{1em}*kS > |\textsuperscript{\textipa{P}}S|
\item Initial *o- becomes |u-|: \hspace{4em} *o- 	>	|u-|
\item *R manifests as |d|: \hspace{6em} *R	>	|d|
\item *y\textsubscript{1} merges with *r as |r|: \hspace{ 4em} *y\textsubscript{1}	>	*|r|	<	*r
\item *y\textsubscript{2} merges with *\v{z} as |y|: \hspace{4em} *y\textsubscript{2}	>	|y|	<	\v{z}

\item *t-series affricates before *i/*e: \hspace{1em} *ti  > *|\v{c}i| 

\hspace{14em} *te  >  *|\v{c}e| 

\hspace{14em} *hte  >  *|h\v{c}e|

\hspace{14em} *t\textsuperscript{\textipa{P}}e >  *|\v{c}\textsuperscript{\textipa{P}}e| 

\hspace{14em} etc. 

\end{itemize}

\subsubsection{\ili{Dhegiha} Reflexes}

\ili{Dhegiha} is characterized by substantial shifts and mergers in its vowel structure.  The nasal \ili{Proto-Siouan} vowel *\k{u} merges with *\k{a}, producing a variably pronounced low back vowel with minimal rounding.  The oral vowel *u also shifts forward to become |\"u|.  In \ili{Dhegiha}, Siouan *y merges completely with MVS *\v{z}.  Unlike the other MVS languages, the \isi{preaspirate stops} do not merge with another stop series.  In most \ili{Dhegiha} languages, these manifest as `tense', or double-long unaspirated stops, but in \ili{Osage} they manifest as preaspirates.  \ili{Proto-Siouan} *rh becomes |th|.\footnote{\citealt[165]{Rankinetal2006PDF}.} MVS stop clusters collapse into a single stop, of the preaspirate series.  The clusters *ks and *ps become |s|, and the clusters *kš and pš become |š|.\footnote{\citealt[64, 123, 222, 849]{Rankinetal2006PDF}.} Siouan *xw becomes |ph|.\footnote{\citealt[180]{Rankinetal2006PDF}.} As in \ili{Winnebago}-\ili{Chiwere}, the presumed \isi{cluster} *wR always seems to reduce to simple *|R|.

\begin{itemize}
\item *\k{u} merges with *\k{a}: \hspace{3.1em} *\k{u}	>	|\k{a}|	<	*\k{a}
\item *u becomes *|\"u|:	 \hspace{4.1em} *u	>	|\"u|
\item *y merges with MVS *|\v{z}|: \hspace{1em} *y	>	|\v{z}|	<	*\v{z}
\item *rh merges with *th: \hspace{3em} *rh	>	|th|	<	*th
\item *xw merges with *ph:	\hspace{ 3em} *xw	>	|ph|	<	*ph
\item *ps and *ks merge with *s: \hspace{1em} *ps	>	|s|	<	*s

\hspace{12em} *ks	>	|s|	<	*s
\item *pš and *kš merge with *š: \hspace{1em} *pš	>	|š|	<	*š

\hspace{12em} *kš	>	|š|	<	*š
\item *wR merges with MVS *R: \hspace{1em} *wR	>	*|R|	<	*R
\item Stop clusters merge with preaffricate stops (general pattern):	
%begin{center}


\begin{tabular}[t]{c c c c c c }
*pt & > & *|ht| & < & *ht \\
*pk	& > & *|hk| & < & *hk \\
*tp & > & *|ht| & < & *ht \\
*tk & > & *|ht| & < & *ht \\
*kp	& >	 & *|hp|	& <	& *hp \\
*kt	& >	& *|ht| & < & *ht \\
\end{tabular}
%\end{center} 
\end{itemize} 
 
\subsubsubsection{\ili{Omaha}-\ili{Ponca} reflexes}

\ili{Omaha} and \ili{Ponca} carry the vowel reorganization begun in \ili{Dhegiha} even further.  \ili{Dhegiha} *\"u, from Siouan *u, now loses its rounding and merges completely with Siouan *i.  Behind it, the Siouan *o vowel is raised to |u|.  Siouan *R manifests as |n|, thereby merging with the |n| from Siouan *r before a nasal vowel.  The plain Siouan \isi{glottal} stop disappears, while the glottalized velar clusters *k\textsuperscript{\textipa{P}} and *x\textsuperscript{\textipa{P}} both reduce to |\textsuperscript{\textipa{P}}| as a neo-\isi{glottal} stop.  The preaspirate stop series manifest as tense, while \isi{simple stops} are voiced.  The postaspirate *ph usually, but not always, reduces to |h|.  The Siouan *r phoneme manifests as what I call `ledh', a quick, smooth, flip of the tongue from an apical |l| to edh and off the back of the front teeth.  Linguists generally indicate it with the edh symbol, \textipa{D}, though l and r would be equally reasonable choices.  Additionally, an entire series of new stops is being generated from a custom of affricating the t-series stops as a ``baby talk'' method of suggesting smallness or cuteness.

\begin{itemize}
\item \ili{Dhegiha} *\"u merges with *i: \hspace{1em} *\"u	>	|i|	<	*i
\item *o becomes |u|: \hspace{6em} *o	>	|u|
\item *R manifests as |n|: \hspace{4.2em} *R	>	|n|	<	*n < *r
\item *\textsuperscript{\textipa{P}} disappears: \hspace{6.2em} *V\textsuperscript{\textipa{P}}V	>	|VV|
\item *k\textsuperscript{\textipa{P}} and x\textsuperscript{\textipa{P}} become |\textsuperscript{\textipa{P}}|:	\hspace{3.2em} *k\textsuperscript{\textipa{P}}	>	|\textsuperscript{\textipa{P}}|	<	*x\textsuperscript{\textipa{P}}
\item *ph usually becomes |h|: \hspace{2.2em} *ph	>	|h|

\item Free \isi{simple stops} are voiced:	\hspace{1em}*p	>	|b|

\hspace{13em} *t	>	|d|

\hspace{13em}*k	>	|g|

\item Preaspirate stops are tense: \hspace{1em} *hp	>	|pp|

\hspace{12.5em} *ht	>	|tt|
					
\hspace{12.5em} *hk	>	|kk|

\item Diminutive t-series transform:	\hspace{1.1em} |d|	dim.>	|\v{j}|

\hspace{14em} |t|	dim. >	|\v{c}|
					
\hspace{14em} |tt|	dim. >	|\v{c}\v{c}|

\hspace{14em} |th|	dim. >	|\v{c}h|

\hspace{14em} |t\textsuperscript{\textipa{P}}|	dim.>	|\v{c}\textsuperscript{\textipa{P}}|
\end{itemize}

\subsubsubsection{\ili{Kaw}-\ili{Osage} reflexes}

\ili{Kaw} and \ili{Osage} share a characteristic of dropping the velar stop from the *kr \isi{cluster} and replacing the \isi{cluster} with |l|.  It seems that both of them also merge the glottalized fricatives *s\textsuperscript{\textipa{P}} and *š\textsuperscript{\textipa{P}} into a glottalized dental/alveolar affricate |c\textsuperscript{\textipa{P}}| (|ts\textsuperscript{\textipa{P}}|).\footnote{\citealt[856]{Rankinetal2006PDF}.} As in IOM, the *t-series, including *t\textsuperscript{\textipa{P}}, is affricatized before a front vowel *i or *e.  

\begin{itemize}
\item *kr drops the velar stop: \hspace{3em} *kr	>	|l|
\item *s\textsuperscript{\textipa{P}} and *š\textsuperscript{\textipa{P}} merge as |c\textsuperscript{\textipa{P}}|: \hspace{3em} *s\textsuperscript{\textipa{P}}	>	|c\textsuperscript{\textipa{P}}|	<	*š\textsuperscript{\textipa{P}}

\item *t-series affricates before *i/*e: \hspace{1em} *ti	>	*|\v{c}i|

\hspace{14em} *te	>	*|\v{c}e|

\hspace{14em} *hte	>	*|h\v{c}e|

\hspace{14em} *t\textsuperscript{\textipa{P}}e	>	*|\v{c}\textsuperscript{\textipa{P}}e|

\hspace{14em} etc.
\end{itemize}

4.3.3.2.1  \ili{Kaw} reflexes
\vspace{1em}

\ili{Kaw} agrees with \ili{Omaha} and \ili{Ponca} in voicing the free \isi{simple stops} and in pronouncing the aspirated stops as tense.  In \ili{Kaw}, Siouan free *r manifests as |y|.

\begin{itemize}
\item Free *r manifests as |y|: \hspace{1em} *r	>	|y|
\item *R manifests as |d|: \hspace{3em} *R	>	|d|
\end{itemize}

4.3.3.2.2  \ili{Osage} reflexes
\vspace{1em}

In \ili{Osage}, the preaspirate series is pronounced with preaspiration, and the free \isi{simple stops} are voiceless.  Siouan free *r manifests as edh or ledh (\textipa{D}).  *ph manifests as |pš|.\footnote{ \citealt[64]{Rankinetal2006PDF}.} 

\begin{itemize}
\item Free *r manifests as |\textipa{D}|: \hspace{1em} *r	>	|\textipa{D}|
\item *R manifests as |t|: \hspace{3em} *R	>	|t|
\item *ph manifests as |pš|: \hspace{2em} *ph	>	|pš|
\end{itemize}

\subsubsubsection{\ili{Quapaw} reflexes}

In \ili{Quapaw}, free Siouan *r manifests as |d|.  It seems that \isi{simple stops} sometimes become tense.\footnote{\citealt[833]{Rankinetal2006PDF}.} The Siouan \isi{cluster} *p\textsuperscript{\textipa{P}} is reduced to plain \isi{glottal} stop.\footnote{\citealt[831]{Rankinetal2006PDF}.} 

\begin{itemize}
\item Free *r manifests as |d|: \hspace{5em} *r	>	|d|
\item Simple stops may become tense:	\hspace{1em} *t	>	|tt|
\item *p\textsuperscript{\textipa{P}} becomes |\textsuperscript{\textipa{P}}|: \hspace{8em} *p\textsuperscript{\textipa{P}}	>	|\textsuperscript{\textipa{P}}|
\end{itemize}

\subsection{\ili{Southeastern Siouan} reflexes}

Very few systematic sound shifts characterize \ili{Southeastern Siouan} as a whole.  One mentioned in the CSD is the loss of glottalized fricatives.  Also, it seems that *š usually affricatizes to |\v{c}|.

\begin{itemize}
\item Fricatives lose glottalization and merge with the corresponding plain form.  Thus, *S\textsuperscript{\textipa{P}} > *|S|.\footnote{\citealt[856]{Rankinetal2006PDF}.}

%begin{center}


\begin{tabular}[t]{c c c c c c c c c}
Fricatives deglottalize: & & *s\textsuperscript{\textipa{P}}	& >	 & *|s| & < & *s \\
& & *š\textsuperscript{\textipa{P}}	& > & *|š| & < & *š \\
& & *x\textsuperscript{\textipa{P}}	 & > & *|x| & < & *x \\
\end{tabular}
%\end{center}

\item *š then usually becomes |\v{c}|:\footnote{\citealt[99, 126, 167, 827, 931]{Rankinetal2006PDF}.} \hspace{1em} *š	>	*|\v{c}|	
\end{itemize}

\subsubsection{\ili{Tutelo} reflexes}

\ili{Tutelo} seems conservative.  The only significant change noted involves the \ili{Proto-Siouan} *š and *s phonemes.

\begin{itemize}
\item *š normally becomes |\v{c}|: \hspace{4.2em} *š	>	|\v{c}|
\item Sometimes, *š becomes |s|:\footnote{\citealt[912]{Rankinetal2006PDF}.} \hspace{2.9em} *š	>	|s|	<	*s
\item *s is indifferently pronounced:\footnote{\citealt[54, 931]{Rankinetal2006PDF}.} \hspace{1.2em} *s	>	|s| or |š|
\end{itemize}

\subsubsection{\ili{Ofo}-\ili{Biloxi} reflexes}

In \ili{Ofo} and \ili{Biloxi}, initial \ili{Proto-Siouan} *w or *h before a vowel is lost.\footnote{\citealt[7, 223, 817, 929]{Rankinetal2006PDF}.} 

\begin{itemize}
\item *wV becomes plain |V|: \hspace{1em} *wV	>	|V|
\item *hV becomes plain |V|: \hspace{1em} *hV	>	|V|
\end{itemize}

\subsubsubsection{\ili{Biloxi} reflexes}

\ili{Biloxi} is fairly conservative.  Final *-i and *-e merge as |-i|,\footnote{\citealt[901]{Rankinetal2006PDF}.}  and the \isi{glottal} stop often appears as |h|.\footnote{\citealt[103]{Rankinetal2006PDF}.} 

\begin{itemize}
\item Final *-e merges with *-i: \hspace{2.1em} *-e	>	|-i|	<	*-i
\item The \isi{glottal} stop becomes |h|: \hspace{1em} *\textsuperscript{\textipa{P}}	>	|h|
\end{itemize}

\subsubsubsection{\ili{Ofo} reflexes}

\ili{Ofo} is much more innovative.  \ili{Proto-Siouan} *y becomes aspirated |\v{c}h|,\footnote{\citealt[85, 242]{Rankinetal2006PDF}.} as in \ili{Dakotan}.  The CSD suggests that \ili{Proto-Siouan} *š before an accented syllable may have become aspirated |\v{c}h| as well.\footnote{\citealt[827]{Rankinetal2006PDF}.} Notably, the *s \isi{fricative} changes to |f|, while \ili{Proto-Siouan} *x shifts forward to become a neo-|š|.\footnote{\citealt[174, 299]{Rankinetal2006PDF}.} Several of the \ili{Proto-Siouan} clusters do interesting things as well.  In the case of a glottalized stop consonant, the \isi{glottal} stop seems to shift forward so that it releases prior to the stop.  This phenomenon is suggested in \ili{Ofo} transcriptions as a neutral vowel appearing epenthetically in front of the stop that in other languages is known to be glottalized.  The stop consonant is then aspirated as well.

\begin{itemize}
\item *y becomes |\v{c}h|: \hspace{5em} *y	>	|\v{c}h|
\item Accented *|š| becomes |\v{c}h|: \hspace{1em} *š\'V	>	*|\v{c}\'V|	>	|\v{c}h\'V|
\item *s becomes |f|: \hspace{6em} *s	>	|f|
\item *x becomes |š|: \hspace{6em} 	*x	>	|š|
\item *hs becomes |fh|:\footnote{\citealt[174, 299]{Rankinetal2006PDF}.}  \hspace{4.8em} 			*hs	>	|fh|
\item *Cr becomes |l|:\footnote{\citealt[90]{Rankinetal2006PDF}.}  \hspace{5em} 			*Cr	>	|l|
\item *C\textsuperscript{\textipa{P}} becomes |\textipa{@}Ch|:\footnote{\citealt[229, 232]{Rankinetal2006PDF}.}  \hspace{3.9em} *C\textsuperscript{\textipa{P}}	>	|\textipa{@}Ch|
\end{itemize}

\section*{Abbreviations}

CSD = \isi{Comparative Siouan Dictionary} 2006; IOM = \ili{Iowa}-\ili{Otoe}-Missouria; MVS = \ili{Mississippi Valley Siouan}.

 

\printbibliography[heading=subbibliography,notkeyword=this]
 
\end{document}