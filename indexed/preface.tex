\addchap{Preface}
\begin{refsection}

This volume presents a group of papers representing a range of current work on Siouan\footnote{``Siouan'' is not to be confused with ``\ili{Sioux}'', a controversial term referring to \ili{Lakota} and \ili{Dakota} people, rarely to \il{Assiniboine}\il{Stoney}Nakota/Nakoda people too, but never correctly to people of other traditionally Siouan-language-speaking communities.} languages, in memory of our colleague Robert L. Rankin\ia{Rankin, Robert L.}, a towering figure in Siouan linguistics throughout his long career, who passed away in February of 2014.


Beyond honoring a beloved colleague, our aim in this volume is to bring a variety of issues in Siouan linguistics to the attention of the linguistic community. The Siouan language family is a large and important one, with branches geographically distributed over a broad swath of the North American plains and parts of the southeastern United States. This puts it in contact historically with several other families of languages: \ili{Algonquian}, \ili{Iroquoian}, \ili{Caddoan}, \ili{Uto-Aztecan}, and \ili{Muskogean}. Siouan languages are, or were historically, spoken by the members of at least 25 ethnic/political groups. One Siouan language, \ili{Lakota}, is among the handful of indigenous North American languages with younger speakers today. Siouan languages have occasionally risen to prominence in general linguistics, for instance in the study of reduplication \citep{Shaw1980}; and \ili{Omaha} and Crow (\ili{Apsaalooke}) have lent their names to two of the basic categories of kinship systems in anthropology. Nonetheless, the Siouan family has been underrepresented in the descriptive and typological literature, and most of the languages in the family are severely understudied. The majority of work on Siouan languages is unpublished, existing only in the form of conference papers or manuscripts.\footnote{Many of these unpublished works are collected in the electronic Siouan Archive, maintained by John Boyle\ia{Boyle, John P.} at the University of California at Riverside.} This volume is a step toward making information on Siouan languages more broadly available and encouraging deeper investigation of the myriad issues they raise.


From the perspective of linguistic \isi{typology}, Siouan languages have many notable features. Many of these features stand to challenge typological generalizations. Here we briefly sketch a few of the most characteristic features of the Siouan family.


All Siouan languages possess a rich variety of applicative affixes, confirming \citegen{Polinsky2013} observation that applicatives are common in North America and adding another language family to her list of applicative-rich families in the area. \citet{Helmbrecht2006a} divides the applicatives into three templatic slots: locative applicatives, benefactive applicatives, and applicative markers; all of the Siouan languages sampled by Helmbrecht possess at least two applicative morphemes.


All Siouan languages are strongly head-final, and the consensus among syntacticians working with Siouan languages is that all but the supraclausal projections (and even some of these) are underlyingly head-final in Siouan languages, contra \citegen{Kayne1994} Antisymmetry theory.


All Siouan languages have head-internal relative clauses\is{clauses, internally headed relative}. A series of strong claims regarding the typological implications of head-internal relative clauses (cf. \citealt{Cole1987, Murasugi2000}), including purported distinctions between ``\ili{Japanese}-type'' and ``\ili{Lakota}-type'' constructions (cf. \citealt{Watanabe2004,Williamson1987,Bonneau1992}), propelled \ili{Lakota} into the debates of theoretical \isi{syntax}. It has been pointed out that head-internal relative clauses of the kind found in \ili{Lakota} and other Siouan languages lack the island restrictions found in other languages. On the other hand, \citet{Murasugi2000} argues that languages with head-internal relative \isi{clauses} must also have head-external relative clauses\is{clauses, relative}, which is not true in Siouan languages.


All Siouan languages have verbal affixes which index \isi{subject} possession of or relationship with the \isi{object}. They vary with respect to contexts of obligatoriness of these affixes.


Many Siouan languages have grammaticalized systems of speaker-\isi{gender} marking, with gender-specific morphology for speech-act markers, address terms, and kinship lexemes.\footnote{In the case of kinship terms, lexical choice is driven by the \isi{gender} of the ``ego" deictic center, which coincides with speaker \isi{gender} when there is 1st-person inflection.} Such usage varies depending on stuational factors, however, especially in the case of speech-act markers; see for instance \citet{Trechter1995}.   


Many Siouan languages have a modal CCV morpheme shape. This does not necessarily imply a preference for CCV phonetic realizations, but may indicate such a preference in the distant past. Another unusual prosodic feature is the preference for second-syllable \isi{stress} in most Siouan languages. \ili{Ho-Chunk} may be the only attested language with default third-syllable \isi{stress} in the world.


Most Siouan languages have \isi{ejective} stops. The \ili{Dhegiha} branch is notable for a four-way glottal-state distinction in its stop series (voiced/lenis, tense/pre-aspirated, \isi{ejective} and aspirated). Outside of the \ili{Dhegiha} branch are many Siouan languages which have the unusual feature of a phonemic voicing distinction in \isi{fricative}s but not in stops.


Verbs play some typologically unusual, prominent roles in Siouan languages. Diachronically, many grammatical items which rarely grammaticalize from verbs in other languages tend to derive from verbs in Siouan languages. For instance, \citet{Rankin1977} documents the derivation of classifiers and articles from verbs. In some Siouan languages, the source verbs and target grammatical items continue to exist in parallel with substantial semantic overlap. The \ili{Omaha} positional article \textit{tʰoⁿ} `\textsc{obviative animate specific standing}\is{animacy}', for instance, is homophonous with the root of \textit{\'atʰoⁿ} `stand on'.


This diachronic tendency is mirrored by synchronic flexibility. Siouan languages tend to verb freely --- to use nearly any open-class stem as a verb. Thus \ili{Lakota} \textit{wima\v{c}ha\v{s}a} `I am a man' is derived from the \is{nominals}nominal stem \textit{wi\v{c}ha\v{s}a} `man/ person' with the 1st-person stative pronominal ma-.

\ili{Dhegiha} articles (which have many features in common with positional classifiers in e.g. \ili{Mayan} languages; see \citealt{Gordon2009}) are homophonous with postverbal and postclausal functional items like subordinating\is{subordination} \isi{conjunction}s and aspect and evidentiality markers. They have considerable semantic overlap with them too, a fact which comprises another area of blurriness between nominal and verbal \isi{syntax}: In \ili{Ponca}, \textit{ni\'ashiⁿga-ama} may mean `the [\textsc{proximate animate plural specific}\is{animacy}] people', but also may mean either `they are people' or `I am told s/he was a person'. Plurality is a part of the semantics of -ama in both the nominal and the first clausal interpretation. To make matters more interesting, these kinds of ambiguity are not always easily resolved by context alone, and may suggest a ``simultaneity'' (cf. \citealt{Woolard1998}) at work as part of speakers' competence. 


This flexibility, that is, the ability of one and the same root to function in both nominal and verbal contexts, has led to some discussion on the status and quality of the noun/verb distinction in Siouan languages (see e.g. \citealt{Helmbrecht2002}  and \citet{Ingham2001}). 

Nominal arguments in general are not required in Siouan languages, thematic relations being signaled by pronominal or agreement markers within the verb --- including zero markers. This makes Siouan languages relevant to debates about the existence of ``pronominal argument'' languages \citep{Jelinek1984} and to the related issues of whether there are languages with truly nonconfigurational\is{configurationality} or \isi{flat structure}. The preponderance of evidence in Siouan is for the existence of hierarchical structure, specifically including a VP\is{verb phrase} (for instance, \citealt{West2003,Johnson2016, JohnsonEtAl2016, Rosen2016}).


Although Siouan languages have many remarkable features in common, they vary on many others.  Some Siouan languages have \isi{noun incorporation}, while others do not. Some Siouan languages have stress-accent systems, and others have pitch-accent systems. \ili{Dhegiha} languages are notable in having as many as eleven definite/specific articles indexing features such as animacy\is{animacy}, proximacy/obviation (or case), posture/position, number, visibility, motion and dispersion; meanwhile other Siouan languages have no fully grammaticalized articles at all.


Some Siouan languages reflect longtime cultural presence on the Plains, while others are located as far east as the Atlantic Coast, and many more show cultural aspects of both regions. \ili{Dhegiha}-speaking peoples (\ili{Quapaw}, \ili{Osage}, \ili{Kanza}, \ili{Omaha} and \ili{Ponca}, and likely \ili{Michigamea} as well \citep{Kasak2016,Koontz1995a} likely lived at the metropolis at Cahokia, perhaps at a time before any of the descendant groups had separated, and have many Eastern Woodlands-style features of traditional governance and religion, in sharp contrast with the more Plains-typical cultural features of close \ili{Lakota} and \ili{Dakota} neighbors and relatives.

One seemingly minor but in fact quite significant issue in Siouan linguistics is the matter of language names and their spelling. Often this involves a self-designation in competition with a name imposed by outsiders. Even when an autonym gains currency among linguists there is sometimes no agreed spelling; so for instance the \ili{Otoe} self-designation is written <Jiwere> or <Chiwere>. For the most part in this volume the choice of language designations has been left to the individual chapter authors. However, after a volume reviewer pointed out that the language of the \ili{Ho-Chunk} or ``Winnebago'' people was spelled no less than ten different ways in various chapters, we encouraged authors to choose one of the two spellings used on the tribe's web site:  <\ili{Ho-Chunk}> or <Hooc\k{a}k>.  Most have voluntarily complied.  In a related move, we decided to retranscribe all \ili{Lakota} data throughout the volume using the \isi{orthography} of the \textit{New \ili{Lakota} Dictionary} \citep{Ullrich2012}.

The volume is divided into four broad areas (Historical, Applied, Formal/Analy\-tical, and Comparative/Cross-Siouan) described in more detail in separate introductions to each part of the volume. Part I consists of five chapters on historical themes: Ryan Kasak\ia{Kasak, Ryan M.} evaluates the evidence for a relationship between \ili{Yuchi} and Siouan; David Kaufman\ia{Kaufman, David V.} discusses the participation of some Siouan languages in a Southeastern \isi{sprachbund}; Rory Larson\ia{Larson, Rory} summarizes current knowledge of Siouan sound changes; and Kathleen Danker\ia{Danker, Kathleen} and Anthony Grant\ia{Grant, Anthony} investigate early attempts to write \ili{Ho-Chunk}, \ili{Kanza}, and \ili{Osage}.  Part II opens with Linda Cumberland\ia{Cumberland, Linda A.}'s interview with Robert Rankin\ia{Rankin, Robert L.} about his work with \ili{Kanza} language programs. Jimm Goodtracks\ia{Goodtracks, Jimm G.}, Saul Schwartz\ia{Schwartz, Saul}, and Bryan Gordon\ia{Gordon, Bryan James} present three different perspectives on \ili{Baxoje-Jiwere} language retention. Justin McBride\ia{McBride, Justin T.} applies formal \isi{syntax} to the solution of a pedagogical problem in teaching \ili{Kanza}. This applied-linguistics section ends with Jill Greer\ia{Greer, Jill D.}'s sketch grammar of \ili{Baxoje-Jiwere}. Part III contains formal analyses of individual Siouan languages. David Rood\ia{Rood, David S.} proposes an analysis of /b/ and /g/ in \ili{Lakota} using the tools of \isi{autosegmental phonology} and \isi{feature geometry}. John Boyle\ia{Boyle, John P.} elucidates the structure of relative \isi{clauses} in \ili{Hidatsa}. Meredith Johnson\ia{Johnson, Meredith}, Bryan Rosen\ia{Rosen, Bryan}, and Mateja Schuck\ia{Schuck, Mateja}, in a series of three interrelated chapters, discuss syntactic constructions in \ili{Ho-Chunk} including resultatives and \is{verb phrase}VP \isi{ellipsis}, which they argue show the language has \isi{VP} and an \isi{adjective} category. Part IV consists of three chapters which take a broader view of grammar, considering data from across the Siouan family. Catherine Rudin\ia{Rudin, Catherine} compares \isi{coordination} constructions across Siouan; Bryan Gordon\ia{Gordon, Bryan James} does the same with \isi{information structure} and \isi{intonation}, and Johannes Helmbrecht\ia{Helmbrecht, Johannes} with nominal possession constructions. 

All four of the areas represented by this volume are ones to which Bob Rankin\ia{Rankin, Robert L.} contributed. His scholarly publications centered primarily around Siouan historical \isi{phonology}, but included works ranging from dictionaries to toponym studies, from philological investigation of early Siouanists  to description of grammaticalization pathways. He was deeply involved in language retention efforts with the \ili{Kanza} Language Project. Other interests included archeology, linguistic \isi{typology}, \ili{Iroquoian} and \ili{Muskogean} languages, and the history of linguistics. 

Bob\ia{Rankin, Robert L.} was a major figure in Siouan linguistics, a mentor to nearly all living Siouanists, and a mainstay of the annual Siouan and \ili{Caddoan} Linguistics Conference meetings for decades. Trained in Romance and \ili{Indo-European} linguistics, with a specialty in \ili{Romanian} (Ph.D. University of Chicago 1972), he shifted gears soon after leaving graduate school, and became an expert in Siouan languages, especially the \ili{Dhegiha} branch, with special focus on \ili{Kanza}. From the mid 1970s through the end of his life, he devoted himself to Siouan studies, both practical and scholarly. His long association with the Kaw Nation led to a grammar and \isi{dictionary} of that language (see \citealt{Cumberland2016}), and he also produced a grammar of \ili{Quapaw}, and briefly conducted \isi{field work} on \ili{Omaha}-\ili{Ponca} and \ili{Osage}. At the University of Kansas he directed dissertations on \ili{Lakota} \citep{Trechter1995} and \ili{Tutelo} \citep{Oliverio1996} as well as several M.A. theses on Siouan languages, and taught a wide variety of courses including field methods and structure of \ili{Lakota} and \ili{Kanza} as well as more theoretical courses in \isi{phonology}, \isi{syntax}, and historical linguistics. Perhaps Bob's greatest gift to the field was his encouragement of others. At conferences and on the \isi{Siouan List} email forum, he was unfailingly patient and encouraging, answering all questions seriously, explaining linguistic terms to non-linguist participants and basic facts of Siouan languages to general linguists with equal enthusiasm and lack of condescension.

Following his untimely passing, a special session was held at the 2014 Siouan and Caddoan Linguistics Conference to organize several projects in Bob's honor: The first of these was publication of the \isi{Comparative Siouan Dictionary}, an immense project comparing \isi{cognates} across all the Siouan languages, undertaken by Rankin and a group of colleagues in the 1980s. It had been circulated in various manuscript forms but never published. Thanks to David Rood\ia{Rood, David S.} (another founding member of the CSD project), with help from Iren Hartmann\ia{Hartmann, Iren}, the CSD is now available online \citep{RankinEtAl2015AccessSeptember}. The second project was a volume of Bob's conference papers and other previously unpublished or less accessible work, to be collected and edited by a group headed by John P. Boyle\ia{Boyle, John P.} and David Rood; that volume, tentatively titled \textit{Siouan Studies: Selected Papers by Robert L. Rankin}, is currently in progress. The third project was a volume of Siouan linguistic work in Bob's memory, which has taken the shape of the present volume.

\printbibliography[heading=subbibliography,notkeyword=this]

%\end{document}

%\printbibliography[heading=subbibliography]
\end{refsection}

