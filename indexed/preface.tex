\addchap{Preface}
\begin{refsection}

This volume presents a group of papers representing a range of current work on \ili{Siouan}\footnote{``\ili{Siouan}'' is not to be confused with ``\ili{Sioux}'', a controversial term referring to \ili{Lakota} and \ili{Dakota} people, rarely to \ili{Nakota}/\ili{Nakoda} people too, but never correctly to people of other traditionally \ili{Siouan}-language-speaking communities.} languages, in memory of our colleague Robert L. Rankin, a towering figure in \ili{Siouan} linguistics throughout his long career, who passed away in February of 2014.


Beyond honoring a beloved colleague, our aim in this volume is to bring a variety of issues in \ili{Siouan} linguistics to the attention of the linguistic community. The \ili{Siouan} language family is a large and important one, with branches geographically distributed over a broad swath of the North American plains and parts of the southeastern United States. This puts it in contact historically with several other families of languages: \ili{Algonquian}, \ili{Iroquoian}, \ili{Caddoan}, \ili{Uto-Aztecan}, and \ili{Muskogean}. \ili{Siouan} languages are, or were historically, spoken by the members of at least 25 ethnic/political groups. One \ili{Siouan} language, \ili{Lakota}, is among the handful of indigenous North American languages with younger speakers today. \ili{Siouan} languages have occasionally risen to prominence in general linguistics, for instance in the study of reduplication (Shaw 1980); and \ili{Omaha} and \ili{Crow} (\ili{Apsaalooke}) have lent their names to two of the basic categories of kinship systems in anthropology. Nonetheless, the \ili{Siouan} family has been underrepresented in the descriptive and typological literature, and most of the languages in the family are severely understudied. The majority of work on \ili{Siouan} languages is unpublished, existing only in the form of conference papers or manuscripts.\footnote{Many of these unpublished works are collected in the electronic \ili{Siouan} Archive, maintained by John Boyle at the University of California at Riverside.} This volume is a step toward making information on \ili{Siouan} languages more broadly available and encouraging deeper investigation of the myriad issues they raise.


From the perspective of linguistic \isi{typology}, \ili{Siouan} languages have many notable features. Many of these features stand to challenge typological generalizations. Here we briefly sketch a few of the most characteristic features of the \ili{Siouan} family.


All \ili{Siouan} languages possess a rich variety of applicative affixes, confirming Polinsky's (2013) observation that applicatives are common in North America and adding another language family to her list of applicative-rich families in the area. \citet{Helmbrecht2006} divides the applicatives into three templatic slots: locative applicatives, benefactive applicatives, and applicative markers; all of the \ili{Siouan} languages sampled by Helmbrecht possess at least two applicative morphemes.


All \ili{Siouan} languages are strongly head-final, and the consensus among syntacticians working with \ili{Siouan} languages is that all but the supraclausal projections (and even some of these) are underlyingly head-final in \ili{Siouan} languages, contra Kayne's (1994) Antisymmetry theory.


All \ili{Siouan} languages have head-internal relative \isi{clauses}. A series of strong claims regarding the typological implications of head-internal relative \isi{clauses} (cf. Cole 1987; Murasugi 2000), including purported distinctions between ``\ili{Japanese}-type'' and ``\ili{Lakota}-type'' constructions (cf. Watanabe 2004; Williamson 1987; Bonneau 1992), propelled \ili{Lakota} into the debates of theoretical \isi{syntax}. It has been pointed out that head-internal relative \isi{clauses} of the kind found in \ili{Lakota} and other \ili{Siouan} languages lack the island restrictions found in other languages. On the other hand, \citet{Murasugi2000} argues that languages with head-internal relative \isi{clauses} must also have head-external relative \isi{clauses}, which is not true in \ili{Siouan} languages.


All \ili{Siouan} languages have \isi{verbal affixes} which index \isi{subject} possession of or relationship with the \isi{object}. They vary with respect to contexts of obligatoriness of these affixes.


Many \ili{Siouan} languages have grammaticalized systems of speaker-\isi{gender} marking, with \isi{gender}-specific morphology for speech-act markers, address terms, and kinship lexemes.\footnote{In the case of kinship terms, lexical choice is driven by the \isi{gender} of the ``ego" deictic center, which coincides with speaker \isi{gender} when there is 1st-person inflection.} Such usage varies depending on stuational factors, however, especially in the case of speech-act markers; see for instance \citet{Trechter1995}.   


Many \ili{Siouan} languages have a modal CCV morpheme shape. This does not necessarily imply a preference for CCV phonetic realizations, but may indicate such a preference in the distant past. Another unusual prosodic feature is the preference for second-syllable \isi{stress} in most \ili{Siouan} languages. Hooc\k{a}k may be the only attested language with default third-syllable \isi{stress} in the world.


Most \ili{Siouan} languages have \isi{ejective} stops. The \ili{Dhegiha} branch is notable for a four-way \isi{glottal}-state distinction in its stop series (voiced/lenis, tense/pre-aspirated, \isi{ejective} and aspirated). Outside of the \ili{Dhegiha} branch are many \ili{Siouan} languages which have the unusual feature of a phonemic voicing distinction in fricatives but not in stops.


Verbs play some typologically unusual, prominent roles in \ili{Siouan} languages. Diachronically, many grammatical items which rarely grammaticalize from verbs in other languages tend to derive from verbs in \ili{Siouan} languages. For instance, \citet{Rankin1977} documents the derivation of classifiers and articles from verbs. In some \ili{Siouan} languages, the source verbs and target grammatical items continue to exist in parallel with substantial semantic overlap. The \ili{Omaha} positional article \textit{tʰoⁿ} `obviative animate specific standing', for instance, is homophonous with the root of \textit{\'atʰoⁿ} `stand on'.


This diachronic tendency is mirrored by synchronic flexibility. \ili{Siouan} languages tend to verb freely --- to use nearly any open-class stem as a verb. Thus \ili{Lakota} \textit{wima\v{c}ha\v{s}a} `I am a man' is derived from the nominal stem \textit{wi\v{c}ha\v{s}a} `man/ person' with the 1st-person stative pronominal ma-.

\ili{Dhegiha} articles (which have many features in common with positional classifiers in e.g. \ili{Mayan} languages; see Gordon, 2009) are homophonous with postverbal and postclausal functional items like subordinating conjunctions and aspect and evidentiality markers. They have considerable semantic overlap with them too, a fact which comprises another area of blurriness between nominal and verbal \isi{syntax}: In \ili{Ponca}, \textit{ni\'ashiⁿga-ama} may mean `the [proximate animate plural specific] people', but also may mean either `they are people' or `I am told s/he was a person'. Plurality is a part of the semantics of -ama in both the nominal and the first clausal interpretation. To make matters more interesting, these kinds of ambiguity are not always easily resolved by context alone, and may suggest a ``simultaneity'' (cf. Woolard 1998) at work as part of speakers' competence. 


This flexibility, that is, the ability of one and the same root to function in both nominal and verbal contexts, has led to some discussion on the status and quality of the noun/verb distinction in \ili{Siouan} languages (see e.g. Helmbrecht, 2002, and Ingham, 2001). 

Nominal arguments in general are not required in \ili{Siouan} languages, thematic relations being signaled by pronominal or agreement markers within the verb --- including zero markers. This makes \ili{Siouan} languages relevant to debates about the existence of ``pronominal argument'' languages (Jelinek 1984) and to the related issues of whether there are languages with truly nonconfigurational or \isi{flat structure}. The preponderance of evidence in \ili{Siouan} is for the existence of hierarchical structure, specifically including a \isi{VP} (for instance, West, 1998; Johnson, this volume; Johnson et al, this volume; and Rosen, this volume).


Although \ili{Siouan} languages have many remarkable features in common, they vary on many others.  Some \ili{Siouan} languages have \isi{noun incorporation}, while others do not. Some \ili{Siouan} languages have \isi{stress}-accent systems, and others have pitch-accent systems. \ili{Dhegiha} languages are notable in having as many as eleven definite/specific articles indexing features such as animacy, proximacy/ obviation (or case), posture/position, number, visibility, motion and dispersion; meanwhile other \ili{Siouan} languages have no fully grammaticalized articles at all.


Some \ili{Siouan} languages reflect longtime cultural presence on the Plains, while others are located as far east as the Atlantic Coast, and many more show cultural aspects of both regions. \ili{Dhegiha}-speaking peoples (\ili{Quapaw}, \ili{Osage}, \ili{Kaw}, \ili{Omaha} and \ili{Ponca}, and likely \ili{Michigamea} as well (Kasak (this volume), Koontz 1995) likely lived at the metropolis at Cahokia, perhaps at a time before any of the descendant groups had separated, and have many Eastern Woodlands-style features of traditional governance and religion, in sharp contrast with the more Plains-typical cultural features of close \ili{Lakota} and \ili{Dakota} neighbors and relatives.

One seemingly minor but in fact quite significant issue in \ili{Siouan} linguistics is the matter of language names and their spelling. Often this involves a self-designation in competition with a name imposed by outsiders. Even when an autonym gains currency among linguists there is sometimes no agreed spelling; so for instance the \ili{Otoe} self-designation is written \ili{Jiwere} or \ili{Chiwere}. For the most part in this volume the choice of language designations has been left to the individual chapter authors. However, after a volume reviewer pointed out that the language of the \ili{Ho-Chunk} or \ili{Winnebago} people was spelled no less than ten different ways in various chapters, we encouraged authors to choose one of the two spellings used on the tribe's web site:  \ili{Ho-Chunk} or Hooc\k{a}k.  Most have voluntarily complied.  In a related move, we decided to retranscribe all \ili{Lakota} data throughout the volume using the now-standard \isi{orthography} of the \textit{New \ili{Lakota} Dictionary} (Ullrich et al, 2008).

The volume is divided into four broad areas (Historical, Applied, Formal/Analy-tical, and Comparative/Cross-\ili{Siouan}) described in more detail in separate introductions to each part of the volume. Part I consists of five chapters on historical themes: Ryan Kasak evaluates the evidence for a relationship between \ili{Yuchi} and \ili{Siouan}; David Kaufman discusses the participation of some \ili{Siouan} languages in a Southeastern \isi{sprachbund}; Rory Larson summarizes current knowledge of \ili{Siouan} sound changes; and Kathleen Danker and Anthony Grant investigate early attempts to write Hooc\k{a}k, \ili{Kanza}, and \ili{Osage}.  Part II opens with Linda Cumberland's interview with Robert Rankin about his work with \ili{Kaw} language programs. Jimm Goodtracks, Saul Schwartz, and Bryan Gordon present three different perspectives on \ili{Baxoje-Jiwere} language retention. Justin McBride applies formal \isi{syntax} to the solution of a pedagogical problem in teaching \ili{Kaw}. This applied-linguistics section ends with Jill Greer's sketch \isi{grammar} of \ili{Baxoje-Jiwere}. Part III contains formal analyses of individual \ili{Siouan} languages. David Rood proposes an analysis of /b/ and /g/ in \ili{Lakota} using the tools of \isi{autosegmental phonology} and \isi{feature geometry}. John Boyle elucidates the structure of relative \isi{clauses} in \ili{Hidatsa}. Meredith Johnson, Bryan Rosen, and Mateja Schuck, in a series of three interrelated chapters, discuss syntactic constructions in Hooc\k{a}k including resultatives and \isi{VP} \isi{ellipsis}, which they argue show the language has \isi{VP} and an \isi{adjective} category. Part IV consists of three chapters which take a broader view of \isi{grammar}, considering data from across the \ili{Siouan} family.  Catherine Rudin compares \isi{coordination} constructions across \ili{Siouan}; Bryan Gordon does the same with \isi{information structure} and \isi{intonation}, and Johannes Helmbrecht with nominal possession constructions. 

All four of the areas represented by this volume are ones to which Bob Rankin contributed. His scholarly publications centered primarily around \ili{Siouan} historical \isi{phonology}, but included works ranging from dictionaries to toponym studies, from philological investigation of early Siouanists  to description of grammaticalization pathways. He was deeply involved in language retention efforts with the \ili{Kaw} Language Project. Other interests included archeology, linguistic \isi{typology}, \ili{Iroquoian} and \ili{Muskogean} languages, and the history of linguistics. 

Bob was a major figure in \ili{Siouan} linguistics, a mentor to nearly all living Siouanists, and a mainstay of the annual \ili{Siouan} and \ili{Caddoan} Linguistics Conference meetings for decades. Trained in \ili{Romance} and \ili{Indo-European} linguistics, with a specialty in \ili{Romanian} (Ph.D. University of Chicago 1972), he shifted gears soon after leaving graduate school, and became an expert in \ili{Siouan} languages, especially the \ili{Dhegiha} branch, with special \isi{focus} on \ili{Kaw}. From the mid 1970s through the end of his life, he devoted himself to \ili{Siouan} studies, both practical and scholarly. His long association with the \ili{Kaw} Tribe led to a \isi{grammar} and \isi{dictionary} of that language (see Cumberland, this volume), and he also produced a \isi{grammar} of \ili{Quapaw}, and briefly conducted \isi{field work} on \ili{Omaha}-\ili{Ponca} and \ili{Osage}. At the University of Kansas he directed dissertations on \ili{Lakota} (Trechter, 1995) and \ili{Tutelo} (Oliverio, 1996) as well as several M.A. theses on \ili{Siouan} languages, and taught a wide variety of courses including field methods and structure of \ili{Lakota} and \ili{Kansa} as well as more theoretical courses in \isi{phonology}, \isi{syntax}, and historical linguistics. Perhaps Bob's greatest gift to the field was his encouragement of others. At conferences and on the \ili{Siouan} List email forum, he was unfailingly patient and encouraging, answering all questions seriously, explaining linguistic terms to non-linguist participants and basic facts of \ili{Siouan} languages to general linguists with equal enthusiasm and lack of condescension.

Following his untimely passing, a special session was held at the 2014 \ili{Siouan} and \ili{Caddoan} Linguistics Conference to organize several projects in Bob's honor: The first of these was publication of the Comparative \ili{Siouan} Dictionary, an immense project comparing \isi{cognates} across all the \ili{Siouan} languages, undertaken by Rankin and a group of colleagues in the 1980s. It had been circulated in various manuscript forms but never published. Thanks to David Rood (another founding member of the CSD project), with help from Iren Hartmann, the CSD is now available online (Rankin et al, 2015). The second project was a volume of Bob's conference papers and other previously unpublished or less accessible work, to be collected and edited by a group headed by John P. Boyle and David Rood; that volume, tentatively titled \textit{\ili{Siouan} Studies: Selected Papers by Robert L. Rankin}, is currently in progress. The third project was a volume of \ili{Siouan} linguistic work in Bob's memory, which has taken the shape of the present volume.

\section*{References}
 

\printbibliography[heading=subbibliography,notkeyword=this]
\begin{reflist}

Bonneau, Jos\'e. 1992. \textit{The structure of internally headed relative \isi{clauses}.} Montreal: McGill University. (Doctoral dissertation.) 

Cole, Peter. 1987. The structure of internally headed relative \isi{clauses}. \textit{Natural Language and Linguistic Theory} 5. 277-302

Gordon, Bryan J. 2009. ``Artifiers'' in \ili{Mississippi Valley Siouan}: A novel determiner class. (Paper presented at the annual meeting of the Linguistic Society of America, San Francisco, 8-11 January 2009.)

Helmbrecht, Johannes. 2002. Nouns and verbs in Hocank (\ili{Winnebago}). \textit{International Journal of American Linguistics} 68(1). 1-27.

Helmbrecht, Johannes. 2006. Applicatives in \ili{Siouan} languages: A study in comparative \ili{Siouan} \isi{grammar}. (Paper presented at \ili{Siouan} and \ili{Caddoan} Languages Conference, Billings, Montana.)

Ingham, Bruce. 2001. Nominal and verbal status in Lakhota: A lexicographical study. \textit{International Journal of American Linguistics} 67(2). 167-192.

Jelinek, Eloise. 1984. Empty categories, case, and \isi{configurationality}. \textit{Natural Language and Linguistic Theory} 2. 39-76.

Kayne, Richard S. 1994. \textit{The antisymmetry of syntax}. Cambridge, Massachusetts: MIT Press.
                 
Koontz, John E. 1995. \ili{Michigamea} as a \ili{Siouan} language. (Paper presented at the 15th \ili{Siouan} and \ili{Caddoan} Languages Conference, Albuquerque.)

Murasugi, Keiko. 2000. An Antisymmetry analysis of \ili{Japanese} relative \isi{clauses}. In Artemis Alexiadou, Paul Law, Andr\'e Meinunger and Chris Wilder, eds., \textit{The \isi{syntax} of relative clauses}, 231-64. Amsterdam: John Benjamins. 

Oliverio, Giulia. 1996. \textit{A \isi{grammar} and \isi{dictionary} of Tutelo}. Lawrence: University of Kansas. (Doctoral dissertation)

Polinsky, Maria. 2013. Applicative constructions. In: Dryer, Matthew S. and Haspelmath, Martin (eds.) \textit{The world atlas of language structures online}. Leipzig: Max Planck Institute for Evolutionary Anthropology. (Available online at http://wals.info/chapter/109, Accessed on 2015-08-23.)

Rankin, Robert L. 1977. From verb, to auxiliary, to noun classifier and definite article: Grammaticalization of the \ili{Siouan} verbs sit, stand, lie. In R. L. Brown et al., eds., \textit{Proceedings of the 1976 MidAmerica Linguistics Conference}, 273-83. Minneapolis: University of Minnesota.  

Rankin, Robert L.; Richard T. Carter; A. Wesley Jones; John E. Koontz; David S. Rood and Iren Hartmann, eds. 2015. \textit{Comparative \ili{Siouan} Dictionary}. Leipzig: Max Planck Institute for Evolutionary Anthropology. (Available online at http://csd.clld.org, Accessed on 2015-09-25.) 

Shaw, Patricia A. 1980. \textit{Theoretical issues in \ili{Dakota} \isi{phonology} and morphology}. New York: Garland.

Trechter, Sara. 1995. \textit{The Pragmatic Functions of Gendered Clitics in Lakhota}. Lawrence: University of Kansas. (Doctoral dissertation.) 

Watanabe, Akira. 2004. Parametrization of quantificational determiners and head-internal relatives. \textit{Language and Linguistics} 5. 59-97.

West, Shannon. 2003. \textit{Subjects and objects in \ili{Assiniboine} Nakoda}. Victoria: University of Victoria. (Doctoral dissertation.)

Williamson, Janis. 1987. An \isi{indefiniteness restriction} for relative \isi{clauses} in Lakhota. In Eric Reuland and Alice ter Meulen, eds., \textit{The representation of (in)definiteness}, 168-190. Cambridge: MIT Press. 

Woolard, Kathryn. 1998. Simultaneity and bivalency as strategies in bilingualism. \textit{Journal of Linguistic Anthropology} 8(1). 3-29.


\end{reflist}

%\end{document}

%\printbibliography[heading=subbibliography]
\end{refsection}

