% 15
\documentclass[output=paper]{LSP/langsci}
\author{Catherine Rudin}
\title{Coordination and related constructions in {Omaha-Ponca} and in {Siouan} languages}

\abstract{Syntactic\is{syntax} constructions expressing semantic \isi{coordination} vary widely across the Siouan language family. A case study of possible coordinating \isi{conjunction}s in Omaha-Ponca demonstrates that distinguishing \isi{coordination} from other means of expressing `and' relations is a non-trivial problem. A survey of words translated as `and,' `or,' or `but' in Siouan languages leads to the conclusion that neither coordinating \isi{conjunction}s nor the syntactic\is{syntax} structures containing them are reconstructable across the Siouan family. It is likely that Proto-Siouan lacked syntactic\is{syntax} \isi{coordination}. 
% KEYWORDS: [Siouan, coordination, subordination, conjunction, comitative]
}
\ChapterDOI{10.17169/langsci.b94.178}

\maketitle

\begin{document}

\section{Introduction}\label{sec:rudin:1}

All languages have ways of expressing additive, disjunctive, and adversative relations among entities or propositions. In European languages these relations are expressed by two distinct syntactic\is{syntax} means: \isi{coordination} and \isi{subordination}. In Siouan languages these two types of \isi{conjunction} construction are also present, but the distinction between them is less robust and less clear; \isi{coordination} may not have existed at all historically. Neither coordinating conjunctions (`and,' `or,' `but') nor the syntactic\is{syntax} structures containing them are reconstructable across the Siouan family.

I begin this examination of \isi{coordination} in Siouan by defining \isi{coordination} and discussing some of the issues involved in distinguishing coordinate from subordinate\is{subordination} \isi{conjunction} (\sectref{sec:rudin:2}). This is followed in \sectref{sec:rudin:3} by a case study of additive \isi{coordination} and coordinate-like constructions in \ili{Omaha-Ponca}, the Siouan language with which I am most familiar. \sectref{sec:rudin:4} is a survey of available data on \isi{coordination} across all branches and most of the languages in the Siouan language family, with a summary table. \sectref{sec:rudin:5} concludes the chapter with a discussion of the (non)universality of \isi{coordination} constructions and some speculations on the history and origins of \isi{coordination} in Siouan.

\section{Issues in defining and identifying \isi{coordination}}\label{sec:rudin:2}
\subsection{The \isi{syntax} of \isi{coordination}}\label{sec:rudin:2.1}

Traditionally, \isi{coordination} is a structure of the type shown in \REF{ex:rudin:1}:

\begin{exe}
\ex\label{ex:rudin:1} 	
\Tree [ .X [ .X ] [ .X ] ] 
\end{exe}	

In this structure two or more conjuncts\is{conjunction} of identical grammatical category together constitute a larger syntactic\is{syntax} unit of the same category. These conjuncts might for instance be noun phrases, verbs, or \isi{clauses}:

\begin{exe}
\ex\label{ex:rudin:2} 		
\begin{minipage}[b]{0.2\textwidth}
\Tree
[ .NP [ .NP ] [ .NP ] ]
\end{minipage}
\begin{minipage}[b]{0.2\textwidth}
\Tree
[ .V [ .V ] [ .V ] ]
\end{minipage}
\begin{minipage}[b]{0.2\textwidth}
\Tree
[ .CP [ .CP ] [ .CP ] ]
\end{minipage}
\end{exe}

The conjuncts\is{conjunction} are sisters, of equal syntactic\is{syntax} status, in a symmetrical constituent. Neither coordinate\is{coordination} is subordinate\is{subordination} to or included in the other. Equality of status is seen by coordinate NPs bearing the same case and triggering plural agreement in languages where those categories are overtly marked. In addition coordinate phrases resist extraction (\citealt{Ross1967}, Coordinate Structure Constraint), and any movement out of them must be ``across the board'' movement out of all the conjuncts. Thus in standard \ili{English} when two pronouns are coordinated\is{coordination}, as in \REF{ex:rudin:3}; they must both be nominative in \isi{subject} position, they require a plural verb, and they cannot be separated.

\begin{exe}
\ex\label{ex:rudin:3} \begin{xlist}
\ex[ ] {[She and I] were chosen.}
\ex[*] {[She and me] were chosen.}
\ex[*] {[She and I] was chosen.}
\ex[*] {[She] was chosen [and I].}
\end{xlist} 
\end{exe}

This contrasts with a non-coordinate\is{coordination} construction like that in \REF{ex:rudin:4}, in which the two pronouns are different cases, the verb is singular, agreeing with only the first pronoun, and the subordinate\is{subordination} portion of the construction can be moved.

\begin{exe} 
\ex\label{ex:rudin:4} 
\begin{xlist}
\ex {[She] [along with me] was chosen.}
\ex {[She] was chosen [along with me].}
\end{xlist}
\end{exe}

Coordinate\is{coordination} constructions may or may not contain an overt coordinating \isi{conjunction}, a word translating as `and,' `or,' `but', etc. If there is one, it may occur between the conjuncts, or after the last one, or may be repeated (before or after each conjunct):

\begin{exe}
\ex\label{ex:rudin:5}			
\begin{minipage}[b]{0.2\textwidth}
\Tree
[ .X [ .X ] [ .conj ] [ .X ] ]
\end{minipage}
\begin{minipage}[b]{0.2\textwidth}
\Tree
[ .X [ .X ] [ .X ] [ .conj ] ]
\end{minipage}
\begin{minipage}[b]{0.2\textwidth}
\Tree
[ .X [ .X ] [ .conj ] [ .X ] [ .conj ] ]
\end{minipage}
\end{exe}

In recent theories of \isi{syntax} (i.e. Minimalism), coordinate\is{coordination} structures are instead treated as asymmetric constructions headed by the coordinator: ``CoordP'' or ``\\isi{\&P},'' or the similar ``Boolean Phrase'' structure argued for by \citet{Munn1993}. This type of structure is adopted partly for theory-internal reasons such as Kayne's Linear Correspondence Axiom (\citeyear{Kayne1994}), but also for reasons having to do with \isi{intonation}, \isi{ellipsis}, and other phenomena which often suggest that the \isi{conjunction} is more closely associated with one conjunct than with the other. See \citet{Citko2011} for detailed discussion. Under this view coordinate\is{coordination} structures look something like those in \REF{ex:rudin:6}; presumably Siouan languages, being strongly head-final, would tend to have the left-branching variant shown on the right:

\begin{exe}
\ex\label{ex:rudin:6}	
\begin{minipage}[b]{0.3\textwidth}
\Tree
[ .\\isi{\&P} [ .\isi{XP} ] [ .\&$'$ [ .\& ] [ .\isi{XP} ] ] ]
\end{minipage}
\begin{minipage}[b]{0.3\textwidth}
\Tree
[ .\\isi{\&P} [ .\&$'$ [ .\isi{XP} ] [ .\& ] ] [ .\isi{XP} ] ]
\end{minipage}
\end{exe}

Issues of whether the \isi{conjunction} forms a constituent with either the preceding or following X, and whether there is such a thing as a Coordination Phrase, are obviously important if one is concerned with distinguishing ``true'' \isi{coordination} from other constructions such as comitatives which have similar meanings. Under the ``\\isi{\&P}'' analysis \isi{coordination} has a syntactic\is{syntax} configuration much like \isi{comitative} or \isi{subordination} structures, with one conjunct higher than the other, making it less straightforward to explain the distinctive behavior of coordinate structures, as well as less clear what criteria distinguish coordinate from subordinate\is{subordination} structures. Numerous works have wrestled with these issues theoretically and across languages, e.g. \citet{Wesche1995} and  \citet{FabriciusHansenRamm2008}. I lack data to deal with such questions in most of the Siouan languages, so the exact structure of apparently coordinate\is{coordination} phrases is left vague in what follows. Detailed research within each language will be needed to sort it out.

It is likely that many of the structures which translate `and/or/but' in various Siouan languages are actually not coordinate. Several other types of syntactic\is{syntax} constructions often express semantic \isi{coordination}. These include at least the following: \REF{ex:rudin:1} \isi{comitative}s (prepositional phrases or subordinate\is{subordination} \isi{clauses} expressing `accompaniment' or a `with' relation); \REF{ex:rudin:2} adverbial \isi{clauses} with temporal or other subordinate relations to a matrix clause (`when,' `although,' `having done X,' etc.); \REF{ex:rudin:3} simple listing of nouns, verbs, or \isi{clauses} (that is, concatenation of separate items which do not form a larger constituent of any kind, sometimes with elements meaning `too,' `also,' `furthermore,' `however' or a phrase which sums them up (`both,' `all'); \REF{ex:rudin:4} co-subordinate\is{subordination} or clause\is{clauses}-chaining constructions, (see e.g. \citealt{Graczyk2007}; \citealt{Boyle2007}).

There are a number of problematic \isi{coordination} constructions in languages of the world, for instance a coordinator analyzed as a transitive verb in a Papua New Guinean language (\citealt{BrownDryer2009}), partial/covert \isi{coordination} of the \textit{nie s Ivan} 'we with Ivan' = `Ivan and I ' type in \ili{Slavic} (e.g. \citealt{McNally1993}; \citealt{Larson2014}), special treatment of commonly linked items \citep{Walchli2005}, and overlaps with serial constructions (\citealt{Carstens2002}). I do not deal with these specifically, but mention them just as a further reminder that the \isi{syntax} of \isi{coordination} is not necessarily a simple issue. For a useful typological\is{typology} overview of \isi{coordination}, see \citet{Haspelmath2007}; other general treatments include \citet{Johannessen1998} and  \citet{VanOirsouw1987}. 

\subsection{The semantics of \isi{coordination}}

Coordinators join constituents with diverse semantic relations, though the semantic aspects of \isi{coordination} have received less attention than its \isi{syntax}. Different authors use widely varying terminology for the meanings \isi{coordination} can express; see for instance \citegen{Citko2011} discussion of Andrej Malchukov's system of classification of \isi{coordination} constructions into semantic types. Among the terms used in the literature are \textit{additive, adversative, \isi{comitative}, consecutive, concessive, contrastive, correction, disjunctive, mirative} and others.

In the cursory survey of the Siouan data below I will for the most part ignore issues of semantics beyond the gross level of meaning indicated by being translated in a grammar or \isi{dictionary} as `and' versus `but' or `or' --- roughly additive, adversative, and disjunctive. From the data available it is often not clear precisely what range of meanings are covered by a given \isi{conjunction}. Semantic classification of the conjunctions will require detailed investigation of usage in each individual language, and will surely interact with numerous factors, including modality, adverbial modifiers, same or different \isi{subject} of conjoined \isi{clauses}, and so on. I leave this entire area for future research. For the present I simply list all elements which seem to translate `and,' `but,' or `or' in any of their meanings.

\subsection{Identifying lexical coordinators}

Another issue is that some of these lexical items, although they translate \ili{English} coordinators, may in fact not be coordinators\is{coordination}. This is yet another area which provides fertile ground for future, deeper research into each individual Siouan language. Coordinating \isi{conjunction}s can be difficult to distinguish from sentence-initial or sentence-final elements (complementizers, \isi{discourse} particles, switch-reference markers, and other clause\is{clauses}-linking morphemes), and from \isi{comitative} or adverbial words. Coordinators often develop historically into sentence-initial or -final elements, presumably by way of a stage involving elided conjuncts. Historical change can go the other way too: as \citet{Mithun1988} and \citet{Stassen2000} both point out, many languages have coordinating conjunctions which are recently and transparently derived from various sources, including \isi{comitative} prepositions, adverbial particles, aspect markers, and clausal (subordinating\is{subordination}) \isi{conjunction}s. This leads to situations in which the same word is sometimes a coordinator, sometimes not, and teasing apart the two usages is tricky; such is the case for example with \ili{Bulgarian} \textit{no}, \textit{ama}, \textit{ami} (\citealt{Fielder2008}) and Australian \ili{English} \textit{but} (\citealt{MulderThompson2008}). Given the slipperiness of this issue in well-studied European languages, it should be no surprise that identifying coordinators\is{coordination} can be problematic when dealing with spoken or inconsistently written data in a language with no tradition of written prose or punctuation conventions.

\section{Additive \isi{coordination} in \ili{Omaha}-Ponca}\label{sec:rudin:3}

My interest in \isi{coordination} in Siouan was sparked not by theoretical considerations but by a practical problem of language teaching. In an \ili{Omaha} language class\is{language classes} in 2002, a student's question of how to say `and' turned out to be unexpectedly hard to answer, with no one word corresponding to \ili{English} \textit{and}. There are several clause\is{clauses} connectors which are at least plausible candidates for coordinators in \ili{Omaha-Ponca}, but nothing which syntactically\is{syntax} coordinates nominal or other non-clausal phrases. To say things like `I have a cat and two dogs' or `That dress is black and white' our \ili{Omaha}-speaking consultants rephrased with non-coordinate constructions, to the sometimes frustrated bewilderment of the \ili{English}-dominant students.  In this section I examine various options for expressing additive \isi{coordination} (`and') in \ili{Omaha-Ponca} and consider whether they are true \isi{coordination} or involve some other strategy such as adverbial modifiers or \isi{subordination}. This case study illustrates both the richness and complexity of the data and the difficulty of conclusively distinguishing \isi{coordination} from non-coordinate structures in a Siouan language.

\subsection{Coordination\is{coordination} of \isi{clauses}: \textit{shi}  and similar words}  

    The word most commonly offered by \ili{Omaha} consultants as a translation for \textit{and} is \textit{shi}, which often occurs as an apparent sentence conjoiner, or at least a \isi{discourse} link between sentences. \citet[52]{Koontz1984} lists \textit{shi} along with \textit{ki}, \textit{goⁿ}, \textit{goⁿki}, \textit{oⁿska}, and \textit{egithe} in a table of ``sentence introducers'' culled from James Owen Dorsey's\ia{Dorsey, James Owen} 19\textsuperscript{th}-century \ili{Omaha} and \ili{Ponca} materials; the same words are found in my field recordings from 100 years later.  It is an open question whether these words start a new sentence or not; i.e. whether the structure is [S \textit{shi} S] or [S][\textit{shi} S], with [\textit{shi} S] constituting a separate sentence.\footnote{David Rood\ia{Rood, David S.} (pc) points out that [S shi] [S] might be a more expected split into two sentences in a verb final language, but shi is not sentence final, in written texts or spoken prosodic contours.}  Dorsey apparently considered them to be the start of a new sentence, but it is unclear why. Presumably he heard a preceding pause, or speakers when dictating to him tended to pronounce \textit{shi} with the following sentence. But there is often a pause or break before a \isi{conjunction} in \ili{English} as well, sentences do begin with coordinating conjunctions (in spite of prescriptive prohibitions), and in the more recent view of \isi{coordination}, the \isi{conjunction} does form a tighter unit with one of the joined \isi{clauses}. Even if we assume the entire string [S \textit{shi} S] is a single sentence, it is unclear whether the two smaller sentences so joined are syntactically\is{syntax} coordinated or one subordinate to the other. \ili{Omaha} has no clear markers of \isi{subordination} that I know of (e.g. no nonfinite verb forms).  

The precise meaning of \textit{shi} is another issue:  Koontz states that \textit{shi} differs from the other ``introducers'' in that it has a meaning of `again' or `marks repetition', but this meaning is not always apparent to me. \textit{Shi} sometimes seems to indicate repetition, but not always.  In the examples below,\footnote{These examples are from my field tapes, recorded in the late 1980s and 1990s, in Macy Nebraska. I am grateful to the National Science Foundation and Wenner-Gren Foundation for support, and to the speakers quoted here, Clifford Wolfe Sr.\ia{Wolfe, Sr., Clifford}, Bertha Wolfe\ia{Wolfe, Bertha}, Mary Clay\ia{Clay, Mary}, and Coolidge Stabler\ia{Stabler, Coolidge}, for sharing their language with me. The \isi{orthography} used in this paper is the ``Macy Standard'' spelling\is{orthography} used at Umoⁿhoⁿ Nation School and the University of Nebraska.}   \textit{shi} (boldfaced) seems to mark not so much repetition as simple additive \isi{coordination} semantics --- `and also' --- or even contrast, as in \REF{ex:rudin:8} or \REF{ex:rudin:10}.  In some discourses \textit{shi} strings together several sentences or \isi{clauses} in a row, as in \REF{ex:rudin:10} and \REF{ex:rudin:11}.  Example \REF{ex:rudin:11} in particular is a fairly extended \isi{discourse} in which nearly every sentence after the first starts with \textit{shi}, and the \isi{discourse} is a list of items, with no sense of repetition except the continued idea of praying for something. Note that \textit{shi} cooccurs with other ``sentence introducers,'' for example, \textit{goⁿki} (in \REF{ex:rudin:10} and \REF{ex:rudin:11}), and with arguably subordinating\is{subordination} adverbial \textit{ki} (in \REF{ex:rudin:11}).\footnote{`?' in examples marks words which were unclear when transcribing\is{transcription} field tapes or whose meaning is unknown. Since the morphological breakdown of most words is immaterial for the purposes of the paper, glosses are not necessarily morpheme-by-morpheme. Clitics are separated with an equal sign.}

\ea \label{ex:rudin:7}
\gll Thíshti  xtáwithe.  \textbf{Shi} thíshti xtóⁿthathe.  éshti  xtóⁿtha=i  ge \textbf{shi}  wíshti xtáathe.\\
 	you   	\textsc{1sgA}.like.2P  and 	you     	2A.like.{1sgA}  	s/he   	like.1P=\textsc{prox}  	?  	and  	I.too  	\textsc{sgA}.like.3P\\
\trans `I like you.  And you like me.  S/he likes me and I like her/him too.'

\ex\label{ex:rudin:8} 
\gll  Zhiⁿgá  ama águdishti údon  wánoⁿ\textipa{P}oⁿ=noⁿ, \textbf{shi}  águdishti  wánoⁿ\textipa{P}oⁿ=bazhi=noⁿ.\\
children 	the 	some        	good  	listen.to.\textsc{1plP=hab} 	and some listen.to.\textsc{1plP}=\textsc{neg=hab}\\
\trans `Some of the children are good; they listen to us, but some of them don't listen to us.'

\ex\label{ex:rudin:9} 
\gll  \textbf{Shi}  góⁿki 	shaóⁿ 	ama ... shaóⁿ  	xé=ta=i  á=bi=ama.\\          
    	and  	then 	Sioux\il{Dakota}\il{Lakota} 	the 	 ... Sioux\il{Dakota}\il{Lakota}  	bury=\textsc{fut=prox} 	say=\textsc{prox=quot}\\
\trans `And the Sioux\il{Dakota}\il{Lakota} ...  His wish was for the Sioux\il{Dakota}\il{Lakota} to bury him.'

\ex\label{ex:rudin:10} 
\gll  Góⁿki 	\textbf{shi} 	gá=tʰe  oⁿgáhi  ki    \textbf{shi} 	wachʰígagha ama  shóⁿ-gagha=i=tʰe 	ki 	\textbf{shi} 	shóⁿshoⁿ    \textbf{shi} 	zhuáwagthe 	agthé=tʰa=ama.\\
    	then 	and that=the \textsc{1plA}.go.there when 	and dancers the end-do=\textsc{prox=evid} when 	and right.away and 	together 	took.\textsc{1plP}.home=\textsc{evid=aux}\\
\trans `We would go there but as soon as the dancers quit they took us right home.'
\z 

\ea                 \label{ex:rudin:11}
\ea
\gll  Wakóⁿda 	thiⁿkʰe 	shti  	btháha=ta=miⁿkʰe\\
	god          	the       	too  	\textsc{1sgA}.pray=\textsc{fut=1sg.aux}\\
\trans `I'm going to pray to God.'
\ex 
\gll	\textbf{Shi} gáge 	iⁿdádoⁿ thé 	amá 	níkashiⁿga 	amá 	shti ewéwaha=tʰe\\
and 	that   what     	this 	the 	person    	the 	too 	\textsc{1sgA}.pray.for.it=\textsc{evid}\\
\trans `I (will) pray for the people who had these things.'

\ex 
\gll	\textbf{Shi}  umóⁿhoⁿ 	ti    	thoⁿ 	shti 	agíwahoⁿ.\\
and 	\ili{Omaha}  	house 	the 	too 	\textsc{1sgA}.pray.for.it.\textsc{refl}\\
\trans `And I (will) pray for my \ili{Omaha} camp/village.'  (i.e. for the present-day reservation)
\ex 
\gll	\textbf{Shi} tʰóⁿwoⁿgtha dúba  édi moⁿthíⁿ umóⁿhoⁿ shti ewéwaha. \\         
and town several there 3A.walk \ili{Omaha}  	too 	\textsc{1sgA}.pray.for.3P\\
\trans `And I (will) pray for the \ili{Omaha} who are in various cities.'  (i.e. off reservation)
\ex 
\gll	Gáge \textbf{shi} 	gahí 	nikashiⁿga.\\
	this 	and 	chief  	person\\
\trans `And for the council.'
\ex 
\gll	\textbf{Shi} uzhóⁿge  	oⁿgáthe 	dshtoⁿ.\\
	and  	road/path 	\textsc{lA}.go  	maybe\\
\trans `And for the path we will take.'  (i.e. for our lives)
\ex 
\gll	Awóⁿhoⁿ 	egóⁿ  	é=ta=miⁿkʰe\\
	\textsc{1sgA}.pray 	thus 	that=\textsc{fut=1sg.aux}\\
\trans `I will pray for those things.' 
\z
\z

The other ``sentence introducers'' listed by Dorsey\ia{Dorsey, James Owen} and Koontz\ia{Koontz, John E.} include \textit{ki}, \textit{goⁿ}, \textit{góⁿki}, and \textit{kigóⁿki}, all meaning `and, and then'. Their distribution is similar to that of \textit{shi}; both in Dorsey's texts and in mine, they occur written at the beginning of sentences as well as joining two sentences or \isi{clauses}, and they indicate a range of connections between those \isi{clauses}, sometimes temporal and sometimes not. 

\subsection{\is{coordination}Coordination of non-sentential categories: does it exist?}

\textit{Shi} and the other sentence conjoiner/introducers generally do not occur in conjoining contexts other than linking sentences. That is, they appear not to coordinate\is{coordination} nominals or other non-clausal categories (though see \REF{ex:rudin:35} below).  In the case of nominals, several patterns occur, generally consisting of a string of NPs with a word meaning something like `also' at the end, sometimes with some element between the individual NPs as well.  

\citet[201]{Koontz1984} gives the formula \underline{NP, NP \textit{éthoⁿba}} for conjoined nominals in Dorsey.  This pattern is found in modern materials as well.  Example \REF{ex:rudin:12} is a sentence from the story \textit{Jimmy and Blackie}, translated into \ili{Omaha} as a school booklet in the 1980s, and \REF{ex:rudin:13} is an example from a conversation I recorded in 1990. \textit{Ethoⁿba} is etymologically related to the number two (\textit{noⁿba}) and probably best treated as an element meaning `both' or `the two of them' instead of as a \isi{conjunction}.

\begin{exe}
\ex\label{ex:rudin:12} 
\gll  Iⁿnoⁿha  	akʰá, iⁿdadi    \textbf{éthoⁿba} 	théthudi 	gthíⁿ 	é=shti.  \\
	my.mother 	the  	my.father 	also        	here       	live   	they=too\\
\trans `My mom and also my dad, they live here too.'

\ex\label{ex:rudin:13} 
\gll  Ivan 	akʰá Silas 	\textbf{éthoⁿba} ukíkizhi. \\         
Ivan the   Silas	 also       	brothers\\
\trans`Ivan and Silas, those two were brothers.'
\end{exe}

Ardis Eschenberg\ia{Eschenberg, Ardis} (pc) reports that the elders/language teachers at Umoⁿhoⁿ Nation school in the early 2000s generally used \underline{NP, NP \textit{shti}} for conjoined nominals.  I have found some examples of this too, but actually very few with this exact pattern.  Example \REF{ex:rudin:14} is one.  Most sentences with \textit{shti} in my data have variations on the pattern such as \textit{shti} after a single NP \REF{ex:rudin:15}, or repeated \textit{shti} \REF{ex:rudin:16}, \REF{ex:rudin:17}.   Note that \textit{shti} cooccurs with \textit{shi} in \REF{ex:rudin:16} to coordinate three NPs:  \underline{NP \textit{shti}, NP \textit{shti, shi} NP}.  In \REF{ex:rudin:17} the second conjunct\is{conjunction} looks like a postverbal afterthought. The word \textit{shti} `too, also' could perhaps be analyzed as a coordinator\is{coordination}, but seems more likely to be an adverbial element, perhaps related to \textit{xti} `very'.

\ea\label{ex:rudin:14}
\gll  Ithádi, 	ihóⁿ  	akʰa 	\textbf{shti} 	gínita  	ezhé 	goⁿkí  	ithádi 	ama, 	 ihóⁿ 	akʰá 	zhúgigtha=bazhí.\\
	his.father 	his.mother 	the 	too 	living 	?  	and  	his.father 	the 	his.mother 	the 	together=\textsc{neg}\\
\trans `His father and his mother are both alive, but his father and mother do not live 	together.'

\ex\label{ex:rudin:15} 
\gll Tim 	akʰá 	iwíkoⁿ=ta=akʰa     Clifford \textbf{shti} utháha 	uwíkoⁿ=ta-akʰa\\
Tim the	 \textsc{3A}.help.\textsc{1sgP}=\textsc{fut=3aux}  Clifford too 	? \textsc{3A}.help.\textsc{3P}=\textsc{fut=3aux}\\
\trans `Tim will help me and Clifford.'
  
\ex\label{ex:rudin:16} 
\gll Shi  níkashiⁿga hútoⁿga  wa'ú  \textbf{shti} shaóⁿ \textbf{shti} \textbf{shi} wáxe dúba 	edí 	atʰí-ama.\\
and 	person  winnebago 	woman too sioux  too  and white 	some there \textsc{3A}.arrive-\textsc{pl.aux} \\
\trans `And a \il{Ho-Chunk}Winnebago woman, some Sioux\il{Dakota}\il{Lakota}, and some whites were also there.'

\ex\label{ex:rudin:17}	
\gll Shi 	wóndoⁿ 	ithádi  \textbf{shti} hóⁿdi 	ugíkitha  	ihóⁿ  	akʰá  	\textbf{shti}.\\
	and 	both 	his.father 	too 	last.night 	\textsc{3A}.was.talking.to.\textsc{3P} 	his.mother 	the 	too\\
\trans`And last night he was talking to both his father and his mother.'
\z

In my elicited data conjoined nominals most often take the form \underline{NP (\textit{egoⁿ}), NP} \underline{\textit{shenoⁿ}}, with degree elements literally meaning `so much' or `that much' as in examples \REF{ex:rudin:18} through \REF{ex:rudin:24}. The awkward literal gloss with `as ... that extent' could perhaps be better rendered `as well as ... all of those'. In any case, this seems unlikely to be a coordinate\is{coordination} construction.

\begin{exe}
\ex\label{ex:rudin:18}
\gll Téska 	tanúka 	\textbf{égoⁿ} 	wazhíⁿga 	\textbf{égoⁿ}	nú \textbf{shénoⁿ} thatʰé  xtáathe.\\
	cow   	meat    	as     	chicken   	as     	potato 	that.extent 	eat      	\textsc{1sgA}.like\\
\trans `I like to eat beef and chicken and potatoes.'    
 
 \ex\label{ex:rudin:19}
\gll Watʰé 	zhíde 	\textbf{égoⁿ} 	hiⁿbé 	ská    \textbf{	shénoⁿ}  	bthíwiⁿ.\\
	dress   	red  	as   	shoe 	white 	that.extent 	\textsc{1sgA}.buy\\
\trans I bought a red dress and white shoes.'  

\ex\label{ex:rudin:20}
\gll Watʰé  zhíde,  hiⁿbé	ská,  watháde  pézhitu \textbf{shénoⁿ} abthiⁿ.\\ 
dress   	red   	shoe 	white 	hat  	green 	that.extent 	\textsc{1sgA}.have \\
\trans `I have a red dress, white shoes, and a green hat.'   

\ex\label{ex:rudin:21}
\gll Sézi  	tʰe 	shé  	\textbf{shénoⁿ} 	áhige 	oⁿgáthiⁿ. \\
	orange 	the 	apple 	that.extent 	much  	\textsc{1plA}.have \\
\trans `We have plenty of (both) oranges and apples.'

\ex\label{ex:rudin:22}
\gll Mary akʰá  	\textbf{égoⁿ} wi \textbf{shénoⁿ} Macy ata oⁿgátha. \\
 Mary	the  as  	I 	that.extent  	Macy to 	\textsc{1plA}.go.there \\
\trans `Mary and I went to Macy.'

\ex\label{ex:rudin:23}
\gll John akʰá \textbf{égoⁿ} Mary akʰá \textbf{shénoⁿ} Macy ata ahí=tʰe. \\
 John the 	as  Mary the 	that.extent  Macy to \textsc{3plA}.arrive.there=\textsc{evid} \\
\trans `John and Mary went to Macy.'
 
\ex\label{ex:rudin:24} 
\gll Tim akʰá Cliffford \textbf{égoⁿ} wi \textbf{shénoⁿ} iwíkoⁿ=ta=akʰa. \\
Tim the Clifford as 	I  that.extent  help.\textsc{1sgP}=\textsc{fut=3aux} \\
\trans `Tim will help Clifford and me.'
\end{exe}

This \underline{NP \textit{égoⁿ}, NP \textit{shénoⁿ}} pattern also occurs in bilingual booklets produced by the Umoⁿhoⁿ Nation school; the translations are from the booklets as well:
	
\begin{exe}	
\ex\label{ex:rudin:25}
\gll Jimmy 	akʰá \textbf{égoⁿ} Sabe akʰá \textbf{shénoⁿ} \\
Jimmy the  	as 	black 	the  	that.extent \\
\trans `Jimmy and Blackie' (title of booklet)

\ex\label{ex:rudin:26} 
\gll Núzhiⁿga ga tʰoⁿ é=\textbf{egoⁿ} mízhiⁿga ga  tʰoⁿ e=shti 	\textbf{shénoⁿ}  \\
boy 	this 	the 	he=as   	girl   this the she=too  that.extent 	 \\

\textit{uwáwakizhi}.
my.younger.siblings
\trans `This is my little brother and sister.' 	
\end{exe}	 
	
A more literal translation of \REF{ex:rudin:26} would be `Like this boy, this girl also, as a group they are my little siblings.' Another pattern combines the previous two:   \underline{NP \textit{égoⁿ}, NP \textit{shti}}; \REF{ex:rudin:27} is an elicited example from my field tapes, \REF{ex:rudin:28} a spontaneously produced sentence. 

\begin{exe}	
\ex\label{ex:rudin:27}
\gll Mary akʰá \textbf{égoⁿ}  wí=\textbf{shti} Macy 	ata 	oⁿgátha. \\
Mary the as I=too Macy to 	\textsc{1plA}.go.there \\
\trans`Mary and  I went to Macy.'

\ex\label{ex:rudin:28}
\gll Ihóⁿ  wiáxchi  \textbf{égoⁿ} ithádi  \textbf{shti} wiáxchi. \\
their.mother  just.one so their.father too just.one \\
\trans `They have the same mother and the same father too.'
\end{exe}	

Simply juxtaposing a string of nominals is another \isi{coordination} strategy, and quite a common one, though I will not give any examples. In fact, all of the nominal \isi{coordination} patterns we have seen so far could be interpreted as simple listing of noun phrases, with some kind of focus element following one or more of the nominals and/or a summing-up element at the end of the nominal string. Given the lack of case marking and near-absence of number agreement  in \ili{Omaha-Ponca},\footnote{Third person plural is not audibly marked in many verbs, and in those where it is, it is homophonous with
proximate singular marking.} as well as the likely status of most if not all lexical noun phrases as adjuncts in this language, the usual tests for coordinate\is{coordination} as opposed to other structures tend not to apply, and it is difficult to distinguish for example coordinate from \isi{comitative} constructions.

 A final, very common way of expressing \ili{English} `and' in situations involving two participants acting together is with the verb \textit{zhugthe} `be with, accompany, be together'. This verb sometimes occurs following two nouns which could be seen as coordinated but are probably just listed; \REF{ex:rudin:29} is more literally `Mary, John, being together they went to Macy.'

\begin{exe}	
\ex\label{ex:rudin:29}
\gll Mary akʰá  John	Macy ata 	\textbf{zhúgthe} 	ahí. \\
Mary the John Macy to  together arrive.there \\
\trans `Mary went to Macy with John. /Mary and John went to Macy.'
\end{exe}

In non-elicited examples, there is almost never more than one lexical noun phrase with \textit{zhugthe}; instead one nominal is given and the other is understood as accompanying it. In \REF{ex:rudin:30} only the woman is mentioned; the other participant is already present in the \isi{discourse}. In \REF{ex:rudin:31} the unmentioned participant is the speaker, and interestingly the verb is first person singular, not plural, indicating that the construction is definitely \isi{comitative} and not \isi{coordination} of an overt with a null NP.\footnote{In playback speakers commented that the second verb, \textit{atʰí}, could have been \textit{oⁿgátʰi} (first person plural), like the verb of the next clause\is{clauses}; \textit{zhuágithe} however would still be first person singular.}   

\ea\label{ex:rudin:30}
\gll Agthí (i)tʰediki shi wa'ú shtewiⁿ \textbf{zhúgthe} agthí=itʰe.\\
\textsc{3A}.came.home 	when and woman whatsoever together came.home=\textsc{evid}\\
\trans `When he came home, he came home with a woman.' (He and some woman came home.)
\z

\ea\label{ex:rudin:31}
\gll Wa'ú  wiwíta Tésóⁿwiⁿ \textbf{zhuágithe}  atʰí, she=kʰe oⁿgátʰi.\\
woman 	my  White.Buffalo together.1\textsc{suus}  \textsc{1sgA}.arrive	this=the 	\textsc{1plA}.arrive\\
\trans `My wife White Buffalo and I are both here; we came here.' (more literally, `My wife White Buffalo, together with my own, I came here...')
\z

There are thus several ways of expressing semantic \isi{coordination} of nominals in \ili{Omaha-Ponca}, but none for which a strong case can be made that it is a syntactic\is{syntax} coordinate construction or any clear candidate for a coordinating construction. The nominal ``\isi{coordination}'' patterns above are all basically lists of NPs with the option of adding a word or words stressing repetition or accompaniment. The picture is even more dubious for adverbs, nominal modifiers, and other non-clausal constituent types. \citet{Koontz1984} does not mention \isi{conjunction} of categories other than nominals. I did not think to elicit them in \isi{field work}, and have not found naturally produced examples.  The kind of sentences my \ili{Omaha}-language-class\is{language classes} students wanted to say, like `I'm wearing a red and yellow shirt,' seem impossible to express without resorting to multiple \isi{clauses} (`My shirt is red and it is also yellow.')

\subsection{Discussion: Once more on \textit{shi}}

Having concluded that \ili{Omaha-Ponca} has no clear coordinating \isi{conjunction} or \isi{coordination} construction for non-clausal \isi{coordination}, I return briefly to my best candidate for clausal coordinating \isi{conjunction}, \textit{shi}. In \sectref{sec:rudin:2.1} I presented a number of examples of \textit{shi} apparently linking \isi{clauses} together; however, it may actually be an adverbial of some sort, not a \isi{conjunction}, in which case \ili{Omaha-Ponca} would not have any true \isi{coordination}, even of \isi{clauses}. It often appears in positions other than clause-initial, most often preverbal, as in the following examples. Here it is clearly not conjoining anything, but does have an `again' sense:  

\begin{exe}	
\ex\label{ex:rudin:32}
\gll óⁿba 	wéthabthiⁿ 	ki 	\textbf{shi}  	wat'éxe=ta=ama.  \\
day  	third          	at 	and 	funeral=\textsc{fut=aux}  \\
\trans `There'll be another funeral Wednesday.'   (Wednesday again will be a funeral.)

\ex\label{ex:rudin:33} 
\gll Oⁿwóⁿthatʰoⁿ thíshtʰoⁿ=i 	tʰedi 	tápuska 	ta 	\textbf{shi}  háthe 	oⁿgákʰi. \\
\textsc{1plA}.eat  	finish=\textsc{prox} 	when 	school 	to 	and ?  	\textsc{1plA}.arrive.back \\
\trans `After dinner we went back (again) to the school.'

\ex\label{ex:rudin:34}
\gll óⁿba 	wiⁿ 	Ishtíⁿthiⁿkhe 	akʰá \textbf{shi} edí=bi=ama. \\
day	one 	Monkey 	the 	and 	there=\textsc{pl=quot} \\
\trans `One day Monkey was there (again), they say.'  (traditional story opening)
\end{exe}

However, it is possible that this is a different \textit{shi} from the sentence-coordinating one. Further research is obviously needed. My data contain a few examples in which \textit{shi} might be interpreted as conjoining nominal phrases, following the last in a string of NPs:  \underline{NP, NP \textit{shi}}. None are very convincing, however, and \textit{shi} in them can plausibly be taken as an adverbial expressing repetition. In \REF{ex:rudin:35}, for instance, fighting was a regular occurrence.

\begin{exe}
\ex\label{ex:rudin:35}
\gll Umóⁿhoⁿ kʰe shaóⁿ kʰe \textbf{shi}  wóⁿdoⁿ 	kikína=noⁿ=i \\
\ili{Omaha}    	the 	Sioux\il{Dakota}\il{Lakota}  	the 	and both \textsc{3A}.\textsc{refl}.fight=\textsc{hab}=\textsc{prox} \\
\trans `The \ili{Omaha} and the Sioux\il{Dakota}\il{Lakota} tribes used to fight each other.'    
\end{exe}

This section thus concludes rather inconclusively: \ili{Omaha-Ponca} apparently has no non-clausal \isi{coordination}, and may or may not have \isi{coordination} of \isi{clauses}. 

\section{Siouan languages: An overview}\label{sec:rudin:4}

At this point we leave the details of \ili{Omaha-Ponca} and turn to a shallow but broad survey of the Siouan family. In spite of limited data on many members of the family and the challenges of interpretation and analysis, there is quite a lot we can say about \isi{coordination} in Siouan languages. In several of the languages \isi{coordination} has been described in some detail. Nearly all of the languages have recorded equivalents of `and,' and many have equivalents for `or' or `but,' though their morpho-syntactic\is{syntax} status is often unclear. In many of the languages \isi{coordination} of \isi{clauses} is different than \isi{coordination} of noun phrases or other categories, as we saw in \ili{Omaha-Ponca}. Perhaps the most interesting result of a survey of Siouan \isi{coordination} is the lack of unity within the family. No coordinators are reconstructable, there are no widespread \isi{cognates}, and strategies for expressing \isi{coordination} differ from language to language. It appears likely that \ili{Proto-Siouan} had no true \isi{coordination}. In this section I briefly describe the data from each sub-branch of Siouan (starting with \ili{Dhegiha} because it is most familiar to me; information on \ili{Omaha-Ponca} is repeated in brief form for completeness). No examples are given in this section and no attempt is made to justify the lexical items given as (possible) coordinators; instead, anything mentioned in sources is listed.

\subsection{\ili{Dhegiha}}
 
\textbf{\ili{Omaha-Ponca}} (data from \citealt{DorseyNDPonka} \citealt{DorseyNDOmahaPonca},  \citealt{Koontz1984}, \citealt{Rudin2003} and my own fieldwork)\footnote{These sources use several different orthographies. In the interest of consistency I have spelled all \ili{Omaha-Ponca} words in the modern ``Macy Standard'' spelling\is{orthography}.} has several ways of expressing `and'. As discussed above, different \isi{conjunction}s are used to coordinate\is{coordination} \isi{clauses} and NPs. Clauses may be conjoined with \textit{ki}, \textit{goⁿ}, \textit{shi} `again, and then,' \textit{goⁿki}, \textit{kigoⁿki} `and then'. Dorsey considers \textit{ki} to be \ili{Ponca} and \textit{goⁿ} to be \ili{Omaha}; both of these are said to join ``substantive \isi{clauses}''. \textit{Goⁿ} is likely the same as subordinating\is{subordination} \textit{(e)goⁿ} `having (done),' related to postposition \textit{egoⁿ} `like, as'. \textit{Goⁿki} and \textit{kigoⁿki} are pretty clearly combinations of these two conjunctions. \textit{Shi} is perhaps the best candidate for a true coordinator, although it, like the others listed here, occurs most often sentence initially (conjoining the sentence to the preceding \isi{discourse} semantically if not syntactically\is{syntax}). NPs are occasionally joined by \textit{goⁿ}; this may actually be the postposition mentioned above. More commonly two NPs are followed by \textit{edoⁿba/éthoⁿba} `also, both;' literally `the two of them'. A string of three or more NPs may be followed by \textit{edabe} `also'. Two or more NPs can be followed by \textit{shti} `too'. Although Dorsey does not list it, one of the most common strategies for coordinating\is{coordination} NPs in my data is \textit{egoⁿ ... shenoⁿ} `both ... and;' literally `as ... that-extent', Ardis Eschenberg (p.c.) finds \textit{egoⁿ ... thoⁿzhoⁿ} used in the same way. The most common translation of `and' with NPs is clearly not syntactic\is{syntax} \isi{coordination}: a \isi{comitative} construction with the verb \textit{zhugthe} `be with'. Simple juxtaposition (listing) of conjuncts with no \isi{conjunction} is common for both S and NP \isi{coordination}. `Or' and `but' in \ili{Omaha-Ponca} are formed with the `and' \isi{conjunction}s for joining \isi{clauses}, and to the best of my knowledge do not exist at all for NPs. Dorsey lists \textit{goⁿ ... ite ki} `either ... or', \textit{shoⁿ doⁿste} `either-or, perhaps', and \textit{doⁿste} at end of clause `or' (the latter two in Dorsey's slip file). `But' is commonly expressed by \textit{shi} `and' connecting two \isi{clauses}, the second of which is negative\is{negation} or contrasts in some way.

\textbf{Osage} (data from \citealt{Quintero2004}) coordinates NPs using \textit{ée\textipa{D}\k{o}\k{o}pa} `the two of them' following two or more NPs. (Compare \ili{Omaha-Ponca} \textit{ethoⁿba}.) Verb agreement suggests that this is a true \isi{coordination} structure; however, it is possible that the two NPs are appositive and the plural verb actually agrees with pronominal \textit{ée-}. Another possible NP coordinator is \textit{\v{s}ki} `also'. Clauses\is{clauses} are coordinated by juxtaposition without a \isi{conjunction}: ``There is no \ili{Osage} equivalent to the \ili{English} use of \textit{and} to conjoin sentences; rather, the elements are strung together with no intervening forms of any kind'' (455). Quintero gives no information on `or' or `but'.

\textbf{Kaw (\il{Kanza}Kansa)} (data from   \citealt{CumberlandRankin2012}; Justin McBride\ia{McBride, Justin T.} p.c.; Robert Rankin\ia{Rankin, Robert L.} p.c.) has an `and' coordinator\is{coordination}, \textit{\v{s}i}, which is used in a variety of syntactic\is{syntax} environments (postverbal, preverbal, postnominal, clause\is{clauses}-initial) and apparently can conjoin both \isi{clauses} and nominals. McBride states that it usually seems to be used adverbially (`again') or adjectivally (`another'), but can also symmetrically coordinate\is{coordination} \isi{clauses}. Numerous \isi{conjunction}s with meanings like `and, then, so' exist, but all seem to be subordinating\is{subordination} rather than coordinating. The \isi{conjunction} \textit{d\k{a}} `and, then, so' occurs between \isi{clauses} and in other coordinating situations; Rankin, in a 2012 email, states that ``Kaw\il{Kanza} ... seems to allow the \isi{conjunction} \textit{d\k{a}} (often reduced to d-schwa ...) in exactly the same places \ili{English} would allow `and'''; he suggests this is a result of adopting \ili{Spanish} \isi{coordination} structures. Further evidence of \ili{Spanish} influence is the clearly borrowed coordinator\is{coordination} \textit{pero} `but'. I have no information on `or' in Kaw\il{Kanza}.

\textbf{Quapaw} (data from \citealt{Rankin2002,Rankin2005b}) probably has \isi{conjunction}s similar to those in the other \ili{Dhegiha} languages, but I have very little information. Rankin's grammar and \isi{dictionary} list \textit{\c{s}i} `and' (cf. \ili{Omaha-Ponca} \textit{shi}, Kaw\il{Kanza} \textit{\v{s}i}), but give no indication of how it is used.

\subsection{\il{Ho-Chunk-Jiwere}Winnebago-Chiwere}
 
\textbf{\isi{Ho-Chunk}} (data from \citealt{Helmbrecht2004}; confirmed by Iren Hartmann\ia{Hartmann, Iren} p.c.) has three apparently straightforward coordinating\is{coordination} \isi{conjunction}s, which Helmbrecht labels as follows: \textit{án\k{a}ga} `and' (coordinate); \textit{n\k{i}\k{i}gé\v{s}ge} (\textit{n\k{i}gee\v{s}ge}) `or' (disjunction); \textit{n\k{u}n\k{i}ge} `but' (adversative). The `and' and `or' words are used to conjoin all types of syntactic\is{syntax} constituents: NP, \isi{VP}, S, ``obliques'' (\isi{adjunct} phrases), and AdvP. The conjunctions are placed between the coordinated phrases, or in the case of three coordinated NPs, preceding the last NP (X Y \textit{án\k{a}ga} Z `X, Y and Z'). Helmbrecht argues that \textit{án\k{a}ga} \isi{conjunction} is true \isi{coordination}: the resulting constituent requires plural agreement, and an overt pronoun is needed to conjoin a 1\textsuperscript{st} or 2\textsuperscript{nd} person. \isi{Ho-Chunk} also has a \isi{comitative} construction with the verb \textit{haki\v{z}u} `to be together,' as well as some other, presumably subordinating\is{subordination} conjunctions: \textit{n\k{a}ga, hirean\k{a}ga} `along with' conjoins animate\is{animacy} \isi{subject}s or \isi{object}s, and \isi{clauses} can be conjoined with \textit{`eegi} `and then' or \textit{\v{s}ge/hi\v{s}ge} `also, even' (placed after 2\textsuperscript{nd} conjunct). Helmbrecht also discusses \isi{negation} of one or both conjuncts; a special \isi{conjunction} \textit{h\k{a}ké}, used at the beginning of S or NP, expresses `and not/but not'.

\textbf{Chiwere\is{Ioway, Otoe-Missouria}} (data from \citealt{Goodtracks1992}; \citealt{Greer2016} (this volume); Bryan Gordon\ia{Gordon, Bryan James}, p.c.) has several ways of expressing `and'. These include words meaning `with' (\textit{tógre, insúⁿ, inúⁿki}), `also' (\textit{hedaⁿ, -daⁿ, na, -ku}), `again' (\textit{\v{s}ige}), and a set of \isi{discourse} connectives in the form of clefts, with copula \textit{aré: aréda, edá, arédare, édare, hédare}. In addition, a string of nominals can be followed by \textit{inuⁿki} or \textit{bróge}. Gordon also lists `bracketing' \isi{conjunction}s: \textit{\v{s}uⁿ, gasúⁿ, nahé\v{s}uⁿ}, and a number of subordinating\is{subordination} connectives. `But' is \textit{núna}.

\subsection{\ili{Dakotan}}
 
There is information available on several of the \ili{Dakotan} languages and \isi{dialects}; some sources include data from more than one dialect. I have found no information on \ili{Stoney}.

\textbf{Assiniboine} (data from \citealt{West2003}; \citealt{Cumberland2005}; \citealt{Levin1964}) has two main `and' coordinators\is{coordination}, \textit{h\~ik} and \textit{h\~ikná}, but sources differ somewhat in their descriptions of how these are used. West argues explicitly that \textit{h\~ikná} conjoins \isi{VP} or V, not \isi{clauses}; i.e. it occurs in the context \isi{VP} \textit{h\~ikná} \isi{VP} or V \textit{h\~ikná} V. She analyzes it as head of a CoordP with the first conjunct\is{conjunction} \isi{VP}/V as \isi{complement} and the second one as \isi{specifier} (pp. 32-38). Clauses\is{clauses} are joined by \textit{h\~ik} repeated after each clause\is{clauses}: S \textit{h\~ik} S \textit{h\~ik}. Cumberland, on the other hand, shows all categories joined by non-repeating \textit{h\~ik}: NP \textit{h\~ik} NP, V \textit{h\~ik} V, \isi{VP} \textit{h\~ik} \isi{VP}. \citet{Levin1964}, cited in \citet[36]{Stassen2000} discusses a third coordinator, \textit{ka}, which conjoins NP. There is also a \isi{comitative} construction with \textit{kici} `with' at the end of a string of NPs. I have no information on `but' or `or' in \ili{Assiniboine}.

\textbf{Lakota}\footnote{In general I have used the \isi{orthography} of the source in this paper. However, in the case of \ili{Lakota}, I have standardized all the disparate orthographies of the various sources to the modern standard spelling system\is{orthography} used by the \ili{Lakota} Language Consortium.} (data from  \citealt{RoodTaylor1996}; \citealt{Ingham2003}; \citealt{Ullrich2016};  \citealt{BoasDeloria1941}) has several `and' \isi{conjunction}s: \textit{na, nahá\textipa{N}} `and also'; \textit{\v{c}ha, \v{c}ha\textipa{N}khé} `and so'; \textit{yu\textipa{N}k\v{h}á\textipa{N}} `and then'; \textit{na} can coordinate\is{coordination} nouns or \isi{clauses}, while the others appear to coordinate only \isi{clauses}. \ili{Lakota} also has a word meaning `or': \textit{naí\textipa{N}\v{s}}, and several expressing contrastive \isi{coordination} `but': \textit{éya\v{s}, k'éya\v{s}, tk\v{h}á, khé\v{s}, \v{s}k\v{h}á. éya\v{s}} is also listed as an interjection meaning `well, but'. Numerous other conjunctions are listed, including \textit{ho, honá} `furthermore', \textit{nakú\textipa{N}} `also', \textit{hé u\textipa{N}} `therefore', \textit{tk\v{h}á\v{s}} `but indeed', and others. Ulrich gives examples of an apparent \isi{comitative}, \textit{ki\v{c}hí}, as well. It is not entirely clear to me whether the `and/or/but' conjunctions are all coordinators\is{coordination} or whether some (or all) are subordinating\is{subordination} conjunctions. Rood and Taylor define ``\isi{conjunction}'' as connecting two sentences, but at least the `and' and `or' words can also conjoin ``parts of a sentence, such as nominals or verbs''. The position of all the \isi{conjunction}s is between conjuncts in their examples, but they state there are ``two possible positions: in the second slot from the beginning or in the last slot in the sentence.'' David Rood (p.c.) points out that obligatory \isi{ablaut} before \textit{na} and \textit{naí\textipa{N}\v{s}} suggests a strong bond between the \isi{conjunction} and the preceding verb.

\textbf{Dakota} (data from \citealt{Riggs1851};  \citealt{BoasDeloria1941}) has unsurprisingly some similar \isi{conjunction}s to \ili{Lakota}, though some also differ. Several words translate `and': \textit{k'a, \v{c}ha, u\textipa{N}khá\textipa{N}, nakú\textipa{N}. U\textipa{N}khá\textipa{N}} conjoins \isi{clauses} with different \isi{subject}s, while \textit{k'a} conjoins nouns and \isi{clauses} with same \isi{subject}; no details are given of the usage of the other `and' words. `But' is \textit{tukhá}, and `or' is \textit{k'a i\v{s}}. \citeauthor{BoasDeloria1941} give forms from several \isi{dialects}; alongside the \ili{Lakota} forms in the previous paragraph they also list \ili{Dakota} forms, usually labelled as ``\ili{Yankton}'' and/or ``\ili{Santee}'' dialect, including \textit{k'a} 'and, \textit{u\textipa{N}khá\textipa{N}} `and then'.     

\subsection{\il{Missouri Valley Siouan}Missouri Valley}
 
\textbf{\il{Apsaalooke}Crow} (data from \citealt{Graczyk2007}) has very different strategies for conjoining \isi{clauses} and
nominals. For coordinate\is{coordination} nominals, the \isi{conjunction}s are \textit{-dak} `and' and \textit{-xxo} `or'. Both are suffixes (or enclitics), but at different levels: \textit{-dak} suffixes to NP, while \textit{-xxo} suffixes to N$'$. Both conjunctions are repeated after each conjunct; \textit{-dak} may and \textit{-xxo} must be omitted after the final conjunct. There is also a \isi{comitative} construction involving the transitive verb \textit{áxpa} `be with' (also `marry') with same-\isi{subject} marking or an incorporation\is{noun incorporation} structure. Clauses\is{clauses} in \il{Apsaalooke}Crow are linked by switch-reference marking rather than \isi{conjunction}. Graczyk analyzes apparently coordinate\is{coordination} \isi{clauses} as `co-\isi{subordination}' or clause-chaining: a string of \isi{clauses} with switch-reference markers but no sentence final clitic, except for the last clause, which determines the speech-act type of the entire string (eg. declarative). The adversative `but' relation between \isi{clauses} is marked with \textit{-htaa} (suffix on clause\is{clauses}) or \textit{hehtaa} (sentence connector).

\textbf{Hidatsa} (data from \citealt{Boyle2005,Boyle2007,Boyle2011}) has significantly changed its \isi{coordination} constructions in quite recent times. Boyle points out that \il{Apsaalooke}Crow and \ili{Hidatsa} share some cognate\is{cognates} morphology in the area of \isi{conjunction}s (e.g. \ili{Hidatsa} \textit{-k} is cognate with \il{Apsaalooke}Crow \textit{-dak}), but \ili{Hidatsa} has innovated a semantic distinction involving \isi{specificity} and inclusiveness of NPs. In the area of clausal/verbal \isi{coordination}, \ili{Hidatsa}'s former switch-reference markers have evolved into \ili{English}-like coordinators (\citealt{Boyle2011}). At present, the following morphemes express `and': \textit{hii} coordinates S's; \textit{-k} coordinates NP (with a nonspecific reading when suffixed to both NPs and a specific reading when suffixed only to the first NP); \textit{-\v{s}ek} coordinates NPs with a non-specific reading; \textit{-a} coordinates V in serial verb construction; \textit{-ak} (the old Same Subject marker) coordinates V or \isi{VP}. There is apparently no `but' coordinator\is{coordination}; adversative meaning ``is shown with juxtaposition with one element being negated'' (John Boyle p.c.).

\textbf{Mandan} (data from \citealt{Clarkson2012}; Randolph Graczyk\ia{Graczyk, Randolph} p.c.) links \isi{clauses} via a switch reference system similar to that of \il{Apsaalooke}Crow. The morpheme \textit{ni} is used both as a same-\isi{subject} marker for \isi{clauses} and as a NP coordinator. NP \isi{coordination} is accomplished with a coordinator following each NP; coordinating \isi{conjunction}s used in this way include \textit{eheni, -kini, -hini, -kiri}, all meaning `and'. In modern usage two new coordinators appear, not found in older texts: \textit{hi(i)} with NPs and \textit{ush} with \isi{clauses}. Both occur between conjuncts rather than after each conjunct\is{conjunction}. Clarkson claims that \isi{coordination} is much more common in recent texts than in those from the early 20\textsuperscript{th} century, suggesting that \ili{Mandan} \isi{syntax}, like that of \ili{Hidatsa}, has been restructured under pressure from \ili{English}. I have no information about alternative or adversative \isi{coordination} in \ili{Mandan}.  

\subsection{\ili{Southeastern Siouan}}
 
\textbf{Biloxi} (data from \citealt{Zenes2009}; based on \citealt{DorseySwanton1912}) has an NP coordinator\is{coordination} \textit{y\k{a}} `and' which suffixes either to each NP or just the last one; it is also possible for NPs simply to be listed. Clauses\is{clauses} are coordinated by simple juxtaposition. Zenes treats the latter two constructions (concatenated NPs and S's) as CoordP with a zero coordinator. \is{coordination}Coordination of a series of \isi{object} NPs is expressed by coordinating \isi{clauses} with the same verb repeated (`I planted onions, I planted potatoes, I planted turnips'). Disjunction of NPs is expressed by \textit{ha} `or' following the second NP. Zenes gives no information about `or' with sentences or \isi{clauses}. \ili{Biloxi} also has a \isi{comitative} construction with \textit{n\k{o}pa} following the second NP.

\textbf{Ofo} (\citealt{DorseySwanton1912}; Robert Rankin\ia{Rankin, Robert L.}, p.c.) apparently coordinates \isi{clauses} only by juxtaposition with no \isi{conjunction}. I have no further information about \ili{Ofo} \isi{coordination}, and none at all about \ili{Tutelo}.

\subsection{Summary}

The known possibly-coordinating\is{coordination} \isi{conjunction}s of the Siouan languages are summarized in Tables \ref{coord} and \ref{morecoord} To give some sense of their \isi{syntax}, the conjunctions are shown with the type of constituents they conjoin when this is known; for instance S \textit{\textbf{ki}} S means \textit{ki} can occur between two \isi{clauses}; NP NP \textit{\textbf {shti}} means \textit {shti} occurs at the end of a string of NPs.

\begin{table}
\caption{Coordinating(?)\is{coordination} \isi{conjunction}s} \label{coord}
\small
\begin{tabular}{ l  l  l  l  }
\lsptoprule
Language & Additive \textbf{\textit{and}} & Disjunctive \textbf{\textit{or}} & Adversative \textbf{\textit{but}} \\
\midrule
\ili{Omaha}- & S \textbf{\textit{ki}} S; NP \textbf{\textit{ki}} NP & \textbf{\textit{goⁿ}}  S \textbf{\textit{ite ki}} &  S \textbf{\textit{shi}} S-NEG \\
\ili{Ponca} & S \textbf{\textit{goⁿ}} S; NP \textbf{\textit{goⁿ}} NP & `either ... or' & \\
& S \textbf{\textit{shi}} S & S  \textbf{\textit{dshtoⁿ shi}}  S \textbf{\textit{dshtoⁿ}} & \\
& NP NP \textbf{\textit{edoⁿba}}/ \textbf{\textit{éthoⁿba}} & `maybe and maybe' & \\
& NP NP NP \textbf{\textit{edabe}} & S \textbf{\textit{shoⁿ}} S \textbf{\textit{doⁿste}} & \\
& NP \textbf{\textit{egoⁿ}} NP \textbf{\textit{shenoⁿ}} & `either-or, perhaps' & \\ \vspace{1em}

& NP NP \textbf{\textit{shti}} & S \textbf{\textit{doⁿste}} `or' &  \\  \vspace{1em}
\ili{Osage}	& NP NP \textbf{\textit{ée\textipa{D}\k{o}\k{o}pa}} &   &   \\ 

Kaw\il{Kanza} & S \textbf{\textit{\v{s}i}} S &   & \textbf{\textit{pero}} \\  \vspace{1em}
& S \textbf{\textit{d\k{a}}} S & & \\  \vspace{1em}

\ili{Quapaw} & \textbf{\textit{\c{c}i}} &   &   \\

\isi{Ho-Chunk} & S \textbf{\textit{án\k{a}ga}} S & S \textbf{\textit{n\k{i}\k{i}gé\v{s}ge}} S & \textbf{\textit{n\k{u}n\k{i}ge}} \\
& also conjoins NP, \isi{VP},  & also conjoins NP, \isi{VP}, & \\  \vspace{1em}
& AdvP, oblique & AdvP, oblique & \\

\ili{Chiwere} & \textbf{\textit{\v{s}ige}} & & \textbf{\textit{núna}} \\
& \textbf{\textit{hedaⁿ}}, -\textbf{\textit{daⁿ}} & & \\
& NP NP \textbf{\textit{inuⁿki}} & & \\  \vspace{1em}
& NP NP \textbf{\textit{bróge}} & & \\

\ili{Assiniboine} & V \textbf{\textit{h\~ikná}} V, \isi{VP} \textbf{\textit{h\~ikná}} \isi{VP} &   &   \\
& S \textbf{\textit{h\~ik}} S; also with NP, & & \\
&  V, \isi{VP}, etc. & & \\
& S \textbf{\textit{h\~ik}} S \textbf{\textit{h\~ik}}; also with & & \\
& NP, V, \isi{VP}, etc. & & \\ \vspace{1em}
& NP \textbf{\textit{ka}} NP & & \\

\ili{Lakota}	& S \textbf{\textit{na}} S, NP \textbf{\textit{na}} NP,  & S \textbf{\textit{naí\textipa{N}\v{s}}} S & S \textbf{\textit{éya\v{s}}} S, NP \textbf{\textit{éya\v{s}}} \\
& V \textbf{\textit{na}} V & & NP, V \textbf{\textit{éya\v{s}}} V \\
& S \textbf{\textit{yu\textipa{N}k\v{h}á\textipa{N}}} S & & \textbf{\textit{k'éya\v{s}}}  \\
& S \textbf{\textit{\v{c}ha}} S & & \textbf{\textit{tk\v{h}á}} \\
& S \textbf{\textit{\v{c}ha\textipa{N}khé}} S & & \textbf{\textit{khé\v{s}}} \\ \vspace{1em}
& & & \textbf{\textit{\v{s}k\v{h}á}} \\

\ili{Dakota} & S \textbf{\textit{k'a}} S, NP \textbf{\textit{k'a}} NP & NP \textbf{\textit{naí\textipa{N}\v{s}}} NP & \textbf{\textit{tukhá}} \\
& S \textbf{\textit{u\textipa{N}k\v{h}á\textipa{N}}} S  & \textbf{\textit{k'a i\v{s}}} & \\
& \textbf{\textit{nakú\textipa{N}}} & & \\
& \textbf{\textit{\v{c}ha}} & & \\
\lspbottomrule
\end{tabular}
\end{table}

\begin{table}
\caption{Coordinating(?)\is{coordination} \isi{conjunction}s continued} \label{morecoord}
\small
\begin{tabular}{ l  l  l  l  }
\lsptoprule
Language & Additive \textbf{\textit{and}} & Disjunctive \textbf{\textit{or}} & Adversative \textbf{\textit{but}} \\
\midrule  \vspace{1em}
\il{Apsaalooke}Crow & NP \textbf{\textit{dak}} NP \textbf{\textit{dak}} & N$'$ \textbf{\textit{xxo}} N$'$ \textbf{\textit{xxo}} &  \\

\ili{Hidatsa} & S \textbf{\textit{hii}} S &   & juxtaposition with  \\
& NP-\textbf{\textit{k}}; NP-\textbf{\textit{k}} NP-\textbf{\textit{k}} & & \isi{negation} \\
& NP -\textbf{\textit{\v{s}ek}} NP & & \\
& V-a V (serial verb) & & \\ \vspace{1em}
& V-\textbf{\textit{ak}} V; VP-\textbf{\textit{ak}} \isi{VP} & & \\

\ili{Mandan} & S-\textbf{\textit{ni}} S &   & \\
& \textbf{\textit{ush}} S \textbf{\textit{ush}} S & & \\
& NP \textbf{\textit{eheni}} NP (\textbf{\textit{eheni}}) & & \\
& NP-\textbf{\textit{kini}} NP-\textbf{\textit{hini}} & & \\
& NP-\textbf{\textit{kiri}} NP(-\textbf{\textit{kiri}}) & & \\ \vspace{1em}
& NP \textbf{\textit{hii}} NP & & \\
 \vspace{1em}
\ili{Biloxi} & NP NP \textbf{\textit{y\k{a}}}; NP \textbf{\textit{y\k{a}}} NP \textbf{\textit{y\k{a}}} & NP NP \textbf{\textit{ha}} & \\
 \vspace{1em}
\ili{Ofo}	 & --- & & \\

\ili{Tutelo} & ---  & & \\
\lspbottomrule
\end{tabular}  
\end{table}

A partial list of \isi{comitative} (`with') subordinators\is{subordination} is given in \tabref{comitative}.  Presumably the other Siouan languages also have \isi{comitative} constructions; I list here only those which were mentioned in one of my sources as a common way to express `and' \isi{coordination}.
 
\begin{table}
\caption{Comitative\is{comitative} words} \label{comitative} 

\begin{tabular} [t]{ l  l  }
\lsptoprule
\ili{Omaha-Ponca}	& \textit{zhugthe} \\
\ili{Chiwere} &  \textit{tógre},  \textit{inúⁿ}, \textit{inúⁿki} \\
\ili{Assiniboine} &  \textit{kici} \\
\ili{Lakota} & \textit{ki\v{c}hi} \\
\il{Apsaalooke}Crow & \textit{áxpa} \\
\ili{Biloxi} & \textit{n\k{o}pa} \\
\lspbottomrule
\end{tabular}
\end{table}

\section{Conclusion}\label{sec:rudin:5}
 
What can we learn from the array of facts above? The most striking conclusion that emerges from the data is the lack of unity among the Siouan languages. Even within subfamilies, the Siouan languages are quite diverse in their treatment of \isi{coordination}. We can identify several areas of disagreement: \REF{ex:rudin:1} The languages differ in the types of constituents that can be coordinated, some having only clausal \isi{coordination}, while others can coordinate NPs and other types of constituents as well, and some may have no true
\isi{coordination} at all, but use various types of \isi{subordination}, co-\isi{subordination}, or simple concatenation to express the relations \ili{English} expresses with `and'/`or'/`but'. \REF{ex:rudin:2} They differ in the constituent order within \isi{coordination} constructions, with the \isi{conjunction} following the first conjunct (\isi{XP} \& \isi{XP}), the second conjunct (\isi{XP} \isi{XP} \&) or each of the conjuncts (\isi{XP} \& \isi{XP} \&), and may also differ in whether the \isi{conjunction} forms a constituent with a following or preceding conjunct ((\isi{XP} \&) \isi{XP}); (\isi{XP} (\&\isi{XP})). The hierarchical structure of each of these configurations has not been studied in most of the languages. Given the generally head-final nature of phrase structure in Siouan languages, if the \isi{conjunction} heads a \isi{coordination} phrase it is expected that the \isi{complement} of the ``\&'' head would be to its left; an \isi{XP} occurring to the right could be a \isi{specifier}, which we would expect to be less closely associated with the \isi{conjunction} than the \isi{complement}. \REF{ex:rudin:3} They differ in the lexical items expressing additive, disjunctive, and adversative \isi{coordination}. Some of the words or suffixes for `and'/`or'/`but' are cognate\is{cognates} among subfamilies --- for instance, most of the \ili{Dhegiha} branch have [\v{s}i] or something similar, and the \ili{Dakotan} branch share something like [na]. But no coordinators appear to be cognate across the family. \REF{ex:rudin:4} Finally, the languages differ also in the expression of \isi{comitative} and other ``semantically coordinated\is{coordination}'' phrases.

In short, there does not seem to be a ``typical Siouan'' \isi{coordination} pattern, nor does it look like we can reconstruct proto-Siouan coordinators. Clearly there has been innovation in at least some of the languages -- perhaps all -- and at least in one or two cases there has been borrowing\is{borrowing} of coordinators and/or \isi{coordination} patterns from European languages, suggesting quite recent change in this semantic field. In at least some languages the most common way to conjoin NPs is with a \isi{comitative}, not a coordinate construction. (This is my impression in \ili{Omaha-Ponca}, and Cumberland (p.c.) has the same impression in \ili{Assiniboine}, for example.) Is it possible there was no morphosyntactic \isi{coordination} in proto-Siouan?

In fact, this is not as unlikely as it might first appear. \citet{Mithun1988} suggests overt \isi{coordination} tends to come with literacy: in spoken language simple concatenation tends to be common, while in writing, where \isi{intonation}al cues are lacking and one cannot assume the same degree of common knowledge with one's audience, explicit morphosyntactic \isi{coordination} is more useful. It is certainly not the case that unwritten languages never have true \isi{coordination}, but as a statistical tendency it makes some sense. Many languages, Mithun says, seem to have developed coordinating \isi{conjunction}s after exposure to written languages\is{orthography} or after developing an indigenous tradition of writing. Since Siouan languages were, until recently, not written, perhaps lack of an inherited \isi{coordination} construction and associated morphology is not surprising. The borrowing\is{borrowing} or innovation of coordinators as speakers became literate in \ili{English} or other European languages (as well as perhaps in the Native languages) seems logical under this view.

In spite of the lack of overt morphological or lexical coordinators\is{coordination} in some languages, Mithun considers \isi{coordination} as a syntactic\is{syntax} and semantic structure to be universal. \citet{Stassen2000}, on the other hand, claims \isi{coordination}, or at least nominal \isi{coordination}, is not universal. He divides languages into two types: ``\textsc{with}-languages,'' which have only a \isi{comitative} (NP with NP) or subordinating\is{subordination} strategy for conjoining NPs , and ``\textsc{and}-Languages,'' which also have a coordinate strategy. Stassen acknowledges that Native American languages tend to be problematic and difficult to classify into his two categories. This preliminary study of the Siouan family certainly bears out the elusiveness of \isi{coordination} constructions in these languages.

\section*{Acknowledgment}

Earlier versions of much of this material were presented at the \isi{Comparative Siouan Grammar} Workshop/Siouan and \ili{Caddoan} Languages Conference, Lincoln, NE 2009, and at the 2015 SSILA meeting in Portland. I would like to thank the participants at both events, as well as the two reviewers for this volume, for helpful comments and data. As was so often true, at so many meetings, Bob Rankin's insightful comments at and after the 2009 workshop were especially valuable, and data he gathered was a major source of information on \ili{Dhegiha} languages. I am grateful to have known Professor Rankin\ia{Rankin, Robert L.} and hope that this paper in some small way contributes to his legacy.

\section*{Abbreviations}

1, 2, 3 = first, second, third person; A = agent; \textsc{aux} = auxiliary; \textsc{evid} = evidential; \textsc{fut} = future; \textsc{hab} = habitual; \textsc{neg} = negative; P = patient; \textsc{pl} = plural; \textsc{prox} = proximate; \textsc{quot} = quotative; \textsc{refl} = reflexive; \textsc{suus} = suus (reflexive possessive). 


\printbibliography[heading=subbibliography,notkeyword=this]

\begin{reflist}
   
\end{reflist}
\end{document}

