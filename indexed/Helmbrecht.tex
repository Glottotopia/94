% 17
\documentclass[output=paper]{LSP/langsci}
\author{Johannes Helmbrecht}
\title{{NP}-internal possessive constructions in {Hoocąk} and other {Siouan} languages}
\abstract{Languages usually have more than one construction to express a possessive relationship. Possessive constructions in an individual language usually express semantically different relations, which are traditionally subsumed under the notion of possession such as part-whole relationships, kinship relationships, prototypical ownership, and others. \il{Ho-Chunk}Hoocąk and the other Siouan languages are no exception from this many-to-many relationship between possessive constructions and semantic kinds of possession. The present paper deals with NP-internal types of possession in Siouan languages leaving aside constructions that express possession on the clause level such as benefactive applicatives, reflexive possessives and the predicative possession. The NP-internal possessive constructions will be examined according to the semantic/syntactic nature of the possessor (regarding the Animacy Hierarchy\is{animacy}), and the semantic nature of the possessed (alienable\is{alienable possession}/inalienable\is{inalienable possession} distinction). I will begin with an analysis of \il{Ho-Chunk}Hoocąk and will then compare the \il{Ho-Chunk}Hoocąk constructions with the corresponding ones in some other Siouan languages. At least one language of each sub-branch of Siouan will be discussed. It will be shown that the choice of differlakota
ent NP-internal possessive constructions depends on both semantic scales (the Animacy Hierarchy\is{animacy} and the alienable/inalienable distinction), but in each Siouan language in very individual ways.
% KEYWORDS: [Siouan, possessive  constructions, alienable vs. inalienable, possessor, possessed, possessive pronouns]
}
\ChapterDOI{10.17169/langsci.b94.180}
\maketitle

\begin{document}

\section{The structure of NP-internal possessive constructions}
It may safely be assumed that all languages have grammatical and lexical means to express a possessive relation between an entity A and an entity B. Semantically, possession is a cover term for a broad range of distinct relations, which are expressed by possessive constructions (PC) in the languages of the world. Central to the notion of grammatical possession are the relations of ownership, whole-part relations, and kinship relations. Less central to the general notion of possession are attribution of a property, spatial relations, association, and perhaps nominalization. All these relations may be expressed by NP-internal possessive constructions in \ili{English} as exemplified in \tabref{semantics}.\footnote{See \citet[262--267]{Dixon2010} for a more detailed discussion of these relations.}

\begin{table}
\caption{Semantics of possessive relation in the broad sense (cf. \citealt[262--267]{Dixon2010})} \label{semantics}
\small
\begin{tabular}{ c c c l }
\lsptoprule
\textbf{Entity A}  & \textbf{Possessive relation}	 & \textbf{Entity B} & \textbf{\ili{English} example} \\
\midrule 
\textbf{Possessor} & $\overleftrightarrow{\hspace{3cm}}$  & \textbf{Possessed} & \\
& Ownership	&  & my car/ Peter's house \\
& Whole-part	 & & Mary's teeth/ the teeth of the bear \\
& Kinship	 & & Peter's wife/ my daughter \\
& Attribution of property & & her sadness/ his age \\
& Spatial relation & & the front of the house/ the inside   \\
& & & of the church \\
& Association & & Jane's teacher/ her former school \\
\lspbottomrule
\end{tabular}
\end{table}

Not all of the different kinds of relations in \tabref{semantics} can be expressed by possessive constructions in all languages, but in most cases ownership, whole-part, and kinship relations are covered by their NP-internal PCs. It is still an open question, whether there is a general semantic notion of possession that covers all relations expressed by PCs. There is at least one prominent approach to possession which claims that there is a semantic prototype with a core and a periphery (cf. Seiler's prototype approach (\citealt{Seiler1983,Seiler2001}); and a critical examination of it in \citealt{Helmbrecht2003}). Others reject this idea (cf. for instance \citealt{Heine1997}; \citealt[263]{Dixon2010} and others).

Languages usually have more than one syntactic construction expressing possessive relations on the clause level as well as on the NP level. In Tables \ref{typology1} and \ref{typology2}, there are examples of different possessive constructions from \il{Ho-Chunk}Hoocąk,\footnote{\il{Ho-Chunk}Hoocąk, formerly also known as \emph{Winnebago}, is a Siouan language still spoken in Wisconsin. \il{Ho-Chunk}Hoocąk together with \ili{Otoe}, \ili{Ioway}, and \ili{Missouria} forms the \il{Ho-Chunk-Chiwere}Winnebago-Chiwere sub-branch of Mississippi-Valley Siouan. For the widely accepted classification of Siouan languages see, for instance, \citet{Rood1979}, \citet[501]{Mithun1999}, and  \citet{ParksRankin2001}.} \ili{English} and \ili{German} for illustrative purposes.

\begin{table}
\caption{A brief \isi{typology} of possessive constructions (part 1, Clausal)} \label{typology1}
\begin{tabular}{l l l }
\lsptoprule
level & construction type & examples \\
\midrule
Clause& predicative possession 	& \ili{English} \\ 
& & \textit{I \textbf{have} a blue car}. \\
 & & \textit{The blue car \textbf{belongs} to me}. \\
 & \\
& external possession\footnote{See, for instance,  \citet{PayneBarshi1999} on types of external possession.} & \il{Ho-Chunk}Hoocąk (BO979)\\     
& (\isi{possessor} raising) & \textit{Huuporo=ra \textbf{h\k{i}}-teek-ire} ... \\ 
& & knee=\textsc{def}   \textbf{\textsc{1e.u}}-hurt-\textsc{sbj.3.pl} \\
& & `When my knees hurt, ... ' \\
& \\
& dative of interest & \ili{German} \\
& & \textit{Sie schneidet \textbf{ihm}           die Haare} \\
& & she cuts         \textbf{him.\textsc{dat}} the hair \\
& & `She cuts his hair.' \\
& \\
&beneficiary-\isi{possessor} & \il{Ho-Chunk}Hoocąk \citep[28]{Helmbrecht2003} \\
& polysemy  & \textit{Waž\k{a}tírera hįįgí'een\k{a}}.\\
& & waž\k{a}tíre=ra    hi-< \textbf{hį-gí}>-'e=n\k{a} \\
& & car=\textsc{def}  <\textbf{\textsc{1e.u-appl.ben}}>-find=\textsc{decl} \\
& & `He found the car for me.'/ 'He found my car.' \\
& \\
& possessive reflexive & \il{Ho-Chunk}Hoocąk \citep{HelmbrechtLehmann2010} \\
& & \textit{Hinįk=ra n\k{a}\k{a}<\textbf{kara}>t’ųp='an\k{a}ga} \\
& & son=\textsc{def} <\textbf{\textsc{poss.rfl}}> embrace(\textsc{sbj.3sg}\&\textsc{obj} \\
& & \textsc{3sg})=and \\
& &  `He (i.e. the father) embraced his son, and ...'\\
\lspbottomrule
\end{tabular}
\end{table} 

\begin{table}
\caption{A brief \isi{typology} of possessive constructions (part 2, Non-clausal)} \label{typology2}
\begin{tabularx}{\textwidth}{l l X }
\lsptoprule
level & construction type & examples \\
\midrule
NP  & juxtaposition & \il{Ho-Chunk}Hoocąk \citep[13]{Helmbrecht2003}  \\
& (no marking at all) & \textit{Peter=gá        šųųk=rá} \\
&  & Peter=\textsc{prop}  dog=\textsc{def} \\
& & `Peter's dog' \\
& \\
& genitive attribute & \ili{English} \\
& (genitive case marker & \textit{Peter\textbf{'s} dog} \\
&  on \isi{possessor}) & \\
& \\
& prepositional attribute & \ili{German} \\
& & \textit{der Hund \textbf{von} Peter} \\
& & \textsc{def} dog    \textbf{of}   P. \\
& & `Peter's dog' \\
& \\
 & pronominal index on  & \ili{Mam} \citep[142]{England1983} \\
& \isi{possessed} noun & \textit{\textbf{t-}kamb'    meenb'a} \\
& (\isi{possessor} marking on & \textbf{\textsc{3sg}-}prize orphan \\
& \isi{possessed}) & \mbox{`the orphan's prize'}  \\
& \\
& mixed strategy (genitive & \ili{Turkish} \citep[633]{Kornfilt1990} \\
& case marking plus &\textit{Ayşe\textbf{-nin} araba\textbf{-sı}} \\
& pronominal index) & 	A.\textbf{-\textsc{gen}}    car\textbf{-\textsc{3sg}} \\
& & `Ayşe's car' \\
& \\
& nominalized predicative  & \il{Ho-Chunk}Hoocąk \citep[19]{Helmbrecht2003} \\
& possession & \textit{hicųwį́ waháara} \\
& & hicųwį́  \textbf{wa-háa=ra} \\
& & aunt      \textsc{obj.3pl}-have.kin(\textsc{1e.u})=\textsc{def} \\
& & `my aunts' \\
& \\
Word & nominal compounds & \ili{German} \\
& & \textit{das Regierungsauto} \\
& & das Regierung-s-auto \\
& & \textsc{def} government-\textsc{linker}-car \\
& & `the car of the government' \\
\lspbottomrule
\end{tabularx}
\end{table}

The present paper deals only with PCs on the NP level. Languages often possess more than just one NP-internal PC, as it is the case for instance in \ili{English}. \ili{English} has the \textit{of}-construction and the genitive \textit{=s} construction to express possession NP-internally; similarly for \ili{German}. If there are two or more NP-internal PCs in a language, the choice of these constructions often depends on the semantic and syntactic category of the \isi{possessor} and/or the semantic type of the \isi{possessed} entity. 

With regard to the \isi{possessor}, the choice of the PC may depend on the specific NP type of the \isi{possessor}. For instance, if the \isi{possessor} is expressed by means of a \isi{possessive pronoun} a different construction may be used than with a \isi{possessor} expressed by a lexical noun phrase. If the \isi{possessor} is a proper name or kinship term this may determine the selection of a specific PC, too. Animacy proper of the \isi{possessor}, i.e. \isi{possessor} NPs with a human, animate\is{animacy} or inanimate common noun, may play a role as well. The implicational scale that brings together these different NP-types that may be relevant for the choice of different NP-internal PCs in Siouan languages as well is called Animacy Hierarchy\is{animacy} (AH). The AH is a scale that describes many different grammatical phenomena cross-linguistically. The AH is usually considered as: 1/2 > 3 > proper noun/kin term > human common noun > animate\is{animacy} common noun > inanimate common noun (cf. for instance \citealt{Dixon1979}; \citealt{Comrie1981}; \citealt{Croft2003}).

With regard to the \isi{possessed}, it can be observed that the choice of the NP-internal PC depends on the semantic class of the \isi{possessed}, i.e. these languages often have two sets of nouns, so that set\textsubscript{1} nouns designating the \isi{possessed} entity require one type of the PC, and set\textsubscript{2} nouns designating the \isi{possessed} entity the other. This classification of nouns with regard to PCs is better known under the heading alienable\is{alienable possession} versus inalienable\is{inalienable possession} distinction. Alienable nouns usually designate entities that can be owned in the prototypical sense implying that the \isi{possessor} has full control over these \isi{possessed} entities; for instance the \isi{possessor} can sell them, give them away, and so on. The class of inalienable nouns is much more heterogeneous with regard to the semantics; inalienable nouns designate entities that bear a close association to the \isi{possessor} implying that the \isi{possessor} has no or only a limited control over them. Often, kinship terms, body-part terms\is{body-part term}, and other relational nouns (local/spatial nouns) belong to this class. 
With regard to the formal marking of the respective PCs, the following possibilities can be distinguished (cf. \citealt[286--290]{Dixon2010}):

\vspace{1em}
i.	the alienable PC is similar to that for \isi{inalienable possession} with an added grammatical element;

ii.	the grammatical marking for \isi{alienable possession} is longer than that for \isi{inalienable possession};

iii.	the alienable PC requires a classifier, the inalienable construction does not;

iv.	overt marking only in an alienable PC;
\vspace{1em}

The possibilities i-iv cover the cross-linguistic observation that inalienable PCs tend to be shorter and morphological less complex than alienable PCs. In other words, the PC for \isi{alienable possession} is always more marked than the PC for \isi{inalienable possession}. In what follows it will be shown that this observation also holds in general for the different NP-internal PCs in Siouan.

\section{Methodical remarks}

The goal of this study is to search for all different NP-internal PCs in selected Siouan languages and to describe the conditioning factors for their choice. The guiding hypothesis is that the syntactic/semantic properties of the \isi{possessor} (Animacy Hierarchy\is{animacy}) and the semantic properties of the \isi{possessed} (alienable\is{alienable possession} vs. inalienable\is{inalienable possession}) is a fruitful notional frame for the discovery and the description of the splits in the expression of possession; cf. Table 3. 
	
Typological studies on possession show that the properties of the possessive relation itself such as actual possession vs. possession in the past, temporary possession vs. permanent possession, close possession vs. loose possession and so on, may trigger a constructional split too, in some languages (cf. \citealt[274--276]{Dixon2010}). This is, as far as I can see, not the case in Siouan languages. Therefore, these semantic parameters won't play a role in the rest of the paper. 

\begin{table}
\caption{Semantic/syntactic parameters for constructional splits in NP-internal possessive constructions.} \label{parameters}
\small
\begin{tabular}[h]{ l l l }

\lsptoprule
Semantic-syntactic &	Semantic properties of the & Semantic properties of \\
properties of the \isi{possessor} & possessive relation &  the \isi{possessed} \\

(Animacy Hierarchy\is{animacy}) & & (alienable\is{alienable possession} - inalienable\is{inalienable possession}) \\
\midrule
pronoun (SAP/3rd person) & temporal/ closeness & A) ownership \\
 
proper name & temporary/ permanent & B) \isi{whole-part relation} \\
 
kinship term & close/ loose & C) \isi{kinship relation} \\
 
common noun & general type of possession & D) attribution \\
 
{[human]} & & E) orientation/ location \\
 
{[animate\is{animacy}]} & & F) association \\
 
{[inanimate\is{animacy}]} & & G) nominalization \\
\lspbottomrule
\end{tabular}
\end{table}

The properties of the \isi{possessor} and the \isi{possessed} as summarized in \tabref{parameters} serve as a kind of questionnaire or guideline for the search for constructional splits in the various grammatical descriptions of Siouan languages that are used here. The data and descriptions of PCs are taken from the grammars that are available for the different Siouan languages. For \il{Ho-Chunk}Hoocąk, data from a text \isi{corpus} and from fieldwork sessions will be taken. 

I will exclude the question of the relation between NP-internal PCs and the clause level PCs for later research. My own experience with text data from \il{Ho-Chunk}Hoocąk makes me think that clause level PCs are often preferred over the NP-internal PCs at least in \il{Ho-Chunk}Hoocąk, but this needs to be shown in more detail. 

\section{NP-internal possession in \il{Ho-Chunk}Hoocąk}

\il{Ho-Chunk}Hoocąk and \il{Ioway, Otoe-Missouria}Chiwere (\ili{Missouria}, \ili{Otoe}, \ili{Ioway}) are closely related and constitute the \il{Ho-Chunk-Chiwere}Winnebago-Chiwere sub-branch of the Mississippi Valley group of the Siouan languages. \il{Ho-Chunk}Hoocąk is taken as a representative of this sub-branch, then. 

\il{Ho-Chunk}Hoocąk has no possessive pronouns comparable to \ili{English} \textit{my, your, his, her}, etc., no nominal case marking in general, and no genitive case marker in particular. In addition, there are no connectives, linkers or possessive markers, i.e. grammatical forms that indicate a possessive relation between two \isi{nominals}. \il{Ho-Chunk}Hoocąk has in principal two different types of NP-internal PCs. The first one is a simple juxtaposition of two nouns without any special possessive marking, see example \REF{petersdog} below. The second type of PC is a complex construction with an inflected verb of possession, e.g. \textit{=hii} 'have.kin' plus a definite article nominalizing the entire construction illustrated in example \REF{myaunts}. Without this definite article, we have a clause expressing a kind of predicative possession. 

\ea  \label{petersdog}
\langinfo{\il{Ho-Chunk}Hoocąk} {}{\citealt[16]{Helmbrecht2003}} \\
\ea
\glll Petergá šųųkrá\\
Peter=gá     šųųk=rá \\
P.=\textsc{prop}    dog=\textsc{def} \\
\glt `Peter's dog'

\ex	 \label{myaunts}
\langinfo{\il{Ho-Chunk}Hoocąk} {}{\citealt[19]{Helmbrecht2003}} \\
\glll hicųwį́ wahaará\\
hicųwį́  wa-haa=rá \\
aunt     \textsc{obj.3pl}-have.kin(\textsc{1e.u})=\textsc{def} \\
\glt `my aunts'
\z
\z

Both types of NP-internal PC will be discussed in more detail in the subsequent sections.

\subsection{Juxtaposition}\label{sec:helmbrecht:3.1}

The semantic/syntactic type of the \isi{possessor} does not require the choice of the juxtaposition PC in \il{Ho-Chunk}Hoocąk with one exception. If the \isi{possessor} is a speech act participant or a third person, the second construction type with the nominalized possessive verb has to be chosen obligatorily (see \sectref{nomvposs} below).

The following series of examples demonstrates that neither the AH – except with regard the pronoun/noun distinction – nor the distinction between alienable\is{alienable possession} vs. inalienable\is{inalienable possession} nouns have any effect on the expression of attributive possession in \il{Ho-Chunk}Hoocąk. The example in \REF{petersfather} is an attributive possessive relation with a proper name as \isi{possessor} and a kinship term as \isi{possessed} noun. The relation is inherent and inalienable. The definite article is required.

\ea \il{Ho-Chunk}Hoocąk (\citealt[16]{Helmbrecht2003}) \label{petersfather}

\glll Petergá hi'\k{a}crá\\
Peter=gá       hi'\k{a}c=rá \\
  P.=\textsc{prop} father=\textsc{def} \\
\glt`Peter's father'
\z

The possessive relation in \REF{womansfacea} is a part-whole relationship with a human \isi{possessor} and a \isi{body-part term} as \isi{possessed}. The possessive relation is inherent and inalienable\is{inalienable possession}. The same holds for the examples in \REF{womansfaceb}-\REF{womansfacec}. The whole PC needs to be specified by a determiner, i.e. the definite article, or a demonstrative pronoun. If there is a definite article following the \isi{possessor} (cf. \REF{womansfaceb}), then it is the possession of a specific and definite \isi{possessor}. If the indefinite article follows the \isi{possessor} (cf. \REF{womansfacec}), it is the possession of an indefinite \isi{possessor}. 

\ea \il{Ho-Chunk}Hoocąk (\citealt[13]{Helmbrecht2003})\label{womansface}
\ea \label{womansfacea}
\gll hinų́k    hišja=rá \\
woman face=\textsc{def} \\
\glt `the woman's face'
\ex\label{womansfaceb}
\gll hinų́́k=\textbf{rá}         hišja=rá \\
woman=\textbf{\textsc{def}} face=\textsc{def} \\
\trans `the face of the (specific/definite) woman'
\ex\label{womansfacec}
\gll hinų́́k=\textbf{íz\k{a}} hišja=ra \\
woman=\textbf{\textsc{def}}  face=\textsc{def} \\ 
\glt `the face of an (indefinite) woman'
\z
\z

The PCs in \REF{johnscar} and \REF{petersdog} (above) are alienable\is{alienable possession}. Both contain alienable \isi{possessed} nouns, the inanimate noun \textit{hiráati} `car' and the animate\is{animacy} noun \textit{šųųk} `dog'. The \isi{possessor} is a human being (proper name) in both cases. 

\ea \il{Ho-Chunk}Hoocąk (\citealt[13]{Helmbrecht2003}) \label{johnscar}

\gll John=ga      hiráati=ra \\
J.=\textsc{prop} car=\textsc{def} \\
\glt `John's car'
\z

The possessive relation in \REF{leg} includes a \isi{body-part term} as \isi{possessed} noun (inseparable, inalienable\is{inalienable possession}) with a non-human \isi{possessor}. The example in \REF{wheel} represents a part-whole relation with an inanimate\is{animacy} object as \isi{possessor} and an inanimate object as \isi{possessed} (separable, alienable\is{alienable possession}). Both possessors in \REF{leg} and \REF{wheel} can be interpreted either as specific or as generic. 

\ea \il{Ho-Chunk}Hoocąk (\citealt[13]{Helmbrecht2003}) \label{leg}

\gll wijúk huu=rá \\
cat      leg=\textsc{def} \\
\glt `the leg(s) of the/a cat'
\z

\ea \il{Ho-Chunk}Hoocąk (\citealt[13]{Helmbrecht2003}) \label{wheel}

\gll waž\k{a}tíre hogis=rá \\
car  circular.part=\textsc{def} \\
\glt `the wheel(s) of a/the car'
\z

Note that the constructions in \REF{leg} and \REF{wheel} often resemble a nominal \isi{compound} with the first noun specifying the second noun thus creating a new word and concept instead of expressing a possessive relation. For instance, the \il{Ho-Chunk}Hoocąk word \textit{n\k{a}\k{a}h\'a} `bark' is a \isi{compound} of the noun \textit{n\k{a}\k{a}} `tree' and \textit{haa} `skin, pelt, hide' thus giving the new concept `tree skin' which corresponds to `bark' in \ili{English}. This combination of two nouns is a nominal \isi{compound} on phonological grounds. The vowel in the second noun is shortened, which is a normal word-internal process in \il{Ho-Chunk}Hoocąk. However, the boundary between \isi{compound} and juxtaposition is often blurred and the function "the first noun specifies the second" can be found in phrasal juxtaposition as well as in nominal compounding. The expressions in \REF{leg} and \REF{wheel} are certainly phrasal in nature. Both words in these expressions have their own primary accent and there are no sandhi processes between the two nouns. 

The same type of construction employed for the expression of possession in the preceding examples is also used for the expression of spatial relations. There are numerous local nouns such as \textit{coow\'e}  `front part', \textit{n\k{a}\k{a}k\'e} `back part', \textit{rook} `inside', \textit{hihák} `top, surface', and so on, which are used to express the specific local/ spatial relation of an object vis-à-vis the spatial region of another object. The local nouns are the \isi{possessed} nouns in these constructions. They designate the spatial position of the \isi{possessor}. The \isi{possessor} functions as the reference point (cf. \citealt{Langacker1993}) of the localization, it represents the object with regard to which another one is localized, cf. the examples in \REF{dogposition}. The clitic \textit{=eja} `there' is a local adverb almost obligatorily used in these constructions. 
 
\ea \il{Ho-Chunk}Hoocąk (\citealt[14]{Helmbrecht2003}) \label{dogposition}

\ea 
\glll \v{s}\k{u}\k{u}kr\'a hirar\'uti  coow\'eja `ak\v{s}\k{a}n\k{a}  \\
\v{s}\k{u}\k{u}k=r\'a \textbf{hirar\'uti}  \textbf{coow=\'eja} `ak=\v{s}\k{a}n\k{a} \\ 
dog=\textsc{def} \textbf{car} \textbf{front=there} be.lying=\textsc{decl} \\
\glt `The dog is (in a lying position) in front of the car.'

\ex 
\glll šųųkrá hirarúti hihákeja jeen\k{a}\\
\v{s}\k{u}\k{u}k=r\'a \textbf{hirar\'uti} \textbf{hih\'ak=eja} jee=n\k{a} \\
dog=\textsc{def}  \textbf{car} \textbf{top=there} be.standing=\textsc{decl} \\
\glt `The dog is (in a standing position) on the top of the car.'

\ex 
\glll šųųkrá hirarúti rookéja n\k{a}k\v{s}\k{a}n\k{a}\\
\v{s}\k{u}\k{u}k=r\'a   \textbf{hirar\'uti} \textbf{rook=\'eja} n\k{a}k=\v{s}\k{a}n\k{a} \\
dog=\textsc{def} \textbf{car} \textbf{inside=there} be.sitting=\textsc{decl} \\
\glt` The dog is (in a sitting position) inside of the car.'
\z
\z

The expressions in examples \REF{petersfather} through \REF{dogposition} show that the semantic nature of a lexical \isi{possessor} does not trigger a shift to another construction type: this holds if the \isi{possessor} is a proper name (=\textit{ga} \textsc{prop}), human noun (=\textsc{def/=indef/}=Ø), animate\is{animacy} noun (=\textsc{def/=indef}/=Ø), or inanimate noun (=\textsc{def/=indef}/=Ø). In addition, the expressions in \REF{petersfather} through \REF{dogposition} show that there is no alienable\is{alienable possession}-inalienable\is{inalienable possession} distinction: the same construction type is chosen with kinship terms, body-part terms\is{body-part term}, relational spatial nouns, as well as with alienable nouns. The \isi{possessor} noun may be marked by a definite (\textsc{def}), an indefinite (\textsc{indef}) article, or by zero. If the \isi{possessor} is a proper name (\textsc{prop}), it will be marked by the proper name marker. The entire PC is always definite (\textsc{def}) (marked on the \isi{possessed} noun) except with spatial nouns. They are usually marked by means of a local adverb clitic =\textit{eja} 'there' which – in this respect – could also be analyzed as a general local postposition. The examples also show that this type of PC may express real ownership, part-whole relations, kinship relations, and spatial relations.
 
\subsection{Nominalized verbal possessive constructions}\label{sec:helmbrecht:3.2} \label{nomvposs}

The juxtaposition of two \isi{nominals} is a general construction type to express possession and other binary relations in \il{Ho-Chunk}Hoocąk. There is, however, an alternative NP-internal \isi{possessive construction}, which indeed exhibits a classification of nouns: inalienable\is{inalienable possession} nouns such as kinship terms, domestic (pet) animals, and alienable\is{alienable possession} nouns. These alternative constructions are in each case a nominalized version of the possessive predication employing different possessive verbs for different types of \isi{possessed} entities. The nominalized possessive \isi{clauses} appear in the same syntactic position as the juxtaposed nouns, i.e. in a noun phrase position of the clause. 

\ea \il{Ho-Chunk}Hoocąk (\citealt[16]{Helmbrecht2003}) \label{seecar}

\ea 
\gll \textbf{John=g\'a hir\'aati=ra}  hac\'aa=n\k{a} \\
\textbf{J.=\textsc{prop} car=\textsc{def}} see(\textsc{1e.a}\&\textsc{obj.3sg}=\textsc{decl} \\
\glt `I see John's car.'  

\ex 
\gll J\textbf{ohn=g\'a hir\'aati han\k{i}=r\'a} hac\'aa=n\k{a}  \\
\textbf{J.=\textsc{prop} car own(\textsc{sbj.3sg}\&\textsc{obj.3sg})=\textsc{def}} see(\textsc{1e.a}\&\textsc{obj.3sg})=\textsc{decl}  \\
\glt `I see John's car.'
\z
\z 

Both \isi{clauses} in \REF{seecar} have the same translation, but speakers indicate that they prefer the nominalized variant over the juxtaposed variant. The same constructional pairs exist for possessive constructions with kinship terms and pet animals (domestic animals). These nominalized possessive \isi{clauses} represent a kind of transition from attributive to predicative possession. The general structure of these nominalized possessive \isi{clauses} is given in \REF{generalstructure}.

\ea General structure of the nominalized verbal \isi{possessive construction} \label{generalstructure}

[(N-POSSESSOR\textsubscript{i})  N-POSSESSED\textsubscript{j}    PRO\textsubscript{j}-PRO\textsubscript{i}-Verb of possession=DET]\textsubscript{NP}
\z

If the \isi{possessor} is a speech act participant or third person, these nominalized PCs are the only possibility. Since the \isi{possessor} is often a \isi{topic} (given and definite) in \isi{discourse} and hence expressed pronominally as a \textsc{3sg}, this type of PC prevails in \isi{discourse} over the alternative juxtaposition. Note that \textsc{3sg} arguments are always marked zero on the verbs. Both entities X\textsubscript{possessor} and Y\textsubscript{possessed} are cross-referenced in the verb of possession utilizing the two different series of pronominal prefixes, the actor/\isi{subject} series for the \isi{possessor} and the undergoer/\isi{object} series for the \isi{possessed}. The verbs of possession are treated as regular transitive verbs.

If the \isi{possessor} is a lexical human noun, this construction type competes with the juxtaposition type of PCs dealt with in the preceding section; cf. the following examples in \REF{petersthings}.
 
\ea \il{Ho-Chunk}Hoocąk (\citealt[16]{Helmbrecht2003}) \label{petersthings}

\ea 
\glll Peterga hi'ą́c hiirá \\
Peter=ga        hi’ą́́c   \textbf{hii}=ra \\
P.=\textsc{prop}  father have.kin=\textsc{def} \\
\glt `Peter's father'

\ex 
\glll Peterga šų́ųk nįįhíra \\
Peter=ga      šų́́ųk   \textbf{nįįhí}=ra \\
P.=\textsc{prop} dog   have.pet=\textsc{def} \\
\glt `Peter's dog'

\ex \gll John=gá      hiráati \textbf{hanį}=rá \\
J.=\textsc{prop} car       own=\textsc{def} \\
\glt `John's car'

\z \z

The verbs of possession that are used in the PCs in \REF{petersthings} are restricted in their usage. The verb \textit{=hii} `X has Y as kin' can only be used with kinship terms or with terms designating close friends. This verb is homophonous with the causative auxiliary \textit{=hii} `to cause'. There are reasons to believe that both verbs are historically cognate, and that they should be considered as different usages of one verb rather than homonyms. The main reason for this analysis is that the causative verb \textit{=hii} has an irregular personal inflection, and the possessive verb \textit{=hii} shows exactly the same pattern. 

The possessive verb \textit{nįįh\'i} `X has Y as pet' is used only with pet animals. Usually, pet animals are domesticated animals such as cats, and dogs, etc. The semantic boundaries of this class are not clear-cut. Historically, \textit{nįįhí} is presumably a combination of *\textit{nį} `to live, living thing', which does not occur independently in \il{Ho-Chunk}Hoocąk and the causative auxiliary \textit{=hii}.\footnote{*\textit{n\k{i}} is the reconstructed Proto-Mississippi-Valley Siouan form for `live, be alive' (cf. \citealt{RankinEtAl2015AccessMay}). This form can be found in other verbs in \il{Ho-Chunk}Hoocąk such as \textit{n\k{i}\k{i}h\'a} `to breathe'  or in \textit{n\k{i}\k{i}'\'{\k{a}}p} `be alive'.} The verb \textit{nįįhí} shows the same inflectional irregularities as the causative verb \textit{=hii}. 

The possessive verb \textit{han\k{i}} `to own' is a regular (lexical) transitive verb designating the possession of alienable\is{alienable possession} entities such as inanimate\is{animacy} objects, artifacts, animals, and so on. Body parts\is{body-part term} belong to this group of nouns, too. It is restricted to human possessors. Part-whole relations with inanimate possessors, on the other hand, are never expressed with this construction. Cf. the summary in \tabref{alienabilityinhochunk}.

\begin{table}
\caption{Alienable\is{alienable possession} vs. inalienable\is{inalienable possession} distinction in \il{Ho-Chunk}Hoocąk} \label{alienabilityinhochunk} 
\begin{tabular}[h]{c  c  c }
\multicolumn{2}{c }{inalienable/ inseparable} & alienable/ separable \\
\lsptoprule
set\textsubscript{1}: =híi & set\textsubscript{2}:  nįįh\'i & set\textsubscript{3}: hanį́ \\
\midrule
\multicolumn{1}{l }{kinship (including} & \multicolumn{1}{l }{pet animals (usually} &  \multicolumn{1}{l }{animate\is{animacy} and inanimate} \\
\multicolumn{1}{l }{close social} & \multicolumn{1}{l }{domestic animals such} & \multicolumn{1}{l }{objects such as non-domestic} \\
\multicolumn{1}{l }{relations such as} & \multicolumn{1}{l }{as dog, cat, horse,} &  \multicolumn{1}{l }{animals, artifacts, and so on} \\
\multicolumn{1}{l }{friendship)} & \multicolumn{1}{l }{etc.)} &  \multicolumn{1}{l }{including body parts\is{body-part term}} \\
\lspbottomrule
\end{tabular}
\end{table}
 
All three verbs in \tabref{alienabilityinhochunk} form the same type of nominalized verbal PC with pronominal and lexical human possessors. There is no difference between them with regard to structural markedness or with regard to the iconic relationship observed for the inalienable\is{inalienable possession} vs. alienable distinction\is{alienable possession} and the size of the corresponding PCs. The paradigms for all three verbs of possession are given below; cf. \tabref{havekin}, \tabref{havepet}, and \tabref{haveparadigm}. The paradigms contain only constructions with a \textsc{3sg} \isi{possessed} noun. If the \isi{possessed} nouns were plural (`aunts', `dogs', and `cars') the verbs of possession would be inflected for the third person plural \isi{object} (wa- \textsc{obj.3pl}). 

\begin{table}
\caption{Paradigm of the possessive verb \textit{hii} 'to have.kin'} \label{havekin}
\begin{tabular}{ l l l }
\lsptoprule
\isi{possessor}	& \isi{possessed} N \textit{hicųwį́} &  \\
\midrule
\textsc{1sg}	& hicųwį́ haa=rá	 & `my aunt' (father's sister) \\
 
\textsc{2sg} & hicųwį́ raa=ra/=gá	& `your aunt' \\
 
\textsc{3sg} & hicųwį́ hii=rá	 & `his aunt' \\
 
\textsc{1i.d} & hicųwį́ hįhi=rá /=ga & `my and your aunt' \\
 
\textsc{1i.pl} & hicųwį́ hįhiwí=ra & `our  aunt' \\ 
 
\textsc{1e.pl} & hicųwį́ haawí=ra & `our aunt' \\
 
\textsc{2pl} & hicųwį́ raawí=ra/=ga & `your aunt'\\
 
\textsc{3pl} & hicųwį́ hiíre=ra & `their aunt' \\
\lspbottomrule 
\end{tabular}
\end{table}

\begin{table}
\caption{Paradigm of the possessive verb \textit{nįįhi} 'to have.pet'} \label{havepet}
\begin{tabularx}{.67\textwidth}{ lXl }
\lsptoprule
\isi{possessor}	& \isi{possessed} N \textit{šųųk} &  \\
\midrule	
\textsc{1sg} & šųųk nįįháa=ra & `my dog' \\
 
\textsc{2sg} & šųųk nįįná=ra/=ga & `your dog' \\
 
\textsc{3sg} & šųųk nįįhí=ra	& `his dog' \\
 
\textsc{1i.d} & šųųk nįį́hi=ra/=ga & `our dog' \\
 
\textsc{1i.pl} & šųųk nįįháwi=ra & `our  dog' \\
 
\textsc{1e.pl} & šųųk nįį́hiwi=ra & `our dog' \\
 
\textsc{2pl} & šųųk nįįnáwira/=ga & `your dog' \\
 
\textsc{3pl} & šųųk nįįhíre=ra & `their dog' \\
\lspbottomrule
\end{tabularx}
\end{table}

\begin{table} 
\caption{Paradigm of the possessive verb \textit{hanį́} 'to have'} \label{haveparadigm}
\begin{tabularx}{.67\textwidth}{ lXl }
\lsptoprule
\isi{possessor}	& \isi{possessed} N \textit{waž\k{a}tíre} \\
\midrule
\textsc{1sg}	 & waž\k{a}tíre haanį́=n\k{a}\footnote{There are two phonological rules in \il{Ho-Chunk}Hoocąk a) that underlying /r/ becomes [n] after nasal vowels and b) that oral vowels are nasalized after nasal consonants. Sometimes rule a) is indicated orthographically by a ha\v{c}ek/caron <\v{n}>.} & `my car' \\
 
\textsc{2sg}	 & waž\k{a}tíre hašįnį́=n\k{a} & `your car' \\
 
\textsc{3sg}	 & waž\k{a}tíre hanį́=n\k{a} & `his car' \\
 
\textsc{1i.d}	& waž\k{a}tíre hįįnį́=n\k{a} & `our car' \\
 
\textsc{1i.pl} & waž\k{a}tíre hįįnį́wį=n\k{a} & `our car' \\
 
\textsc{1e.pl} & waž\k{a}tíre haanį́wį=n\k{a} & `our car' \\
 
\textsc{2pl}	& waž\k{a}tíre hašįnį́wį=n\k{a} & `your car' \\
 
\textsc{3pl}	& waž\k{a}tíre hanį́įne=ra & `their car' \\
\lspbottomrule
\end{tabularx}
\end{table}

The kinship term \textit{hicųwį́} `aunt (father's sister)' has a variant form that is used for address purposes, \textit{cųwį́} `(my) aunt!'. These address forms of kinship terms --- often simply lacking the initial syllable \textit{hi-} --- cannot occur in a \isi{possessive construction}. This seems to be a general rule for obvious reasons. The usage of kinship terms as address terms usually presupposes that such a \isi{kinship relation} holds between speaker and hearer. 

There is another kind of variation in the paradigm of kinship possession that may be rooted in the mutual knowledge of the interlocutors. The common determiner in possessive constructions with a kinship term is the definite article \textit{=ra}. However, in the second person singular and plural the determiner is \textit{=ga}, a deictic element also used for the indication of proper names. Lipkind claims that \textit{=ga} has to be used exclusively in these instances (cf. \citealt[31]{Lipkind1945}), but \il{Ho-Chunk}Hoocąk speakers gave me forms that show that there is actually a choice between \textit{=ra} and \textit{=ga} in the second person and in the first person inclusive dual form;\footnote{I am particularly grateful to Henning Garvin\ia{Garvin, Henning} helping me to collect the relevant forms here.} \textit{=ga} is ungrammatical in all other person categories. One of my most important language consultants, Phil Mike\ia{Mike, Phil}, indicated to me that this choice has to do with the mutual knowledge of the kinsman by both interlocutors. The definite article is used in the second person, if the speaker does not know the kinsman (assuming that the hearer knows his or her kinsmen), but \textit{=ga} is used when both interlocutors know the person talked about (which is more naturally the case if the speaker talks about the kinsman of the hearer). This could also explain why \textit{=ga} is not allowed if the \isi{possessed} is plural. The deictic suffix \textit{=ga} is also used with the address forms of kinship terms indicating the first person as \isi{possessor}. \citet[31]{Lipkind1945} says that all kin terms with initial \textit{hi-} take \textit{haará} `my' in the first person; the few ones without it take solely \textit{=ga} instead; the reason is that the shorter forms are terms of address while the \textit{hi-} forms are terms for reference. For instance, the form \textit{cųwį́} is the address term corresponding to \textit{hicųwį́} 'aunt (father's sister)'. Hence the \textsc{1sg} possessive form is \textit{cųwį́-gá} which translates literally `that aunt' implying that everybody knows that she is the aunt of the speaker (EGO). It is a kind of reduced form of speaking. The address term implies that the person so addressed has the kin relation designated by the term toward the speaker. It is an effect of the Animacy Hierarchy\is{animacy}. Shared background knowledge of the \isi{possessor} plays an important role here (cf. also \citealt[26f]{Heine1997}).\footnote{This can also be interpreted as an instance where the inherent relationality of kin terms leads to a structural reduction of the expression of possession confirming the prediction of the prototype approach.}

\section{Constructional splits in the other Siouan languages}

In what follows a few other Siouan languages are examined with regard to constructional splits that have to do with the NP type of the \isi{possessor} and the semantics of the \isi{possessed}. I will begin with the Northwestern Siouan languages \il{Apsaalooke}Crow, \ili{Hidatsa}, and \ili{Mandan} (\sectref{crow}--\ref{mandan}), then I will continue with \ili{Lakota} (the \ili{Dakotan} sub-branch of Mississippi-Valley Siouan; \sectref{lakota}) and \ili{Osage} (\ili{Dhegiha} sub-branch of Mississippi-Valley Siouan; \sectref{osage}), and I will close this investigation with \ili{Biloxi} as a representative of the South-Eastern branch of Siouan (Ohio-Valley Siouan; \sectref{biloxi}).
 
\subsection{Crow}\label{sec:helmbrecht:4.1} \label{crow} 
\subsubsection{The possessor}
\il{Apsaalooke}Crow has four different NP-internal PCs depending on the semantic/syntactic nature of the \isi{possessor}; cf. the examples in \REF{crowfather} through \REF{crowdeer}. 

\ea \il{Apsaalooke}Crow (\citealt[234]{Graczyk2007}) \label{crowfather}
\ea {[Poss.Pro — N\textsubscript{possessed}]}
\ex
\gll Ø-iilápxe \\
\textsc{3sg.poss}-father \\
\glt `his father'
\z
\z

\ea \il{Apsaalooke}Crow (\citealt[234]{Graczyk2007}) \label{crowcharlie}
\ea {[N\textsubscript{possessor}(-DET/-Ø) Poss.Pro --- N\textsubscript{possessed}]}
\ex
\gll Charlie-sh 		Ø-iilápxe \\
C.-\textsc{det}      \textsc{3sg.poss}-father \\
\glt `Charlie's father'
\z \z

\ea	\il{Apsaalooke}Crow \citep[235]{Graczyk2007} \label{crowbrother}
\ea {[Emphatic PRO-POSS.PRO-N\textsubscript{possessed}]}
\ex 
\gll bii-w- achuuké \\
\textsc{1sg.emph-1sg.poss}-younger.brother \\
\glt `MY younger brother'
\z \z

\ea \il{Apsaalooke}Crow (\citealt[236]{Graczyk2007}) \label{crowdeer}
\ea {[[[N\textsubscript{possessor}] [N\textsubscript{possessor}]] [N\textsubscript{possessed}]]}
\ex 
\gll úuxbishke            chíis-uua iía \\
white.tailed.deer tail-\textsc{pl}      hair \\
\glt `hair from the tail of the white-tail deer'
\z \z

No matter whether the \isi{possessed} noun is alienable\is{alienable possession} or inalienable\is{inalienable possession}, there has to be a \isi{possessive pronoun} attached to the \isi{possessed} noun indicating the \isi{possessor} (cf. example \REF{crowfather}). The same is true if there is a lexical \isi{possessor} in addition (cf. example \REF{crowcharlie}). The possessive prefix may be emphasized by means of a bound emphatic pronoun prefixed to the possessive prefix (cf. example \REF{crowbrother}). Interestingly, there are also PCs that do not show any possessive marking and hence look like a juxtaposition expressing a whole-part relationship, cf. the example in \REF{crowdeer}. I did not find more examples like this in Graczyk's grammar, so I cannot say if this is generally an alternative possibility or required for non-human possessors.

\subsubsection{The possessed}
\il{Apsaalooke}Crow has different paradigms of proper bound possessive pronouns distinguishing different sets of \isi{possessed} nouns according to the alienable versus inalienable distinction. The paradigm of possessive pronouns for \isi{alienable possession} is given in \tabref{crowalienablepossession}; the paradigm of \isi{inalienable possession} is given in \tabref{crowinalienablepossession}.

\begin{table}
\caption{Alienable possession in \il{Apsaalooke}Crow (\citealt[53]{Graczyk2007})} \label{crowalienablepossession}
\begin{tabular}{ l l l}
\lsptoprule
& stem \textit{íilaalee} & \\
\midrule 	
\textsc{1sg} & \textbf{ba-s}-íilaalee	& `my car(s)' \\
 
\textsc{2sg} & \textbf{dí-s}-iilaalee & `your car(s)' \\
 
\textsc{3sg} & \textbf{i-s}-íilaalee & `his/her car(s)' \\
 
\textsc{1i.pl} & \textbf{balee-is}-íilaalee & `our car(s)' \\
 
\textsc{1e.pl} & \textbf{ba-s}-íilaalee-\textbf{o} & `our car(s)' \\
 
\textsc{2pl} & \textbf{dí-s}-iilaalee-\textbf{o} & `your car(s)' \\
 
\textsc{3pl} & \textbf{i-s}-íilaalee-\textbf{o} & `their car(s)' \\
\lspbottomrule
\end{tabular}
\end{table}

\begin{table}
\caption{Inalienable possession\is{inalienable possession} in \il{Apsaalooke}Crow (\citealt[52]{Graczyk2007})} \label{crowinalienablepossession}
\begin{tabular}{ l l l }
\lsptoprule
& stem \textit{apá}  & \\
\midrule	
\textsc{1sg} & \textbf{b}-apé & `my nose' \\
 
\textsc{2sg} & \textbf{d}-ápe	 & `your nose' \\
 
\textsc{3sg} & \textbf{Ø}-apé & `his/her nose' \\
 
\textsc{1i.pl}	& -	& - \\
 
\textsc{1pl} & \textbf{b}-ap-\textbf{úua}	& `our noses' \\
 
\textsc{2pl} & \textbf{d}-áp-\textbf{uua}	& `your noses' \\
 
\textsc{3pl} & \textbf{Ø}-ap-\textbf{úua}	& `their noses' \\
\lspbottomrule
\end{tabular}
\end{table}

The possessive pronouns of \isi{alienable possession} in \tabref{crowalienablepossession} are formally invariable; they have an additional /\textit{-s}/ thus being phonologically more marked than the prefixes of the inalienable paradigm. The \textsc{2sg} \isi{possessive pronoun} of the alienable paradigm shows a shift of the primary \isi{stress} from the stem to the prefix, a pattern which is found also in some of the active verb paradigms. The \textsc{1i.pl} prefix \textit{balee-} is taken from the B-set pronominal paradigm for stative verbs. This form is added to the \textsc{3sg.poss} \textit{is-} prefix, probably a late innovation introducing a \textsc{1pl} inclusive-exclusive distinction into the alienable paradigm. This distinction is lacking in the inalienable\is{inalienable possession} paradigm of possessive pronouns as well as in the verbal paradigms. The suffixes in both paradigms (\textit{-o} in the alienable possessive paradigm, \textit{-úua} in the inalienable possessive paradigm) indicate the plurality of the \isi{possessor}.

The paradigm of \isi{inalienable possession} varies in form depending on the stem-initial sounds. There are three phonologically conditioned allomorphic paradigms, for stems in /d-/, /i+consonant-/, and /vowel-/. As can be seen in \tabref{crowinalienablepossession}, the stem itself also undergoes some sound changes.

There are, however, three additional paradigms of \isi{inalienable possession}: a) one that marks possession with the undergoer series of pronominal prefixes (called B-set of pronominal prefixes in Graczyk's grammar), b) one with an irregular paradigm, and c) one residual paradigm that shows stem suppletion. \citet[57]{Graczyk2007} finds the following classification of nouns associated with these three different inalienable paradigms.

\begin{description}
\item[a)] Inalienable possession\is{inalienable possession} with the B-set prefixes is used with nouns referring to internal body parts\is{body-part term} such as `gland', `joint', `limb', `hip', `bone', `lung', `stomach', etc. (cf. \citealt[57]{Graczyk2007}).

\item[b)] There are not enough nouns requiring the irregular paradigm for a semantic classification, but they all seem to belong semantically rather to the inalienable class of nouns, though;

\item[c)] The nouns that require suppletive stems refer to kinship relations, clothing, and some culturally important possessions, cf. the examples in \tabref{crowsuppletion}. The first column shows the nouns in citation form, the second column in a \isi{possessive construction}. The corresponding stems are clearly suppletive. 
\end{description}

\begin{table}
\caption{Suppletive stems in \il{Apsaalooke}Crow (\citealt[58]{Graczyk2007})} \label{crowsuppletion}
\begin{tabular}[h]{ l l l l}
\lsptoprule
 ihkáa	 & `mother'	& is-ahká	& `his mother \\
 huupá & `shoe' &  is-ahpá	 &  `his shoe'\\
alúuta &  `arrow' & is-aá &  `his arrow'\\
buú & `song'	&  is-huú & `his song'\\
\lspbottomrule
\end{tabular}
\end{table}

There is also a prefix \textit{bale-} that is used if inalienable\is{inalienable possession} nouns are used without indicating a \isi{possessor}. This form is called depossessivizer in \citet[53/234]{Graczyk2007} and it is obligatorily used with un\isi{possessed} body-part nouns\is{body-part term}. This form is not used with kinship terms.

\tabref{crowalienability} summarizes the findings with regard to the alienable\is{alienable possession}/inalienable distinction. Inalienable nouns are a closed class of nouns in \il{Apsaalooke}Crow. It is clear that the semantic classification of the nouns with regard to the different PCs is not sharp. There are even body-part nouns\is{body-part term} that belong to the alienable class (set\textsubscript{5}). Gross modo, however, the nouns in set\textsubscript{1} - set\textsubscript{4} could be subsumed under a class of inalienable nouns semantically.

\begin{table}
\caption{Alienable vs. inalienable distinction in Crow} \label{crowalienability}  
\begin{tabularx}{\textwidth}{ Xp{2cm}XXcX }
\lsptoprule
\multicolumn{4}{c}{inalienable } && \multicolumn{1}{c}{alienable} \\
\hhline{----~-}
\multicolumn{1}{c}{set\textsubscript{1}} &
\multicolumn{1}{c}{ set\textsubscript{2}}	& 
\multicolumn{1}{c}{set\textsubscript{3}} & 
\multicolumn{1}{c}{set\textsubscript{4}} && 
\multicolumn{1}{c}{set\textsubscript{5}} \\
\midrule
phonologically conditioned inalienable\is{inalienable possession} paradigm & 	B-set  \mbox{prefixes} & irregular  paradigm & suppletive \isi{possessed} forms && alienable\is{alienable possession} paradigm \\ 
\\

body parts\is{body-part term}, kinship & 
 \raggedright closed class  of nouns  referring to  internal body  parts & 
`chest', `tail', `husband'	  & 
\raggedright closed class of nouns  referring to objects  closely associated to a person (e.g. clothing, a few kin terms, culturally  important possessions)  && 
open class of nouns not inherently \isi{possessed}; exceptions are: \textit{huli}  `bone', \textit{íili} `blood',  \textit{kahkahká} `forearm' and a  few others. \\ 
\lspbottomrule
\end{tabularx}  
\end{table}
 
\subsection{Hidatsa}\label{sec:helmbrecht:4.2} \label{hidatsa} 
\subsubsection{The possessor}
\ili{Hidatsa} and \il{Apsaalooke}Crow are closely related and constitute the Missouri Valley sub-branch of Siouan. Although they belong to the same sub-branch of Siouan, there are differences in the expression of possession. \ili{Hidatsa} has different PCs depending on the syntactic/semantic type of the \isi{possessor}. As in \il{Apsaalooke}Crow, there is an obligatory marking of the \isi{possessor} on the \isi{possessed} noun no matter whether the \isi{possessed} noun is alienable\is{alienable possession} or inalienable\is{inalienable possession}; cf. the alienable PC in \REF{hidatsahisdog}). If there is an additional lexical \isi{possessor}, the structure of the PC in \ili{Hidatsa} is analog to the one in \il{Apsaalooke}Crow, cf. the alienable PC in \REF{hidatsamansdog}.

\ea \ili{Hidatsa} (\citealt[81]{Boyle2007})
\ea \label{hidatsamansdog}
\glll macée idawashúga \\ 
wacée ita=wašúka \\
man  \textsc{3sg.poss}=dog \\
\glt `man's dog'
\ex \label{hidatsahisdog}
\glll idawashúga \\
ita=wašúka \\
\textsc{3sg.poss}=dog \\
\glt `his dog'
\z \z

\citet{Boyle2007} does not mention in his grammar of \ili{Hidatsa} whether there exists a juxtaposition of \isi{possessor}-\isi{possessed} as another possible PC in \ili{Hidatsa}. One of the peculiarities of PCs in \ili{Hidatsa} is that they can freely be modified by a definite article and/or a demonstrative pronoun. Since there are a lot of similarities between \il{Apsaalooke}Crow and \ili{Hidatsa}, the discussion of the properties of the \isi{possessed} will be brief.

\subsubsection{The possessed}

As in \il{Apsaalooke}Crow, there are two paradigms of possessive pronouns in \ili{Hidatsa}, one indicating \isi{inalienable possession}, the other \isi{alienable possession}; cf. \tabref{hidatsapronouns}.

\begin{table}
\caption{Alienable and inalienable possessive pronouns (\citealt[72]{Boyle2007}; 80)} \label{hidatsapronouns}
\begin{tabular}{l l l l l}
\lsptoprule
& \multicolumn{2}{c }{inalienable possessive pronouns} & \multicolumn{2}{c}{alienable possessive pronouns} \\
\midrule
1 & ma-  /wa-/	 & `my'	 & mada= /wa-ta=/ & `my' \\
 
2 & ni-     /ri-/	& `your	& nida=   /ri-ta=/	& `your \\
 
3 & i-       /i-/	& `his, her'	& ida=     /i-ta=/	& `his, her' \\
\lspbottomrule
\end{tabular}
\end{table}


The paradigm for \isi{inalienable possession} shows --- as with \il{Apsaalooke}Crow set\textsubscript{1} nouns --- phonologically conditioned allomorphy (stem-initial vowel vs. stem-initial consonant, and /r/-initial stems). It seems that there is no semantic sub-classification associated with the allomorphy in the inalienable prefixes and the corresponding irregularities. Therefore, I lumped these different formal properties of inalienable nouns together in one set\textsubscript{1} class of nouns in \tabref{hidatsaalienability}.
	
However, there are also differences. For instance, the 2POSS forms do not trigger a shift in \isi{stress} assignment as in \il{Apsaalooke}Crow, and the inalienable\is{inalienable possession} possessive prefixes are true prefixes, whereas the corresponding alienable\is{alienable possession} forms are analyzed as clitics.  The alienable forms are identical to the ones for \isi{inalienable possession} plus /\textit{ta-}/ which can be found in other Siouan languages as well (cf. e.g. in \ili{Lakota} alienable PCs of set\textsubscript{4} nouns which have a \textit{-t\textsuperscript{h}a} prefix added to the undergoer pronominal prefix; cf. \tabref{lakotaalienability} below). There is no mention of a depossessivizer in Boyle's grammar of \ili{Hidatsa}.

\begin{table}
\caption{Alienable vs. inalienable distinction in Hidatsa} \label{hidatsaalienability}
\begin{tabular}{ l l }
\lsptoprule
inalienable\is{inalienable possession} & alienable\is{alienable possession} \\
 \multicolumn{1}{c}{set\textsubscript{1}} &  \multicolumn{1}{c}{set\textsubscript{2}} \\
\midrule
&\\
inalienable paradigm (including & alienable paradigm (no allomorphy) \\
phonologically conditioned allomorphy & \\
and some irregular forms) & \\
 & \\
 closed class of nouns: body parts\is{body-part term}, & open class of nouns not inherently \\
many kinship terms, some clothing &  \isi{possessed} \\
items & \\
\lspbottomrule
\end{tabular}
\end{table}
 
\subsection{Mandan}\label{sec:helmbrecht:4.3} \label{mandan}

\ili{Mandan} is considered a proper sub-branch of Siouan neither belonging to the Missouri Valley nor the Mississippi Valley group of Siouan. 

The semantic/syntactic properties of the \isi{possessor} and their possible effects on the choice of the PC are not discussed and described in Mixco's grammatical sketch (\citealt{Mixco1997a}). However, looking into the appended \ili{Mandan} text, it seems that juxtapositions are possible in case  the \isi{possessor} is a lexical noun. There is at least one clear example of this construction (cf. \REF{villagechief}) that shows that association may be expressed by this PC.

\ea \ili{Mandan} (\citealt[70: text line 24]{Mixco1997a}) \label{villagechief}

\gll 'wį=ti   rų'wąʔk=ši-s \\
village man=good-\textsc{def} \\
\glt `the village chief'
\z

If the \isi{possessor} is a speech act participant or a third person, one of the following distinct PCs has to be used. In one construction the possessive pronominal affixes, which are in principle identical to the undergoer series of pronominal affixes (called `stative' in \citealt[44]{Mixco1997a}) are attached directly to the noun stem that designates the \isi{possessed} [\textsc{poss}-N\textsubscript{stem}]\textsubscript{inalienable possession}. This construction is used for \isi{inalienable possession}; see the relevant forms in \tabref{mandanpossaffixes}.

\begin{table}
\caption{Possessor affixes in \ili{Mandan} (\citealt[16f,44]{Mixco1997a})} \label{mandanpossaffixes}
\begin{tabular}{l l l }
\lsptoprule
& \textsc{sg} & \textsc{pl} \\
\midrule
1 & wį-\footnote{Note that this form of the \textsc{1sg.poss} differs from the corresponding form of the undergoer series, which is w\k{a}-. Mixco speculates that the w\k{i}- form is a contraction of w\k{a}- + i- for the third person, but provides no evidence for this idea.}  & ro:- \\
 
2 & rį- & rį-stem-rįt \\
 
3 & i- & -kræ\footnote {Mixco does not give the full paradigm, neither for the stative or undergoer affixes nor for the possessive affixes. This is the reason for the question mark. In addition I did not find a single example in Mixco's sketch of \ili{Mandan} that corresponds to `their Y'. Note, however, that \citet[8]{Kennard1936} gives the form \textit{-k\textipa{E}r\textipa{E}} for the \textsc{3pl} possessive affix. The forms are identical, but the \isi{transcription} is different.} \\
\lspbottomrule
\end{tabular}
\end{table}

The second PC inserts a prefix \textit{ta-} between the stem and the possessive prefix [\textsc{poss}-\textit{ta}-N\textsubscript{stem}]\textsubscript{alienable possession}. This construction is used for \isi{alienable possession}. The form \textit{ta-} as an alienable marker is cognate to \ili{Lakota} \textit{t\textsuperscript{h}á-}, see below. The possessive prefixes are the same as in the inalienable PC, see \tabref{mandanpossaffixes}.

There are some peculiarities with PC for \isi{inalienable possession}. First, there are some kinship terms that require a prefix \textit{ko-} for third person \isi{possessor}. I suppose this form is related historically to \textit{ku-/tku-} in \ili{Lakota}. Secondly, there are kinship terms and a few other alienable terms (old nominalized verb forms) that take the actor series of pronominal prefixes in order to express the \isi{possessor}. For instance, the kinship term for `mother' takes the usual undergoer series of prefixes for \isi{inalienable possession}, but requires \textit{ko-} for the third person \isi{possessor}; cf. \REF{mandanmother}. 

\ea \ili{Mandan} (\citealt[45]{Mixco1997a}) \label{mandanmother}
\ea
\gll wį-hų:-s \\			
\textsc{1sg.poss}-mother-\textsc{def} \\
\glt `my mother'
\ex \gll rų-hų:-s \\
\textsc{2sg.poss}-mother-\textsc{def} \\
\glt `your mother'
\ex \gll ko-hų:-s \\
\textsc{3sg.poss}-mother-\textsc{def} \\
\glt `his mother'
\z \z

The term for `father', on the other hand, requires the actor series of pronominal affixes in \ili{Mandan} in order to express the \isi{possessor}, cf. the examples in \REF{mandanfather}.

\ea \ili{Mandan} (\citealt[45]{Mixco1997a}) \label{mandanfather}
\ea \gll wa-aʔt-s \\
\textsc{1sg.a}-father-\textsc{def} \\
\glt `my father'
\ex \gll a-aʔt-s \\
\textsc{2sg.a}-father-\textsc{def} \\
\glt `your father'
\ex \gll ko-aʔt-s \\
\textsc{3sg.a}-father-\textsc{def} \\
\glt `his father'
\z \z

Interestingly, no mention is made of the way body parts\is{body-part term} are \isi{possessed} in \ili{Mandan}. A quick look into the \ili{Mandan} text (cf. \citealt[66ff]{Mixco1997a}) reveals that body-part nouns never occur in one of the above described PCs with possessive affixes. They appear always without the \textit{ta-} form and never carry any possessive affixes. The \isi{possessor} always has to be inferred from the text.
 
\subsection{Lakota}\label{sec:helmbrecht:4.4} \label{lakota} 
\subsubsection{The possessor}
\ili{Lakota} is a language of the \ili{Mississippi Valley Siouan} languages, more specifically of the \ili{Dakotan} sub-branch of this group. \ili{Lakota} does employ possessive pronouns, which are almost entirely identical to the set of undergoer pronominal prefixes in stative/inactive verbs. If the \isi{possessor} is a SAP/pronoun and the \isi{possessed} noun belongs to the class of alienable\is{alienable possession} nouns, the following constructions may be used. Note that the \textsc{1sg.poss} \textit{mi-} is a special form that does not correspond to the regular \textsc{1sg} form of the pronominal undergoer prefixes (\textit{ma-}).\footnote{Data in this section has been re-spelled in the current \ili{Lakota} \isi{orthography}.}

\vspace{1em}
\textbf{a)	Ownership}

[N\textsubscript{possessed-inanim} PRO.POSS-HAVE DET]
\vspace{1em}

\ea \ili{Lakota} (\citealt[98]{Buechel1939}) \label{lakotahouse}

\gll thípi     mi-t\v{h}áwa ki\textipa{N} \\
house \textsc{1sg}-have  \textsc{def} \\
\glt `my house'
\z
\vspace{1em}
\textbf{b) Ownership, attribution of property}

[PRO-\textit{t\v{h}a}-N\textsubscript{possessed-inanim/abstr} DEF]
\vspace{1em}
\ea \ili{Lakota} (\citealt[98]{Buechel1939}) \label{lakotalandwisdom}

\ea \gll mi-t\v{h}á-mak\v{h}o\v{c}he   ki\textipa{N} \\
\textsc{1sg.poss}-land   \textsc{def} \\
\glt `my land'
\ex  
\glll nit\v{h}óksape ki\textipa{N} \\
 ni-t\v{h}á-wóksape   ki\textipa{N} \\
\textsc{2sg-poss}-wisdom  \textsc{def} \\
\glt `thy wisdom'
\z \z

There is no information about the conditions or the differences between the two constructions; it is clear that the one in \REF{lakotahouse} contains a stative verb of possession \textit{t\v{h}áwa-} `have' that is nominalized in this context inflecting for the person and number of the \isi{possessor} and the number of the \isi{possessed}. In \citet[458]{RoodTaylor1996} it is said that the stative verb of possession \textit{it\v{h}áwa} `have' depends only on the category of the \isi{possessor} in this PC and not on the number of the \isi{possessed}. It seems that this stative verb of possession has been grammaticalized towards a marker of possession quite recently in \ili{Lakota}. 

The PCs in \REF{lakotalandwisdom} contain a marker for possession \textit{t\v{h}á-} `POSS' which is attached to the \isi{possessed} noun and preceded by the pronominal affix of the \isi{possessor}. This marker is common Siouan (cf. \citealt{RankinEtAl2015AccessMay}). If there are lexical nouns expressing the \isi{possessor}, the following PCs are used.

\vspace{1em}
\textbf{c)	Ownership}

[N\textsubscript{possessed-anim} N\textsubscript{possessor-PROP} PRO.POSS-HAVE DEF] 
\ea	\ili{Lakota} (\citealt[91]{Buechel1939}) \label{lakotadavidshorse}

\gll {\v{s}\'u\textipa{N}ka   wak\v{h}\'a\textipa{N}}   David   Ø-t\v{h}áwa    ki\textipa{N} \\
horse      D.   \textsc{3sg}-have  \textsc{def} \\
\glt `David's horse'
\z

[N\textsubscript{possessed-anim} N\textsubscript{possessor-PROP} PRO.POSS-HAVE DEF]

\ea \ili{Lakota} (\citealt[91]{Buechel1939})

\gll {\v{s}\'u\textipa{N}ka wak\v{h}\'a\textipa{N}} Peter  na  Paul   Ø-t\v{h}áwa-pi     ki\textipa{N}  \\
horse    P.       and P.      \textsc{3sg}-have-\textsc{pl} \textsc{def} \\
\glt `Peter and Paul's horses (or horse)'
\z
\textbf{d)	Association}

[N\textsubscript{possessor-PROP} PRO.POSS-\textit{t\v{h}a}-N\textsubscript{possessed-hum} DEF]

\ea	\ili{Lakota} (\citealt[92]{Buechel1939})

\gll It\v{h}\'a\textipa{N}\v{c}ha\textipa{N}    Ø-t\v{h}a-wóilake   ki\textipa{N}  \\
Lord   \textsc{3sg-poss}-servant   \textsc{def} \\
\glt `the Lord's servant'
\z

[N\textsubscript{possessor-PROP} PRO.POSS-\textit{t\v{h}a}-N\textsubscript{possessed-hum} DEF]

\ea	\ili{Lakota} (\citealt[92]{Buechel1939}) \label{lakotaabraham}

\gll Abraham Ø-t\v{h}a-w\'amak\v{h}a\v{s}ka\textipa{N}-pi ki\textipa{N}  \\
 A. \textsc{3sg-poss}-animal-\textsc{pl}  \textsc{def} \\
\glt `Abraham's animals'
\z

Again we have two different PCs in the examples \REF{lakotadavidshorse}-\REF{lakotaabraham} with a lexical \isi{possessor}, one with a verb of possession that is nominalized, and the other exhibiting a morphological \isi{possessor} marking on the \isi{possessed} noun. These examples represent \isi{alienable possession}s. It can be concluded that the syntactic status of the \isi{possessor} does not play a role for the choice of the PCs.

If the relation between the \isi{possessor} and the \isi{possessed} is a \isi{whole-part relation}, or a partitive relation, or the \isi{possessor} noun is an abstract noun or a nominalization, the following constructions are used. 

\vspace{1em}
\textbf{e)	Whole-part relationships}

[N\textsubscript{possessor-inanim} N\textsubscript{possessed-anim} DEF] (juxtaposition)

\ea \ili{Lakota} (\citealt[92]{Buechel1939})

\gll ma\v{h}píya zitkála-pi  ki\textipa{N}  \\
cloud      bird-\textsc{pl} \textsc{def} \\
\glt `the birds of the air'
\z

[N\textsubscript{possessor-inanim} N\textsubscript{possessed-inanim} INDEF] 

\ea \ili{Lakota} (\citealt[92]{Buechel1939})

\gll  \v{c}he\v{h} \'ik\v{h}a\textipa{N} wa\textipa{N} \\  
bucket rope \textsc{indef} \\
\glt `a bucket handle, rope of a bucket' 
\z 

\textbf{f)	Partitive}

\ea \ili{Lakota} (\citealt[93]{Buechel1939}) \label{lakotamanychiefs}

\gll it\v{h}\'a\textipa{N}\v{c}ha\textipa{N}pi k\k{i} \'ota \\
chiefs \textsc{def} many \\
\glt `many of the chiefs'
\z

Example \REF{lakotamanychiefs} is not really a PC, but a regular quantified NP. The same holds for \REF{lakotagoodworks}. It can hardly be considered a PC. It is rather a juxtaposition expressing a NP (`good works') modifying another NP (`man').

\vspace{1em}

\textbf{g)	With an abstract \isi{possessor} N}

\ea \ili{Lakota} (\citealt[93]{Buechel1939}) \label{lakotagoodworks}

\gll wi\v{c}háša o\v{h}'a\textipa{N}     wašté ki\textipa{N}     hé\v{c}ha \\
man       in.actions good  \textsc{def} such \\
\glt `a man of good works'
\z

\subsubsection{The possessed}
There are different PCs according to the semantic type of the \isi{possessed} noun; body-part terms\is{body-part term} are simply affixed by the pronominal series of undergoer prefixes. Among the body-part terms, there is a split between body parts that are ``conceived as particularly subject to willpower" (\citealt[128]{BoasDeloria1941}), and the others. \citet[100]{Buechel1939} describes this difference as ``possession of one's incorporeal constituents" versus ``possession of one's body and its physical parts"; compare the examples in \REF{lakotaincorporeal} and \REF{lakotabody}.

\ea \ili{Lakota} (\citealt[101]{Buechel1939}) \label{lakotaincorporeal}

\textit{mi-ná\v{g}i ki\textipa{N} } \hspace{2.1em}		`my souls'

\textit{mi-\v{c}há\v{z}e  ki\textipa{N} } \hspace{1.9em} `my name'

\textit{mi-ó\v{h}'a\textipa{N} ki\textipa{N}}	\hspace{2em}		`my occupation'

etc.

\ex \ili{Lakota} (\citealt[100]{Buechel1939}) \label{lakotabody}

\textit{ma-\v{c}hé\v{z}i ki\textipa{N} } \hspace{1.7em}	`my tongue'

\textit{ma-íšta ki\textipa{N} }	 \hspace{2.1em}	`my eye'

\textit{ma-sí ki\textipa{N} }		 \hspace{3em}	`my foot'

etc.
\z

Note that this distinction has become partially obsolete in contemporary \ili{Lakota}. \citet[458]{RoodTaylor1996} note that this distinction is semantically maintained only in the Oglala variety of \ili{Lakota}. There \textit{ma-} (\textsc{1sg.poss}) is used for ``concrete visible possessions", and \textit{mi-} (\textsc{1sg.poss}) for ``intangibles" (cf. \citealt[458]{RoodTaylor1996}). Otherwise, both forms are in free variation.

Kinship relations with a \isi{possessor} of the first and second person are expressed solely by the possessive prefixes. A \isi{possessor} of the third person requires an additional marker \textit{–ku, -tku, -\v{c}u} which is suffixed to the \isi{possessed} kinship term; cf. \REF{lakotamygrandfather}.

\ea	\ili{Lakota} (\citealt[102]{Buechel1939}) \label{lakotamygrandfather}

\textit{mi-t\v{h}\'u\textipa{N}kašila} \hspace{3.6em}		`my grandfather'
 
\textit{ni-t\v{h}\'u\textipa{N}kašila ki\textipa{N} } \hspace{2.3em}	`thy grandfather'

\textit{Ø-t\v{h}\'u\textipa{N}kaši\textbf{tku} ki\textipa{N} }	\hspace{1.8em} `his/her grandfather'
\z

\tabref{lakotaalienability} summarizes the findings. As was mentioned above, the set\textsubscript{1} and set\textsubscript{2} \isi{possessed} nouns are no longer separated formally in \ili{Lakota} (except for Oglala).

\begin{table}
\caption{Alienable\is{alienable possession} vs. inalienable distinction\is{inalienable possession} in \ili{Lakota} (\citealt[127--133]{BoasDeloria1941})} \label{lakotaalienability}
\small
\begin{tabular}{ l l l l }
\lsptoprule
\multicolumn{2}{l}{inseparable/inalienable } & \multicolumn{2}{l} {separable/alienable} \\
\midrule
 \multicolumn{1}{c}{set\textsubscript{1}} &  \multicolumn{1}{c}{set\textsubscript{2}}	&  \multicolumn{1}{c}{set\textsubscript{3}} &  \multicolumn{1}{c}{set\textsubscript{4}} \\
\midrule
body-part terms\is{body-part term}  & body-part terms & kinship relations & distal affinal kinship  \\
{[+control]} & [-control] & & terms prototypical  \\
{[incorporeal} & [physical parts] & ownership & \\
constituents] & kidney, knee, liver, & & \\
mouth, lips, facial & lungs, blood, etc. & & \\
expression, eye, & & & \\
arm, voice, hand, & & & \\
spirit, etc. 	 & & & \\
\midrule
 \multicolumn{1}{c}{PC}	&  \multicolumn{1}{c}{PC} &  \multicolumn{1}{c}{PC} &  \multicolumn{1}{c}{PC} \\
\midrule
{[\textsc{pro.poss}-noun]} & [\textsc{pro.poss}-noun] & [1./2.\textsc{poss}-noun] & [\textsc{pro.poss} -\textit{t\textsuperscript{h}a}-noun] \\
with a special form & & [\textsc{3.poss}-noun\textit{-ku}] & [noun \textsc{pro.poss}\textit{-t\textsuperscript{h}a'wa}] \\
in the \textsc{1sg.poss}  & & -\textit{tku}]  & \\
(\textit{mi-}) &  &  \textit{-cu}]	& \\
\lspbottomrule
\end{tabular}
\end{table}

As in \il{Ho-Chunk}Hoocąk, the causative verb is used for the clause-level predicative expression of possession of a kinship term, cf. \REF{lakotahavegrandfather}.

\ea \ili{Lakota} (\citealt[102]{Buechel1939}) \label{lakotahavegrandfather}

\ea \gll t\v{h}u\textipa{N}kášila-wa-ya \\
grandfather-\textsc{1sg.a}-have.kin \\
\trans `I have (him) as grandfather'
\ex \gll t\v{h}u\textipa{N}kášila-u\textipa{N}-ya\textipa{N}-pi \\
grandfather-\textsc{1i.a}-have.kin-\textsc{pl} \\
\glt `We have (him) as grandfather'
\z \z

I found no example showing that this verb of possession could be used like the alienable\is{alienable possession} verb of possession \textit{t\v{h}\'awa} illustrated in \REF{lakotahouse} above. If this were the case, we would have a quite similar opposition of verbs of possession in \ili{Lakota} as we found in \il{Ho-Chunk}Hoocąk. 

In addition, it should be mentioned that \ili{Lakota} allows the non-modifying auto-referential usage of the possessive pronouns, however only the expressions based on the verb of possession \textit{t\v{h}\'awa} plus the definite article. This could be interpreted as a nominalized possessive predication; cf. \REF{lakotatookmine}.\footnote{One of the reviewers mentioned that \textit{mit\v{h}áwa kį he} could be analyzed as a null head \is{clauses, relative}relative clause. This is probably the best way to treat it. It does not, however, change the argument here. The example only demonstrates that a nominal expression for the \isi{possessed} is not required in this \isi{possessive construction}.}

\ea	\ili{Lakota} (\citealt[22]{Buechel1939}) \label{lakotatookmine}

\gll mit\v{h}áwa ki\textipa{N}      h\'e   ahí     i\v{c}ú \\
mine        \textsc{def} she came take \\
\glt `She came and took mine'
\z

Interestingly, this is a PC in which there is no \isi{possessed} noun. All other PCs discussed so far require a \isi{possessed} lexical noun.
 
\subsection{Osage}\label{sec:helmbrecht:4.5} \label{osage}

\ili{Osage} is taken as a representative of the \ili{Dhegiha} sub-branch of \ili{Mississippi Valley Siouan}. It was chosen because there is a recent extensive grammatical description of this language (\citealt{Quintero2004}). Unfortunately, it is difficult to find the relevant data out of Quintero's grammar of \ili{Osage}. There is no specific chapter on possession, and there is no index in the grammar. Quintero uses the terms alienable\is{alienable possession} and inalienable\is{inalienable possession}, but it is not made explicit which nouns are alienable and which are inalienable. However, some conclusions about this question can be drawn from the numerous examples provided by the grammar.
There is a special construction for PCs with \isi{possessed} kinship nouns. Kinship nouns are inflected with a series of inalienable pronominal prefixes, cf. \tabref{osageinalienable}.

\begin{table}
\caption{Inalienable possessive prefixes for kinship terms in \ili{Osage} (\citealt[481]{Quintero2004}f)} \label{osageinalienable}
\begin{tabular}{ l l l l }
\lsptoprule
Possessor	 & inalienable & example & translation \\
& prefix paradigm & & \\
\midrule
\textsc{1sg} & wi-	& wi-sǫ́ka & `my (male's) younger brother' \\
 
\textsc{2sg} & ði- & ði-sǫ́ka & `your (male's) younger brother' \\
 
\textsc{3sg} & i- & i-sǫ́ka & `his (male's) younger brother' \\
 
\textsc{1pl} & does not exist &	- & - \\
 
\textsc{2pl} & ? & ? & \\
 
\textsc{3pl} & ? & ? & \\
\lspbottomrule
\end{tabular}
\end{table}
 
The question marks in \tabref{osageinalienable} indicate that Quintero did not provide the expected forms. In addition, PCs with \isi{possessed} body-part nouns are not provided either.

Alienable nouns require another construction, which has the following properties. There is a pronominally inflected (bound) stem \textit{-hta}, which marks possession.\footnote{Again, this is the Common Siouan marker for \isi{alienable possession} (cf. \citealt{RankinEtAl2015AccessMay}).} The pronominal prefixes resemble the ones used for the PCs with \isi{possessed} kinship terms, with one exception. There is a dual and plural form for the first person, which does not exist in the PCs with \isi{possessed} kinship terms. The inflected possessive form follows the \isi{possessed} noun; cf. the examples in \REF{osagegroceries} and \REF{osagemaryandjohn}. The full paradigm is given in \tabref{osagealienableposs}.

\ea	\ili{Osage} (\citealt[298]{Quintero2004}) \label{osagegroceries}

\gll ówe  che    \textbf{hc\'i} \textbf{ \k{a}kóhta-api}  aðį́-ahi-a \\
groceries  those \textbf{house} \textbf{\textsc{1pl.poss-pl}}   have-arrive.there-\textsc{imp} \\
\glt `Bring those groceries to our house!'

\ex \ili{Osage} (\citealt[299]{Quintero2004}) \label{osagemaryandjohn}

\gll Máry Jóhn-a  \textbf{hc\'i}  \textbf{íhta-api} \\
M.      J.-\textsc{syl}   \textbf{house} \textbf{\textsc{3sg.poss-pl}} \\
\glt `Mary and John's house'
\z

\begin{table}
\caption{Alienable possession in \ili{Osage} (\citealt[297]{Quintero2004}f)} \label{osagealienableposs}
\begin{tabular}{ l l l l }
\lsptoprule
& \isi{possessed} & \isi{possessor} & translation \\
\midrule
\textsc{1sg} & hcí  'house'	& wihta ? (<wi-hta) & `my house' \\
 
\textsc{2sg} & hcí  'house'	& ðíhta (<ðí-hta)	& `your house' \\
 
\textsc{3sg} & hcí  'house'	& ihta (<i-hta)	& `his/her house' \\
 
\textsc{1du} & hcí  'house' & ąkóhta (< ąkó-hta) & `our house' \\
 
\textsc{1pl} & hcí  'house' & ąkóhtapi (<ąkó-hta-api) & `our house' \\
 
\textsc{2pl} & hcí  'house' & ðíhtaapi (<ðí-hta-api) & `your house' \\
 
\textsc{3pl} & hcí  'house' & ihta-api (<i-hta-api) & `their house' \\
\lspbottomrule
\end{tabular}
\end{table}

Quintero analyzes the possessive form \textit{-hta} as a noun or nominal element for two reasons: first, this stem is inflected by the same prefixes as the inalienable\is{inalienable possession} nouns (kinship terms), and secondly, if it would be analyzed as a verbal stem, the possessive inflection would be quite irregular (cf. \citealt[317]{Quintero2004}f). 

One problem with this reasoning is that one would have to expect that the nominal stem \textit{-hta} belongs to the group of inalienable nouns because it requires the inalienable\is{inalienable possession} series of prefixes. There is, however, no evidence for that. Secondly, the order of elements suggests that the \textit{-hta} stem is of verbal origin. If it would be nominal, it should precede the \isi{possessed} noun. Attributive nouns always precede the head nouns; all other modifying elements follow the head noun. That the pronominal prefixes are different from the ones for stative/inactive verbs is not necessarily an argument for the non-verbal character of the stem --- there are often deviations in possessive paradigms. Furthermore, this possessive form may be used autonomously without a \isi{possessed} noun, cf. the example in \REF{osagebroken}. This construction is not possible in \il{Ho-Chunk}Hoocąk. The utterance in \REF{osagebroken} would require the reflexive possessive prefix \textit{k-/kara-} in \il{Ho-Chunk}Hoocąk.

\ea	\ili{Osage} (\citealt[413]{Quintero2004}) \label{osagebroken}

\gll ąkóhta     akxa     Ø-xǫ́-api-ðe \\
\textsc{1pl.poss} \textsc{sbj} \textsc{3sg.sbj}-break-\textsc{pl-decl} \\
\glt `Ours is broken'
\z 

Part-whole relationships - at least with regard to inanimate\is{animacy} parts - seem to be expressed by means of a simple juxtaposition. However, I found only one example illustrating this in Quintero's grammar, cf. example \REF{osagegarage}.

\ea	\ili{Osage} (\citealt[423]{Quintero2004}) \label{osagegarage}

\gll oðíhtą hci      hcíže áðiitą-a \\
car       house door close-\textsc{imp} \\
\glt `Close the garage door!'
\z

To summarize: there is an alienable\is{alienable possession}/inalienable\is{inalienable possession} distinction in \ili{Osage} and it seems that kinship terms belong to the inalienable set of nouns (set\textsubscript{1}), while all other nouns belong to the alienable set of noun (set\textsubscript{2}); cf. \tabref{osagealienability}.

\begin{table}
\caption{Alienable vs. inalienable distinction in Osage} \label{osagealienability}
\begin{tabular}{ l l }
\lsptoprule
inalienable\is{inalienable possession} & alienable\is{alienable possession} \\
 \multicolumn{1}{c}{set\textsubscript{1}} &  \multicolumn{1}{c}{set\textsubscript{2}} \\
\midrule
 
kinship terms & 	all other nouns ? \\
\midrule
 \multicolumn{1}{c}{PC}	&  \multicolumn{1}{c}{PC} \\
\midrule
PRO-N\textsubscript{possessed} & (N\textsubscript{possessor}) N\textsubscript{possessed} PRO.POSS-\textit{hta} \\
\lspbottomrule
\end{tabular}
\end{table}

\subsection{Biloxi}\label{sec:helmbrecht:4.6}\label{biloxi}
\ili{Biloxi} was chosen as a representative of the Ohio Valley sub-branch of Siouan. The standard reference work with respect to a grammatical description is \citet{Einaudi1976}. She mentions two NP internal PC types in her grammar of \ili{Biloxi}, a) a juxtaposition of two \isi{nominals} to be used for all kinds of \isi{possessed} nouns, and b) pronominally inflected nouns designating body parts\is{body-part term} and kinship relations (cf. \citealt[57--68]{Einaudi1976}). Concerning a) the order of nouns in the juxtaposition PC is \isi{possessor} precedes \isi{possessed}. Concerning b) if body parts and kinship terms are \isi{possessed}, the \isi{possessed} nouns have to be inflected obligatorily with pronominal prefixes that are identical to the ones in verbs. This holds also for some intimate personal possessions such as `house', `clothing', etc. See two examples for the juxtaposed PC construction in \REF{biloxihouse} and two examples of the inflected PC construction in \REF{biloxiuncle}.

\ea \ili{Biloxi} (\citealt[139]{Einaudi1976}f) \label{biloxihouse}

\ea
\gll \k{a}ya   ti-k		\\				
man house-\textsc{det} \\
\glt `the man's house'

\ex 
\gll ama tupe k\k{a} \\
ground hole \textsc{det} \\
\glt `the ground's hole'
\z \z

\ea	\ili{Biloxi} (\citealt[139]{Einaudi1976}f) \label{biloxiuncle}

\ea \gll tuhe   Ø-tukąni       yandi	\\  
T.      \textsc{3sg}-uncle    \textsc{det} \\
\glt `Tuhe's uncle (mother's brother)'
\ex \gll ąya   Ø-anahį    k\k{a} \\
man \textsc{3sg}-hair   \textsc{det} \\
\glt `	people's hair'
\z \z

Full paradigms of \isi{inalienable possession} are given in \tabref{biloxiparadigm}.

\begin{table}
\caption{Paradigm of \isi{inalienable possession} in \ili{Biloxi} (\citealt[57]{Einaudi1976}f/62f)} \label{biloxiparadigm}
\begin{tabular}{l l l }
\lsptoprule
\isi{possessor}	& kinship term & \isi{body-part term} \\
& adi `father' & cake `hand' \\
\midrule
 
\textsc{1sg} & nk-adi	& nk-cake \\
 
\textsc{2sg} & iy-adi & i-cake \\
 
\textsc{3sg} & Ø-adi & Ø-cake \\
 
\textsc{1pl} & nk-ax-tu & nk-cak-tu \\
 
\textsc{2pl} & iy-adi-tu & i-cak-tu \\
 
\textsc{3pl} & ax-tu & cak-tu \\
\lspbottomrule
\end{tabular}
\end{table}

\begin{table}
\caption{Alienable vs. inalienable distinction in Biloxi} \label{biloxialienability}
\begin{tabular}{ l l }
\lsptoprule
inalienable\is{inalienable possession} & alienable\is{alienable possession} \\
\midrule
set\textsubscript{1} & set\textsubscript{2} \\
\midrule
kinship terms & all other nouns \\
body-part terms\is{body-part term} &  \\
intimate personal possessions & \\
such as 'house', clothing' & \\	
\midrule
PC &	PC \\
\midrule
PRO-N\textsubscript{possessed} DET & N\textsubscript{possessor}-N\textsubscript{possessed} DET \\
\lspbottomrule
\end{tabular}
\end{table}

I did not find any examples that illustrate how alienable\is{alienable possession} nouns are \isi{possessed} by SAP possessors, something like `my horse', `your car', etc.

\section{Conclusions}
There is an alienable\is{alienable possession}-inalienable\is{inalienable possession} distinction in one way or other in all Siouan languages, even in \ili{Biloxi}, as seen in \tabref{biloxialienability}, but there, the inalienable nouns (kinship, body parts\is{body-part term}) are inflected by means of the \isi{subject} prefixes. As the examination of PCs in the various Siouan languages shows, there are at least four kinds of constructions that are used to express possession on the NP level. The simplest construction is  juxtaposition, which is used in all sample languages except for \ili{Hidatsa}, for which no data were available. In\isi{alienable possession} is expressed in all sample languages with a series of possessive affixes directly attached to the \isi{possessed}. The sole exception is \il{Ho-Chunk}Hoocąk, which has no possessive affixes. There are two principal constructions that express \isi{alienable possession} in the sample Siouan languages. There is a construction that has a possessive marker attached to the stem indicating \isi{alienable possession}. The same set of possessive affixes appears with these constructions. This construction is not available in \il{Ho-Chunk}Hoocąk and \ili{Biloxi}. The second construction utilizes a verb of possession that is nominalized by a determiner and inflected by the same paradigm of possessive affixes. It follows the \isi{possessed} noun. This construction is missing in \ili{Missouri Valley Siouan} and in \ili{Biloxi}. I have no clear data for \ili{Osage}. The principle types of constructions that are used in Siouan languages to express possession are summarized in \tabref{siouandistribution} together with the semantic kinds of \isi{possessed} nouns. 

\begin{sidewaystable}
\caption{Distribution of NP-internal possessive constructions among Siouan languages} \label{siouandistribution}
\footnotesize
\begin{tabularx}{\textheight}{ lp{2cm}XXX }	
\lsptoprule
& \multicolumn{4}{c} {less marked \hspace{2em}  $\overrightarrow{\hspace{6cm}}$ \hspace{2em} more marked} \\
& juxtaposition N\textsubscript{poss'or} N\textsubscript{poss'ed} &POSS.PRO-N\textsubscript{poss'ed} & POSS.PRO-POSS-N\textsubscript{poss'ed} & N\textsubscript{poss'ed} POSS.PRO-verb.poss-DET \\
\midrule
\il{Apsaalooke}Crow & 1) part-whole & 1) body parts\is{body-part term}, kinship terms & rest, plus some exceptions &	Ø \\
& 2) others?	&  2) internal body parts\is{body-part term}  &  & \\
& & 3) `chest', `tail', `husband'  & & \\
& & 4) closely associated with \isi{possessor}, e.g.  clothing items, kin terms, cultural possession	  & & \\ 
\\[-.8em]
\ili{Hidatsa} & ? & 1) many kinship terms& rest & Ø \\
& & 2) body parts\is{body-part term} & & \\
& & 3) some clothing  items & & \\
\\[-.8em]
\ili{Mandan} & 1) association & 1) kinship terms & 1) kinship terms & ? \\
& 2) body parts\is{body-part term} &  2) ? & 2) ? & \\
\\[-.8em]
\ili{Lakota} & 1) ownership & 1) body parts\is{body-part term} & 1) kinship terms & 1) ownership \\
& 2) part-whole & 2) internal body parts\is{body-part term} & 2) ownership & 2) kinship \\
& & 3) kinship terms	 & 3) attribution of property & \\
& & & 4) association & \\
\\[-.8em]
\il{Ho-Chunk}Hoocąk & 1) part-whole & 	Ø	& Ø	& 1) kinship \\
& 2) boFmike
dy parts & &  & 2) domestic/ pet animals  \\
& 3) kinship & & & 3) rest \\
& 4) local nouns & & & \\
\\[-.8em]
\ili{Osage}	& part-whole & 	kinship terms	& ownership? & \\
\\[-.8em]
\ili{Biloxi}	& 1) part-whole & 1) kinship & 	Ø	& Ø \\
& 2) ownership & 2) body parts\is{body-part term} & & \\
& 3) rest	& 3) intimate personal belongings & & \\
& & (`house', `clothing') & & \\
\lspbottomrule
\end{tabularx}
\end{sidewaystable}


The nominalized verbs of possession appear only in \ili{Mississippi Valley Siouan}, most prominently in \il{Ho-Chunk}Hoocąk. \il{Ho-Chunk}Hoocąk is particular also with regard to the lack of the two middle construction types in Table 20; one could perhaps say that \il{Ho-Chunk}Hoocąk has not really grammaticalized a NP-internal \isi{possessive construction}:  juxtaposition is semantically the most abstract means, hence able to subsume all kinds of binary relations (among them also real ownership) and the verbal expression of possession is semantically the most concrete one, hence excluding many binary relations that are often expressed by means of possessive constructions (there is no possibility to express association, whole-part, attribution of property relations with these PCs).

Another interesting observation is that there is no neat classification of nouns with respect to the alienable\is{alienable possession}/inalienable\is{inalienable possession} distinction. Alienable and inalienable nouns are distributed over all kinds of PCs and it seems that the often observed markedness relations between alienable and inalienable PCs do not really hold in Siouan. For instance, juxtapositions as the least marked PCs comprise real ownership (\ili{Lakota}, \ili{Biloxi}) as well as body parts\is{body-part term} (\ili{Mandan}, \il{Ho-Chunk}Hoocąk) and kinship terms (\il{Ho-Chunk}Hoocąk). On the other hand, nominalized predicative PCs, which are the most complex PCs in this study, include not only real ownership (\ili{Lakota}, \il{Ho-Chunk}Hoocąk) but also kinship terms which are inalienable nouns. The two construction types in the middle columns in Table 20 show a markedness relation between inalienable and alienable nouns that is much clearer. The PC with the possessive pronouns attached to the \isi{possessed} nouns (second column from left) are chosen primarily for \isi{inalienable possession} (all languages except \il{Ho-Chunk}Hoocąk) and the PC with the added possession marker (\textsc{POSS}) are used overwhelmingly for \isi{alienable possession} such as real ownership or as a kind of rest category that always includes alienable nouns (all languages except \il{Ho-Chunk}Hoocąk). In \ili{Lakota} and \ili{Mandan}, however, kinship terms as \isi{possessed} nouns are included, which blurs this distinction to some degree. 
   
\section*{Abbreviations}

1, 2, 3, = first, second, third person; \textsc{a} = actor; AH = Animacy Hierarchy\is{animacy}; \textsc{appl.ben} = benefactive applicative; \textsc{appl.supess} = locative applicative superessive; \textsc{dat} = dative; \textsc{decl} = declarative; \textsc{def} = definite article; \textsc{e} = exclusive; \textsc{emph} = emphatic; \textsc{gen} = genitive; \textsc{i} = inclusive; \textsc{indef} = indefinite article; \textsc{obj} = \isi{object}; PC = \isi{possessive construction}; \textsc{pl} = plural; \textsc{poss pro} = \isi{possessive pronoun}; \textsc{prep} = preposition; \textsc{prop} = proper name; \textsc{refl.poss} = reflexive possession; SAP = speech act participant; \textsc{sbj} = \isi{subject}; \textsc{sg} = singular; \textsc{u} = undergoer.
 \printbibliography[heading=subbibliography,notkeyword=this]
 
\end{document}
